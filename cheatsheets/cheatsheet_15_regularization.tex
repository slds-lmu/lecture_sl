\documentclass{beamer}
\newcommand \beameritemnestingprefix{}


\usepackage[orientation=landscape,size=a0,scale=1.4,debug]{beamerposter}
\mode<presentation>{\usetheme{mlr}}


\usepackage[utf8]{inputenc} % UTF-8
\usepackage[english]{babel} % Language
\usepackage{hyperref} % Hyperlinks
\usepackage{ragged2e} % Text position
\usepackage[export]{adjustbox} % Image position
\usepackage[most]{tcolorbox}
\usepackage{amsmath}
\usepackage{mathtools}
\usepackage{dsfont}
\usepackage{verbatim}
\usepackage{amsmath}
\usepackage{amsfonts}
\usepackage{csquotes}
\usepackage{multirow}
\usepackage{longtable}
\usepackage[absolute,overlay]{textpos}
\usepackage{psfrag}
\usepackage{algorithm}
\usepackage{algpseudocode}
\usepackage{eqnarray}
\usepackage{arydshln}
\usepackage{tabularx}
\usepackage{placeins}
\usepackage{tikz}
\usepackage{setspace}
\usepackage{colortbl}
\usepackage{mathtools}
\usepackage{wrapfig}
\usepackage{bm}


\input{../latex-math/basic-math.tex}
% machine learning
\newcommand{\Xspace}{\mathcal{X}} % X, input space
\newcommand{\Yspace}{\mathcal{Y}} % Y, output space
\newcommand{\Zspace}{\mathcal{Z}} % Z, space of sampled datapoints
\newcommand{\nset}{\{1, \ldots, n\}} % set from 1 to n
\newcommand{\pset}{\{1, \ldots, p\}} % set from 1 to p
\newcommand{\gset}{\{1, \ldots, g\}} % set from 1 to g
\newcommand{\Pxy}{\mathbb{P}_{xy}} % P_xy
\newcommand{\Exy}{\mathbb{E}_{xy}} % E_xy: Expectation over random variables xy
\newcommand{\xv}{\mathbf{x}} % vector x (bold)
\newcommand{\xtil}{\tilde{\mathbf{x}}} % vector x-tilde (bold)
\newcommand{\yv}{\mathbf{y}} % vector y (bold)
\newcommand{\xy}{(\xv, y)} % observation (x, y)
\newcommand{\xvec}{\left(x_1, \ldots, x_p\right)^\top} % (x1, ..., xp)
\newcommand{\Xmat}{\mathbf{X}} % Design matrix
\newcommand{\allDatasets}{\mathds{D}} % The set of all datasets
\newcommand{\allDatasetsn}{\mathds{D}_n}  % The set of all datasets of size n
\newcommand{\D}{\mathcal{D}} % D, data
\newcommand{\Dn}{\D_n} % D_n, data of size n
\newcommand{\Dtrain}{\mathcal{D}_{\text{train}}} % D_train, training set
\newcommand{\Dtest}{\mathcal{D}_{\text{test}}} % D_test, test set
\newcommand{\xyi}[1][i]{\left(\xv^{(#1)}, y^{(#1)}\right)} % (x^i, y^i), i-th observation
\newcommand{\Dset}{\left( \xyi[1], \ldots, \xyi[n]\right)} % {(x1,y1)), ..., (xn,yn)}, data
\newcommand{\defAllDatasetsn}{(\Xspace \times \Yspace)^n} % Def. of the set of all datasets of size n
\newcommand{\defAllDatasets}{\bigcup_{n \in \N}(\Xspace \times \Yspace)^n} % Def. of the set of all datasets
\newcommand{\xdat}{\left\{ \xv^{(1)}, \ldots, \xv^{(n)}\right\}} % {x1, ..., xn}, input data
\newcommand{\ydat}{\left\{ \yv^{(1)}, \ldots, \yv^{(n)}\right\}} % {y1, ..., yn}, input data
\newcommand{\yvec}{\left(y^{(1)}, \hdots, y^{(n)}\right)^\top} % (y1, ..., yn), vector of outcomes
\newcommand{\greekxi}{\xi} % Greek letter xi
\renewcommand{\xi}[1][i]{\xv^{(#1)}} % x^i, i-th observed value of x
\newcommand{\yi}[1][i]{y^{(#1)}} % y^i, i-th observed value of y
\newcommand{\xivec}{\left(x^{(i)}_1, \ldots, x^{(i)}_p\right)^\top} % (x1^i, ..., xp^i), i-th observation vector
\newcommand{\xj}{\xv_j} % x_j, j-th feature
\newcommand{\xjvec}{\left(x^{(1)}_j, \ldots, x^{(n)}_j\right)^\top} % (x^1_j, ..., x^n_j), j-th feature vector
\newcommand{\phiv}{\mathbf{\phi}} % Basis transformation function phi
\newcommand{\phixi}{\mathbf{\phi}^{(i)}} % Basis transformation of xi: phi^i := phi(xi)

%%%%%% ml - models general
\newcommand{\lamv}{\bm{\lambda}} % lambda vector, hyperconfiguration vector
\newcommand{\Lam}{\Lambda}	 % Lambda, space of all hpos
% Inducer / Inducing algorithm
\newcommand{\preimageInducer}{\left(\defAllDatasets\right)\times\Lam} % Set of all datasets times the hyperparameter space
\newcommand{\preimageInducerShort}{\allDatasets\times\Lam} % Set of all datasets times the hyperparameter space
% Inducer / Inducing algorithm
\newcommand{\ind}{\mathcal{I}} % Inducer, inducing algorithm, learning algorithm

% continuous prediction function f
\newcommand{\ftrue}{f_{\text{true}}}  % True underlying function (if a statistical model is assumed)
\newcommand{\ftruex}{\ftrue(\xv)} % True underlying function (if a statistical model is assumed)
\newcommand{\fx}{f(\xv)} % f(x), continuous prediction function
\newcommand{\fdomains}{f: \Xspace \rightarrow \R^g} % f with domain and co-domain
\newcommand{\Hspace}{\mathcal{H}} % hypothesis space where f is from
\newcommand{\Hall}{\mathcal{H}_{\text{all}}} % unrestricted hypothesis space
\newcommand{\fbayes}{f^{\ast}} % Bayes-optimal model
\newcommand{\fxbayes}{f^{\ast}(\xv)} % Bayes-optimal model
\newcommand{\fkx}[1][k]{f_{#1}(\xv)} % f_j(x), discriminant component function
\newcommand{\fhspace}{\hat f_{\Hspace}} % fhat_H
\newcommand{\fh}{\hat{f}} % f hat, estimated prediction function
\newcommand{\fxh}{\fh(\xv)} % fhat(x)
\newcommand{\fxt}{f(\xv ~|~ \thetav)} % f(x | theta)
\newcommand{\fxi}{f\left(\xv^{(i)}\right)} % f(x^(i))
\newcommand{\fxih}{\hat{f}\left(\xv^{(i)}\right)} % f(x^(i))
\newcommand{\fxit}{f\big(\xv^{(i)} ~|~ \thetav\big)} % f(x^(i) | theta)
\newcommand{\fhD}{\fh_{\D}} % fhat_D, estimate of f based on D
\newcommand{\fhDtrain}{\fh_{\Dtrain}} % fhat_Dtrain, estimate of f based on D
\newcommand{\fhDnlam}{\fh_{\Dn, \lamv}} %model learned on Dn with hp lambda
\newcommand{\fhDlam}{\fh_{\D, \lamv}} %model learned on D with hp lambda
\newcommand{\fhDnlams}{\fh_{\Dn, \lamv^\ast}} %model learned on Dn with optimal hp lambda
\newcommand{\fhDlams}{\fh_{\D, \lamv^\ast}} %model learned on D with optimal hp lambda

% discrete prediction function h
\newcommand{\hx}{h(\xv)} % h(x), discrete prediction function
\newcommand{\hh}{\hat{h}} % h hat
\newcommand{\hxh}{\hat{h}(\xv)} % hhat(x)
\newcommand{\hxt}{h(\xv | \thetav)} % h(x | theta)
\newcommand{\hxi}{h\left(\xi\right)} % h(x^(i))
\newcommand{\hxit}{h\left(\xi ~|~ \thetav\right)} % h(x^(i) | theta)
\newcommand{\hbayes}{h^{\ast}} % Bayes-optimal classification model
\newcommand{\hxbayes}{h^{\ast}(\xv)} % Bayes-optimal classification model

% yhat
\newcommand{\yh}{\hat{y}} % yhat for prediction of target
\newcommand{\yih}{\hat{y}^{(i)}} % yhat^(i) for prediction of ith targiet
\newcommand{\resi}{\yi- \yih}

% theta
\newcommand{\thetah}{\hat{\theta}} % theta hat
\newcommand{\thetav}{\bm{\theta}} % theta vector
\newcommand{\thetavh}{\bm{\hat\theta}} % theta vector hat
\newcommand{\thetat}[1][t]{\thetav^{[#1]}} % theta^[t] in optimization
\newcommand{\thetatn}[1][t]{\thetav^{[#1 +1]}} % theta^[t+1] in optimization
\newcommand{\thetahDnlam}{\thetavh_{\Dn, \lamv}} %theta learned on Dn with hp lambda
\newcommand{\thetahDlam}{\thetavh_{\D, \lamv}} %theta learned on D with hp lambda
\newcommand{\mint}{\min_{\thetav \in \Theta}} % min problem theta
\newcommand{\argmint}{\argmin_{\thetav \in \Theta}} % argmin theta

% densities + probabilities
% pdf of x
\newcommand{\pdf}{p} % p
\newcommand{\pdfx}{p(\xv)} % p(x)
\newcommand{\pixt}{\pi(\xv~|~ \thetav)} % pi(x|theta), pdf of x given theta
\newcommand{\pixit}[1][i]{\pi\left(\xi[#1] ~|~ \thetav\right)} % pi(x^i|theta), pdf of x given theta
\newcommand{\pixii}[1][i]{\pi\left(\xi[#1]\right)} % pi(x^i), pdf of i-th x

% pdf of (x, y)
\newcommand{\pdfxy}{p(\xv,y)} % p(x, y)
\newcommand{\pdfxyt}{p(\xv, y ~|~ \thetav)} % p(x, y | theta)
\newcommand{\pdfxyit}{p\left(\xi, \yi ~|~ \thetav\right)} % p(x^(i), y^(i) | theta)

% pdf of x given y
\newcommand{\pdfxyk}[1][k]{p(\xv | y= #1)} % p(x | y = k)
\newcommand{\lpdfxyk}[1][k]{\log p(\xv | y= #1)} % log p(x | y = k)
\newcommand{\pdfxiyk}[1][k]{p\left(\xi | y= #1 \right)} % p(x^i | y = k)

% prior probabilities
\newcommand{\pik}[1][k]{\pi_{#1}} % pi_k, prior
\newcommand{\pih}{\hat{\pi}} % pi hat, estimated prior (binary classification)
\newcommand{\pikh}[1][k]{\hat{\pi}_{#1}} % pi_k hat, estimated prior
\newcommand{\lpik}[1][k]{\log \pi_{#1}} % log pi_k, log of the prior
\newcommand{\pit}{\pi(\thetav)} % Prior probability of parameter theta

% posterior probabilities
\newcommand{\post}{\P(y = 1 ~|~ \xv)} % P(y = 1 | x), post. prob for y=1
\newcommand{\postk}[1][k]{\P(y = #1 ~|~ \xv)} % P(y = k | y), post. prob for y=k
\newcommand{\pidomains}{\pi: \Xspace \rightarrow \unitint} % pi with domain and co-domain
\newcommand{\pibayes}{\pi^{\ast}} % Bayes-optimal classification model
\newcommand{\pixbayes}{\pi^{\ast}(\xv)} % Bayes-optimal classification model
\newcommand{\piastxtil}{\pi^{\ast}(\xtil)} % Bayes-optimal classification model
\newcommand{\piastkxtil}{\pi^{\ast}_k(\xtil)} % Bayes-optimal classification model for k-th class
\newcommand{\pix}{\pi(\xv)} % pi(x), P(y = 1 | x)
\newcommand{\piv}{\bm{\pi}} % pi, bold, as vector
\newcommand{\pikx}[1][k]{\pi_{#1}(\xv)} % pi_k(x), P(y = k | x)
\newcommand{\pikxt}[1][k]{\pi_{#1}(\xv ~|~ \thetav)} % pi_k(x | theta), P(y = k | x, theta)
\newcommand{\pixh}{\hat \pi(\xv)} % pi(x) hat, P(y = 1 | x) hat
\newcommand{\pikxh}[1][k]{\hat \pi_{#1}(\xv)} % pi_k(x) hat, P(y = k | x) hat
\newcommand{\pixih}{\hat \pi(\xi)} % pi(x^(i)) with hat
\newcommand{\pikxih}[1][k]{\hat \pi_{#1}(\xi)} % pi_k(x^(i)) with hat
\newcommand{\pdfygxt}{p(y ~|~\xv, \thetav)} % p(y | x, theta)
\newcommand{\pdfyigxit}{p\left(\yi ~|~\xi, \thetav\right)} % p(y^i |x^i, theta)
\newcommand{\lpdfygxt}{\log \pdfygxt } % log p(y | x, theta)
\newcommand{\lpdfyigxit}{\log \pdfyigxit} % log p(y^i |x^i, theta)

% probabilistic
\newcommand{\bayesrulek}[1][k]{\frac{\P(\xv | y= #1) \P(y= #1)}{\P(\xv)}} % Bayes rule
\newcommand{\muv}{\bm{\mu}} % expectation vector of Gaussian
\newcommand{\muk}[1][k]{\bm{\mu_{#1}}} % mean vector of class-k Gaussian (discr analysis)
\newcommand{\mukh}[1][k]{\bm{\hat{\mu}_{#1}}} % estimated mean vector of class-k Gaussian (discr analysis)

% residual and margin
\newcommand{\rx}{r(\xv)} % residual 
\newcommand{\eps}{\epsilon} % residual, stochastic
\newcommand{\epsv}{\bm{\epsilon}} % residual, stochastic, as vector
\newcommand{\epsi}{\epsilon^{(i)}} % epsilon^i, residual, stochastic
\newcommand{\epsh}{\hat{\epsilon}} % residual, estimated
\newcommand{\epsvh}{\hat{\epsv}} % residual, estimated, vector
\newcommand{\yf}{y \fx} % y f(x), margin
\newcommand{\yfi}{\yi \fxi} % y^i f(x^i), margin
\newcommand{\Sigmah}{\hat \Sigma} % estimated covariance matrix
\newcommand{\Sigmahj}{\hat \Sigma_j} % estimated covariance matrix for the j-th class
\newcommand{\nux}{\nu(\xv)} % nu(x) = y * f(x)

% ml - loss, risk, likelihood
\newcommand{\Lyf}{L\left(y, f\right)} % L(y, f), loss function
% \newcommand{\Lypi}{L\left(y, \pi\right)} % L(y, pi), loss function
\newcommand{\Lxy}{L\left(y, \fx\right)} % L(y, f(x)), loss function
\newcommand{\Lxyi}{L\left(\yi, \fxi\right)} % loss of observation
\newcommand{\Lxyt}{L\left(y, \fxt\right)} % loss with f parameterized
\newcommand{\Lxyit}{L\left(\yi, \fxit\right)} % loss of observation with f parameterized
\newcommand{\Lxym}{L\left(\yi, f\left(\bm{\tilde{x}}^{(i)} ~|~ \thetav\right)\right)} % loss of observation with f parameterized
\newcommand{\Lpixy}{L\left(y, \pix\right)} % loss in classification
% \newcommand{\Lpiy}{L\left(y, \pi\right)} % loss in classification
\newcommand{\Lpiv}{L\left(y, \piv\right)} % loss in classification
\newcommand{\Lpixyi}{L\left(\yi, \pixii\right)} % loss of observation in classification
\newcommand{\Lpixyt}{L\left(y, \pixt\right)} % loss with pi parameterized
\newcommand{\Lpixyit}{L\left(\yi, \pixit\right)} % loss of observation with pi parameterized
% \newcommand{\Lhy}{L\left(y, h\right)} % L(y, h), loss function on discrete classes
\newcommand{\Lhxy}{L\left(y, \hx\right)} % L(y, h(x)), loss function on discrete classes
\newcommand{\Lr}{L\left(r\right)} % L(r), loss defined on residual (reg) / margin (classif)
\newcommand{\lone}{|y - \fx|} % L1 loss
\newcommand{\ltwo}{\left(y - \fx\right)^2} % L2 loss
\newcommand{\lbernoullimp}{\ln(1 + \exp(-y \cdot \fx))} % Bernoulli loss for -1, +1 encoding
\newcommand{\lbernoullizo}{- y \cdot \fx + \log(1 + \exp(\fx))} % Bernoulli loss for 0, 1 encoding
\newcommand{\lcrossent}{- y \log \left(\pix\right) - (1 - y) \log \left(1 - \pix\right)} % cross-entropy loss
\newcommand{\lbrier}{\left(\pix - y \right)^2} % Brier score
\newcommand{\risk}{\mathcal{R}} % R, risk
\newcommand{\riskbayes}{\mathcal{R}^\ast}
\newcommand{\riskf}{\risk(f)} % R(f), risk
\newcommand{\riskdef}{\E_{y|\xv}\left(\Lxy \right)} % risk def (expected loss)
\newcommand{\riskt}{\mathcal{R}(\thetav)} % R(theta), risk
\newcommand{\riske}{\mathcal{R}_{\text{emp}}} % R_emp, empirical risk w/o factor 1 / n
\newcommand{\riskeb}{\bar{\mathcal{R}}_{\text{emp}}} % R_emp, empirical risk w/ factor 1 / n
\newcommand{\riskef}{\riske(f)} % R_emp(f)
\newcommand{\risket}{\mathcal{R}_{\text{emp}}(\thetav)} % R_emp(theta)
\newcommand{\riskr}{\mathcal{R}_{\text{reg}}} % R_reg, regularized risk
\newcommand{\riskrt}{\mathcal{R}_{\text{reg}}(\thetav)} % R_reg(theta)
\newcommand{\riskrf}{\riskr(f)} % R_reg(f)
\newcommand{\riskrth}{\hat{\mathcal{R}}_{\text{reg}}(\thetav)} % hat R_reg(theta)
\newcommand{\risketh}{\hat{\mathcal{R}}_{\text{emp}}(\thetav)} % hat R_emp(theta)
\newcommand{\LL}{\mathcal{L}} % L, likelihood
\newcommand{\LLt}{\mathcal{L}(\thetav)} % L(theta), likelihood
\newcommand{\LLtx}{\mathcal{L}(\thetav | \xv)} % L(theta|x), likelihood
\newcommand{\logl}{\ell} % l, log-likelihood
\newcommand{\loglt}{\logl(\thetav)} % l(theta), log-likelihood
\newcommand{\logltx}{\logl(\thetav | \xv)} % l(theta|x), log-likelihood
\newcommand{\errtrain}{\text{err}_{\text{train}}} % training error
\newcommand{\errtest}{\text{err}_{\text{test}}} % test error
\newcommand{\errexp}{\overline{\text{err}_{\text{test}}}} % avg training error

% lm
\newcommand{\thx}{\thetav^\top \xv} % linear model
\newcommand{\olsest}{(\Xmat^\top \Xmat)^{-1} \Xmat^\top \yv} % OLS estimator in LM

\input{../latex-math/ml-trees.tex}
\input{../latex-math/ml-nn.tex}


\title{Supervised Learning :\,: CHEAT SHEET} % Package title in header, \, adds thin space between ::
\newcommand{\packagedescription}{ % Package description in header
%	The \textbf{I2ML}: Introduction to Machine Learning course offers an introductory and applied overview of "supervised" Machine Learning. It is organized as a digital lecture.
}

\newlength{\columnheight} % Adjust depending on header height
\setlength{\columnheight}{84cm} 

\newtcolorbox{codebox}{%
	sharp corners,
	leftrule=0pt,
	rightrule=0pt,
	toprule=0pt,
	bottomrule=0pt,
	hbox}

\newtcolorbox{codeboxmultiline}[1][]{%
	sharp corners,
	leftrule=0pt,
	rightrule=0pt,
	toprule=0pt,
	bottomrule=0pt,
	#1}

\begin{document}
\begin{frame}[fragile]{}
\begin{columns}
	\begin{column}{.31\textwidth}
		\begin{beamercolorbox}[center]{postercolumn}
			\begin{minipage}{.98\textwidth}
				\parbox[t][\columnheight]{\textwidth}{

					\begin{myblock}{Regularization}

            Regularization is an effective technique to reduce overfitting.
  $$
  \riskrf = \riskef + \lambda \cdot J(f) = \sumin \Lxyi + \lambda \cdot J(f)
  $$
\begin{itemize}[$\bullet$]
  \setlength{\itemindent}{+.3in}
  \item $J(f)$: \textbf{complexity penalty}, \textbf{roughness penalty} or \textbf{regularizer}
  \item $\lambda \geq 0$: \textbf{complexity control} parameter
  \item The higher $\lambda$, the more we penalize complexity

  \item $\lambda = 0$: We just do simple ERM; $\lambda \to \infty$: we don't care about loss, models become as \enquote{simple} as possible

\item $\lambda$ is hard to set manually and is usually selected via CV

  \item As for $\riske$, $\riskr$ and $J$ are often defined in terms of $\thetav$: \\
  
  $$\riskrt = \risket + \lambda \cdot J(\thetav)$$

\end{itemize}

\end{myblock}

\begin{myblock}{Ridge Regression}

Use L2 penalty in linear regression:
\begin{eqnarray*}  
\thetah_{\text{ridge}} &=& \argmin_{\thetav} \sumin \left(\yi - \thetav^T \xi \right)^2 + \lambda \sum_{j=1}^{p} \theta_j^2 \\
%&=& \argmin_{\thetav} \left(\yv - \Xmat \thetav\right)^\top \left(\yv - \Xmat \thetav\right) + \lambda \thetav^\top \thetav \\
&=& \argmin_{\thetav} \| \yv - \Xmat \thetav \|_2^2  + \lambda \|\thetav\|_2^2
\end{eqnarray*}

Can still analytically solve this:
$$\thetah_{\text{ridge}} = ({\Xmat}^T \Xmat  + \lambda \id)^{-1} \Xmat^T\yv$$\\

Equivalent to solving the following constrained optimization problem:
\begin{eqnarray*}
\min_{\thetav} && \sumin \left(\yi - \fxit\right)^2 \\
  \text{s.t. } && \|\thetav\|_2^2  \leq t \\
\end{eqnarray*}

For special case of orthonormal design $\Xmat^{\top}\Xmat=\id$, $\thetah_{\text{OLS}}=\Xmat^{\top}\yv$:
$$\thetah_{\text{Ridge}}= ({\Xmat}^T \Xmat  + \lambda \id)^{-1} \Xmat^T\yv=((1+\lambda)\id)^{-1}\thetah_{\text{OLS}} = \frac{\thetah_{\text{OLS}}}{1+\lambda}\quad (\text{no sparsity})\vspace{-0.22cm}$$

\end{myblock}
				}
			\end{minipage}
		\end{beamercolorbox}
	\end{column}
	
%%%%%%%%%%%%%%%%%%%%%%%%%%%%%%%%%%%%%%%%%%%%%%%%%%%%%%%%%%%%%%%%%%%%%

\begin{column}{.31\textwidth}
\begin{beamercolorbox}[center]{postercolumn}
\begin{minipage}{.98\textwidth}
\parbox[t][\columnheight]{\textwidth}{
\begin{myblock}{}

\begin{codebox}
\textbf{Geometric Analysis}
\end{codebox}
Quadratic Taylor approx of unregularized $\risket$ around its minimizer $\thetah$, where $\bm{H}$ is the Hessian of $\risket$ at $\thetah$:
$$ \mathcal{\tilde R}_{\text{emp}}(\thetav)= \mathcal{R}_{\text{emp}}(\thetah) + \nabla_{\thetav} \mathcal{R}_{\text{emp}}(\thetah)\cdot(\thetav - \thetah) + \ \frac{1}{2} (\thetav - \thetah)^T \bm{H} (\thetav - \thetah) $$

Since we want a minimizer, first-order term is 0 and $\bm{H}$ is positive semidefinite:
$$ \mathcal{\tilde R}_{\text{emp}}(\thetav)= \mathcal{R}_{\text{emp}}(\thetah) + \ \frac{1}{2} (\thetav - \thetah)^T \bm{H} (\thetav - \thetah) $$

$$\nabla_{\thetav}\mathcal{\tilde R}_{\text{reg}}(\thetav) = 0 \rightarrow \hat{\thetav}_{\text{ridge}} = (\bm{H} + \lambda \id)^{-1}\bm{H} \thetah$$

$\bm{H}$ is a real symmetric matrix, it can be decomposed as $\bm{H} = \bm{Q} \bm{\Sigma} \bm{Q}^\top$:
\begin{aligned} 
    \hat{\thetav}_{\text{ridge}} &=\left(\bm{Q} \bm{\Sigma} \bm{Q}^{\top}+\lambda \id\right)^{-1} \bm{Q} \bm{\Sigma} \bm{Q}^{\top} \thetah \\ 
              &=\left[\bm{Q}(\bm{\Sigma}+\lambda \id) \bm{Q}^{\top}\right]^{-1} \bm{Q} \bm{\Sigma} \bm{Q}^{\top} \thetah \\ 
              &=\bm{Q}(\bm{\Sigma} + \lambda \id)^{-1} \bm{\Sigma} \bm{Q}^{\top} \thetah 
    \end{aligned}
  \end{myblock}
\begin{myblock}{Lasso Regression}

Use L1 penalty in linear regression:
\begin{align*}
\thetah_{\text{lasso}}&= \argmin_{\thetav} \sumin \left(\yi - \thetav^T \xi\right)^2 + \lambda \sum_{j=1}^{p} \vert\theta_j\vert\\
&= \argmin_{\thetav}\left(\yv - \Xmat \thetav\right)^\top \left(\yv - \Xmat \thetav\right) + \lambda \|\thetav\|_1
\end{align*}

Lasso can shrink some coeffs to zero, which gives sparse solutions. However, it has difficulties handling correlated predictors.\\

Equivalent to solving the following constrained optimization problem:
\begin{eqnarray*}
\min_{\thetav} && \sumin \left(\yi - \fxit\right)^2 \\
  \text{s.t. } && \|\thetav\|_1  \leq t \\
\end{eqnarray*}

For special case of orthonormal design $\Xmat^{\top}\Xmat=\id$, $\thetah_{\text{OLS}}=\Xmat^{\top}\yv$:
$$\thetah_{\text{lasso}}=\text{sign}(\thetah_{\text{OLS}})(\vert \thetah_{\text{OLS}} \vert - \lambda)_{+}\quad(\text{sparsity})\vspace{-0.1cm}.$$
Function $S(\theta,\lambda):=\text{sign}(\theta)(|\theta|-\lambda)_{+}$ is called \textbf{soft thresholding} operator: for $|\theta|\leq\lambda$ it returns $0$, whereas params $|\theta|>\lambda$ are shrunken toward $0$ by $\lambda$.\\
\end{myblock}
}
\end{minipage}
\end{beamercolorbox}
\end{column}

%%%%%%%%%%%%%%%%%%%%%%%%%%%%%%%%%%%%%%%%%%%%%%%%%%%%%%%%%%%%%%%%%%%%%%%%%%%%%%%%%%%%%%%%%%%%%%%%%%%%

\begin{column}{.31\textwidth}
\begin{beamercolorbox}[center]{postercolumn}
\begin{minipage}{.98\textwidth}
\parbox[t][\columnheight]{\textwidth}{
\begin{myblock}{}
\begin{codebox}
\textbf{Geometric Analysis}
\end{codebox}
$$ \mathcal{\tilde R}_{\text{emp}}(\thetav)= \mathcal{R}_{\text{emp}}(\thetah) + \ \frac{1}{2} (\thetav - \thetah)^T \bm{H} (\thetav - \thetah) $$
We assume the $\bm{H}$ is diagonal, with $H_{j,j} \geq 0$
$$\mathcal{\tilde R}_{\text{reg}}(\thetav) = \mathcal{R}_{\text{emp}}(\thetah) + \sum_j \left[ \frac{1}{2} H_{j,j} (\theta_j - \hat{\theta}_j)^2 \right] + \sum_j \lambda |\theta_j|$$

Minimize analytically:
     \begin{align*}\hat{\theta}_{\text{lasso},j} &= \sign(\hat{\theta}_j) \max \left\{ |\hat{\theta}_j| - \frac{\lambda}{H_{j,j}},0 \right\} \\
     &= \begin{cases} 
     \hat{\theta}_j + \frac{\lambda}{H_{j,j}} &, \text{if}   \;\hat{\theta}_j < -\frac{\lambda}{H_{j,j}} \\
       0 &, \text{if}   \;\hat{\theta}_j \in [-\frac{\lambda}{H_{j,j}}, \frac{\lambda}{H_{j,j}}] \\
     \hat{\theta}_j - \frac{\lambda}{H_{j,j}} &, \text{if}   \;\hat{\theta}_j > \frac{\lambda}{H_{j,j}} \\
     \end{cases}
     \end{align*}
If $H_{j,j} = 0$ exactly, $\thetah_{\text{lasso},j} = 0$
  
\end{myblock}

\begin{myblock}{More Regularization Methods}

\begin{codebox}
\textbf{Elastic Net Regression}
\end{codebox}
\begin{align*}
\mathcal{R}_{\text{elnet}}(\thetav) &=  \sumin (\yi - \thetav^\top \xi)^2 + \lambda_1 \|\thetav\|_1 + \lambda_2 \|\thetav\|_2^2 \\
&= \sumin (\yi - \thetav^\top \xi)^2 + \lambda \left( (1-\alpha) \|\thetav\|_1 + \alpha \|\thetav\|_2^2\right),\\
 \alpha=\frac{\lambda_2}{\lambda_1+\lambda_2}, \lambda=\lambda_1+\lambda_2
\end{align*}


\end{myblock}

  
  }
  
  \end{minipage}
  \end{beamercolorbox}
  \end{column}
  
\end{columns}
\end{frame}

\begin{frame}[fragile]{}
\begin{columns}
	\begin{column}{.31\textwidth}
		\begin{beamercolorbox}[center]{postercolumn}
			\begin{minipage}{.98\textwidth}
				\parbox[t][\columnheight]{\textwidth}{

\begin{myblock}{}
        \begin{codebox}
\textbf{Other Examples}
\end{codebox}

\begin{itemize}[$\bullet$]
  \setlength{\itemindent}{+.3in}
\item \textbf{L0}: not continuous or convex, NP-hard
$$\lambda \|\thetav\|_0 = \lambda \sum_j |\theta_j|^0$$

\item Smoothly Clipped Absolute Deviations (SCAD): non-convex, $\gamma>2$ controlls how fast penalty ``tapers off''
$$
\text{SCAD}(\theta \mid \lambda, \gamma)= \begin{cases}\lambda|\theta| & \text { if }|\theta| \leq \lambda \\ \frac{2 \gamma \lambda|\theta|-\theta^2-\lambda^2}{2(\gamma-1)} & \text { if } \lambda<|\theta|<\gamma \lambda \\ \frac{\lambda^2(\gamma+1)}{2} & \text { if }|\theta| \geq \gamma \lambda\end{cases}
$$

\item Minimax Concave Penalty (MCP): non-convex, $\gamma>1$ controlls how fast penalty ``tapers off''
$$
MCP(\theta | \lambda, \gamma)= \begin{cases}\lambda|\theta|-\frac{\theta^2}{2 \gamma}, & \text { if }|\theta| \leq \gamma \lambda \\ \frac{1}{2} \gamma \lambda^2, & \text { if }|\theta|>\gamma \lambda\end{cases}
$$
\end{itemize}

\end{myblock}
\begin{myblock}{Equivalence of Regularization}

  \begin{codebox}
\textbf{RRM vs MAP}
\end{codebox}
Regularized risk minimization (RRM) is the same as a maximum a posteriori (MAP) estimate in Bayes.\\

From Bayes theorem:
$$
p(\thetav | \xv, y) = \frac{p(y | \thetav, \xv) q(\thetav) }{p(y | \xv)} \propto 
p(y | \thetav, \xv) q(\thetav)
$$

The maximum a posteriori (MAP) estimator of $\thetav$ is now the minimizer of
$$
- \log p\left(y ~|~ \thetav, \xv\right) - \log q(\thetav).
$$

Identify the loss $\Lxyt$ with $-\log(p(y | \thetav, \xv))$:
\begin{itemize}[$\bullet$]
  \setlength{\itemindent}{+.3in}
 \item If $q(\thetav)$ is constant (i.e., we used a uniform, non-informative 
  prior), the second term is irrelevant and we arrive at ERM.
  \item If not, we can identify $J(\thetav) \propto -\log(q(\thetav))$, i.e., 
  the log-prior corresponds to the regularizer, and the additional $\lambda$, which controls the strength of our
  penalty, usually influences the peakedness / inverse variance / strength of our prior.
\end{itemize}

\end{myblock}

				}
			\end{minipage}
		\end{beamercolorbox}
	\end{column}
	
%%%%%%%%%%%%%%%%%%%%%%%%%%%%%%%%%%%%%%%%%%%%%%%%%%%%%%%%%%%%%%%%%%%%%

\begin{column}{.31\textwidth}
\begin{beamercolorbox}[center]{postercolumn}
\begin{minipage}{.98\textwidth}
\parbox[t][\columnheight]{\textwidth}{

\begin{myblock}{ }
  
   \begin{codebox}
\textbf{L2 vs Weight Decay}
\end{codebox}

$L2$ regularization with GD is equivalent to weight decay.\\

Optimize $L2$-regularized risk of a model $\fxt$ by GD:
$$
\min_{\thetav} \riskrt = \min_{\thetav} \risket + \frac{\lambda}{2} \|\thetav\|^2_2
$$

The gradient is
$$
\nabla_{\thetav} \riskrt = \nabla_{\thetav} \risket + \lambda \thetav
$$

We iteratively update $\thetav$ by step size \(\alpha\) times the negative gradient:
\begin{align*}
\thetav^{[\text{new}]} &= \thetav^{[\text{old}]} - \alpha \left(\nabla_{\thetav} \riske(\thetav^{[\text{old}]}) + \lambda \thetav^{[\text{old}]}\right) \\&=
\thetav^{[\text{old}]} (1 - \alpha \lambda) - \alpha \nabla_{\thetav} \riske(\thetav^{[\text{old}]})
\end{align*}
{\small
We see how $\thetav^{[old]}$ decays in magnitude -- for small $\alpha$ and $\lambda$.}
\end{myblock}

\begin{myblock}{Early stopping}
Early stopping is another technique to aboid overfitting, which makes traning process stop when validation error stops decreasing.
\begin{enumerate}
									\setlength{\itemindent}{+.3in}
									\item Split training data $\Dtrain$ into $\mathcal{D}_{\text{subtrain}}$ and $\mathcal{D}_{\text{val}}.$ 
									\item Train on $\mathcal{D}_{\text{subtrain}}$ and evaluate model using the validation set $\mathcal{D}_{\text{val}}$.
									\item Stop training when validation error stops decreasing.
									\item Use parameters of the previous step for the actual model.
								\end{enumerate}


\end{myblock}
}
\end{minipage}
\end{beamercolorbox}
\end{column}

%%%%%%%%%%%%%%%%%%%%%%%%%%%%%%%%%%%%%%%%%%%%%%%%%%%%%%%%%%%%%%%%%%%%%%%%%%%%%%%%%%%%%%%%%%%%%%%%%%%%

\begin{column}{.31\textwidth}
\begin{beamercolorbox}[center]{postercolumn}
\begin{minipage}{.98\textwidth}
\parbox[t][\columnheight]{\textwidth}{

 \begin{myblock}{}


\end{myblock}
  }
  
  \end{minipage}
  \end{beamercolorbox}
  \end{column}
  
\end{columns}
\end{frame}

\end{document}