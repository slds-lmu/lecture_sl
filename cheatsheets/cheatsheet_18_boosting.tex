\documentclass{beamer}
\newcommand \beameritemnestingprefix{}


\usepackage[orientation=landscape,size=a0,scale=1.4,debug]{beamerposter}
\mode<presentation>{\usetheme{mlr}}


\usepackage[utf8]{inputenc} % UTF-8
\usepackage[english]{babel} % Language
\usepackage{hyperref} % Hyperlinks
\usepackage{ragged2e} % Text position
\usepackage[export]{adjustbox} % Image position
\usepackage[most]{tcolorbox}
\usepackage{amsmath}
\usepackage{mathtools}
\usepackage{dsfont}
\usepackage{verbatim}
\usepackage{amsmath}
\usepackage{amsfonts}
\usepackage{csquotes}
\usepackage{multirow}
\usepackage{longtable}
\usepackage[absolute,overlay]{textpos}
\usepackage{psfrag}
\usepackage{algorithm}
\usepackage{algpseudocode}
\usepackage{eqnarray}
\usepackage{arydshln}
\usepackage{tabularx}
\usepackage{placeins}
\usepackage{tikz}
\usepackage{setspace}
\usepackage{colortbl}
\usepackage{mathtools}
\usepackage{wrapfig}
\usepackage{bm}


\input{../latex-math/basic-math.tex}
% machine learning
\newcommand{\Xspace}{\mathcal{X}} % X, input space
\newcommand{\Yspace}{\mathcal{Y}} % Y, output space
\newcommand{\Zspace}{\mathcal{Z}} % Z, space of sampled datapoints
\newcommand{\nset}{\{1, \ldots, n\}} % set from 1 to n
\newcommand{\pset}{\{1, \ldots, p\}} % set from 1 to p
\newcommand{\gset}{\{1, \ldots, g\}} % set from 1 to g
\newcommand{\Pxy}{\mathbb{P}_{xy}} % P_xy
\newcommand{\Exy}{\mathbb{E}_{xy}} % E_xy: Expectation over random variables xy
\newcommand{\xv}{\mathbf{x}} % vector x (bold)
\newcommand{\xtil}{\tilde{\mathbf{x}}} % vector x-tilde (bold)
\newcommand{\yv}{\mathbf{y}} % vector y (bold)
\newcommand{\xy}{(\xv, y)} % observation (x, y)
\newcommand{\xvec}{\left(x_1, \ldots, x_p\right)^\top} % (x1, ..., xp)
\newcommand{\Xmat}{\mathbf{X}} % Design matrix
\newcommand{\allDatasets}{\mathds{D}} % The set of all datasets
\newcommand{\allDatasetsn}{\mathds{D}_n}  % The set of all datasets of size n
\newcommand{\D}{\mathcal{D}} % D, data
\newcommand{\Dn}{\D_n} % D_n, data of size n
\newcommand{\Dtrain}{\mathcal{D}_{\text{train}}} % D_train, training set
\newcommand{\Dtest}{\mathcal{D}_{\text{test}}} % D_test, test set
\newcommand{\xyi}[1][i]{\left(\xv^{(#1)}, y^{(#1)}\right)} % (x^i, y^i), i-th observation
\newcommand{\Dset}{\left( \xyi[1], \ldots, \xyi[n]\right)} % {(x1,y1)), ..., (xn,yn)}, data
\newcommand{\defAllDatasetsn}{(\Xspace \times \Yspace)^n} % Def. of the set of all datasets of size n
\newcommand{\defAllDatasets}{\bigcup_{n \in \N}(\Xspace \times \Yspace)^n} % Def. of the set of all datasets
\newcommand{\xdat}{\left\{ \xv^{(1)}, \ldots, \xv^{(n)}\right\}} % {x1, ..., xn}, input data
\newcommand{\ydat}{\left\{ \yv^{(1)}, \ldots, \yv^{(n)}\right\}} % {y1, ..., yn}, input data
\newcommand{\yvec}{\left(y^{(1)}, \hdots, y^{(n)}\right)^\top} % (y1, ..., yn), vector of outcomes
\newcommand{\greekxi}{\xi} % Greek letter xi
\renewcommand{\xi}[1][i]{\xv^{(#1)}} % x^i, i-th observed value of x
\newcommand{\yi}[1][i]{y^{(#1)}} % y^i, i-th observed value of y
\newcommand{\xivec}{\left(x^{(i)}_1, \ldots, x^{(i)}_p\right)^\top} % (x1^i, ..., xp^i), i-th observation vector
\newcommand{\xj}{\xv_j} % x_j, j-th feature
\newcommand{\xjvec}{\left(x^{(1)}_j, \ldots, x^{(n)}_j\right)^\top} % (x^1_j, ..., x^n_j), j-th feature vector
\newcommand{\phiv}{\mathbf{\phi}} % Basis transformation function phi
\newcommand{\phixi}{\mathbf{\phi}^{(i)}} % Basis transformation of xi: phi^i := phi(xi)

%%%%%% ml - models general
\newcommand{\lamv}{\bm{\lambda}} % lambda vector, hyperconfiguration vector
\newcommand{\Lam}{\Lambda}	 % Lambda, space of all hpos
% Inducer / Inducing algorithm
\newcommand{\preimageInducer}{\left(\defAllDatasets\right)\times\Lam} % Set of all datasets times the hyperparameter space
\newcommand{\preimageInducerShort}{\allDatasets\times\Lam} % Set of all datasets times the hyperparameter space
% Inducer / Inducing algorithm
\newcommand{\ind}{\mathcal{I}} % Inducer, inducing algorithm, learning algorithm

% continuous prediction function f
\newcommand{\ftrue}{f_{\text{true}}}  % True underlying function (if a statistical model is assumed)
\newcommand{\ftruex}{\ftrue(\xv)} % True underlying function (if a statistical model is assumed)
\newcommand{\fx}{f(\xv)} % f(x), continuous prediction function
\newcommand{\fdomains}{f: \Xspace \rightarrow \R^g} % f with domain and co-domain
\newcommand{\Hspace}{\mathcal{H}} % hypothesis space where f is from
\newcommand{\Hall}{\mathcal{H}_{\text{all}}} % unrestricted hypothesis space
\newcommand{\fbayes}{f^{\ast}} % Bayes-optimal model
\newcommand{\fxbayes}{f^{\ast}(\xv)} % Bayes-optimal model
\newcommand{\fkx}[1][k]{f_{#1}(\xv)} % f_j(x), discriminant component function
\newcommand{\fhspace}{\hat f_{\Hspace}} % fhat_H
\newcommand{\fh}{\hat{f}} % f hat, estimated prediction function
\newcommand{\fxh}{\fh(\xv)} % fhat(x)
\newcommand{\fxt}{f(\xv ~|~ \thetav)} % f(x | theta)
\newcommand{\fxi}{f\left(\xv^{(i)}\right)} % f(x^(i))
\newcommand{\fxih}{\hat{f}\left(\xv^{(i)}\right)} % f(x^(i))
\newcommand{\fxit}{f\big(\xv^{(i)} ~|~ \thetav\big)} % f(x^(i) | theta)
\newcommand{\fhD}{\fh_{\D}} % fhat_D, estimate of f based on D
\newcommand{\fhDtrain}{\fh_{\Dtrain}} % fhat_Dtrain, estimate of f based on D
\newcommand{\fhDnlam}{\fh_{\Dn, \lamv}} %model learned on Dn with hp lambda
\newcommand{\fhDlam}{\fh_{\D, \lamv}} %model learned on D with hp lambda
\newcommand{\fhDnlams}{\fh_{\Dn, \lamv^\ast}} %model learned on Dn with optimal hp lambda
\newcommand{\fhDlams}{\fh_{\D, \lamv^\ast}} %model learned on D with optimal hp lambda

% discrete prediction function h
\newcommand{\hx}{h(\xv)} % h(x), discrete prediction function
\newcommand{\hh}{\hat{h}} % h hat
\newcommand{\hxh}{\hat{h}(\xv)} % hhat(x)
\newcommand{\hxt}{h(\xv | \thetav)} % h(x | theta)
\newcommand{\hxi}{h\left(\xi\right)} % h(x^(i))
\newcommand{\hxit}{h\left(\xi ~|~ \thetav\right)} % h(x^(i) | theta)
\newcommand{\hbayes}{h^{\ast}} % Bayes-optimal classification model
\newcommand{\hxbayes}{h^{\ast}(\xv)} % Bayes-optimal classification model

% yhat
\newcommand{\yh}{\hat{y}} % yhat for prediction of target
\newcommand{\yih}{\hat{y}^{(i)}} % yhat^(i) for prediction of ith targiet
\newcommand{\resi}{\yi- \yih}

% theta
\newcommand{\thetah}{\hat{\theta}} % theta hat
\newcommand{\thetav}{\bm{\theta}} % theta vector
\newcommand{\thetavh}{\bm{\hat\theta}} % theta vector hat
\newcommand{\thetat}[1][t]{\thetav^{[#1]}} % theta^[t] in optimization
\newcommand{\thetatn}[1][t]{\thetav^{[#1 +1]}} % theta^[t+1] in optimization
\newcommand{\thetahDnlam}{\thetavh_{\Dn, \lamv}} %theta learned on Dn with hp lambda
\newcommand{\thetahDlam}{\thetavh_{\D, \lamv}} %theta learned on D with hp lambda
\newcommand{\mint}{\min_{\thetav \in \Theta}} % min problem theta
\newcommand{\argmint}{\argmin_{\thetav \in \Theta}} % argmin theta

% densities + probabilities
% pdf of x
\newcommand{\pdf}{p} % p
\newcommand{\pdfx}{p(\xv)} % p(x)
\newcommand{\pixt}{\pi(\xv~|~ \thetav)} % pi(x|theta), pdf of x given theta
\newcommand{\pixit}[1][i]{\pi\left(\xi[#1] ~|~ \thetav\right)} % pi(x^i|theta), pdf of x given theta
\newcommand{\pixii}[1][i]{\pi\left(\xi[#1]\right)} % pi(x^i), pdf of i-th x

% pdf of (x, y)
\newcommand{\pdfxy}{p(\xv,y)} % p(x, y)
\newcommand{\pdfxyt}{p(\xv, y ~|~ \thetav)} % p(x, y | theta)
\newcommand{\pdfxyit}{p\left(\xi, \yi ~|~ \thetav\right)} % p(x^(i), y^(i) | theta)

% pdf of x given y
\newcommand{\pdfxyk}[1][k]{p(\xv | y= #1)} % p(x | y = k)
\newcommand{\lpdfxyk}[1][k]{\log p(\xv | y= #1)} % log p(x | y = k)
\newcommand{\pdfxiyk}[1][k]{p\left(\xi | y= #1 \right)} % p(x^i | y = k)

% prior probabilities
\newcommand{\pik}[1][k]{\pi_{#1}} % pi_k, prior
\newcommand{\pih}{\hat{\pi}} % pi hat, estimated prior (binary classification)
\newcommand{\pikh}[1][k]{\hat{\pi}_{#1}} % pi_k hat, estimated prior
\newcommand{\lpik}[1][k]{\log \pi_{#1}} % log pi_k, log of the prior
\newcommand{\pit}{\pi(\thetav)} % Prior probability of parameter theta

% posterior probabilities
\newcommand{\post}{\P(y = 1 ~|~ \xv)} % P(y = 1 | x), post. prob for y=1
\newcommand{\postk}[1][k]{\P(y = #1 ~|~ \xv)} % P(y = k | y), post. prob for y=k
\newcommand{\pidomains}{\pi: \Xspace \rightarrow \unitint} % pi with domain and co-domain
\newcommand{\pibayes}{\pi^{\ast}} % Bayes-optimal classification model
\newcommand{\pixbayes}{\pi^{\ast}(\xv)} % Bayes-optimal classification model
\newcommand{\piastxtil}{\pi^{\ast}(\xtil)} % Bayes-optimal classification model
\newcommand{\piastkxtil}{\pi^{\ast}_k(\xtil)} % Bayes-optimal classification model for k-th class
\newcommand{\pix}{\pi(\xv)} % pi(x), P(y = 1 | x)
\newcommand{\piv}{\bm{\pi}} % pi, bold, as vector
\newcommand{\pikx}[1][k]{\pi_{#1}(\xv)} % pi_k(x), P(y = k | x)
\newcommand{\pikxt}[1][k]{\pi_{#1}(\xv ~|~ \thetav)} % pi_k(x | theta), P(y = k | x, theta)
\newcommand{\pixh}{\hat \pi(\xv)} % pi(x) hat, P(y = 1 | x) hat
\newcommand{\pikxh}[1][k]{\hat \pi_{#1}(\xv)} % pi_k(x) hat, P(y = k | x) hat
\newcommand{\pixih}{\hat \pi(\xi)} % pi(x^(i)) with hat
\newcommand{\pikxih}[1][k]{\hat \pi_{#1}(\xi)} % pi_k(x^(i)) with hat
\newcommand{\pdfygxt}{p(y ~|~\xv, \thetav)} % p(y | x, theta)
\newcommand{\pdfyigxit}{p\left(\yi ~|~\xi, \thetav\right)} % p(y^i |x^i, theta)
\newcommand{\lpdfygxt}{\log \pdfygxt } % log p(y | x, theta)
\newcommand{\lpdfyigxit}{\log \pdfyigxit} % log p(y^i |x^i, theta)

% probabilistic
\newcommand{\bayesrulek}[1][k]{\frac{\P(\xv | y= #1) \P(y= #1)}{\P(\xv)}} % Bayes rule
\newcommand{\muv}{\bm{\mu}} % expectation vector of Gaussian
\newcommand{\muk}[1][k]{\bm{\mu_{#1}}} % mean vector of class-k Gaussian (discr analysis)
\newcommand{\mukh}[1][k]{\bm{\hat{\mu}_{#1}}} % estimated mean vector of class-k Gaussian (discr analysis)

% residual and margin
\newcommand{\rx}{r(\xv)} % residual 
\newcommand{\eps}{\epsilon} % residual, stochastic
\newcommand{\epsv}{\bm{\epsilon}} % residual, stochastic, as vector
\newcommand{\epsi}{\epsilon^{(i)}} % epsilon^i, residual, stochastic
\newcommand{\epsh}{\hat{\epsilon}} % residual, estimated
\newcommand{\epsvh}{\hat{\epsv}} % residual, estimated, vector
\newcommand{\yf}{y \fx} % y f(x), margin
\newcommand{\yfi}{\yi \fxi} % y^i f(x^i), margin
\newcommand{\Sigmah}{\hat \Sigma} % estimated covariance matrix
\newcommand{\Sigmahj}{\hat \Sigma_j} % estimated covariance matrix for the j-th class
\newcommand{\nux}{\nu(\xv)} % nu(x) = y * f(x)

% ml - loss, risk, likelihood
\newcommand{\Lyf}{L\left(y, f\right)} % L(y, f), loss function
% \newcommand{\Lypi}{L\left(y, \pi\right)} % L(y, pi), loss function
\newcommand{\Lxy}{L\left(y, \fx\right)} % L(y, f(x)), loss function
\newcommand{\Lxyi}{L\left(\yi, \fxi\right)} % loss of observation
\newcommand{\Lxyt}{L\left(y, \fxt\right)} % loss with f parameterized
\newcommand{\Lxyit}{L\left(\yi, \fxit\right)} % loss of observation with f parameterized
\newcommand{\Lxym}{L\left(\yi, f\left(\bm{\tilde{x}}^{(i)} ~|~ \thetav\right)\right)} % loss of observation with f parameterized
\newcommand{\Lpixy}{L\left(y, \pix\right)} % loss in classification
% \newcommand{\Lpiy}{L\left(y, \pi\right)} % loss in classification
\newcommand{\Lpiv}{L\left(y, \piv\right)} % loss in classification
\newcommand{\Lpixyi}{L\left(\yi, \pixii\right)} % loss of observation in classification
\newcommand{\Lpixyt}{L\left(y, \pixt\right)} % loss with pi parameterized
\newcommand{\Lpixyit}{L\left(\yi, \pixit\right)} % loss of observation with pi parameterized
% \newcommand{\Lhy}{L\left(y, h\right)} % L(y, h), loss function on discrete classes
\newcommand{\Lhxy}{L\left(y, \hx\right)} % L(y, h(x)), loss function on discrete classes
\newcommand{\Lr}{L\left(r\right)} % L(r), loss defined on residual (reg) / margin (classif)
\newcommand{\lone}{|y - \fx|} % L1 loss
\newcommand{\ltwo}{\left(y - \fx\right)^2} % L2 loss
\newcommand{\lbernoullimp}{\ln(1 + \exp(-y \cdot \fx))} % Bernoulli loss for -1, +1 encoding
\newcommand{\lbernoullizo}{- y \cdot \fx + \log(1 + \exp(\fx))} % Bernoulli loss for 0, 1 encoding
\newcommand{\lcrossent}{- y \log \left(\pix\right) - (1 - y) \log \left(1 - \pix\right)} % cross-entropy loss
\newcommand{\lbrier}{\left(\pix - y \right)^2} % Brier score
\newcommand{\risk}{\mathcal{R}} % R, risk
\newcommand{\riskbayes}{\mathcal{R}^\ast}
\newcommand{\riskf}{\risk(f)} % R(f), risk
\newcommand{\riskdef}{\E_{y|\xv}\left(\Lxy \right)} % risk def (expected loss)
\newcommand{\riskt}{\mathcal{R}(\thetav)} % R(theta), risk
\newcommand{\riske}{\mathcal{R}_{\text{emp}}} % R_emp, empirical risk w/o factor 1 / n
\newcommand{\riskeb}{\bar{\mathcal{R}}_{\text{emp}}} % R_emp, empirical risk w/ factor 1 / n
\newcommand{\riskef}{\riske(f)} % R_emp(f)
\newcommand{\risket}{\mathcal{R}_{\text{emp}}(\thetav)} % R_emp(theta)
\newcommand{\riskr}{\mathcal{R}_{\text{reg}}} % R_reg, regularized risk
\newcommand{\riskrt}{\mathcal{R}_{\text{reg}}(\thetav)} % R_reg(theta)
\newcommand{\riskrf}{\riskr(f)} % R_reg(f)
\newcommand{\riskrth}{\hat{\mathcal{R}}_{\text{reg}}(\thetav)} % hat R_reg(theta)
\newcommand{\risketh}{\hat{\mathcal{R}}_{\text{emp}}(\thetav)} % hat R_emp(theta)
\newcommand{\LL}{\mathcal{L}} % L, likelihood
\newcommand{\LLt}{\mathcal{L}(\thetav)} % L(theta), likelihood
\newcommand{\LLtx}{\mathcal{L}(\thetav | \xv)} % L(theta|x), likelihood
\newcommand{\logl}{\ell} % l, log-likelihood
\newcommand{\loglt}{\logl(\thetav)} % l(theta), log-likelihood
\newcommand{\logltx}{\logl(\thetav | \xv)} % l(theta|x), log-likelihood
\newcommand{\errtrain}{\text{err}_{\text{train}}} % training error
\newcommand{\errtest}{\text{err}_{\text{test}}} % test error
\newcommand{\errexp}{\overline{\text{err}_{\text{test}}}} % avg training error

% lm
\newcommand{\thx}{\thetav^\top \xv} % linear model
\newcommand{\olsest}{(\Xmat^\top \Xmat)^{-1} \Xmat^\top \yv} % OLS estimator in LM

\input{../latex-math/ml-trees.tex}
\input{../latex-math/ml-nn.tex}
\input{../latex-math/ml-ensembles.tex}


\title{Supervised Learning :\,: CHEAT SHEET} % Package title in header, \, adds thin space between ::
\newcommand{\packagedescription}{ % Package description in header
%	The \textbf{I2ML}: Introduction to Machine Learning course offers an introductory and applied overview of "supervised" Machine Learning. It is organized as a digital lecture.
}

\newlength{\columnheight} % Adjust depending on header height
\setlength{\columnheight}{84cm} 

\newtcolorbox{codebox}{%
	sharp corners,
	leftrule=0pt,
	rightrule=0pt,
	toprule=0pt,
	bottomrule=0pt,
	hbox}

\newtcolorbox{codeboxmultiline}[1][]{%
	sharp corners,
	leftrule=0pt,
	rightrule=0pt,
	toprule=0pt,
	bottomrule=0pt,
	#1}

\begin{document}
\begin{frame}[fragile]{}
\begin{columns}
	\begin{column}{.31\textwidth}
		\begin{beamercolorbox}[center]{postercolumn}
			\begin{minipage}{.98\textwidth}
				\parbox[t][\columnheight]{\textwidth}{

					\begin{myblock}{AdaBoost}
						\textbf{Boosting} is a homogeneous ensemble method that takes a weak classifier and sequentially apply it to modified versions of the training data.


							\begin{algorithm}[H]
								\begin{algorithmic}[1]
									\State Initialize observation weights: $w^{[1](i)} = \frac{1}{n} \quad \forall i \in \nset$
									\For {$m = 1 \to M$}
									\State Fit classifier to training data with weights $\wm$ and get $\blh$
									\State Calculate weighted in-sample misclassification rate
									$$
									\errm = \sumin \wmi \cdot \mathds{1}_{\{\yi \,\neq\, \blh(\xi)\}}
									$$
									\State Compute: $ \betamh = \frac{1}{2} \log \left( \frac{1 - \errm}{\errm}\right)$
									\State Set: $w^{[m+1](i)} = \wmi \cdot \exp\left(- \betamh \cdot
									\yi \cdot \blh(\xi)\right) $
									\State Normalize $w^{[m+1](i)}$ such that $\sumin w^{[m+1](i)} = 1$
									\EndFor
									\State Output: $\fxh = \sum_{m=1}^{M} \betamh \blh(\xv)$
								\end{algorithmic}
								\caption{AdaBoost $\Yspace = \setmp$}
							\end{algorithm}
							
\fbox{
									\parbox{\dimexpr\textwidth-2\fboxsep-2\fboxrule}{
										\begin{table}[] 
											\small
											\renewcommand{\arraystretch}{1.25} %<- modify value to suit your needs
											\begin{tabular}{c|lll}
												 & Random Forest & AdaBoost\\ \hline
												Base Learners & typically deeper decision trees & weak learners, e.g. only stumps \\
												Weights & equal & different, depending on predictive accuracy\\
												Structure & independent BLs  & sequential, order matter \\
												Aim & variance reduction & bias and variance reduction\\
												Overfit & tends not to & tends to \\
												\end{tabular}
										\end{table}
									}
								}\\

							
						\end{myblock} 


            \begin{myblock}{Gradient Boosting}

We want to learn an additive model: $
\fx = \sum_{m=1}^M \alpha^{[m]} \blxt$

Hence, we minimize the empirical risk:
$$
\riskef = \sum_{i=1}^n L\left(\yi,\fxi \right) =
\sum_{i=1}^n L\left(\yi, \sum_{m=1}^M \alpha^{[m]} b(\xi, \thetav^{[m]}) %\blxt
\right)
$$

And add additive components in a greedy fashion by sequentially minimizing the risk only w.r.t. the next additive component:
$$ \min \limits_{\alpha, \bm{\theta}} \sum_{i=1}^n L\left(\yi, \fmdh\left(\xi\right) + \alpha b\left(\xi, \bm{\theta}\right)\right) $$

Doing this iteratively is called \textbf{forward stagewise additive modeling}.


	\end{myblock} 

				}
			\end{minipage}
		\end{beamercolorbox}
	\end{column}
	
%%%%%%%%%%%%%%%%%%%%%%%%%%%%%%%%%%%%%%%%%%%%%%%%%%%%%%%%%%%%%%%%%%%%%

\begin{column}{.31\textwidth}
\begin{beamercolorbox}[center]{postercolumn}
\begin{minipage}{.98\textwidth}
\parbox[t][\columnheight]{\textwidth}{

\begin{myblock}{}
  \begin{algorithm}[H]
							%	\begin{footnotesize}
									\begin{center}
										\caption{Gradient Boosting Algorithm}
										\begin{algorithmic}[1]
											\State Initialize $\hat{f}^{[0]}(\xv) = \argmin_{\bm{\theta}} \sumin L(\yi, b(\xi, \bm{\theta}))$
											%\State Set the learning rate $\beta$ to a small constant value
											\For{$m = 1 \to M$}
											\State For all $i$: $\rmi = -\left[\pd{\Lxyi}{\fxi}\right]_{f=\fmdh}$
											\State Fit a regression base learner to the pseudo-residuals $\rmi$:
											\State $\thetamh = \argmin \limits_{\bm{\theta}} \sumin (\rmi - b(\xi, \bm{\theta}))^2$
											%\State Line search: $\betamh = \argmin_{\beta} \sumin L(\yi, \fmd(\xv) + \beta b(\xv, \thetamh))$
											\State Set $\betam$ to $\beta$ being a small constant value or via line search
											\State Update $\fmh(\xv) = \fmdh(\xv) + \betam b(\xv, \thetamh)$
											\EndFor
											\State Output $\fh(\xv) = \hat{f}^{[M]}(\xv)$
										\end{algorithmic}
									\end{center}
							%	\end{footnotesize}
							\end{algorithm}

              \begin{codebox} 
            \textbf{Gradient Boosting with Trees}
            \end{codebox}

            Tree can be seen as additive model: $ b(\xv) = \sum_{t=1}^{T} c_t \mathds{1}_{\{\xv \in R_t\}} $, $R_t$ are the terminal regions, $c_t$ are terminal constants.
 
            GB with trees is still additive:
$
  \fm(\xv) = \fmd(\xv) +  \alpha^{[m]} \bl(\xv) 
         = \fmd(\xv) +  \alpha^{[m]} \sum_{t=1}^{\Tm} \ctm \mathds{1}_{\{\xv \in \Rtm\}}
        &= \fmd(\xv) +  \sum_{t=1}^{\Tm} \ctmt \mathds{1}_{\{\xv \in \Rtm\}}$

With $\ctmt = \alpha^{[m]} \cdot \ctm = \argmin_{c} \sum_{\xi \in \Rtm} L(\yi, \fmd(\xi) + c)$.

\begin{algorithm}[H]
  %\begin{footnotesize}
  \begin{center}
  \caption{Gradient Boosting for $g$-class Classification.}
    \begin{algorithmic}[1]
      \State Initialize $f_{k}^{[0]}(\xv) = 0,\ k = 1,\ldots,g$
      \For{$m = 1 \to M$}
          \State Set $\pikx = \frac{\exp(f_k^{[m]}(\xv))}{\sum_j \exp(f_j^{[m]}(\xv))}, k = 1,\ldots,g$
            \For{$k = 1 \to g$}
              \State For all $i$: Compute $\rmi_k = \mathds{1}_{\{\yi = k\}} - \pi_k(\xi)$
              \State Fit regr. tree to the $\rmi_k$ giving terminal regions $R_{tk}^{[m]}$
              \State Compute
              \State \hskip\algorithmicindent\relax $\hat{c}_{tk}^{[m]} =
                \frac{g-1}{g}\frac{\sum_{\xi \in R_{tk}^{[m]}} \rmi_k}{\sum_{\xi \in R_{tk}^{[m]}} \left|\rmi_k\right|\left(1 - \left|\rmi_k\right|\right)}$
              \State Update $\hat{f}_k^{[m]}(\xv) = \hat{f}_k^{[m-1]}(\xv) + \sum_t \hat{c}_{tk}^{[m]} \mathds{1}_{\{\xv \in R_{tk}^{[m]}\}}$
            \EndFor
      \EndFor
    \State Output $\hat{f}_1^{[M]}, \ldots, \hat{f}_g^{[M]}$
    \end{algorithmic}
    \end{center}
    %\end{footnotesize}
\end{algorithm}
  \end{myblock}


  \begin{myblock}{XGBoost}
XGBoost uses stochastic GB for data subsampling with regularization:   
$$
    \riskr^{[m]} = \sum_{i=1}^{n} L(\yi, \fmd(\xi) + \bl(\xi))
    + \lambda_1 J_1(\bl) + \lambda_2 J_2(\bl) + \lambda_3 J_3(\bl)
$$


\end{myblock}

}
\end{minipage}
\end{beamercolorbox}
\end{column}

%%%%%%%%%%%%%%%%%%%%%%%%%%%%%%%%%%%%%%%%%%%%%%%%%%%%%%%%%%%%%%%%%%%%%%%%%%%%%%%%%%%%%%%%%%%%%%%%%%%%

\begin{column}{.31\textwidth}
\begin{beamercolorbox}[center]{postercolumn}
\begin{minipage}{.98\textwidth}
\parbox[t][\columnheight]{\textwidth}{
\begin{myblock}{}
\begin{itemize}[$\bullet$] 
  \setlength{\itemindent}{+.3in}
    \item $J_1(\bl) = T^{[m]}$:  Nr of leaves to penalize tree depth
    \item $J_2(\bl) = \left\|\mathbf{c}^{[m]}\right\|^2_2$:  L2 penalty over leaf values 
    \item $J_3(\bl) = \left\|\mathbf{c}^{[m]}\right\|_1$: L1 penalty over leaf values 
  \end{itemize}

\end{myblock}

\begin{myblock}{Component Wise Boosting}
For CWB we generalize to multiple base learner sets $\{ \mathcal{B}_1, ... \mathcal{B}_J \}$ with associated parameter spaces
$\{ \bm{\Theta}_1, ... \bm{\Theta}_J \}$.


\begin{algorithm}[H]
  \begin{center}
  \caption{Componentwise Gradient Boosting.}
    \begin{algorithmic}[1]
      \State Initialize $\fm[0](\xv) = \argmin_{\theta_0\in\R} \sum  \limits_{i=1}^n L(\yi, \theta_0)$
      \For{$m = 1 \to M$}
        \State For all $i$: $\rmi = -\left[\pd{L(y, f)}{f}\right]_{f=\fmd(\xi),y=\yi}$
        \For {$j= 1\to J$}
          \State Fit regression base learner $b_j \in \mathcal{B}_j$ to the vector of pseudo-residuals $\rmm$:
          \State $\thetamh_j = \argmin_{\thetav \in \bm{\Theta_j}} \sum  \limits_{i=1}^n
          (\rmi - b_j(\xi, \thetav))^2$
        \EndFor
        \State $j^{[m]} = \argmin_{j} \sum  \limits_{i=1}^n (\rmi - \hat{b}_j(\xi, \thetamh_j))^2$
        \State Update $\fm(\xv) = \fmd(\xv) + \alpha \hat{b}_{\hat{j}}(\xv, \thetamh_{j^{[m]}})$
      \EndFor
      \State Output $\fh(\xv) = \fm[M](\xv)$
    \end{algorithmic}
    \end{center}
\end{algorithm}

\begin{codebox}
  \textbf{Handling Categorical Features in CWB}
\end{codebox}
\begin{itemize}[$\bullet$] 
  \setlength{\itemindent}{+.3in}
    \item 
        One base learner to simultaneously estimate all categories: 
        $$b_j(x_j | \thetav_j) = \sum_{g=1}^G \theta_{j,g}\mathds{1}_{\{g = x_j\}} = (\mathds{1}_{\{x_j = 1\}}, ..., \mathds{1}_{\{x_j = G\}}) \thetav_j$$
        Hence, $b_j$ incorporates a one-hot encoded feature with group means $\thetav\in\R^G$ as estimators. 
    
    \item 
        One binary base learner per category: $b_{j,g}(x_j | \theta_{j,g}) = \theta_{j,g}\mathds{1}_{\{g = x_j\}}$\\
        Including all categories of the feature means adding $G$ base learners $b_{j,1}, \dots, b_{j,G}$ %with each accounting for one specific class.
  \end{itemize}
  
  \begin{codebox}
  \textbf{Handling Intercept in CWB}
\end{codebox}
  Loss-optimal constant $\fm[0](\xv)$ as an initial model intercept.
%\vspace{0.3cm}
\begin{itemize}[$\bullet$] 
  \setlength{\itemindent}{+.3in}
\item Include an intercept BL
\item Add BL $b_{\text{int}} = \theta$ as potential candidate considered in each iteration and remove intercept from all linear BLs, i.e., $b_j(\xv) = \theta_j x_j$.
    Final intercept is given as $\fm[0](\xv) + \hat{\theta}$.
   \end{itemize}

\end{myblock}
  }
  
  \end{minipage}
  \end{beamercolorbox}
  \end{column}
  
  
  
\end{columns}
\end{frame}
\end{document}