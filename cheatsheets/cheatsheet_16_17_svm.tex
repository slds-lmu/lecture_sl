\documentclass{beamer}
\newcommand \beameritemnestingprefix{}


\usepackage[orientation=landscape,size=a0,scale=1.4,debug]{beamerposter}
\mode<presentation>{\usetheme{mlr}}


\usepackage[utf8]{inputenc} % UTF-8
\usepackage[english]{babel} % Language
\usepackage{hyperref} % Hyperlinks
\usepackage{ragged2e} % Text position
\usepackage[export]{adjustbox} % Image position
\usepackage[most]{tcolorbox}
\usepackage{amsmath}
\usepackage{mathtools}
\usepackage{dsfont}
\usepackage{verbatim}
\usepackage{amsmath}
\usepackage{amsfonts}
\usepackage{csquotes}
\usepackage{multirow}
\usepackage{longtable}
\usepackage[absolute,overlay]{textpos}
\usepackage{psfrag}
\usepackage{algorithm}
\usepackage{algpseudocode}
\usepackage{eqnarray}
\usepackage{arydshln}
\usepackage{tabularx}
\usepackage{placeins}
\usepackage{tikz}
\usepackage{setspace}
\usepackage{colortbl}
\usepackage{mathtools}
\usepackage{wrapfig}
\usepackage{bm}


\input{../latex-math/basic-math.tex}
% machine learning
\newcommand{\Xspace}{\mathcal{X}} % X, input space
\newcommand{\Yspace}{\mathcal{Y}} % Y, output space
\newcommand{\Zspace}{\mathcal{Z}} % Z, space of sampled datapoints
\newcommand{\nset}{\{1, \ldots, n\}} % set from 1 to n
\newcommand{\pset}{\{1, \ldots, p\}} % set from 1 to p
\newcommand{\gset}{\{1, \ldots, g\}} % set from 1 to g
\newcommand{\Pxy}{\mathbb{P}_{xy}} % P_xy
\newcommand{\Exy}{\mathbb{E}_{xy}} % E_xy: Expectation over random variables xy
\newcommand{\xv}{\mathbf{x}} % vector x (bold)
\newcommand{\xtil}{\tilde{\mathbf{x}}} % vector x-tilde (bold)
\newcommand{\yv}{\mathbf{y}} % vector y (bold)
\newcommand{\xy}{(\xv, y)} % observation (x, y)
\newcommand{\xvec}{\left(x_1, \ldots, x_p\right)^\top} % (x1, ..., xp)
\newcommand{\Xmat}{\mathbf{X}} % Design matrix
\newcommand{\allDatasets}{\mathds{D}} % The set of all datasets
\newcommand{\allDatasetsn}{\mathds{D}_n}  % The set of all datasets of size n
\newcommand{\D}{\mathcal{D}} % D, data
\newcommand{\Dn}{\D_n} % D_n, data of size n
\newcommand{\Dtrain}{\mathcal{D}_{\text{train}}} % D_train, training set
\newcommand{\Dtest}{\mathcal{D}_{\text{test}}} % D_test, test set
\newcommand{\xyi}[1][i]{\left(\xv^{(#1)}, y^{(#1)}\right)} % (x^i, y^i), i-th observation
\newcommand{\Dset}{\left( \xyi[1], \ldots, \xyi[n]\right)} % {(x1,y1)), ..., (xn,yn)}, data
\newcommand{\defAllDatasetsn}{(\Xspace \times \Yspace)^n} % Def. of the set of all datasets of size n
\newcommand{\defAllDatasets}{\bigcup_{n \in \N}(\Xspace \times \Yspace)^n} % Def. of the set of all datasets
\newcommand{\xdat}{\left\{ \xv^{(1)}, \ldots, \xv^{(n)}\right\}} % {x1, ..., xn}, input data
\newcommand{\ydat}{\left\{ \yv^{(1)}, \ldots, \yv^{(n)}\right\}} % {y1, ..., yn}, input data
\newcommand{\yvec}{\left(y^{(1)}, \hdots, y^{(n)}\right)^\top} % (y1, ..., yn), vector of outcomes
\newcommand{\greekxi}{\xi} % Greek letter xi
\renewcommand{\xi}[1][i]{\xv^{(#1)}} % x^i, i-th observed value of x
\newcommand{\yi}[1][i]{y^{(#1)}} % y^i, i-th observed value of y
\newcommand{\xivec}{\left(x^{(i)}_1, \ldots, x^{(i)}_p\right)^\top} % (x1^i, ..., xp^i), i-th observation vector
\newcommand{\xj}{\xv_j} % x_j, j-th feature
\newcommand{\xjvec}{\left(x^{(1)}_j, \ldots, x^{(n)}_j\right)^\top} % (x^1_j, ..., x^n_j), j-th feature vector
\newcommand{\phiv}{\mathbf{\phi}} % Basis transformation function phi
\newcommand{\phixi}{\mathbf{\phi}^{(i)}} % Basis transformation of xi: phi^i := phi(xi)

%%%%%% ml - models general
\newcommand{\lamv}{\bm{\lambda}} % lambda vector, hyperconfiguration vector
\newcommand{\Lam}{\Lambda}	 % Lambda, space of all hpos
% Inducer / Inducing algorithm
\newcommand{\preimageInducer}{\left(\defAllDatasets\right)\times\Lam} % Set of all datasets times the hyperparameter space
\newcommand{\preimageInducerShort}{\allDatasets\times\Lam} % Set of all datasets times the hyperparameter space
% Inducer / Inducing algorithm
\newcommand{\ind}{\mathcal{I}} % Inducer, inducing algorithm, learning algorithm

% continuous prediction function f
\newcommand{\ftrue}{f_{\text{true}}}  % True underlying function (if a statistical model is assumed)
\newcommand{\ftruex}{\ftrue(\xv)} % True underlying function (if a statistical model is assumed)
\newcommand{\fx}{f(\xv)} % f(x), continuous prediction function
\newcommand{\fdomains}{f: \Xspace \rightarrow \R^g} % f with domain and co-domain
\newcommand{\Hspace}{\mathcal{H}} % hypothesis space where f is from
\newcommand{\Hall}{\mathcal{H}_{\text{all}}} % unrestricted hypothesis space
\newcommand{\fbayes}{f^{\ast}} % Bayes-optimal model
\newcommand{\fxbayes}{f^{\ast}(\xv)} % Bayes-optimal model
\newcommand{\fkx}[1][k]{f_{#1}(\xv)} % f_j(x), discriminant component function
\newcommand{\fhspace}{\hat f_{\Hspace}} % fhat_H
\newcommand{\fh}{\hat{f}} % f hat, estimated prediction function
\newcommand{\fxh}{\fh(\xv)} % fhat(x)
\newcommand{\fxt}{f(\xv ~|~ \thetav)} % f(x | theta)
\newcommand{\fxi}{f\left(\xv^{(i)}\right)} % f(x^(i))
\newcommand{\fxih}{\hat{f}\left(\xv^{(i)}\right)} % f(x^(i))
\newcommand{\fxit}{f\big(\xv^{(i)} ~|~ \thetav\big)} % f(x^(i) | theta)
\newcommand{\fhD}{\fh_{\D}} % fhat_D, estimate of f based on D
\newcommand{\fhDtrain}{\fh_{\Dtrain}} % fhat_Dtrain, estimate of f based on D
\newcommand{\fhDnlam}{\fh_{\Dn, \lamv}} %model learned on Dn with hp lambda
\newcommand{\fhDlam}{\fh_{\D, \lamv}} %model learned on D with hp lambda
\newcommand{\fhDnlams}{\fh_{\Dn, \lamv^\ast}} %model learned on Dn with optimal hp lambda
\newcommand{\fhDlams}{\fh_{\D, \lamv^\ast}} %model learned on D with optimal hp lambda

% discrete prediction function h
\newcommand{\hx}{h(\xv)} % h(x), discrete prediction function
\newcommand{\hh}{\hat{h}} % h hat
\newcommand{\hxh}{\hat{h}(\xv)} % hhat(x)
\newcommand{\hxt}{h(\xv | \thetav)} % h(x | theta)
\newcommand{\hxi}{h\left(\xi\right)} % h(x^(i))
\newcommand{\hxit}{h\left(\xi ~|~ \thetav\right)} % h(x^(i) | theta)
\newcommand{\hbayes}{h^{\ast}} % Bayes-optimal classification model
\newcommand{\hxbayes}{h^{\ast}(\xv)} % Bayes-optimal classification model

% yhat
\newcommand{\yh}{\hat{y}} % yhat for prediction of target
\newcommand{\yih}{\hat{y}^{(i)}} % yhat^(i) for prediction of ith targiet
\newcommand{\resi}{\yi- \yih}

% theta
\newcommand{\thetah}{\hat{\theta}} % theta hat
\newcommand{\thetav}{\bm{\theta}} % theta vector
\newcommand{\thetavh}{\bm{\hat\theta}} % theta vector hat
\newcommand{\thetat}[1][t]{\thetav^{[#1]}} % theta^[t] in optimization
\newcommand{\thetatn}[1][t]{\thetav^{[#1 +1]}} % theta^[t+1] in optimization
\newcommand{\thetahDnlam}{\thetavh_{\Dn, \lamv}} %theta learned on Dn with hp lambda
\newcommand{\thetahDlam}{\thetavh_{\D, \lamv}} %theta learned on D with hp lambda
\newcommand{\mint}{\min_{\thetav \in \Theta}} % min problem theta
\newcommand{\argmint}{\argmin_{\thetav \in \Theta}} % argmin theta

% densities + probabilities
% pdf of x
\newcommand{\pdf}{p} % p
\newcommand{\pdfx}{p(\xv)} % p(x)
\newcommand{\pixt}{\pi(\xv~|~ \thetav)} % pi(x|theta), pdf of x given theta
\newcommand{\pixit}[1][i]{\pi\left(\xi[#1] ~|~ \thetav\right)} % pi(x^i|theta), pdf of x given theta
\newcommand{\pixii}[1][i]{\pi\left(\xi[#1]\right)} % pi(x^i), pdf of i-th x

% pdf of (x, y)
\newcommand{\pdfxy}{p(\xv,y)} % p(x, y)
\newcommand{\pdfxyt}{p(\xv, y ~|~ \thetav)} % p(x, y | theta)
\newcommand{\pdfxyit}{p\left(\xi, \yi ~|~ \thetav\right)} % p(x^(i), y^(i) | theta)

% pdf of x given y
\newcommand{\pdfxyk}[1][k]{p(\xv | y= #1)} % p(x | y = k)
\newcommand{\lpdfxyk}[1][k]{\log p(\xv | y= #1)} % log p(x | y = k)
\newcommand{\pdfxiyk}[1][k]{p\left(\xi | y= #1 \right)} % p(x^i | y = k)

% prior probabilities
\newcommand{\pik}[1][k]{\pi_{#1}} % pi_k, prior
\newcommand{\pih}{\hat{\pi}} % pi hat, estimated prior (binary classification)
\newcommand{\pikh}[1][k]{\hat{\pi}_{#1}} % pi_k hat, estimated prior
\newcommand{\lpik}[1][k]{\log \pi_{#1}} % log pi_k, log of the prior
\newcommand{\pit}{\pi(\thetav)} % Prior probability of parameter theta

% posterior probabilities
\newcommand{\post}{\P(y = 1 ~|~ \xv)} % P(y = 1 | x), post. prob for y=1
\newcommand{\postk}[1][k]{\P(y = #1 ~|~ \xv)} % P(y = k | y), post. prob for y=k
\newcommand{\pidomains}{\pi: \Xspace \rightarrow \unitint} % pi with domain and co-domain
\newcommand{\pibayes}{\pi^{\ast}} % Bayes-optimal classification model
\newcommand{\pixbayes}{\pi^{\ast}(\xv)} % Bayes-optimal classification model
\newcommand{\piastxtil}{\pi^{\ast}(\xtil)} % Bayes-optimal classification model
\newcommand{\piastkxtil}{\pi^{\ast}_k(\xtil)} % Bayes-optimal classification model for k-th class
\newcommand{\pix}{\pi(\xv)} % pi(x), P(y = 1 | x)
\newcommand{\piv}{\bm{\pi}} % pi, bold, as vector
\newcommand{\pikx}[1][k]{\pi_{#1}(\xv)} % pi_k(x), P(y = k | x)
\newcommand{\pikxt}[1][k]{\pi_{#1}(\xv ~|~ \thetav)} % pi_k(x | theta), P(y = k | x, theta)
\newcommand{\pixh}{\hat \pi(\xv)} % pi(x) hat, P(y = 1 | x) hat
\newcommand{\pikxh}[1][k]{\hat \pi_{#1}(\xv)} % pi_k(x) hat, P(y = k | x) hat
\newcommand{\pixih}{\hat \pi(\xi)} % pi(x^(i)) with hat
\newcommand{\pikxih}[1][k]{\hat \pi_{#1}(\xi)} % pi_k(x^(i)) with hat
\newcommand{\pdfygxt}{p(y ~|~\xv, \thetav)} % p(y | x, theta)
\newcommand{\pdfyigxit}{p\left(\yi ~|~\xi, \thetav\right)} % p(y^i |x^i, theta)
\newcommand{\lpdfygxt}{\log \pdfygxt } % log p(y | x, theta)
\newcommand{\lpdfyigxit}{\log \pdfyigxit} % log p(y^i |x^i, theta)

% probabilistic
\newcommand{\bayesrulek}[1][k]{\frac{\P(\xv | y= #1) \P(y= #1)}{\P(\xv)}} % Bayes rule
\newcommand{\muv}{\bm{\mu}} % expectation vector of Gaussian
\newcommand{\muk}[1][k]{\bm{\mu_{#1}}} % mean vector of class-k Gaussian (discr analysis)
\newcommand{\mukh}[1][k]{\bm{\hat{\mu}_{#1}}} % estimated mean vector of class-k Gaussian (discr analysis)

% residual and margin
\newcommand{\rx}{r(\xv)} % residual 
\newcommand{\eps}{\epsilon} % residual, stochastic
\newcommand{\epsv}{\bm{\epsilon}} % residual, stochastic, as vector
\newcommand{\epsi}{\epsilon^{(i)}} % epsilon^i, residual, stochastic
\newcommand{\epsh}{\hat{\epsilon}} % residual, estimated
\newcommand{\epsvh}{\hat{\epsv}} % residual, estimated, vector
\newcommand{\yf}{y \fx} % y f(x), margin
\newcommand{\yfi}{\yi \fxi} % y^i f(x^i), margin
\newcommand{\Sigmah}{\hat \Sigma} % estimated covariance matrix
\newcommand{\Sigmahj}{\hat \Sigma_j} % estimated covariance matrix for the j-th class
\newcommand{\nux}{\nu(\xv)} % nu(x) = y * f(x)

% ml - loss, risk, likelihood
\newcommand{\Lyf}{L\left(y, f\right)} % L(y, f), loss function
% \newcommand{\Lypi}{L\left(y, \pi\right)} % L(y, pi), loss function
\newcommand{\Lxy}{L\left(y, \fx\right)} % L(y, f(x)), loss function
\newcommand{\Lxyi}{L\left(\yi, \fxi\right)} % loss of observation
\newcommand{\Lxyt}{L\left(y, \fxt\right)} % loss with f parameterized
\newcommand{\Lxyit}{L\left(\yi, \fxit\right)} % loss of observation with f parameterized
\newcommand{\Lxym}{L\left(\yi, f\left(\bm{\tilde{x}}^{(i)} ~|~ \thetav\right)\right)} % loss of observation with f parameterized
\newcommand{\Lpixy}{L\left(y, \pix\right)} % loss in classification
% \newcommand{\Lpiy}{L\left(y, \pi\right)} % loss in classification
\newcommand{\Lpiv}{L\left(y, \piv\right)} % loss in classification
\newcommand{\Lpixyi}{L\left(\yi, \pixii\right)} % loss of observation in classification
\newcommand{\Lpixyt}{L\left(y, \pixt\right)} % loss with pi parameterized
\newcommand{\Lpixyit}{L\left(\yi, \pixit\right)} % loss of observation with pi parameterized
% \newcommand{\Lhy}{L\left(y, h\right)} % L(y, h), loss function on discrete classes
\newcommand{\Lhxy}{L\left(y, \hx\right)} % L(y, h(x)), loss function on discrete classes
\newcommand{\Lr}{L\left(r\right)} % L(r), loss defined on residual (reg) / margin (classif)
\newcommand{\lone}{|y - \fx|} % L1 loss
\newcommand{\ltwo}{\left(y - \fx\right)^2} % L2 loss
\newcommand{\lbernoullimp}{\ln(1 + \exp(-y \cdot \fx))} % Bernoulli loss for -1, +1 encoding
\newcommand{\lbernoullizo}{- y \cdot \fx + \log(1 + \exp(\fx))} % Bernoulli loss for 0, 1 encoding
\newcommand{\lcrossent}{- y \log \left(\pix\right) - (1 - y) \log \left(1 - \pix\right)} % cross-entropy loss
\newcommand{\lbrier}{\left(\pix - y \right)^2} % Brier score
\newcommand{\risk}{\mathcal{R}} % R, risk
\newcommand{\riskbayes}{\mathcal{R}^\ast}
\newcommand{\riskf}{\risk(f)} % R(f), risk
\newcommand{\riskdef}{\E_{y|\xv}\left(\Lxy \right)} % risk def (expected loss)
\newcommand{\riskt}{\mathcal{R}(\thetav)} % R(theta), risk
\newcommand{\riske}{\mathcal{R}_{\text{emp}}} % R_emp, empirical risk w/o factor 1 / n
\newcommand{\riskeb}{\bar{\mathcal{R}}_{\text{emp}}} % R_emp, empirical risk w/ factor 1 / n
\newcommand{\riskef}{\riske(f)} % R_emp(f)
\newcommand{\risket}{\mathcal{R}_{\text{emp}}(\thetav)} % R_emp(theta)
\newcommand{\riskr}{\mathcal{R}_{\text{reg}}} % R_reg, regularized risk
\newcommand{\riskrt}{\mathcal{R}_{\text{reg}}(\thetav)} % R_reg(theta)
\newcommand{\riskrf}{\riskr(f)} % R_reg(f)
\newcommand{\riskrth}{\hat{\mathcal{R}}_{\text{reg}}(\thetav)} % hat R_reg(theta)
\newcommand{\risketh}{\hat{\mathcal{R}}_{\text{emp}}(\thetav)} % hat R_emp(theta)
\newcommand{\LL}{\mathcal{L}} % L, likelihood
\newcommand{\LLt}{\mathcal{L}(\thetav)} % L(theta), likelihood
\newcommand{\LLtx}{\mathcal{L}(\thetav | \xv)} % L(theta|x), likelihood
\newcommand{\logl}{\ell} % l, log-likelihood
\newcommand{\loglt}{\logl(\thetav)} % l(theta), log-likelihood
\newcommand{\logltx}{\logl(\thetav | \xv)} % l(theta|x), log-likelihood
\newcommand{\errtrain}{\text{err}_{\text{train}}} % training error
\newcommand{\errtest}{\text{err}_{\text{test}}} % test error
\newcommand{\errexp}{\overline{\text{err}_{\text{test}}}} % avg training error

% lm
\newcommand{\thx}{\thetav^\top \xv} % linear model
\newcommand{\olsest}{(\Xmat^\top \Xmat)^{-1} \Xmat^\top \yv} % OLS estimator in LM

\input{../latex-math/ml-trees.tex}
\input{../latex-math/ml-nn.tex}
\input{../latex-math/ml-svm}
\input{../latex-math/ml-eval}


\title{Supervised Learning :\,: CHEAT SHEET} % Package title in header, \, adds thin space between ::
\newcommand{\packagedescription}{ % Package description in header
%	The \textbf{I2ML}: Introduction to Machine Learning course offers an introductory and applied overview of "supervised" Machine Learning. It is organized as a digital lecture.
}

\newlength{\columnheight} % Adjust depending on header height
\setlength{\columnheight}{84cm} 

\newtcolorbox{codebox}{%
	sharp corners,
	leftrule=0pt,
	rightrule=0pt,
	toprule=0pt,
	bottomrule=0pt,
	hbox}

\newtcolorbox{codeboxmultiline}[1][]{%
	sharp corners,
	leftrule=0pt,
	rightrule=0pt,
	toprule=0pt,
	bottomrule=0pt,
	#1}

\begin{document}
\begin{frame}[fragile]{}
\begin{columns}
	\begin{column}{.31\textwidth}
		\begin{beamercolorbox}[center]{postercolumn}
			\begin{minipage}{.98\textwidth}
				\parbox[t][\columnheight]{\textwidth}{

					\begin{myblock}{Linear hard-margin SVM}
      
For labeled data $\D = \Dset$, with $\yi \in \{-1, +1\}$:
\begin{itemize}[$\bullet$]
  \setlength{\itemindent}{+.3in}
  \item Assume linear separation by $\fx = \thetav^\top \xv + \theta_0$, such that all $+$-observations are in the positive halfspace
$
  \phantom{i}\{\xv \in \Xspace: \fx > 0\}
$
  and all $-$-observations are in the negative halfspace
$
  \phantom{i}\{\xv \in \Xspace : \fx < 0\}.
$

  \item For a linear separating hyperplane, we have
  $$
    \yi \underbrace{\left(\thetav^\top \xi + \theta_0\right)}_{= \fxi} > 0 \quad \forall i \in \{1, 2, ..., n\}.
  $$

  \item 
    % For correctly classified points $\left(\xi, \yi\right)$,
  $$
    d \left(f, \xi \right) = \frac{\yi \fxi}{\|\thetav\|} = \yi \frac{\thetav^T \xi + \theta_0}{\|\thetav\|}
  $$
  computes the (signed) distance to the separating hyperplane $\fx = 0$,
    positive for correct classifications, negative for incorrect.
   \item The distance of $f$ to the whole dataset $\D$
    is the smallest distance
    $
    \gamma = \min\limits_i \Big\{ d \left(f, \xi \right) \Big\}
    $, which represents the \textbf{safety margin}. It is positive if $f$ separates and we want to maximize it.
\end{itemize}
\begin{eqnarray*}
    & \max\limits_{\thetav, \theta_0} & \gamma \\
    & \text{s.t.} & \,\, d \left(f, \xi \right) \geq \gamma \quad \forall\, i \in \nset.
    \end{eqnarray*}

    \begin{codebox}
\textbf{Primal linear hard-margin SVM:}
\end{codebox}				
								\begin{eqnarray*}
									& \min\limits_{\thetav, \theta_0} \quad & \frac{1}{2} \|\thetav\|^2 \\
									& \text{s.t.} & \,\,\yi  \left( \scp{\thetav}{\xi} + \theta_0 \right) \geq 1 \quad \forall\, i \in \nset
								\end{eqnarray*}
	This is a convex quadratic program.\\


								\textbf{Support vectors}: All instances $(\xi, \yi)$ with minimal margin
								$\yi  \fxi = 1$, fulfilling the inequality constraints with equality. 
								All have distance of $\gamma = 1 / \|\thetav\|$ from the separating hyperplane.\\
								
The Lagrange function of the SVM optimization problem is
{\small
\begin{eqnarray*}
&L(\thetav, \theta_0, \alphav) = & \frac{1}{2}\|\thetav\|^2  -  \sum_{i=1}^n \alpha_i \left[\yi  \left( \scp{\thetav}{\xi} + \theta_0 \right) - 1\right]\\
 & \text{s.t.} & \,\, \alpha_i \ge 0 \quad \forall\, i \in \nset.
\end{eqnarray*}
}
The \textbf{dual} form of this problem is
$\max\limits_{\alpha} \min\limits_{\thetav, \theta_0}  L(\thetav, \theta_0,\alphav).$\\

We find the stationary point of $L(\thetav, \theta_0,\alphav)$ w.r.t. $\thetav, \theta_0$ and obtain
$$
    \thetav = \sum_{i=1}^n \alpha_i \yi \xi, 
    0 = \sum_{i=1}^n \alpha_i \yi \quad \forall\, i \in \nset.
$$
\end{myblock}
				}
			\end{minipage}
		\end{beamercolorbox}
	\end{column}
	
%%%%%%%%%%%%%%%%%%%%%%%%%%%%%%%%%%%%%%%%%%%%%%%%%%%%%%%%%%%%%%%%%%%%%

\begin{column}{.31\textwidth}
\begin{beamercolorbox}[center]{postercolumn}
\begin{minipage}{.98\textwidth}
\parbox[t][\columnheight]{\textwidth}{

\begin{myblock}{}

 \begin{codebox}
\textbf{Dual linear hard-margin SVM:}
\end{codebox}

\begin{eqnarray*}
    & \max\limits_{\alphav \in \R^n} & \sum_{i=1}^n \alpha_i - \frac{1}{2}\sum_{i=1}^n\sum_{j=1}^n\alpha_i\alpha_j\yi y^{(j)} \scp{\xi}{\xv^{(j)}} \\
    & \text{s.t.} & \sum_{i=1}^n \alpha_i \yi = 0, \\
    & \quad & \alpha_i \ge 0~\forall i \in \nset,
\end{eqnarray*}

In matrix notation with $\bm{K}:= \Xmat \Xmat^T$:
\begin{eqnarray*}
  & \max\limits_{\alphav \in \R^n} & \one^T \alphav - \frac{1}{2} \alphav^T \diag(\yv)\bm{K} \diag(\yv) \alphav \\
  & \text{s.t.} & \alphav^T \yv = 0, \\
  & \quad & \alphav \geq 0,
\end{eqnarray*}

Solution (if existing):
								%
								$$
								\thetah = \sum\nolimits_{i=1}^n \hat \alpha_i \yi \xi, \quad \theta_0 = \yi - \scp{\thetav}{\xi}.
								$$
  \end{myblock}

  \begin{myblock}{Linear Soft-Margin SVM}
    Allow violations of the margin constraints via slack vars $\sli \geq 0$
    $$
    \yi \left( \scp{\thetav}{\xi} + \thetav_0 \right) \geq 1 - \sli
    $$

    Now we have two distinct and contradictory goals:
    \begin{itemize}[$\bullet$]
      \setlength{\itemindent}{+.3in}
      \item Maximize the margin.
      \item Minimize margin violations.
    \end{itemize}

    \begin{codebox}
      \textbf{Primal linear soft-margin SVM:} 
    \end{codebox}	
							\begin{eqnarray*}
								& \min\limits_{\thetav, \thetav_0,\sli} & \frac{1}{2} \|\thetav\|^2 + C   \sum_{i=1}^n \sli \\
								& \text{s.t.} & \,\, \yi  \left( \scp{\thetav}{\xi} + \thetav_0 \right) \geq 1 - \sli \quad \forall\, i \in \nset,\\
								& \text{and} & \,\, \sli \geq 0 \quad \forall\, i \in \nset,\\
							\end{eqnarray*}
							where the constant $C > 0$ controls trade-off between the two conflicting
							objectives of maximizing the size of the margin and minimizing the
							frequency and size of margin violations.\\

    \end{myblock}
}
\end{minipage}
\end{beamercolorbox}
\end{column}

%%%%%%%%%%%%%%%%%%%%%%%%%%%%%%%%%%%%%%%%%%%%%%%%%%%%%%%%%%%%%%%%%%%%%%%%%%%%%%%%%%%%%%%%%%%%%%%%%%%%

\begin{column}{.31\textwidth}
\begin{beamercolorbox}[center]{postercolumn}
\begin{minipage}{.98\textwidth}
\parbox[t][\columnheight]{\textwidth}{

  \begin{myblock}{}

							
						\begin{codebox}
							\textbf{Dual linear soft-margin SVM:} 	
						\end{codebox}
							\begin{eqnarray*}
								& \max\limits_{\alphav \in \R^n} & \sum_{i=1}^n \alpha_i - \frac{1}{2}\sum_{i=1}^n\sum_{j=1}^n\alpha_i\alpha_j\yi y^{(j)} \scp{\xi}{\xv^{(j)}} \\
								& \text{s.t. } & 0 \le \alpha_i \le C, \forall\, i \in \nset \quad \text{and} \quad  \sum_{i=1}^n \alpha_i \yi = 0
							\end{eqnarray*}

        \begin{itemize}[$\bullet$]
      \setlength{\itemindent}{+.3in}
    \item Non-SVs have $\alpha_i = 0\; (\Rightarrow \mu_i = C \Rightarrow \sli = 0)$ and can be
    removed from the problem without changing the solution. Their margin $y\fx \geq ~ 1.$ 
    They are always classified correctly and are never inside of the margin. 
    
    \item SVs with $0 < \alpha_i < C\; (\Rightarrow \mu_i > 0 \Rightarrow \sli = 0)$ are located exactly on the
    margin and have $y\fx=1$. 
    \item SVs with $\alpha_i = C$ have an associated
     slack $\sli \geq 0.$ They can be on the margin or can be margin violators with $y\fx < 1$ (they can even be misclassified if $\sli \geq 1$).
\end{itemize}
        
Regularized ERM representation with hinge loss:
$$ \risket = \frac{1}{2} \|\thetav\|^2 + C \sumin \Lxyi ;\; \Lxy = \max(1-y\fx, 0)$$
  
\end{myblock}

\begin{myblock}{Optimization}
  \begin{algorithm}[H]
  \caption{Stochastic subgradient descent (without intercept $\theta_0$)}
  \begin{algorithmic}[1]
    \For {$t = 1, 2, ...$}
      \State Pick step size $\alpha$
      \State Randomly pick an index $i$
      \State If $\yfi < 1$ set $\thetatn = (1 - \lambda \alpha) \thetat + \alpha \yi \xi$ 
      \State If $\yfi \geq 1$ set $\thetatn = (1 - \lambda \alpha) \thetat$ 
      \EndFor
  \end{algorithmic}
\end{algorithm}

\begin{algorithm}[H]
  \caption{Pairwise coordinate ascent in the dual}
  \begin{algorithmic}[1]
    \State Initialize $\alphav = 0$ (or more cleverly)
    \For {$t = 1, 2, ...$}
      \State Select some pair $\alpha_i, \alpha_j$ to update next
      \State Optimize dual w.r.t. $\alpha_i, \alpha_j$, while holding $\alpha_k$ ($k\ne i, j$) fixed
      \EndFor
  \end{algorithmic}
\end{algorithm}
  
\end{myblock}

  }
  
  \end{minipage}
  \end{beamercolorbox}
  \end{column}
  
  
  
\end{columns}
\end{frame}

\begin{frame}[fragile]{}
\begin{columns}
	\begin{column}{.31\textwidth}
		\begin{beamercolorbox}[center]{postercolumn}
			\begin{minipage}{.98\textwidth}
				\parbox[t][\columnheight]{\textwidth}{

					\begin{myblock}{Kernel}
Kernel = Feature Map + Inner product\\

\begin{codebox} \textbf{Mercer Kernel}
\end{codebox}
A \textbf{(Mercer) kernel} on a space~$\Xspace$ is a
  continuous function
  $$ k : \Xspace \times \Xspace \to \R $$
  of two arguments with the properties
  \begin{itemize}[$\bullet$]
    \setlength{\itemindent}{+.3in}
    \item Symmetry: $k(\xv, \tilde \xv) = k(\tilde \xv, \xv)$ for all
    $\xv, \tilde \xv \in \Xspace$.
    \item Positive definiteness: For each finite subset $\left\{\xv^{(1)}, \dots, \xv^{(n)}\right\}$
    the \textbf{kernel Gram matrix} $\bm{K} \in \R^{n \times n}$ with entries
    $K_{ij} = k(\xi, \xv^{(j)})$ is positive semi-definite.
  \end{itemize}\\

  \textbf{Reproducing property}: for all kernels, there must exist a Hilbert space, where a map $\phi$ of this space satisfies $k(\xv, \xtil) = \scp{\phix}{\phixt}$.
  The space is called \textbf{reproducing kernel Hilbert space} (RKHS).

\begin{codebox}
      \textbf{Typical Kernels} 
    \end{codebox}	
  A kernel can be constructed from other kernels $k_1$ and~$k_2$:
  \begin{itemize}[$\bullet$]
    \setlength{\itemindent}{+.3in}
      \item For $\lambda \geq 0$, $\lambda \cdot k_1$ is a kernel.
      \item $k_1 + k_2$ is a kernel.
      \item $k_1 \cdot k_2$ is a kernel (thus also $k_1^n$).
    \end{itemize}\\


    Useful kernels:
  \begin{itemize}[$\bullet$]
    \setlength{\itemindent}{+.3in}
      \item Every constant function taking a non-negative value.
      \item \textbf{Linear kernel}: $k(\xv, \tilde \xv) = \xv^\top \tilde \xv$.
      \item \textbf{Polynomial kernel}: $k(\xv, \tilde \xv) = (\xv^\top \tilde \xv + b)^d, \text{ for } b\geq 0, d \in \N$.
      $$\phix = \left( \sqrt{\mat{d \\ k_1, \ldots, k_{p+1}}} x_1^{k_1} \ldots x_p^{k_p} b^{k_{p+1}/2} \right)_{k_i \geq 0, \sum_i k_i = d}$$
      \item \textbf{Gaussian kernel}: $k(\xv, \tilde \xv) = \exp(-\frac{\|\xv - \tilde \xv\|^2}{2\sigma^2})$
or $k(\xv, \tilde \xv) = \exp(-\gamma \|\xv - \tilde \xv\|^2), ~ \gamma>0$
    \end{itemize}


\end{myblock}
				}
			\end{minipage}
		\end{beamercolorbox}
	\end{column}
	
%%%%%%%%%%%%%%%%%%%%%%%%%%%%%%%%%%%%%%%%%%%%%%%%%%%%%%%%%%%%%%%%%%%%%

\begin{column}{.31\textwidth}
\begin{beamercolorbox}[center]{postercolumn}
\begin{minipage}{.98\textwidth}
\parbox[t][\columnheight]{\textwidth}{

\begin{myblock}{ }
\begin{codebox}
\textbf{Dual kernelized soft-margin SVM:}
\end{codebox}
							%	
									\begin{eqnarray*}
										& \max_{\alpha} & \sum_{i=1}^n \alpha_i - \frac{1}{2}\sum_{i=1}^n\sum_{j=1}^n\alpha_i\alpha_j\yi y^{(j)} k(\xi, \xv^{(j)})  \\
										& \text{s.t. } & 0 \le \alpha_i \le C, \forall\, i \in \nset \quad \text{and} \quad  \sum_{i=1}^n \alpha_i \yi = 0
									\end{eqnarray*}	
							
								Kernel representation of separating hyperplane:
									$$ \fx = \sumin \alpha_i \yi k(\xi, \xv)  + \theta_0$$

\end{myblock}

\begin{myblock}{Hyperparameters of SVM}
    SVMs are somewhat sensitive to its hyperparameters and should always be tuned.\\

    \begin{itemize}[$\bullet$]
    \setlength{\itemindent}{+.3in}
      \item The choice of C, the choice of the kernel, the kernel parameters are all up to the user.
\item Small C allows for margin-violating points in favor of a large margin.
\item Large C penalizes margin violators, decision boundary is more wiggly.
    \end{itemize}

\end{myblock}
}
\end{minipage}
\end{beamercolorbox}
\end{column}

%%%%%%%%%%%%%%%%%%%%%%%%%%%%%%%%%%%%%%%%%%%%%%%%%%%%%%%%%%%%%%%%%%%%%%%%%%%%%%%%%%%%%%%%%%%%%%%%%%%%

\begin{column}{.31\textwidth}
\begin{beamercolorbox}[center]{postercolumn}
\begin{minipage}{.98\textwidth}
\parbox[t][\columnheight]{\textwidth}{

  \begin{myblock}{}

\end{myblock}
  }
  
  \end{minipage}
  \end{beamercolorbox}
  \end{column}
  
  
  
\end{columns}
\end{frame}

\end{document}