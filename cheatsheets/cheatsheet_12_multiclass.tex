\documentclass{beamer}
\newcommand \beameritemnestingprefix{}


\usepackage[orientation=landscape,size=a0,scale=1.4,debug]{beamerposter}
\mode<presentation>{\usetheme{mlr}}


\usepackage[utf8]{inputenc} % UTF-8
\usepackage[english]{babel} % Language
\usepackage{hyperref} % Hyperlinks
\usepackage{ragged2e} % Text position
\usepackage[export]{adjustbox} % Image position
\usepackage[most]{tcolorbox}
\usepackage{amsmath}
\usepackage{mathtools}
\usepackage{dsfont}
\usepackage{verbatim}
\usepackage{amsmath}
\usepackage{amsfonts}
\usepackage{csquotes}
\usepackage{multirow}
\usepackage{longtable}
\usepackage[absolute,overlay]{textpos}
\usepackage{psfrag}
\usepackage{algorithm}
\usepackage{algpseudocode}
\usepackage{eqnarray}
\usepackage{arydshln}
\usepackage{tabularx}
\usepackage{placeins}
\usepackage{tikz}
\usepackage{setspace}
\usepackage{colortbl}
\usepackage{mathtools}
\usepackage{wrapfig}
\usepackage{bm}


\input{../latex-math/basic-math.tex}
% machine learning
\newcommand{\Xspace}{\mathcal{X}} % X, input space
\newcommand{\Yspace}{\mathcal{Y}} % Y, output space
\newcommand{\Zspace}{\mathcal{Z}} % Z, space of sampled datapoints
\newcommand{\nset}{\{1, \ldots, n\}} % set from 1 to n
\newcommand{\pset}{\{1, \ldots, p\}} % set from 1 to p
\newcommand{\gset}{\{1, \ldots, g\}} % set from 1 to g
\newcommand{\Pxy}{\mathbb{P}_{xy}} % P_xy
\newcommand{\Exy}{\mathbb{E}_{xy}} % E_xy: Expectation over random variables xy
\newcommand{\xv}{\mathbf{x}} % vector x (bold)
\newcommand{\xtil}{\tilde{\mathbf{x}}} % vector x-tilde (bold)
\newcommand{\yv}{\mathbf{y}} % vector y (bold)
\newcommand{\xy}{(\xv, y)} % observation (x, y)
\newcommand{\xvec}{\left(x_1, \ldots, x_p\right)^\top} % (x1, ..., xp)
\newcommand{\Xmat}{\mathbf{X}} % Design matrix
\newcommand{\allDatasets}{\mathds{D}} % The set of all datasets
\newcommand{\allDatasetsn}{\mathds{D}_n}  % The set of all datasets of size n
\newcommand{\D}{\mathcal{D}} % D, data
\newcommand{\Dn}{\D_n} % D_n, data of size n
\newcommand{\Dtrain}{\mathcal{D}_{\text{train}}} % D_train, training set
\newcommand{\Dtest}{\mathcal{D}_{\text{test}}} % D_test, test set
\newcommand{\xyi}[1][i]{\left(\xv^{(#1)}, y^{(#1)}\right)} % (x^i, y^i), i-th observation
\newcommand{\Dset}{\left( \xyi[1], \ldots, \xyi[n]\right)} % {(x1,y1)), ..., (xn,yn)}, data
\newcommand{\defAllDatasetsn}{(\Xspace \times \Yspace)^n} % Def. of the set of all datasets of size n
\newcommand{\defAllDatasets}{\bigcup_{n \in \N}(\Xspace \times \Yspace)^n} % Def. of the set of all datasets
\newcommand{\xdat}{\left\{ \xv^{(1)}, \ldots, \xv^{(n)}\right\}} % {x1, ..., xn}, input data
\newcommand{\ydat}{\left\{ \yv^{(1)}, \ldots, \yv^{(n)}\right\}} % {y1, ..., yn}, input data
\newcommand{\yvec}{\left(y^{(1)}, \hdots, y^{(n)}\right)^\top} % (y1, ..., yn), vector of outcomes
\newcommand{\greekxi}{\xi} % Greek letter xi
\renewcommand{\xi}[1][i]{\xv^{(#1)}} % x^i, i-th observed value of x
\newcommand{\yi}[1][i]{y^{(#1)}} % y^i, i-th observed value of y
\newcommand{\xivec}{\left(x^{(i)}_1, \ldots, x^{(i)}_p\right)^\top} % (x1^i, ..., xp^i), i-th observation vector
\newcommand{\xj}{\xv_j} % x_j, j-th feature
\newcommand{\xjvec}{\left(x^{(1)}_j, \ldots, x^{(n)}_j\right)^\top} % (x^1_j, ..., x^n_j), j-th feature vector
\newcommand{\phiv}{\mathbf{\phi}} % Basis transformation function phi
\newcommand{\phixi}{\mathbf{\phi}^{(i)}} % Basis transformation of xi: phi^i := phi(xi)

%%%%%% ml - models general
\newcommand{\lamv}{\bm{\lambda}} % lambda vector, hyperconfiguration vector
\newcommand{\Lam}{\Lambda}	 % Lambda, space of all hpos
% Inducer / Inducing algorithm
\newcommand{\preimageInducer}{\left(\defAllDatasets\right)\times\Lam} % Set of all datasets times the hyperparameter space
\newcommand{\preimageInducerShort}{\allDatasets\times\Lam} % Set of all datasets times the hyperparameter space
% Inducer / Inducing algorithm
\newcommand{\ind}{\mathcal{I}} % Inducer, inducing algorithm, learning algorithm

% continuous prediction function f
\newcommand{\ftrue}{f_{\text{true}}}  % True underlying function (if a statistical model is assumed)
\newcommand{\ftruex}{\ftrue(\xv)} % True underlying function (if a statistical model is assumed)
\newcommand{\fx}{f(\xv)} % f(x), continuous prediction function
\newcommand{\fdomains}{f: \Xspace \rightarrow \R^g} % f with domain and co-domain
\newcommand{\Hspace}{\mathcal{H}} % hypothesis space where f is from
\newcommand{\Hall}{\mathcal{H}_{\text{all}}} % unrestricted hypothesis space
\newcommand{\fbayes}{f^{\ast}} % Bayes-optimal model
\newcommand{\fxbayes}{f^{\ast}(\xv)} % Bayes-optimal model
\newcommand{\fkx}[1][k]{f_{#1}(\xv)} % f_j(x), discriminant component function
\newcommand{\fhspace}{\hat f_{\Hspace}} % fhat_H
\newcommand{\fh}{\hat{f}} % f hat, estimated prediction function
\newcommand{\fxh}{\fh(\xv)} % fhat(x)
\newcommand{\fxt}{f(\xv ~|~ \thetav)} % f(x | theta)
\newcommand{\fxi}{f\left(\xv^{(i)}\right)} % f(x^(i))
\newcommand{\fxih}{\hat{f}\left(\xv^{(i)}\right)} % f(x^(i))
\newcommand{\fxit}{f\big(\xv^{(i)} ~|~ \thetav\big)} % f(x^(i) | theta)
\newcommand{\fhD}{\fh_{\D}} % fhat_D, estimate of f based on D
\newcommand{\fhDtrain}{\fh_{\Dtrain}} % fhat_Dtrain, estimate of f based on D
\newcommand{\fhDnlam}{\fh_{\Dn, \lamv}} %model learned on Dn with hp lambda
\newcommand{\fhDlam}{\fh_{\D, \lamv}} %model learned on D with hp lambda
\newcommand{\fhDnlams}{\fh_{\Dn, \lamv^\ast}} %model learned on Dn with optimal hp lambda
\newcommand{\fhDlams}{\fh_{\D, \lamv^\ast}} %model learned on D with optimal hp lambda

% discrete prediction function h
\newcommand{\hx}{h(\xv)} % h(x), discrete prediction function
\newcommand{\hh}{\hat{h}} % h hat
\newcommand{\hxh}{\hat{h}(\xv)} % hhat(x)
\newcommand{\hxt}{h(\xv | \thetav)} % h(x | theta)
\newcommand{\hxi}{h\left(\xi\right)} % h(x^(i))
\newcommand{\hxit}{h\left(\xi ~|~ \thetav\right)} % h(x^(i) | theta)
\newcommand{\hbayes}{h^{\ast}} % Bayes-optimal classification model
\newcommand{\hxbayes}{h^{\ast}(\xv)} % Bayes-optimal classification model

% yhat
\newcommand{\yh}{\hat{y}} % yhat for prediction of target
\newcommand{\yih}{\hat{y}^{(i)}} % yhat^(i) for prediction of ith targiet
\newcommand{\resi}{\yi- \yih}

% theta
\newcommand{\thetah}{\hat{\theta}} % theta hat
\newcommand{\thetav}{\bm{\theta}} % theta vector
\newcommand{\thetavh}{\bm{\hat\theta}} % theta vector hat
\newcommand{\thetat}[1][t]{\thetav^{[#1]}} % theta^[t] in optimization
\newcommand{\thetatn}[1][t]{\thetav^{[#1 +1]}} % theta^[t+1] in optimization
\newcommand{\thetahDnlam}{\thetavh_{\Dn, \lamv}} %theta learned on Dn with hp lambda
\newcommand{\thetahDlam}{\thetavh_{\D, \lamv}} %theta learned on D with hp lambda
\newcommand{\mint}{\min_{\thetav \in \Theta}} % min problem theta
\newcommand{\argmint}{\argmin_{\thetav \in \Theta}} % argmin theta

% densities + probabilities
% pdf of x
\newcommand{\pdf}{p} % p
\newcommand{\pdfx}{p(\xv)} % p(x)
\newcommand{\pixt}{\pi(\xv~|~ \thetav)} % pi(x|theta), pdf of x given theta
\newcommand{\pixit}[1][i]{\pi\left(\xi[#1] ~|~ \thetav\right)} % pi(x^i|theta), pdf of x given theta
\newcommand{\pixii}[1][i]{\pi\left(\xi[#1]\right)} % pi(x^i), pdf of i-th x

% pdf of (x, y)
\newcommand{\pdfxy}{p(\xv,y)} % p(x, y)
\newcommand{\pdfxyt}{p(\xv, y ~|~ \thetav)} % p(x, y | theta)
\newcommand{\pdfxyit}{p\left(\xi, \yi ~|~ \thetav\right)} % p(x^(i), y^(i) | theta)

% pdf of x given y
\newcommand{\pdfxyk}[1][k]{p(\xv | y= #1)} % p(x | y = k)
\newcommand{\lpdfxyk}[1][k]{\log p(\xv | y= #1)} % log p(x | y = k)
\newcommand{\pdfxiyk}[1][k]{p\left(\xi | y= #1 \right)} % p(x^i | y = k)

% prior probabilities
\newcommand{\pik}[1][k]{\pi_{#1}} % pi_k, prior
\newcommand{\pih}{\hat{\pi}} % pi hat, estimated prior (binary classification)
\newcommand{\pikh}[1][k]{\hat{\pi}_{#1}} % pi_k hat, estimated prior
\newcommand{\lpik}[1][k]{\log \pi_{#1}} % log pi_k, log of the prior
\newcommand{\pit}{\pi(\thetav)} % Prior probability of parameter theta

% posterior probabilities
\newcommand{\post}{\P(y = 1 ~|~ \xv)} % P(y = 1 | x), post. prob for y=1
\newcommand{\postk}[1][k]{\P(y = #1 ~|~ \xv)} % P(y = k | y), post. prob for y=k
\newcommand{\pidomains}{\pi: \Xspace \rightarrow \unitint} % pi with domain and co-domain
\newcommand{\pibayes}{\pi^{\ast}} % Bayes-optimal classification model
\newcommand{\pixbayes}{\pi^{\ast}(\xv)} % Bayes-optimal classification model
\newcommand{\piastxtil}{\pi^{\ast}(\xtil)} % Bayes-optimal classification model
\newcommand{\piastkxtil}{\pi^{\ast}_k(\xtil)} % Bayes-optimal classification model for k-th class
\newcommand{\pix}{\pi(\xv)} % pi(x), P(y = 1 | x)
\newcommand{\piv}{\bm{\pi}} % pi, bold, as vector
\newcommand{\pikx}[1][k]{\pi_{#1}(\xv)} % pi_k(x), P(y = k | x)
\newcommand{\pikxt}[1][k]{\pi_{#1}(\xv ~|~ \thetav)} % pi_k(x | theta), P(y = k | x, theta)
\newcommand{\pixh}{\hat \pi(\xv)} % pi(x) hat, P(y = 1 | x) hat
\newcommand{\pikxh}[1][k]{\hat \pi_{#1}(\xv)} % pi_k(x) hat, P(y = k | x) hat
\newcommand{\pixih}{\hat \pi(\xi)} % pi(x^(i)) with hat
\newcommand{\pikxih}[1][k]{\hat \pi_{#1}(\xi)} % pi_k(x^(i)) with hat
\newcommand{\pdfygxt}{p(y ~|~\xv, \thetav)} % p(y | x, theta)
\newcommand{\pdfyigxit}{p\left(\yi ~|~\xi, \thetav\right)} % p(y^i |x^i, theta)
\newcommand{\lpdfygxt}{\log \pdfygxt } % log p(y | x, theta)
\newcommand{\lpdfyigxit}{\log \pdfyigxit} % log p(y^i |x^i, theta)

% probabilistic
\newcommand{\bayesrulek}[1][k]{\frac{\P(\xv | y= #1) \P(y= #1)}{\P(\xv)}} % Bayes rule
\newcommand{\muv}{\bm{\mu}} % expectation vector of Gaussian
\newcommand{\muk}[1][k]{\bm{\mu_{#1}}} % mean vector of class-k Gaussian (discr analysis)
\newcommand{\mukh}[1][k]{\bm{\hat{\mu}_{#1}}} % estimated mean vector of class-k Gaussian (discr analysis)

% residual and margin
\newcommand{\rx}{r(\xv)} % residual 
\newcommand{\eps}{\epsilon} % residual, stochastic
\newcommand{\epsv}{\bm{\epsilon}} % residual, stochastic, as vector
\newcommand{\epsi}{\epsilon^{(i)}} % epsilon^i, residual, stochastic
\newcommand{\epsh}{\hat{\epsilon}} % residual, estimated
\newcommand{\epsvh}{\hat{\epsv}} % residual, estimated, vector
\newcommand{\yf}{y \fx} % y f(x), margin
\newcommand{\yfi}{\yi \fxi} % y^i f(x^i), margin
\newcommand{\Sigmah}{\hat \Sigma} % estimated covariance matrix
\newcommand{\Sigmahj}{\hat \Sigma_j} % estimated covariance matrix for the j-th class
\newcommand{\nux}{\nu(\xv)} % nu(x) = y * f(x)

% ml - loss, risk, likelihood
\newcommand{\Lyf}{L\left(y, f\right)} % L(y, f), loss function
% \newcommand{\Lypi}{L\left(y, \pi\right)} % L(y, pi), loss function
\newcommand{\Lxy}{L\left(y, \fx\right)} % L(y, f(x)), loss function
\newcommand{\Lxyi}{L\left(\yi, \fxi\right)} % loss of observation
\newcommand{\Lxyt}{L\left(y, \fxt\right)} % loss with f parameterized
\newcommand{\Lxyit}{L\left(\yi, \fxit\right)} % loss of observation with f parameterized
\newcommand{\Lxym}{L\left(\yi, f\left(\bm{\tilde{x}}^{(i)} ~|~ \thetav\right)\right)} % loss of observation with f parameterized
\newcommand{\Lpixy}{L\left(y, \pix\right)} % loss in classification
% \newcommand{\Lpiy}{L\left(y, \pi\right)} % loss in classification
\newcommand{\Lpiv}{L\left(y, \piv\right)} % loss in classification
\newcommand{\Lpixyi}{L\left(\yi, \pixii\right)} % loss of observation in classification
\newcommand{\Lpixyt}{L\left(y, \pixt\right)} % loss with pi parameterized
\newcommand{\Lpixyit}{L\left(\yi, \pixit\right)} % loss of observation with pi parameterized
% \newcommand{\Lhy}{L\left(y, h\right)} % L(y, h), loss function on discrete classes
\newcommand{\Lhxy}{L\left(y, \hx\right)} % L(y, h(x)), loss function on discrete classes
\newcommand{\Lr}{L\left(r\right)} % L(r), loss defined on residual (reg) / margin (classif)
\newcommand{\lone}{|y - \fx|} % L1 loss
\newcommand{\ltwo}{\left(y - \fx\right)^2} % L2 loss
\newcommand{\lbernoullimp}{\ln(1 + \exp(-y \cdot \fx))} % Bernoulli loss for -1, +1 encoding
\newcommand{\lbernoullizo}{- y \cdot \fx + \log(1 + \exp(\fx))} % Bernoulli loss for 0, 1 encoding
\newcommand{\lcrossent}{- y \log \left(\pix\right) - (1 - y) \log \left(1 - \pix\right)} % cross-entropy loss
\newcommand{\lbrier}{\left(\pix - y \right)^2} % Brier score
\newcommand{\risk}{\mathcal{R}} % R, risk
\newcommand{\riskbayes}{\mathcal{R}^\ast}
\newcommand{\riskf}{\risk(f)} % R(f), risk
\newcommand{\riskdef}{\E_{y|\xv}\left(\Lxy \right)} % risk def (expected loss)
\newcommand{\riskt}{\mathcal{R}(\thetav)} % R(theta), risk
\newcommand{\riske}{\mathcal{R}_{\text{emp}}} % R_emp, empirical risk w/o factor 1 / n
\newcommand{\riskeb}{\bar{\mathcal{R}}_{\text{emp}}} % R_emp, empirical risk w/ factor 1 / n
\newcommand{\riskef}{\riske(f)} % R_emp(f)
\newcommand{\risket}{\mathcal{R}_{\text{emp}}(\thetav)} % R_emp(theta)
\newcommand{\riskr}{\mathcal{R}_{\text{reg}}} % R_reg, regularized risk
\newcommand{\riskrt}{\mathcal{R}_{\text{reg}}(\thetav)} % R_reg(theta)
\newcommand{\riskrf}{\riskr(f)} % R_reg(f)
\newcommand{\riskrth}{\hat{\mathcal{R}}_{\text{reg}}(\thetav)} % hat R_reg(theta)
\newcommand{\risketh}{\hat{\mathcal{R}}_{\text{emp}}(\thetav)} % hat R_emp(theta)
\newcommand{\LL}{\mathcal{L}} % L, likelihood
\newcommand{\LLt}{\mathcal{L}(\thetav)} % L(theta), likelihood
\newcommand{\LLtx}{\mathcal{L}(\thetav | \xv)} % L(theta|x), likelihood
\newcommand{\logl}{\ell} % l, log-likelihood
\newcommand{\loglt}{\logl(\thetav)} % l(theta), log-likelihood
\newcommand{\logltx}{\logl(\thetav | \xv)} % l(theta|x), log-likelihood
\newcommand{\errtrain}{\text{err}_{\text{train}}} % training error
\newcommand{\errtest}{\text{err}_{\text{test}}} % test error
\newcommand{\errexp}{\overline{\text{err}_{\text{test}}}} % avg training error

% lm
\newcommand{\thx}{\thetav^\top \xv} % linear model
\newcommand{\olsest}{(\Xmat^\top \Xmat)^{-1} \Xmat^\top \yv} % OLS estimator in LM

\input{../latex-math/ml-trees.tex}
\input{../latex-math/ml-nn.tex}


\title{Supervised Learning :\,: CHEAT SHEET} % Package title in header, \, adds thin space between ::
\newcommand{\packagedescription}{ % Package description in header
%	The \textbf{I2ML}: Introduction to Machine Learning course offers an introductory and applied overview of "supervised" Machine Learning. It is organized as a digital lecture.
}

\newlength{\columnheight} % Adjust depending on header height
\setlength{\columnheight}{84cm} 

\newtcolorbox{codebox}{%
	sharp corners,
	leftrule=0pt,
	rightrule=0pt,
	toprule=0pt,
	bottomrule=0pt,
	hbox}

\newtcolorbox{codeboxmultiline}[1][]{%
	sharp corners,
	leftrule=0pt,
	rightrule=0pt,
	toprule=0pt,
	bottomrule=0pt,
	#1}

\begin{document}
\begin{frame}[fragile]{}
\begin{columns}
	\begin{column}{.31\textwidth}
		\begin{beamercolorbox}[center]{postercolumn}
			\begin{minipage}{.98\textwidth}
				\parbox[t][\columnheight]{\textwidth}{

					\begin{myblock}{Multiclass Classification}

      Multiclass classification with $g > 2$ classes
$$\D \subset \left(\Xspace \times \Yspace\right)^n, \Yspace = \{1, ..., g\}$$ 

Goal is to find a model  $f: \Xspace \to \R^g$, where $g$ is the number of classes, that minimizes the expected loss over random variables $\xy \sim \Pxy$ 
$$
 \riskf = \E_{xy}[\Lxy] = \E_{x}\left[\sum_{k \in \Yspace} L(k, f(\bm{x})) \P(y = k| \xv = \xv)\right] 
$$

The optimal model for a loss function $\Lxy$ is
$$ \fxh = \argmin_{f \in \Hspace} \sum_{k \in \Yspace} L(k, f(\bm{x})) \P(y = k| \xv = \xv)\, $$

ERM: $\fh = \argmin_{f \in \Hspace} \riske(f) = \argmin_{f \in \Hspace} \sumin \Lxyi$

\begin{codebox} 
  \textbf{One-Hot Encoding}
  \end{codebox}
$$
\text{with}\quad \mathds{1}_{\{y = k\}} = \begin{cases} 1 & \text{ if } y = k \\
0 & \text{ otherwise}\end{cases}
$$

\begin{codebox} 
  \textbf{Notations}
  \end{codebox}
\begin{itemize}[$\bullet$] 
  \setlength{\itemindent}{+.3in}
    \item Vectors of scores $$f(\xv) = \left(f_1(\xv), ..., f_g(\xv)\right)$$
    \item Vectors of probabilities $$\pi(\xv) = \left(\pi_1(\xv), ..., \pi_g(\xv)\right)$$
    \item Hard labels $$\hx = k, k \in \{1, 2, ..., g\}$$
\end{itemize}

\begin{codebox} 
  \textbf{Loss Functions}
  \end{codebox}

\begin{itemize}[$\bullet$] 
  \setlength{\itemindent}{+.3in}
  \item 0-1 Loss
  $$ L(y, \hx) = \mathds{1}_{\{y \neq \hx\}} $$
  \item Brier Score
  $$ L(y, \pi(x)) = \sum_{k = 1}^g \left(\mathds{1}_{\{y = k\}} - \pi_k(\xv)\right)^2 $$
  \item Logarithmic Loss
  $$ L(y, \pi(x)) = - \sum_{k = 1}^g \mathds{1}_{\{y = k\}} \log\left(\pi_k(\xv)\right) $$
\end{itemize}

\end{myblock}

				}
			\end{minipage}
		\end{beamercolorbox}
	\end{column}
	
%%%%%%%%%%%%%%%%%%%%%%%%%%%%%%%%%%%%%%%%%%%%%%%%%%%%%%%%%%%%%%%%%%%%%

\begin{column}{.31\textwidth}
\begin{beamercolorbox}[center]{postercolumn}
\begin{minipage}{.98\textwidth}
\parbox[t][\columnheight]{\textwidth}{

\begin{myblock}{Softmax Regression}

  Softmax regression gives us a \textbf{linear classifier}.

  We have $g$ linear discriminant functions
  $$
      f_k(\xv) = \thetav_k^\top \xv, \quad k = 1, 2, ..., g,
  $$
  each indicating the confidence in class $k$.\\

  The $g$ score functions are transformed into $g$ probability functions by the \textbf{softmax} function $s:\R^g \to [0,1]^g$ 
  $$
    \pi_k(\xv) = s(\fx)_k = \frac{\exp(\thetav_k^\top \xv)}{\sum_{j = 1}^g \exp(\thetav_j^\top \xv) }\,,
  $$

  The probabilities are well-defined: $\sum \pi_k(\xv) = 1$ and $\pi_k(\xv) \in [0, 1]$ for all $k$.\\

  Use the multiclass \textbf{logarithmic loss}
  $$
    L(y, \pix) = - \sum_{k = 1}^g \mathds{1}_{\{y = k\}} \log\left(\pi_k(\xv)\right).
  $$ \\
  
  The softmax function is 
  \begin{itemize}[$\bullet$] 
  \setlength{\itemindent}{+.3in}
  \item a smooth approximation of the arg max operation: \\
  $s((1, 1000, 2)^T) \approx (0, 1, 0)^T$ (picks out 2nd element).  
  \item invariant to constant offsets in the input:  
    $$ 
    s(\fx + \mathbf{c}) = \frac{\exp(\thetav_k^\top \xv + c)}{\sum_{j = 1}^g \exp(\thetav_j^\top \xv + c)} = 
    \frac{\exp(\thetav_k^\top \xv)\cdot \exp(c)}{\sum_{j = 1}^g \exp(\thetav_j^\top \xv) \cdot \exp(c)} = 
    s(\fx)
    $$  
\end{itemize}
  \end{myblock}


  \begin{myblock}{Binary Reduction}

    Computational effort for one-vs-one is much higher than for one-vs-rest.
  \begin{codebox} 
  \textbf{One-vs-Rest}
  \end{codebox}

Create $g$ binary subproblems, where in each the $k$-th original class is encoded as $+1$, and all other classes (the \textbf{rest}) as $- 1$.

Applying all classifiers to a sample $\xv \in \Xspace$ and predicting the label $k$ for which the corresponding classifier reports the highest confidence: 
    $$
      \hat y = \text{arg max}_{k \in \{1, 2, ..., g\}} \hat f_k(\xv). 
    $$
  
\end{myblock}
}
\end{minipage}
\end{beamercolorbox}
\end{column}

%%%%%%%%%%%%%%%%%%%%%%%%%%%%%%%%%%%%%%%%%%%%%%%%%%%%%%%%%%%%%%%%%%%%%%%%%%%%%%%%%%%%%%%%%%%%%%%%%%%%

\begin{column}{.31\textwidth}
\begin{beamercolorbox}[center]{postercolumn}
\begin{minipage}{.98\textwidth}
\parbox[t][\columnheight]{\textwidth}{

  \begin{myblock}{}
  
    \begin{codebox} 
  \textbf{One-vs-One}
  \end{codebox}

Create $\frac{g(g - 1)}{2}$ binary sub-problems, where each $\D_{k, \tilde k} \subset \D$ only considers observations from a class-pair $\yi \in \{k, \tilde k\}$, other observations are omitted.

Label prediction is done via \textbf{majority voting}. We predict the label of a new $\xv$ with all classifiers and select the class that occurred most often. 

    \begin{codebox} 
  \textbf{Error-Correcting Output Codes (ECOC)}
  \end{codebox}

  \textbf{Codebooks:} The k-th column defines how classes of all observations are encoded in the binary subproblem / for binary classifier $f_k(\xv)$.
								Entry $(m, i)$ takes values $\in \{-1, 0, +1\}$
								\begin{itemize}
									\setlength{\itemindent}{+.3in}
									\item if $0$, observations of class $\yi = m$ are ignored.
									\item if $1$, observations of class $\yi = m$ are encoded as $1$.
									\item if $- 1$, observations of class $\yi = m$ are encoded as $- 1$.
								\end{itemize}
  \begin{table}[]
  \begin{tabular}{|c|r|r|r|r|r|r|r|r|} \hline
  \textbf{Class}  & \textbf{$f_1{(\xv)}$} & \textbf{$f_2{(\xv)}$}  & \textbf{$f_3{(\xv)}$} & \textbf{$f_4{(\xv)}$} & \textbf{$f_5{(\xv)}$} & \textbf{$f_6{(\xv)}$} & \textbf{$f_7{(\xv)}$} & \textbf{$f_8{(\xv)}$}\\ \hline
  \textbf{$1$} & -1 & -1 & -1 & -1 & 1 & 1 & 1 & 1 \\ \hline
  \textbf{$2$} & -1 & - 1 & 1 & 1 & -1 & -1 & 1 & 1 \\ \hline
  \textbf{$3$} & -1 & 1 & -1 & 1 & -1 & 1 & -1 & 1\\ \hline
  \end{tabular}
  \end{table}\\

  We want to maximize distances between rows, and want the distances between columns to not be too small (identical columns) or too high (complementary columns):
  \begin{itemize}[$\bullet$] 
  \setlength{\itemindent}{+.3in}
  \item Row separation: each codeword should be well-separated in Hamming distance from each of the other codewords.
  \item Column separation: columns should be uncorrelated
  \end{itemize}\\

  Train L binary classifiers: only few classes $g \le 11$, exhaustive search can be performed.
  
  Randomized hill-climbing algorithm:
\begin{itemize}[$\bullet$]
  \setlength{\itemindent}{+.3in}
    \item $g$ codewords of length $L$ are randomly drawn. 
    \item Any pair of such random strings will be separated by a Hamming distance that is binomially distributed with mean $\frac{L}{2}$.
    \item Iteratively improves the code: algorithm repeatedly finds the pair of rows closest together and the pair of columns that have the most extreme distance.
    \item Then computes the four codeword bits where these rows and columns intersect and changes them to improve the row and column separations.
    \item When the procedure reaches a local maximum, the algorithm randomly chooses pairs of rows and columns and tries to improve their separation.
  \end{itemize}

\end{myblock}
  }
  
  \end{minipage}
  \end{beamercolorbox}
  \end{column}
  
  
  
\end{columns}
\end{frame}
\end{document}