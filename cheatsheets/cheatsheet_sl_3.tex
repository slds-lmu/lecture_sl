\documentclass{beamer}


\usepackage[orientation=landscape,size=a0,scale=1.4,debug]{beamerposter}
\mode<presentation>{\usetheme{mlr}}


\usepackage[utf8]{inputenc} % UTF-8
\usepackage[english]{babel} % Language
\usepackage{hyperref} % Hyperlinks
\usepackage{ragged2e} % Text position
\usepackage[export]{adjustbox} % Image position
\usepackage[most]{tcolorbox}
\usepackage{amsmath}
\usepackage{mathtools}
\usepackage{dsfont}
\usepackage{verbatim}
\usepackage{amsmath}
\usepackage{amsfonts}
\usepackage{csquotes}
\usepackage{multirow}
\usepackage{longtable}
\usepackage{enumerate}
\usepackage[absolute,overlay]{textpos}
\usepackage{psfrag}
\usepackage{algorithm}
\usepackage{algpseudocode}
\usepackage{eqnarray}
\usepackage{arydshln}
\usepackage{tabularx}
\usepackage{placeins}
\usepackage{tikz}
\usepackage{setspace}
\usepackage{colortbl}
\usepackage{mathtools}
\usepackage{wrapfig}
\usepackage{bm}
\usepackage{nicefrac}

\input{../../latex-math/basic-math.tex}
\input{../../latex-math/basic-ml.tex}
\input{../../latex-math/ml-svm}
\input{../../latex-math/ml-gp}
\input{../../latex-math/ml-ensembles.tex}
\input{../../latex-math/ml-trees.tex}	


\title{SL :\,: BASICS} % Package title in header, \, adds thin space between ::
\newcommand{\packagedescription}{ \invisible{x} % Package description in header
	% The \textbf{I2ML}: Introduction to Machine Learning course offers an introductory and applied overview of "supervised" Machine Learning. It is organized as a digital lecture.
}

\newlength{\columnheight} % Adjust depending on header height
\setlength{\columnheight}{84cm} 

\newtcolorbox{codebox}{%
	sharp corners,
	leftrule=0pt,
	rightrule=0pt,
	toprule=0pt,
	bottomrule=0pt,
	hbox}

\newtcolorbox{codeboxmultiline}[1][]{%
	sharp corners,
	leftrule=0pt,
	rightrule=0pt,
	toprule=0pt,
	bottomrule=0pt,
	#1}
	

	
\begin{document}
\begin{frame}[fragile]{}
\vspace{-8ex}
\begin{columns}
	\begin{column}{.31\textwidth}
		\begin{beamercolorbox}[center]{postercolumn}
			\begin{minipage}{.98\textwidth}
				\parbox[t][\columnheight]{\textwidth}{
%%%%%%%%%%%%%%%%%%%%%%%%%%%%%%%%%%%%%%%%%%%%%%%%%%%%%%%%%%%%%%%%%%%%%%%%%%%%%%%%
% First Column begin
%-------------------------------------------------------------------------------
% Data
%-------------------------------------------------------------------------------
\begin{myblock}{Nonlinear Support Vector Machines}
%	
%	
	
	\textbf{Dual kernelized soft-margin SVM:}
%	
		\begin{eqnarray*}
			& \max_{\alpha} & \sum_{i=1}^n \alpha_i - \frac{1}{2}\sum_{i=1}^n\sum_{j=1}^n\alpha_i\alpha_j\yi y^{(j)} k(\xi, \xv^{(j)})  \\
			& \text{s.t. } & 0 \le \alpha_i \le C, \forall\, i \in \nset \quad \text{and} \quad  \sum_{i=1}^n \alpha_i \yi = 0
		\end{eqnarray*}	
%	
	Kernel representation of separating hyperplane:
%
		$$ \fx = \sumin \alpha_i \yi k(\xi, \xv)  + \theta_0$$
%
\end{myblock}
%
%-------------------------------------------------------------------------------
%   
%-------------------------------------------------------------------------------
\begin{myblock}{Gaussian Processes}
%
	Bayesian Linear Model:
%
	\begin{eqnarray*}
		\yi &=& \fxi + \epsi = \thetab^T \xi + \epsi, \quad \text{for } i \in \{1, \ldots, n\}
	\end{eqnarray*}
	%
	where $\epsi \sim \mathcal{N}(0, \sigma^2).$
%
	Parameter vector $\thetab$ is stochastic and follows a distribution.\\
%	
	
	Gaussian variant: 
%	
	\begin{itemize}
		\setlength{\itemindent}{+.3in}
		\item Prior distribution: $\thetab \sim \mathcal{N}(\zero, \tau^2 \id_p)$ 
		\item Posterior distribution:	$
		\thetab ~|~ \Xmat, \yv \sim \mathcal{N}(\sigma^{-2}\bm{A}^{-1}\Xmat^\top\yv, \bm{A}^{-1})
		$ with $\bm{A}:= \sigma^{-2}\Xmat^\top\Xmat + \frac{1}{\tau^2} \id_p$
		\item Predictive distribution of $y_* = 	\thetab^\top \xv_*$ for a new observations $\xv_*$: 
		$$
		y_* ~|~ \Xmat, \yv, \xv_* \sim \mathcal{N}(\sigma^{-2}\yv^\top \Xmat \Amat^{-1}\xv_*, \xv_*^\top\Amat^{-1}\xv_*)
		$$
	\end{itemize}

%	
 
	
%
	\fbox{
	\begin{table}
	\begin{tabular}{cc}
		\textbf{Weight-Space View} & \textbf{Function-Space View} \vspace{4mm}\\ 
		Parameterize functions & \vspace{1mm}\\
		\footnotesize Example: $\fxt = \thetab^\top \xv$ & \vspace{3mm}\\
		Define distributions on $\thetab$ & Define distributions on $f$ \vspace{4mm}\\
		Inference in parameter space $\Theta$ & Inference in function space $\Hspace$
	\end{tabular}
	\end{table}  }\\
%


$\hphantom{text}$\\

\textbf{Gaussian Processes:} A function $\fx$ is generated by a GP $\gp$ if for \textbf{any finite} set of inputs $\left\{\xv^{(1)}, \dots, \xv^{(n)}\right\}$, the associated vector of function values $\bm{f} = \left(f(\xv^{(1)}), \dots, f(\xv^{(n)})\right)$ has a Gaussian distribution
%
$$
\bm{f} = \left[f\left(\xi[1]\right),\dots, f\left(\xi[n]\right)\right] \sim \mathcal{N}\left(\bm{m}, \bm{K}\right),
$$
%
with 
%
\begin{eqnarray*}
	\textbf{m} &:=& \left(m\left(\xi\right)\right)_{i}, \quad
	\textbf{K} := \left(k\left(\xi, \xv^{(j)}\right)\right)_{i,j}, 
\end{eqnarray*}
%
where $m(\xv)$ is the mean function and $k(\xv, \xv^\prime)$ is the covariance function. 
 
\end{myblock}\vfill
% End First Column
%%%%%%%%%%%%%%%%%%%%%%%%%%%%%%%%%%%%%%%%%%%%%%%%%%%%%%%%%%%%%%%%%%%%%%%%%%%%%%%%
				}
			\end{minipage}
		\end{beamercolorbox}
	\end{column}
	\begin{column}{.31\textwidth}
		\begin{beamercolorbox}[center]{postercolumn}
			\begin{minipage}{.98\textwidth}
				\parbox[t][\columnheight]{\textwidth}{
%%%%%%%%%%%%%%%%%%%%%%%%%%%%%%%%%%%%%%%%%%%%%%%%%%%%%%%%%%%%%%%%%%%%%%%%%%%%%%%%
% Begin Second Column
\begin{myblock}{}  
%
	Types of covariance functions:
	%		
	\begin{itemize}
		\setlength{\itemindent}{+.3in}
		%			
		\item $k(.,.)$ is stationary if it is as a function of $\bm{d} = \bm{x} - \bm{x}^\prime$, $ \leadsto k(\bm{d})$
		\item $k(.,.)$ is isotropic if it is a function of $r = \|\bm{x} - \bm{x}^\prime\|$,  $ \leadsto k(r)$
		\item $k(., .)$ is a dot product covariance function if $k$ is a function of $\bm{x}^T \bm{x}^\prime$
	\end{itemize}
%
	Commonly used covariance functions
	
	\begin{center}
			\fbox{  
			\begin{tabular}{|c|c|}
				\hline
				Name & $k(\bm{x}, \bm{x}^\prime)$\\
				\hline
				constant & $\sigma_0^2$ \\ [1em]
				linear & $\sigma_0^2 + \bm{x}^T\bm{x}^\prime$ \\ [1em]
				polynomial & $(\sigma_0^2 + \bm{x}^T\bm{x}^\prime)^p$ \\ [1em]
				squared exponential & $\exp(- \frac{\|\bm{x} - \bm{x}^\prime\|^2}{2\ls^2})$ \\ [1em]
				Matérn & \begin{footnotesize} $\frac{1}{2^\nu \Gamma(\nu)}\biggl(\frac{\sqrt{2 \nu}}{\ls}\|\bm{x} - \bm{x}^\prime\|\biggr)^{\nu} K_\nu\biggl(\frac{\sqrt{2 \nu}}{\ls}\|\bm{x} - \bm{x}^\prime\|\biggr)$\end{footnotesize}  \\ [1em]
				exponential & $\exp\left(- \frac{\|\bm{x} - \bm{x}^\prime\|}{\ls}\right)$ \\ [1em]
				\hline
		\end{tabular} }\\
	\end{center}
		


	%
	\textbf{Posterior process} 
	
	Assuming a zero-mean GP prior $\mathcal{GP}\left(\bm{0}, k(\xv, \xv^\prime)\right).$ 
%	
	For $ f_* = f\left(\xv_*\right)$ on single unobserved test point $\xv_*$ 
%	
	\begin{eqnarray*}
		f_* ~|~ \xv_*, \Xmat, \bm{f} \sim \mathcal{N}(\bm{k}_{*}^{T}\Kmat^{-1}\bm{f}, \bm{k}_{**} - \bm{k}_*^T \Kmat ^{-1}\bm{k}_*),
	\end{eqnarray*}
%
	where, $\Kmat = \left(k\left(\xi, \xv^{(j)}\right)\right)_{i,j}$, $\bm{k}_* = \left[k\left(\xv_*, \xi[1]\right), ..., k\left(\xv_*, \xi[n]\right)\right]$ and $ \bm{k}_{**}\ = k(\xv_*, \xv_*)$. \\
	
	%
	For multiple unobserved test points
	$
	\bm{f}_* = \left[f\left(\xi[1]_*\right), ..., f\left(\xi[m]_*\right)\right]:
	$
	\begin{eqnarray*}
		\bm{f}_* ~|~ \Xmat_*, \Xmat, \bm{f} \sim \mathcal{N}(\Kmat_{*}^{T}\Kmat^{-1}\bm{f}, \Kmat_{**} - \Kmat_*^T \Kmat ^{-1}\Kmat_*).
	\end{eqnarray*} 
	with $\Kmat_* = \left(k\left(\xi, \xv_*^{(j)}\right)\right)_{i,j}$, $\Kmat_{**} = \left(k\left(\xi[i]_*, \xi[j]_*\right)\right)_{i,j}$.\\
	
	Predictive mean when assuming a non-zero mean GP prior $\gp$ with mean $m(\xv):$ 
	$$
	m(\Xmat_*) + \Kmat_*\Kmat^{-1}\left(\bm{y} - m(\Xmat)\right)
	$$
	Predictive variance remains unchanged. \\
	
	\textbf{Noisy posterior process:}
%	
	Assuming a zero-mean GP prior $\mathcal{GP}\left(\bm{0}, k(\xv, \xv^\prime)\right):$ 
%	
	\begin{eqnarray*}
		\bm{f}_* ~|~ \Xmat_*, \Xmat, \bm{y} \sim \mathcal{N}(\bm{m}_{\text{post}}, \bm{K}_\text{post}).
	\end{eqnarray*}
	with nugget $\sigma^2 $ and
%	 
	\begin{eqnarray*}
		\bm{m}_{\text{post}} &=& \Kmat_{*}^{T} \left(\Kmat+ \sigma^2 \cdot \id\right)^{-1}\bm{y} \\
		\bm{K}_\text{post} &=& \Kmat_{**} - \Kmat_*^T \left(\Kmat  + \sigma^2 \cdot \id\right)^{-1}	\Kmat_*,	
	\end{eqnarray*} 
%	
	
	Predictive mean when assuming a non-zero mean GP prior $\gp$ with mean $m(\xv):$ 
	$$
	m(\Xmat_*) + \Kmat_*(\Kmat +\sigma^2 \id)^{-1}\left(\bm{y} - m(\Xmat)\right)
	$$
	Predictive variance remains unchanged.
%
%
\end{myblock}


 
% End Second Column					
%%%%%%%%%%%%%%%%%%%%%%%%%%%%%%%%%%%%%%%%%%%%%%%%%%%%%%%%%%%%%%%%%%%%%%%%%%%%%%%%
				}
			\end{minipage}
		\end{beamercolorbox}
	\end{column}
	\begin{column}{.31\textwidth}
		\begin{beamercolorbox}[center]{postercolumn}
			\begin{minipage}{.98\textwidth}
				\parbox[t][\columnheight]{\textwidth}{
%%%%%%%%%%%%%%%%%%%%%%%%%%%%%%%%%%%%%%%%%%%%%%%%%%%%%%%%%%%%%%%%%%%%%%%%%%%%%%%%
% Begin Third Column#


%-------------------------------------------------------------------------------
% Regression Losses 
%------------------------------------------------------------------------------- 
\begin{myblock}{Boosting}
%	

\begin{algorithm}[H]
	\begin{algorithmic}[1]
		\State Initialize observation weights: $w^{[1](i)} = \frac{1}{n} \quad \forall i \in \nset$
		\For {$m = 1 \to M$}
		\State Fit classifier to training data with weights $\wm$ and get $\blh$
		\State Calculate weighted in-sample misclassification rate
		$$
		\errm = \sumin \wmi \cdot \mathds{1}_{\{\yi \,\neq\, \blh(\xi)\}}
		$$
		\State Compute: $ \betamh = \frac{1}{2} \log \left( \frac{1 - \errm}{\errm}\right)$
		\State Set: $w^{[m+1](i)} = \wmi \cdot \exp\left(- \betamh \cdot
		\yi \cdot \blh(\xi)\right) $
		\State Normalize $w^{[m+1](i)}$ such that $\sumin w^{[m+1](i)} = 1$
		\EndFor
		\State Output: $\fxh = \sum_{m=1}^{M} \betamh \blh(\xv)$
	\end{algorithmic}
	\caption{AdaBoost}
\end{algorithm}
%

$\hphantom{text}$\\

\begin{algorithm}[H]
%	\begin{footnotesize}
		\begin{center}
			\caption{Gradient Boosting Algorithm}
			\begin{algorithmic}[1]
				\State Initialize $\hat{f}^{[0]}(\xv) = \argmin_{\bm{\theta}} \sumin L(\yi, b(\xi, \bm{\theta}))$
				%\State Set the learning rate $\beta$ to a small constant value
				\For{$m = 1 \to M$}
				\State For all $i$: $\rmi = -\left[\pd{\Lxyi}{\fxi}\right]_{f=\fmdh}$
				\State Fit a regression base learner to the pseudo-residuals $\rmi$:
				\State $\thetamh = \argmin \limits_{\bm{\theta}} \sumin (\rmi - b(\xi, \bm{\theta}))^2$
				%\State Line search: $\betamh = \argmin_{\beta} \sumin L(\yi, \fmd(\xv) + \beta b(\xv, \thetamh))$
				\State Set $\betam$ to $\beta$ being a small constant value or via line search
				\State Update $\fmh(\xv) = \fmdh(\xv) + \betam b(\xv, \thetamh)$
				\EndFor
				\State Output $\fh(\xv) = \hat{f}^{[M]}(\xv)$
			\end{algorithmic}
		\end{center}
%	\end{footnotesize}
\end{algorithm} 

$\hphantom{text}$\\

\begin{algorithm}[H]
%	\begin{footnotesize}
		\begin{center}
			\caption{Gradient Boosting for Multiclass}
			\begin{algorithmic}[1]
				\State Initialize $f_{k}^{[0]}(\xv) = 0,\ k = 1,\ldots,g$
				\For{$m = 1 \to M$}
				\State Set $\pik^{[m]}(\xv) = \frac{\exp(f_k^{[m]}(\xv))}{\sum_j \exp(f_j^{[m]}(\xv))}, k = 1,\ldots,g$
				\For{$k = 1 \to g$}
				\State For all $i$: Compute $\rmi_k = \mathds{1}_{\{\yi = k\}} - \pik^{[m]}(\xi)$
				\State Fit a regression base learner $\hat{b}^{[m]}_k$ to the pseudo-residuals $\rmi_k$
				\State Obtain $\betamh_k$ by constant learning rate or line-search
				\State Update $\hat{f}_k^{[m]} = \hat{f}_k^{[m-1]} + \betamh_k \hat{b}^{[m]}_k$
				\EndFor
				\EndFor
				\State Output $\hat{f}_1^{[M]}, \ldots, \hat{f}_g^{[M]}$
			\end{algorithmic}
		\end{center}
%	\end{footnotesize}
\end{algorithm}
	
\end{myblock} 
%
%
%-------------------------------------------------------------------------------
% Classification 
%------------------------------------------------------------------------------- 

%\begin{myblock}{Classification}
% 				    We want to assign new observations to known categories according to criteria learned from a training set.  
%             \vspace*{1ex}
%             

%$y \in \Yspace = \gset : $ categorical output variable (label)\\ 

%\textbf{Classification} usually means to construct $g$ \textbf{discriminant functions}:
  
%$f_1(\xv), \ldots, \fgx$, so that we choose our class as \\ $h(\xv) = \argmax_{k \in \gset} \fkx$ \\

%\textbf{Linear Classifier:} functions $\fkx$ can be specified as linear functions\\

% \hspace*{1ex}\textbf{Note: }All linear classifiers can represent non-linear decision boundaries \hspace*{1ex}in our original input space if we include derived features. For example: \hspace*{1ex}higher order interactions, polynomials or other transformations of x in \hspace*{1ex}the model.

%\textbf{Binary classification: }If only 2 classes ($\Yspace = \setzo$ or  $\Yspace = \setmp$) exist, we can use a single discriminant function $\fx = f_{1}(\xv) - f_{2}(\xv)$.  \\


% \textbf{Generative approach }models $\pdfxyk$, usually by making some assumptions about the structure of these distributions and employs the Bayes theorem: 
% $\pikx = \postk \propto \pdfxyk \pik$. \\ %It allows the computation of \hspace*{1ex}$\pikx$. \\
% \textbf{Examples}: Linear discriminant analysis (LDA), Quadratic discriminant analysis (QDA), Naive Bayes\\
% 
% \textbf{Discriminant approach }tries to optimize the discriminant functions directly, usually via empirical risk minimization:\\ 
% $ \fh = \argmin_{f \in \Hspace} \riske(f) = \argmin_{f \in \Hspace} \sumin \Lxyi.$\\
% \textbf{Examples}: Logistic/softmax regression, kNN


%\end{myblock}

%-------------------------------------------------------------------------------
% HRO - Components of Learning 
%-------------------------------------------------------------------------------          
%\begin{myblock}{Components of Learning}

%\textbf{Learning = Hypothesis space + Risk + Optimization} \\
%\phantom{\textbf{Learning}} \textbf{= }$ \Hspace + \risket + \argmin_{\thetab \in \Theta} 
%\risket$

% 
% \textbf{Learning &= Hypothesis space &+ Risk  &+ Optimization} \\
% &= $\Hspace &+ \risket &+ \argmin_{\thetab \in \Theta} \risket$
% 
% \textbf{Hypothesis space: } Defines (and restricts!) what kind of model $f$
% can be learned from the data.
% 
% Examples: linear functions, decision trees
% 
% \vspace*{0.5ex}
% 
% \textbf{Risk: } Quantifies how well a model performs on a given
% data set. This allows us to rank candidate models in order to choose the best one.
% 
% Examples: squared error, negative (log-)likelihood
% 
% \vspace*{0.5ex}
% 
% \textbf{Optimization: } Defines how to search for the best model, i.e., the model with the smallest {risk}, in the hypothesis space.
% 
% Examples: gradient descent, quadratic programming


%\end{myblock}
% End Third Column
%%%%%%%%%%%%%%%%%%%%%%%%%%%%%%%%%%%%%%%%%%%%%%%%%%%%%%%%%%%%%%%%%%%%%%%%%%%%%%%%
			  }
			\end{minipage}
		\end{beamercolorbox}
	\end{column}
\end{columns}

\end{frame}
\end{document}
