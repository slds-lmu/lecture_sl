\documentclass{beamer}


\usepackage[orientation=landscape,size=a0,scale=1.4,debug]{beamerposter}
\mode<presentation>{\usetheme{mlr}}


\usepackage[utf8]{inputenc} % UTF-8
\usepackage[english]{babel} % Language
\usepackage{hyperref} % Hyperlinks
\usepackage{ragged2e} % Text position
\usepackage[export]{adjustbox} % Image position
\usepackage[most]{tcolorbox}
\usepackage{amsmath}
\usepackage{mathtools}
\usepackage{dsfont}
\usepackage{verbatim}
\usepackage{amsmath}
\usepackage{amsfonts}
\usepackage{csquotes}
\usepackage{multirow}
\usepackage{longtable}
\usepackage{enumerate}
\usepackage[absolute,overlay]{textpos}
\usepackage{psfrag}
\usepackage{algorithm}
\usepackage{algpseudocode}
\usepackage{eqnarray}
\usepackage{arydshln}
\usepackage{tabularx}
\usepackage{placeins}
\usepackage{tikz}
\usepackage{setspace}
\usepackage{colortbl}
\usepackage{mathtools}
\usepackage{wrapfig}
\usepackage{bm}
\usepackage{nicefrac}

\input{../latex-math/basic-math.tex}
\input{../latex-math/basic-ml.tex}


\title{SL :\,: BASICS} % Package title in header, \, adds thin space between ::
\newcommand{\packagedescription}{ \invisible{x} % Package description in header
	% The \textbf{I2ML}: Introduction to Machine Learning course offers an introductory and applied overview of "supervised" Machine Learning. It is organized as a digital lecture.
}

\newlength{\columnheight} % Adjust depending on header height
\setlength{\columnheight}{84cm} 

\newtcolorbox{codebox}{%
	sharp corners,
	leftrule=0pt,
	rightrule=0pt,
	toprule=0pt,
	bottomrule=0pt,
	hbox}

\newtcolorbox{codeboxmultiline}[1][]{%
	sharp corners,
	leftrule=0pt,
	rightrule=0pt,
	toprule=0pt,
	bottomrule=0pt,
	#1}
	

	
\begin{document}
	\begin{frame}[fragile]{}
		\vspace{-8ex}
		\begin{columns}
			\begin{column}{.31\textwidth}
				\begin{beamercolorbox}[center]{postercolumn}
					\begin{minipage}{.98\textwidth}
						\parbox[t][\columnheight]{\textwidth}{
							%%%%%%%%%%%%%%%%%%%%%%%%%%%%%%%%%%%%%%%%%%%%%%%%%%%%%%%%%%%%%%%%%%%%%%%%%%%%%%%%
							% First Column begin
							%-------------------------------------------------------------------------------
							% Data
							%-------------------------------------------------------------------------------
							\begin{myblock}{Advanced Risk Minimization}
							%	
							%	
								Bayes risk:
							%	
								\begin{eqnarray*}
									\riskbayes_{L} = \inf_{f: \Xspace \to \R^g} \risk_L\left(f\right) 
								\end{eqnarray*}
							%
								Bayes regret:
							%
								{\small
								\begin{eqnarray*}
									\risk_L\left(f \right) - \riskbayes_{L} &=& \underbrace{\left[\risk_L\left( f \right) - \inf_{f \in \Hspace} \risk_L(f)\right]}_{\text{estimation error}} + \underbrace{\left[\inf_{f \in \Hspace} \risk_L(f) - \riskbayes_{L}\right]}_{\text{approximation error}}
								\end{eqnarray*}}
							%
							%	Empirical risk of a hypothesis $f \in \Hspace$ for a loss $L$ on a data set $ \Dset$:
							%%	
							%	\begin{eqnarray*}
							% 		\riskef =  \sumin \Lxyi
							%	\end{eqnarray*}
							%
								Empirical risk minimizer:
							%	
								\begin{eqnarray*}
									\hat f = \argmin_{f \in \Hspace} \riskef = \argmin_{f \in \Hspace} \sumin \Lxyi
								\end{eqnarray*}
							%
								Pseudo-residuals:
								%
								\vspace*{-0.3cm}
								\begin{eqnarray*}
									\tilde r &=& - \frac{\partial \Lxy}{\partial \fx}
								\end{eqnarray*}
								%
								Huber Loss:
								%
								$$
									\Lxy = \begin{cases}
												\frac{1}{2}(y - \fx)^2  & \text{ if } |y - \fx| \le \epsilon \\
												\epsilon |y - \fx|-\frac{1}{2}\epsilon^2 \quad & \text{ otherwise }
											\end{cases}
									, \quad \epsilon > 0
								$$
								Log-cosh Loss:
								%
								$$
									\Lxy = \log \left( \cosh(\left|y - \fx\right|) \right)
								% \log \left(\tfrac{1}{2} (\left|y - \fx\right| / c)^2 + 1\right), \quad c \in \R
								$$
								Cauchy loss:
								%
								$$
									\Lxy = \frac{c^2}{2} \log \left( 1 + \left( \frac{\left|y - \fx\right|}{c}\right)^2 \right), 
									\quad c \in \R
								% \log \left(\tfrac{1}{2} (\left|y - \fx\right| / c)^2 + 1\right), \quad c \in \R
								$$
								Log-Barrier Loss:
								%
								\begin{small}
									\[
									\Lxy = \left\{\begin{array}{lr}
										-\epsilon^{2} \cdot \log \Bigl( 1 - \Bigl(\frac{\left|y - \fx\right|}{\epsilon}\Bigr)^2 \Bigr) & \text{if } \left|y-\fx\right| \leq \epsilon \\
										\infty & \text{if } \left|y-\fx\right|  > \epsilon
												\end{array}
											\right.
									\]
								\end{small}
								%
								$\eps$-Insensitive loss :
								%
								$$
									\Lxy =  \begin{cases}
												0  & \text{if } |y - \fx| \le \epsilon \\
												|y - \fx|-\epsilon & \text{otherwise }
											\end{cases},
									\quad \epsilon \in \R_{+}
								$$
								%
								Quantile Loss / Pinball Loss:
								%
								$$
									\Lxy = \begin{cases}
												(1 - \alpha) (\fx - y) & \text{ if } y < \fx\\
												\alpha (y - \fx) & \text{ if } y \ge \fx
											\end{cases},
									\quad \alpha \in (0, 1)
								$$
								%  
								\fbox{
									\parbox{\dimexpr\textwidth-2\fboxsep-2\fboxrule}{
										\begin{table}[] 
											\small
											\renewcommand{\arraystretch}{1.25} %<- modify value to suit your needs
											\begin{tabular}{c|lll}
												Loss Function & Risk Minimizer & Optimal Constant Model\\ \hline
												L2 & $\fxbayes = \E_{y|x} \left[y ~|~ \xv \right]$ & $\fx = \frac{1}{n} \sumin \yi$ \\
												L1 & $\fxbayes = \text{med}_{y|x} \left[y ~|~ \xv \right]$ & $\fx = \text{med}(\yi)$\\
												0-1 & $\hxbayes = \argmax_{l \in \Yspace} \P(y = l~|~ \xv)$  & $\hx = \text{mode} \left\{\yi\right\}$ \\
												Brier & $\pi^*_k(\xv) = \P(y = k~|~\xv)$ & $\pikx =  \frac{1}{n} \sumin \mathds{1}_{\{\yi = k\}}$\\
												Bernoulli (on probs) & $\pi^*_k(\xv) = \P(y = k~|~\xv)$ & $\pikx =  \frac{1}{n} \sumin \mathds{1}_{\{\yi = k\}}$ \\
												Bernoulli (on scores) & $f^*_k(\xv) = \log\left(\frac{\P(y = k ~|~\xv)}{1 - \P(y = k ~|~\xv)}\right)$ & $\fkx = \log \frac{n_{k}}{n - n_{k}}$  
											\end{tabular}
										\end{table}
									}
								}\\

								%
								Bayes risk for  0-1-loss (also: Bayes error rate):
								%
								\begin{eqnarray*}  
									\riskbayes &=& 1 - \E_x \left[\max_{l \in \Yspace} \P(y = l~|~ \xv = \xv)\right]
								\end{eqnarray*}

							\end{myblock}\vfill
						}
						% End First Column
						%%%%%%%%%%%%%%%%%%%%%%%%%%%%%%%%%%%%%%%%%%%%%%%%%%%%%%%%%%%%%%%%%%%%%%%%%%%%%%%%
					\end{minipage}
				\end{beamercolorbox}
			\end{column}
			\begin{column}{.31\textwidth}
				\begin{beamercolorbox}[center]{postercolumn}
					\begin{minipage}{.98\textwidth}
						\parbox[t][\columnheight]{\textwidth}{
							%%%%%%%%%%%%%%%%%%%%%%%%%%%%%%%%%%%%%%%%%%%%%%%%%%%%%%%%%%%%%%%%%%%%%%%%%%%%%%%%
							% Begin Second Column
							\begin{myblock}{} \vspace{-4ex}

								Hinge loss
								$$\Lxy = \max \{ 0, 1 - y\fx \}$$
								%
								Squared Hinge Loss
								$$\Lxy = \max \{ 0, (1 - y\fx)^2 \}$$
								%
								Exponential Loss
								$$\Lxy = \exp(-y\fx)$$
								%
								AUC-loss
								%
								$$AUC = \frac{1}{n_+ n_-}   \sum\nolimits_{i: \yi = 1} \sum\nolimits_{j: \yi[j] = -1} [f^{(i)} > f^{(j)}]$$
								%
							\end{myblock} 
							%
							%
							\begin{myblock}{Multiclass Classification}
								%	
								\begin{small}
									\fbox{	
										\parbox{\dimexpr\textwidth-2\fboxsep-2\fboxrule}{
											\begin{table}[]
										%		\bgroup
										%		\def\arraystretch{2}%  1 is the default, change whatever you need
												\begin{tabular}{c|cc}
													& Logistic Regression & Softmax Regression \\ \hline
													$\Yspace$ & $\{0, 1\}$ & $\{1, 2, ..., g\}$ \\[0.5cm]
													Discriminant fun. & $f(\xv) = \thetav^\top \xv$ & $f_k(\xv) = \thetav_{k}^{\top} \xv, k = 1, \ldots, g$ \\[0.5cm]
													Probabilities & $\pi(\xv) = \frac{1}{1 + \exp\left(-\thetav^\top \xv\right)}$ & $\pi_k(\xv) = \frac{\exp(\thetav_k^\top \xv)}{\sum_{j = 1}^g \exp(\thetav_j^\top \xv) }$ \\[0.5cm]
													$L(y, \pix)$ & Bernoulli / logarithmic loss & Multiclass logarithmic loss\\[-0.3cm]
													& $-y \log \left(\pix\right) - (1 - y) \log \left(1 - \pix\right)$  & $ - \sum_{k = 1}^g [y = k] \log\left(\pi_k(\xv)\right)$ \\
												\end{tabular}
									%			\egroup
											\end{table}
										}
									}
								\end{small}

								%
								\textbf{Codebooks:}
								The k-th column defines how classes of all observations are encoded in the binary subproblem / for binary classifier $f_k(\xv)$.
								Entry $(m, i)$ takes values $\in \{-1, 0, +1\}$
								\begin{itemize}
									\setlength{\itemindent}{+.3in}
									\item if $0$, observations of class $\yi = m$ are ignored.
									\item if $1$, observations of class $\yi = m$ are encoded as $1$.
									\item if $- 1$, observations of class $\yi = m$ are encoded as $- 1$.
								\end{itemize} 
								k-th row is called codeword for class k. \\
								\textbf{One-vs-Rest:}
								Create $g$ binary subproblems, where in each the $k$-th original class is encoded as $+1$, and all other classes (the \textbf{rest}) as $- 1$. Prediction:
								%
								$
									\hat y = \text{arg max}_{k \in \{1, 2, ..., g\}} \hat f_k(\xv). 
								$\\
								%
								\textbf{One-vs-One:}  Create $\nicefrac{g(g - 1)}{2}$ binary subproblems, where each $\D_{k, \tilde k} \subset \D$ only considers observations from a class-pair $\yi \in \{k, \tilde k\}$, other observations are omitted.  Prediction: majority voting. \\
								%
								\textbf{Error-Correcting Codes (ECOC):} Creating codes that can correct for as many errors as possible by row separation (codewords are well-separated in Hamming distance) and column separation (uncorrelated columns).
							\end{myblock}

							%-------------------------------------------------------------------------------
							% Loss and Risk 
							%-------------------------------------------------------------------------------
							\begin{myblock}{Information Theory (Discrete)}
								%	
								\textbf{Entropy} of a discrete random variable $X$ with domain $\Xspace$ and pmf $p(x)$:
								\begin{equation*}
									\begin{aligned} 
										H(X) := H(p) &= - \E[\log_2(p(X))]           &= -\sum\nolimits_{x \in \Xspace} p(x) \log_2 p(x) 
									\end{aligned} 
								\end{equation*}
								%
								\textbf{Joint entropy} of $n$ discrete random variables $X_1, X_2, \ldots, X_n:$   
								%
								\begin{small}  
									$$ H(X_1, X_2, \ldots, X_n) = - \sum\nolimits_{x_1 \in \Xspace_1} \ldots \sum\nolimits_{x_n \in \Xspace_n} p(x_1,x_2, \ldots, x_n) \log_2(p(x_1,x_2, \ldots, x_n)) $$ 
								\end{small}  

							\end{myblock}
% End Second Column					
%%%%%%%%%%%%%%%%%%%%%%%%%%%%%%%%%%%%%%%%%%%%%%%%%%%%%%%%%%%%%%%%%%%%%%%%%%%%%%%%
						}
					\end{minipage}
				\end{beamercolorbox}
			\end{column}

			\begin{column}{.31\textwidth}
				\begin{beamercolorbox}[center]{postercolumn}
					\begin{minipage}{.98\textwidth}
						\parbox[t][\columnheight]{\textwidth}{
							%%%%%%%%%%%%%%%%%%%%%%%%%%%%%%%%%%%%%%%%%%%%%%%%%%%%%%%%%%%%%%%%%%%%%%%%%%%%%%%%
							% Begin Third Column#


							%-------------------------------------------------------------------------------
							% Regression Losses 
							%------------------------------------------------------------------------------- 
							\begin{myblock}{} \vspace{-4ex}
								Properties of discrete entropy:
								\begin{enumerate}
									\setlength{\itemindent}{+.3in}
									\item Entropy is non-negative, so $H(X) >= 0$.
									\item If one event has probability $p(x) = 1$, then $H(X)=0$. 
									\item Symmetry. Reordering values of $p(x)$ does not change entropy.
									\item Adding or removing an event with $p(x)=0$ does not change entropy.
									\item $H(X)$ is continuous in probabilities $p(x)$.
									\item Entropy is additive for independent RVs.
									\item Entropy is maximal for a uniform distribution.
								\end{enumerate}
								%
								\vspace*{1ex}
								%        \includegraphics[width=1\columnwidth]{img/reg_loss.PNG}
								\textbf{Conditional entropy} of $Y$  given $X$ for $(X, Y) \sim p(x, y):$
								% 
								\vspace{-0.2cm}
								%
								\begin{small}  
									\begin{equation*}
										\begin{aligned}
											H(Y | X) &= \E_X[H(Y|X=x)] = \sum_{x \in \Xspace} p(x) H(Y | X=x) \\
											&=-\sum_{x \in \Xspace} p(x) \sum_{y \in \Yspace} p(y | x) \log p(y | x) 
											=-\sum_{x \in \Xspace} \sum_{y \in \Yspace} p(x, y) \log p(y | x)  
										\end{aligned}
									\end{equation*}
								\end{small}  
								%
								Chain rule for entropy:
								%
								$$H(X, Y)=H(X)+H(Y | X)$$
								%
								n-Variable version:
								$$H\left(X_{1}, X_{2}, \ldots, X_{n}\right)=\sumin H\left(X_{i} | X_{i-1}, \ldots, X_{1}\right)$$
								%
								Other properties:
								%
								\begin{itemize}
									\setlength{\itemindent}{+.3in}
										\item $H(X, X)       = H(X)  $
										\item $H(X | X)      = 0 $
										\item $H(X, Y | Z)   =H(X | Z)+H(Y | X, Z)$
										\item $H(X | Y) \leq H(X) $
										\item  If $H(X|Y) = 0$, then $X$ is a function of $Y$
								\end{itemize}
								%
								\textbf{Mutual information} between $X$ and $Y:$
								%
								\begin{equation*}
									\begin{aligned}
										I(X ; Y) &= \E_{p(x, y)} \left[ \log \frac{p(X, Y)}{p(X) p(Y)} \right] = \sum_{x \in \Xspace} \sum_{y \in \Yspace} p(x, y) \log \frac{p(x, y)}{p(x) p(y)} \\
										&= H(X)-H(X | Y)
									\end{aligned}
								\end{equation*}
								%
								Properties:
								\begin{itemize}
									\setlength{\itemindent}{+.3in}
										\item $ I(X ; Y) = H(X) - H(X | Y) $
											\item $I(X ; Y) = H(Y) - H(Y | X) $
											\item $I(X ; Y) = H(X) + H(Y) - H(X, Y)$
											\item $I(X ; Y) = I(Y ; X) $
											\item $I(X ; X) = H(X)$
											\item $ I(X;Y) \geq 0$, with equality if and only if $X$ and $Y$ are independent
								\end{itemize}
								%
								\textbf{Cross-entropy} of two distributions $p$ and $q$ on the same domain $\Xspace:$
								%
								$$ 
									H_q(p) = \sum_{x \in \Xspace} p(x) \log\left(\frac{1}{q(x)}\right) = - \sum_{x \in \Xspace} p(x) \log\left(q(x)\right) 
								$$
								%
								\textbf{Kullback-Leibler Divergence:}
								%
								$$ 
									D_{KL}(p \| q) = \E_p \left[\log \frac{p(X)}{q(X)}\right] = \sum_{x \in \Xspace} p(x) \cdot \log \frac{p(x)}{q(x)} 
								$$
								%
								Relationship:
								$$
									H_q(p) = H(p) + D_{KL}(p \| q)
								$$
							\end{myblock}



			  			}
					\end{minipage}
				\end{beamercolorbox}
			\end{column}
		\end{columns}

	\end{frame}
\end{document}



%-------------------------------------------------------------------------------
% Classification 
%------------------------------------------------------------------------------- 

%\begin{myblock}{Classification}
% 				    We want to assign new observations to known categories according to criteria learned from a training set.  
%             \vspace*{1ex}
%             

%$y \in \Yspace = \gset : $ categorical output variable (label)\\ 

%\textbf{Classification} usually means to construct $g$ \textbf{discriminant functions}:
  
%$f_1(\xv), \ldots, \fgx$, so that we choose our class as \\ $h(\xv) = \argmax_{k \in \gset} \fkx$ \\

%\textbf{Linear Classifier:} functions $\fkx$ can be specified as linear functions\\

% \hspace*{1ex}\textbf{Note: }All linear classifiers can represent non-linear decision boundaries \hspace*{1ex}in our original input space if we include derived features. For example: \hspace*{1ex}higher order interactions, polynomials or other transformations of x in \hspace*{1ex}the model.

%\textbf{Binary classification: }If only 2 classes ($\Yspace = \setzo$ or  $\Yspace = \setmp$) exist, we can use a single discriminant function $\fx = f_{1}(\xv) - f_{2}(\xv)$.  \\


% \textbf{Generative approach }models $\pdfxyk$, usually by making some assumptions about the structure of these distributions and employs the Bayes theorem: 
% $\pikx = \postk \propto \pdfxyk \pik$. \\ %It allows the computation of \hspace*{1ex}$\pikx$. \\
% \textbf{Examples}: Linear discriminant analysis (LDA), Quadratic discriminant analysis (QDA), Naive Bayes\\
% 
% \textbf{Discriminant approach }tries to optimize the discriminant functions directly, usually via empirical risk minimization:\\ 
% $ \fh = \argmin_{f \in \Hspace} \riske(f) = \argmin_{f \in \Hspace} \sumin \Lxyi.$\\
% \textbf{Examples}: Logistic/softmax regression, kNN


%\end{myblock}

%-------------------------------------------------------------------------------
% HRO - Components of Learning 
%-------------------------------------------------------------------------------          
%\begin{myblock}{Components of Learning}

%\textbf{Learning = Hypothesis space + Risk + Optimization} \\
%\phantom{\textbf{Learning}} \textbf{= }$ \Hspace + \risket + \argmin_{\thetav \in \Theta} 
%\risket$

% 
% \textbf{Learning &= Hypothesis space &+ Risk  &+ Optimization} \\
% &= $\Hspace &+ \risket &+ \argmin_{\thetav \in \Theta} \risket$
% 
% \textbf{Hypothesis space: } Defines (and restricts!) what kind of model $f$
% can be learned from the data.
% 
% Examples: linear functions, decision trees
% 
% \vspace*{0.5ex}
% 
% \textbf{Risk: } Quantifies how well a model performs on a given
% data set. This allows us to rank candidate models in order to choose the best one.
% 
% Examples: squared error, negative (log-)likelihood
% 
% \vspace*{0.5ex}
% 
% \textbf{Optimization: } Defines how to search for the best model, i.e., the model with the smallest {risk}, in the hypothesis space.
% 
% Examples: gradient descent, quadratic programming


%\end{myblock}
% End Third Column
%%%%%%%%%%%%%%%%%%%%%%%%%%%%%%%%%%%%%%%%%%%%%%%%%%%%%%%%%%%%%%%%%%%%%%%%%%%%%%%%