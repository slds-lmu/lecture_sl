\documentclass{beamer}
\newcommand \beameritemnestingprefix{}


\usepackage[orientation=landscape,size=a0,scale=1.4,debug]{beamerposter}
\mode<presentation>{\usetheme{mlr}}


\usepackage[utf8]{inputenc} % UTF-8
\usepackage[english]{babel} % Language
\usepackage{hyperref} % Hyperlinks
\usepackage{ragged2e} % Text position
\usepackage[export]{adjustbox} % Image position
\usepackage[most]{tcolorbox}
\usepackage{amsmath}
\usepackage{mathtools}
\usepackage{dsfont}
\usepackage{verbatim}
\usepackage{amsmath}
\usepackage{amsfonts}
\usepackage{csquotes}
\usepackage{multirow}
\usepackage{longtable}
\usepackage[absolute,overlay]{textpos}
\usepackage{psfrag}
\usepackage{algorithm}
\usepackage{algpseudocode}
\usepackage{eqnarray}
\usepackage{arydshln}
\usepackage{tabularx}
\usepackage{placeins}
\usepackage{tikz}
\usepackage{setspace}
\usepackage{colortbl}
\usepackage{mathtools}
\usepackage{wrapfig}
\usepackage{bm}


\input{../latex-math/basic-math.tex}
% machine learning
\newcommand{\Xspace}{\mathcal{X}} % X, input space
\newcommand{\Yspace}{\mathcal{Y}} % Y, output space
\newcommand{\Zspace}{\mathcal{Z}} % Z, space of sampled datapoints
\newcommand{\nset}{\{1, \ldots, n\}} % set from 1 to n
\newcommand{\pset}{\{1, \ldots, p\}} % set from 1 to p
\newcommand{\gset}{\{1, \ldots, g\}} % set from 1 to g
\newcommand{\Pxy}{\mathbb{P}_{xy}} % P_xy
\newcommand{\Exy}{\mathbb{E}_{xy}} % E_xy: Expectation over random variables xy
\newcommand{\xv}{\mathbf{x}} % vector x (bold)
\newcommand{\xtil}{\tilde{\mathbf{x}}} % vector x-tilde (bold)
\newcommand{\yv}{\mathbf{y}} % vector y (bold)
\newcommand{\xy}{(\xv, y)} % observation (x, y)
\newcommand{\xvec}{\left(x_1, \ldots, x_p\right)^\top} % (x1, ..., xp)
\newcommand{\Xmat}{\mathbf{X}} % Design matrix
\newcommand{\allDatasets}{\mathds{D}} % The set of all datasets
\newcommand{\allDatasetsn}{\mathds{D}_n}  % The set of all datasets of size n
\newcommand{\D}{\mathcal{D}} % D, data
\newcommand{\Dn}{\D_n} % D_n, data of size n
\newcommand{\Dtrain}{\mathcal{D}_{\text{train}}} % D_train, training set
\newcommand{\Dtest}{\mathcal{D}_{\text{test}}} % D_test, test set
\newcommand{\xyi}[1][i]{\left(\xv^{(#1)}, y^{(#1)}\right)} % (x^i, y^i), i-th observation
\newcommand{\Dset}{\left( \xyi[1], \ldots, \xyi[n]\right)} % {(x1,y1)), ..., (xn,yn)}, data
\newcommand{\defAllDatasetsn}{(\Xspace \times \Yspace)^n} % Def. of the set of all datasets of size n
\newcommand{\defAllDatasets}{\bigcup_{n \in \N}(\Xspace \times \Yspace)^n} % Def. of the set of all datasets
\newcommand{\xdat}{\left\{ \xv^{(1)}, \ldots, \xv^{(n)}\right\}} % {x1, ..., xn}, input data
\newcommand{\ydat}{\left\{ \yv^{(1)}, \ldots, \yv^{(n)}\right\}} % {y1, ..., yn}, input data
\newcommand{\yvec}{\left(y^{(1)}, \hdots, y^{(n)}\right)^\top} % (y1, ..., yn), vector of outcomes
\newcommand{\greekxi}{\xi} % Greek letter xi
\renewcommand{\xi}[1][i]{\xv^{(#1)}} % x^i, i-th observed value of x
\newcommand{\yi}[1][i]{y^{(#1)}} % y^i, i-th observed value of y
\newcommand{\xivec}{\left(x^{(i)}_1, \ldots, x^{(i)}_p\right)^\top} % (x1^i, ..., xp^i), i-th observation vector
\newcommand{\xj}{\xv_j} % x_j, j-th feature
\newcommand{\xjvec}{\left(x^{(1)}_j, \ldots, x^{(n)}_j\right)^\top} % (x^1_j, ..., x^n_j), j-th feature vector
\newcommand{\phiv}{\mathbf{\phi}} % Basis transformation function phi
\newcommand{\phixi}{\mathbf{\phi}^{(i)}} % Basis transformation of xi: phi^i := phi(xi)

%%%%%% ml - models general
\newcommand{\lamv}{\bm{\lambda}} % lambda vector, hyperconfiguration vector
\newcommand{\Lam}{\Lambda}	 % Lambda, space of all hpos
% Inducer / Inducing algorithm
\newcommand{\preimageInducer}{\left(\defAllDatasets\right)\times\Lam} % Set of all datasets times the hyperparameter space
\newcommand{\preimageInducerShort}{\allDatasets\times\Lam} % Set of all datasets times the hyperparameter space
% Inducer / Inducing algorithm
\newcommand{\ind}{\mathcal{I}} % Inducer, inducing algorithm, learning algorithm

% continuous prediction function f
\newcommand{\ftrue}{f_{\text{true}}}  % True underlying function (if a statistical model is assumed)
\newcommand{\ftruex}{\ftrue(\xv)} % True underlying function (if a statistical model is assumed)
\newcommand{\fx}{f(\xv)} % f(x), continuous prediction function
\newcommand{\fdomains}{f: \Xspace \rightarrow \R^g} % f with domain and co-domain
\newcommand{\Hspace}{\mathcal{H}} % hypothesis space where f is from
\newcommand{\Hall}{\mathcal{H}_{\text{all}}} % unrestricted hypothesis space
\newcommand{\fbayes}{f^{\ast}} % Bayes-optimal model
\newcommand{\fxbayes}{f^{\ast}(\xv)} % Bayes-optimal model
\newcommand{\fkx}[1][k]{f_{#1}(\xv)} % f_j(x), discriminant component function
\newcommand{\fhspace}{\hat f_{\Hspace}} % fhat_H
\newcommand{\fh}{\hat{f}} % f hat, estimated prediction function
\newcommand{\fxh}{\fh(\xv)} % fhat(x)
\newcommand{\fxt}{f(\xv ~|~ \thetav)} % f(x | theta)
\newcommand{\fxi}{f\left(\xv^{(i)}\right)} % f(x^(i))
\newcommand{\fxih}{\hat{f}\left(\xv^{(i)}\right)} % f(x^(i))
\newcommand{\fxit}{f\big(\xv^{(i)} ~|~ \thetav\big)} % f(x^(i) | theta)
\newcommand{\fhD}{\fh_{\D}} % fhat_D, estimate of f based on D
\newcommand{\fhDtrain}{\fh_{\Dtrain}} % fhat_Dtrain, estimate of f based on D
\newcommand{\fhDnlam}{\fh_{\Dn, \lamv}} %model learned on Dn with hp lambda
\newcommand{\fhDlam}{\fh_{\D, \lamv}} %model learned on D with hp lambda
\newcommand{\fhDnlams}{\fh_{\Dn, \lamv^\ast}} %model learned on Dn with optimal hp lambda
\newcommand{\fhDlams}{\fh_{\D, \lamv^\ast}} %model learned on D with optimal hp lambda

% discrete prediction function h
\newcommand{\hx}{h(\xv)} % h(x), discrete prediction function
\newcommand{\hh}{\hat{h}} % h hat
\newcommand{\hxh}{\hat{h}(\xv)} % hhat(x)
\newcommand{\hxt}{h(\xv | \thetav)} % h(x | theta)
\newcommand{\hxi}{h\left(\xi\right)} % h(x^(i))
\newcommand{\hxit}{h\left(\xi ~|~ \thetav\right)} % h(x^(i) | theta)
\newcommand{\hbayes}{h^{\ast}} % Bayes-optimal classification model
\newcommand{\hxbayes}{h^{\ast}(\xv)} % Bayes-optimal classification model

% yhat
\newcommand{\yh}{\hat{y}} % yhat for prediction of target
\newcommand{\yih}{\hat{y}^{(i)}} % yhat^(i) for prediction of ith targiet
\newcommand{\resi}{\yi- \yih}

% theta
\newcommand{\thetah}{\hat{\theta}} % theta hat
\newcommand{\thetav}{\bm{\theta}} % theta vector
\newcommand{\thetavh}{\bm{\hat\theta}} % theta vector hat
\newcommand{\thetat}[1][t]{\thetav^{[#1]}} % theta^[t] in optimization
\newcommand{\thetatn}[1][t]{\thetav^{[#1 +1]}} % theta^[t+1] in optimization
\newcommand{\thetahDnlam}{\thetavh_{\Dn, \lamv}} %theta learned on Dn with hp lambda
\newcommand{\thetahDlam}{\thetavh_{\D, \lamv}} %theta learned on D with hp lambda
\newcommand{\mint}{\min_{\thetav \in \Theta}} % min problem theta
\newcommand{\argmint}{\argmin_{\thetav \in \Theta}} % argmin theta

% densities + probabilities
% pdf of x
\newcommand{\pdf}{p} % p
\newcommand{\pdfx}{p(\xv)} % p(x)
\newcommand{\pixt}{\pi(\xv~|~ \thetav)} % pi(x|theta), pdf of x given theta
\newcommand{\pixit}[1][i]{\pi\left(\xi[#1] ~|~ \thetav\right)} % pi(x^i|theta), pdf of x given theta
\newcommand{\pixii}[1][i]{\pi\left(\xi[#1]\right)} % pi(x^i), pdf of i-th x

% pdf of (x, y)
\newcommand{\pdfxy}{p(\xv,y)} % p(x, y)
\newcommand{\pdfxyt}{p(\xv, y ~|~ \thetav)} % p(x, y | theta)
\newcommand{\pdfxyit}{p\left(\xi, \yi ~|~ \thetav\right)} % p(x^(i), y^(i) | theta)

% pdf of x given y
\newcommand{\pdfxyk}[1][k]{p(\xv | y= #1)} % p(x | y = k)
\newcommand{\lpdfxyk}[1][k]{\log p(\xv | y= #1)} % log p(x | y = k)
\newcommand{\pdfxiyk}[1][k]{p\left(\xi | y= #1 \right)} % p(x^i | y = k)

% prior probabilities
\newcommand{\pik}[1][k]{\pi_{#1}} % pi_k, prior
\newcommand{\pih}{\hat{\pi}} % pi hat, estimated prior (binary classification)
\newcommand{\pikh}[1][k]{\hat{\pi}_{#1}} % pi_k hat, estimated prior
\newcommand{\lpik}[1][k]{\log \pi_{#1}} % log pi_k, log of the prior
\newcommand{\pit}{\pi(\thetav)} % Prior probability of parameter theta

% posterior probabilities
\newcommand{\post}{\P(y = 1 ~|~ \xv)} % P(y = 1 | x), post. prob for y=1
\newcommand{\postk}[1][k]{\P(y = #1 ~|~ \xv)} % P(y = k | y), post. prob for y=k
\newcommand{\pidomains}{\pi: \Xspace \rightarrow \unitint} % pi with domain and co-domain
\newcommand{\pibayes}{\pi^{\ast}} % Bayes-optimal classification model
\newcommand{\pixbayes}{\pi^{\ast}(\xv)} % Bayes-optimal classification model
\newcommand{\piastxtil}{\pi^{\ast}(\xtil)} % Bayes-optimal classification model
\newcommand{\piastkxtil}{\pi^{\ast}_k(\xtil)} % Bayes-optimal classification model for k-th class
\newcommand{\pix}{\pi(\xv)} % pi(x), P(y = 1 | x)
\newcommand{\piv}{\bm{\pi}} % pi, bold, as vector
\newcommand{\pikx}[1][k]{\pi_{#1}(\xv)} % pi_k(x), P(y = k | x)
\newcommand{\pikxt}[1][k]{\pi_{#1}(\xv ~|~ \thetav)} % pi_k(x | theta), P(y = k | x, theta)
\newcommand{\pixh}{\hat \pi(\xv)} % pi(x) hat, P(y = 1 | x) hat
\newcommand{\pikxh}[1][k]{\hat \pi_{#1}(\xv)} % pi_k(x) hat, P(y = k | x) hat
\newcommand{\pixih}{\hat \pi(\xi)} % pi(x^(i)) with hat
\newcommand{\pikxih}[1][k]{\hat \pi_{#1}(\xi)} % pi_k(x^(i)) with hat
\newcommand{\pdfygxt}{p(y ~|~\xv, \thetav)} % p(y | x, theta)
\newcommand{\pdfyigxit}{p\left(\yi ~|~\xi, \thetav\right)} % p(y^i |x^i, theta)
\newcommand{\lpdfygxt}{\log \pdfygxt } % log p(y | x, theta)
\newcommand{\lpdfyigxit}{\log \pdfyigxit} % log p(y^i |x^i, theta)

% probabilistic
\newcommand{\bayesrulek}[1][k]{\frac{\P(\xv | y= #1) \P(y= #1)}{\P(\xv)}} % Bayes rule
\newcommand{\muv}{\bm{\mu}} % expectation vector of Gaussian
\newcommand{\muk}[1][k]{\bm{\mu_{#1}}} % mean vector of class-k Gaussian (discr analysis)
\newcommand{\mukh}[1][k]{\bm{\hat{\mu}_{#1}}} % estimated mean vector of class-k Gaussian (discr analysis)

% residual and margin
\newcommand{\rx}{r(\xv)} % residual 
\newcommand{\eps}{\epsilon} % residual, stochastic
\newcommand{\epsv}{\bm{\epsilon}} % residual, stochastic, as vector
\newcommand{\epsi}{\epsilon^{(i)}} % epsilon^i, residual, stochastic
\newcommand{\epsh}{\hat{\epsilon}} % residual, estimated
\newcommand{\epsvh}{\hat{\epsv}} % residual, estimated, vector
\newcommand{\yf}{y \fx} % y f(x), margin
\newcommand{\yfi}{\yi \fxi} % y^i f(x^i), margin
\newcommand{\Sigmah}{\hat \Sigma} % estimated covariance matrix
\newcommand{\Sigmahj}{\hat \Sigma_j} % estimated covariance matrix for the j-th class
\newcommand{\nux}{\nu(\xv)} % nu(x) = y * f(x)

% ml - loss, risk, likelihood
\newcommand{\Lyf}{L\left(y, f\right)} % L(y, f), loss function
% \newcommand{\Lypi}{L\left(y, \pi\right)} % L(y, pi), loss function
\newcommand{\Lxy}{L\left(y, \fx\right)} % L(y, f(x)), loss function
\newcommand{\Lxyi}{L\left(\yi, \fxi\right)} % loss of observation
\newcommand{\Lxyt}{L\left(y, \fxt\right)} % loss with f parameterized
\newcommand{\Lxyit}{L\left(\yi, \fxit\right)} % loss of observation with f parameterized
\newcommand{\Lxym}{L\left(\yi, f\left(\bm{\tilde{x}}^{(i)} ~|~ \thetav\right)\right)} % loss of observation with f parameterized
\newcommand{\Lpixy}{L\left(y, \pix\right)} % loss in classification
% \newcommand{\Lpiy}{L\left(y, \pi\right)} % loss in classification
\newcommand{\Lpiv}{L\left(y, \piv\right)} % loss in classification
\newcommand{\Lpixyi}{L\left(\yi, \pixii\right)} % loss of observation in classification
\newcommand{\Lpixyt}{L\left(y, \pixt\right)} % loss with pi parameterized
\newcommand{\Lpixyit}{L\left(\yi, \pixit\right)} % loss of observation with pi parameterized
% \newcommand{\Lhy}{L\left(y, h\right)} % L(y, h), loss function on discrete classes
\newcommand{\Lhxy}{L\left(y, \hx\right)} % L(y, h(x)), loss function on discrete classes
\newcommand{\Lr}{L\left(r\right)} % L(r), loss defined on residual (reg) / margin (classif)
\newcommand{\lone}{|y - \fx|} % L1 loss
\newcommand{\ltwo}{\left(y - \fx\right)^2} % L2 loss
\newcommand{\lbernoullimp}{\ln(1 + \exp(-y \cdot \fx))} % Bernoulli loss for -1, +1 encoding
\newcommand{\lbernoullizo}{- y \cdot \fx + \log(1 + \exp(\fx))} % Bernoulli loss for 0, 1 encoding
\newcommand{\lcrossent}{- y \log \left(\pix\right) - (1 - y) \log \left(1 - \pix\right)} % cross-entropy loss
\newcommand{\lbrier}{\left(\pix - y \right)^2} % Brier score
\newcommand{\risk}{\mathcal{R}} % R, risk
\newcommand{\riskbayes}{\mathcal{R}^\ast}
\newcommand{\riskf}{\risk(f)} % R(f), risk
\newcommand{\riskdef}{\E_{y|\xv}\left(\Lxy \right)} % risk def (expected loss)
\newcommand{\riskt}{\mathcal{R}(\thetav)} % R(theta), risk
\newcommand{\riske}{\mathcal{R}_{\text{emp}}} % R_emp, empirical risk w/o factor 1 / n
\newcommand{\riskeb}{\bar{\mathcal{R}}_{\text{emp}}} % R_emp, empirical risk w/ factor 1 / n
\newcommand{\riskef}{\riske(f)} % R_emp(f)
\newcommand{\risket}{\mathcal{R}_{\text{emp}}(\thetav)} % R_emp(theta)
\newcommand{\riskr}{\mathcal{R}_{\text{reg}}} % R_reg, regularized risk
\newcommand{\riskrt}{\mathcal{R}_{\text{reg}}(\thetav)} % R_reg(theta)
\newcommand{\riskrf}{\riskr(f)} % R_reg(f)
\newcommand{\riskrth}{\hat{\mathcal{R}}_{\text{reg}}(\thetav)} % hat R_reg(theta)
\newcommand{\risketh}{\hat{\mathcal{R}}_{\text{emp}}(\thetav)} % hat R_emp(theta)
\newcommand{\LL}{\mathcal{L}} % L, likelihood
\newcommand{\LLt}{\mathcal{L}(\thetav)} % L(theta), likelihood
\newcommand{\LLtx}{\mathcal{L}(\thetav | \xv)} % L(theta|x), likelihood
\newcommand{\logl}{\ell} % l, log-likelihood
\newcommand{\loglt}{\logl(\thetav)} % l(theta), log-likelihood
\newcommand{\logltx}{\logl(\thetav | \xv)} % l(theta|x), log-likelihood
\newcommand{\errtrain}{\text{err}_{\text{train}}} % training error
\newcommand{\errtest}{\text{err}_{\text{test}}} % test error
\newcommand{\errexp}{\overline{\text{err}_{\text{test}}}} % avg training error

% lm
\newcommand{\thx}{\thetav^\top \xv} % linear model
\newcommand{\olsest}{(\Xmat^\top \Xmat)^{-1} \Xmat^\top \yv} % OLS estimator in LM

\input{../latex-math/ml-trees.tex}
\input{../latex-math/ml-nn.tex}
% ml - Gaussian Process

\newcommand{\fvec}{[f(\xi[1]), \dots, f(\xi[n])]} % function vector
\newcommand{\fv}{\mathbf{f}} % function vector
\newcommand{\mv}{\mathbf{m}} % GP mean vector
\newcommand{\kv}{\mathbf{k}} % cov matrix partition
\newcommand{\kcc}{k(\cdot, \cdot)} % cov of arbitrary inputs
\newcommand{\kxij}[2]{k(\xi, \xi[j])} % cov of x_i, x_j
\newcommand{\Kmat}{\mathbf{K}} % GP cov matrix
\newcommand{\nmk}{\normal(\mv, \Kmat)} % Gaussian w/ mean vec, cov mat
\newcommand{\nzk}{\normal(\zero, \Kmat)} % zero-mean Gaussian
\newcommand{\gpmk}{\mathcal{GP}(m(\cdot), \kcc)} % GP definition
\newcommand{\gpzk}{\mathcal{GP}(\zero, \kcc)} % zero-mean GP
\newcommand{\Xsubset}{\bm{X}} % finite subset from xspace
\newcommand{\fX}{f(\Xsubset)} % Gaussian vector of finite subset
\newcommand{\kXX}{k(\Xsubset, \Xsubset)} % kernel fun for finite subset
\newcommand{\mX}{m(\Xsubset)} % mean fun for finite subset
\newcommand{\ls}{\ell} % length-scale
\newcommand{\xxtnorm}{\| \xv - \xtil\|} % norm of x minus x tilde
\newcommand{\sqexpkernel}{\exp \left(- \frac{\| \xv - \xv^{\prime} \|^2}{2 \ls^2} \right)} % squared exponential kernel

% GP prediction
\newcommand{\xstar}{\xv_\ast} % test obs features
\newcommand{\ystar}{\yv_\ast} % test obs target
\newcommand{\fstar}{\fv_\ast} % test obs fun vector
\newcommand{\Xstar}{\Xmat_\ast} % test design matrix
\newcommand{\fstarvec}{\left[f\left(\xi[1]_{\ast}\right), \dots, f\left(\xi[m]_{\ast}\right) \right]} % pred function vector
\newcommand{\kstar}{\kv_{\ast}} % cov of new obs with x
\newcommand{\kstarstar}{\kv_{\ast \ast}} % cov of new obs
\newcommand{\Kstar}{\Kmat_{\ast}} % cov mat of new obs with x
\newcommand{\Kstarstar}{\Kmat_{\ast \ast}} % cov mat of new obs
\newcommand{\Kmatinv}{\Kmat^{-1}} % inverse cov mat
\newcommand{\Ky}{\Kmat_y} % cov mat of y




\title{Supervised Learning :\,: CHEAT SHEET} % Package title in header, \, adds thin space between ::
\newcommand{\packagedescription}{ % Package description in header
%	The \textbf{I2ML}: Introduction to Machine Learning course offers an introductory and applied overview of "supervised" Machine Learning. It is organized as a digital lecture.
}

\newlength{\columnheight} % Adjust depending on header height
\setlength{\columnheight}{84cm} 

\newtcolorbox{codebox}{%
	sharp corners,
	leftrule=0pt,
	rightrule=0pt,
	toprule=0pt,
	bottomrule=0pt,
	hbox}

\newtcolorbox{codeboxmultiline}[1][]{%
	sharp corners,
	leftrule=0pt,
	rightrule=0pt,
	toprule=0pt,
	bottomrule=0pt,
	#1}

\begin{document}
\begin{frame}[fragile]{}
\begin{columns}
	\begin{column}{.31\textwidth}
		\begin{beamercolorbox}[center]{postercolumn}
			\begin{minipage}{.98\textwidth}
				\parbox[t][\columnheight]{\textwidth}{

					\begin{myblock}{Bayesian Linear Model}
							
								Bayesian Linear Model:
							$$\yi = \fxi + \epsi = \thetav^T \xi + \epsi, \quad \text{for } i \in \{1, \ldots, n\}$$
								
								where $\epsi \sim \mathcal{N}(0, \sigma^2).$
							
								Parameter vector $\thetav$ is stochastic and follows a distribution.\\
							
								
								Gaussian variant:
								\begin{itemize}[$\bullet$]
									\setlength{\itemindent}{+.3in}
									\item Prior distribution: $\thetav \sim \mathcal{N}(\zero, \tau^2 \id_p)$ 
									\item Posterior distribution:	$
									\thetav ~|~ \Xmat, \yv \sim \mathcal{N}(\sigma^{-2}\bm{K}^{-1}\Xmat^\top\yv, \bm{K}^{-1})
									$ with $\bm{K}:= \sigma^{-2}\Xmat^\top\Xmat + \frac{1}{\tau^2} \id_p$
									\item Predictive distribution of $y_* = 	\thetav^\top \xv_*$ for a new observations $\xv_*$: 
									$$
									y_* ~|~ \Xmat, \yv, \xv_* \sim \mathcal{N}(\sigma^{-2}\yv^\top \Xmat \Amat^{-1}\xv_*, \xv_*^\top\Amat^{-1}\xv_*)
									$$
								\end{itemize}
\end{myblock}

                \begin{myblock}{Gaussian Processes}
                  
                  
								\fbox{
									\parbox{\dimexpr\textwidth-2\fboxsep-2\fboxrule}{
										\begin{table}
											\begin{tabular}{cc}
												\textbf{Weight-Space View} & \textbf{Function-Space View} \vspace{4mm}\\ 
												Parameterize functions & Work on functions directly\\
												\footnotesize Example: $\fxt = \thetav^\top \xv$ & \vspace{3mm}\\
												Define distributions on $\thetav$ & Define distributions on $f$ \vspace{4mm}\\
												Inference in parameter space $\Theta$ & Inference in function space $\Hspace$
											\end{tabular}
										\end{table}  
									}
								}\\
							

								\textbf{Gaussian Processes:} A function $\fx$ is generated by a GP $\gp$ if for \textbf{any finite} set of inputs $\left\{\xv^{(1)}, \dots, \xv^{(n)}\right\}$, the associated vector of function values $\bm{f} = \left(f(\xv^{(1)}), \dots, f(\xv^{(n)})\right)$ has a Gaussian distribution
								%
								$$
								\bm{f} = \left[f\left(\xi[1]\right),\dots, f\left(\xi[n]\right)\right] \sim \mathcal{N}\left(\bm{m}, \bm{K}\right),
								$$
								%
								with 
								%
								\begin{eqnarray*}
									\textbf{m} &:=& \left(m\left(\xi\right)\right)_{i}, \quad
									\textbf{K} := \left(k\left(\xi, \xv^{(j)}\right)\right)_{i,j}, 
								\end{eqnarray*}
								%
								where $m(\xv)$ is the mean function and $k(\xv, \xv^\prime)$ is the covariance function. \\
							
                Types of \textbf{covariance functions}:
								\begin{itemize}[$\bullet$]
									\setlength{\itemindent}{+.3in}
									%			
									\item $k(.,.)$ is stationary if it is as a function of $\bm{d} = \bm{x} - \bm{x}^\prime$, $ \leadsto k(\bm{d})$
									\item $k(.,.)$ is isotropic if it is a function of $r = \|\bm{x} - \bm{x}^\prime\|$,  $ \leadsto k(r)$
									\item $k(., .)$ is a dot product covariance function if $k$ is a function of $\bm{x}^T \bm{x}^\prime$
								\end{itemize}

							\end{myblock}
				}
			\end{minipage}
		\end{beamercolorbox}
	\end{column}
	
%%%%%%%%%%%%%%%%%%%%%%%%%%%%%%%%%%%%%%%%%%%%%%%%%%%%%%%%%%%%%%%%%%%%%

\begin{column}{.31\textwidth}
\begin{beamercolorbox}[center]{postercolumn}
\begin{minipage}{.98\textwidth}
\parbox[t][\columnheight]{\textwidth}{

\begin{myblock}{}  
							
								Commonly used covariance functions:
								
								\begin{center}
										\fbox{  
										\begin{tabular}{|c|c|}
											\hline
											Name & $k(\bm{x}, \bm{x}^\prime)$\\
											\hline
											constant & $\sigma_0^2$ \\ [1em]
											linear & $\sigma_0^2 + \bm{x}^T\bm{x}^\prime$ \\ [1em]
											polynomial & $(\sigma_0^2 + \bm{x}^T\bm{x}^\prime)^p$ \\ [1em]
											squared exponential & $\exp(- \frac{\|\bm{x} - \bm{x}^\prime\|^2}{2\ls^2})$ \\ [1em]
											Matérn & \begin{footnotesize} $\frac{1}{2^\nu \Gamma(\nu)}\biggl(\frac{\sqrt{2 \nu}}{\ls}\|\bm{x} - \bm{x}^\prime\|\biggr)^{\nu} K_\nu\biggl(\frac{\sqrt{2 \nu}}{\ls}\|\bm{x} - \bm{x}^\prime\|\biggr)$\end{footnotesize}  \\ [1em]
											exponential & $\exp\left(- \frac{\|\bm{x} - \bm{x}^\prime\|}{\ls}\right)$ \\ [1em]
											\hline
									\end{tabular} }\\
								\end{center}
									

\end{myblock}
							\begin{myblock}{Gaussian Processes Prediction}

                \begin{codebox}
								\textbf{Posterior Process}
                \end{codebox}
								
								Assuming a zero-mean GP prior $\mathcal{GP}\left(\bm{0}, k(\xv, \xv^\prime)\right).$ 
							%	
								For $ f_* = f\left(\xv_*\right)$ on single unobserved test point $\xv_*$ 
							%	
								\begin{eqnarray*}
									f_* ~|~ \xv_*, \Xmat, \bm{f} \sim \mathcal{N}(\bm{k}_{*}^{T}\Kmat^{-1}\bm{f}, \bm{k}_{**} - \bm{k}_*^T \Kmat ^{-1}\bm{k}_*),
								\end{eqnarray*}
							%
								where, $\Kmat = \left(k\left(\xi, \xv^{(j)}\right)\right)_{i,j}$, $\bm{k}_* = \left[k\left(\xv_*, \xi[1]\right), ..., k\left(\xv_*, \xi[n]\right)\right]$ and $ \bm{k}_{**}\ = k(\xv_*, \xv_*)$. \\
								
								%
								For multiple unobserved test points
								$
								\bm{f}_* = \left[f\left(\xi[1]_*\right), ..., f\left(\xi[m]_*\right)\right]:
								$
								\begin{eqnarray*}
									\bm{f}_* ~|~ \Xmat_*, \Xmat, \bm{f} \sim \mathcal{N}(\Kmat_{*}^{T}\Kmat^{-1}\bm{f}, \Kmat_{**} - \Kmat_*^T \Kmat ^{-1}\Kmat_*).
								\end{eqnarray*} 
								with $\Kmat_* = \left(k\left(\xi, \xv_*^{(j)}\right)\right)_{i,j}$, $\Kmat_{**} = \left(k\left(\xi[i]_*, \xi[j]_*\right)\right)_{i,j}$.\\
								
								Predictive mean when assuming a non-zero mean GP prior $\gp$ with mean $m(\xv):$ 
								$$
								m(\Xmat_*) + \Kmat_*\Kmat^{-1}\left(\bm{y} - m(\Xmat)\right)
								$$
								Predictive variance remains unchanged. \\
								
							%
							%
							\end{myblock}

}
\end{minipage}
\end{beamercolorbox}
\end{column}

%%%%%%%%%%%%%%%%%%%%%%%%%%%%%%%%%%%%%%%%%%%%%%%%%%%%%%%%%%%%%%%%%%%%%%%%%%%%%%%%%%%%%%%%%%%%%%%%%%%%

\begin{column}{.31\textwidth}
\begin{beamercolorbox}[center]{postercolumn}
\begin{minipage}{.98\textwidth}
\parbox[t][\columnheight]{\textwidth}{

 \begin{myblock}{}
\begin{codebox}
								\textbf{Noisy Posterior Process}
                \end{codebox}
							%	
								Assuming a zero-mean GP prior $\mathcal{GP}\left(\bm{0}, k(\xv, \xv^\prime)\right):$ 
							%	
								\begin{eqnarray*}
									\bm{f}_* ~|~ \Xmat_*, \Xmat, \bm{y} \sim \mathcal{N}(\bm{m}_{\text{post}}, \bm{K}_\text{post}).
								\end{eqnarray*}
								with nugget $\sigma^2 $ and
							%	 
								\begin{eqnarray*}
									\bm{m}_{\text{post}} &=& \Kmat_{*}^{T} \left(\Kmat+ \sigma^2 \cdot \id\right)^{-1}\bm{y} \\
									\bm{K}_\text{post} &=& \Kmat_{**} - \Kmat_*^T \left(\Kmat  + \sigma^2 \cdot \id\right)^{-1}	\Kmat_*,	
								\end{eqnarray*} 
							%	
								
								Predictive mean when assuming a non-zero mean GP prior $\gp$ with mean $m(\xv):$ 
								$$
								m(\Xmat_*) + \Kmat_*(\Kmat +\sigma^2 \id)^{-1}\left(\bm{y} - m(\Xmat)\right)
								$$
								Predictive variance remains unchanged.
							%
							%
							\end{myblock}

  \begin{myblock}{Train a Gaussian Processes}
  
 We can learn the numerical hyperparameters of a selected covariance function directly during GP training.

 Let us assume 
$$
	y = \fx + \eps, ~ \eps \sim \mathcal{N}\left(0, \sigma^2\right),
$$
where $\fx \sim \mathcal{GP}\left(\bm{0}, k\left(\xv, \xv^\prime | \thetav \right)\right)$. 



Observing $\bm{y} \sim \mathcal{N}\left(\bm{0}, \bm{K} + \sigma^2 \id\right)$, the marginal log-likelihood (or evidence) is
\begin{eqnarray*}
\log p(\bm{y} ~|~ \bm{X}, \thetav) &=& \log \left[\left(2 \pi\right)^{-n / 2} |\bm{K}_y|^{-1 / 2} \exp\left(- \frac{1}{2} \bm{y}^\top \bm{K}_y^{-1} \bm{y}\right) \right]\\
&=& -\frac{1}{2}\bm{y}^T\bm{K}_y^{-1} \bm{y} - \frac{1}{2} \log \left| \bm{K}_y \right| - \frac{n}{2} \log 2\pi. 
\end{eqnarray*}
with $\bm{K}_y:=\bm{K} + \sigma^2 \id$ and $\thetav$ denoting the hyperparameters (the parameters of the covariance function). 
\\


The three terms of the marginal likelihood have interpretable roles, considering that 
the model becomes less flexible as the length-scale increases:
\begin{itemize}[$\bullet$]
\setlength{\itemindent}{+.3in}
\item the data fit $-\frac{1}{2}\bm{y}^T\bm{K}_y^{-1} \bm{y}$, which tends to decrease if the length scale increases
\item the complexity penalty $- \frac{1}{2} \log \left| \bm{K}_y \right|$, which depends on the covariance function only and which increases with the length-scale, because the model gets less complex with growing length-scale
\item a normalization constant $- \frac{n}{2} \log 2\pi$
\end{itemize}
\end{myblock}
  }
  
  \end{minipage}
  \end{beamercolorbox}
  \end{column}
  
  
  
\end{columns}
\end{frame}
\end{document}