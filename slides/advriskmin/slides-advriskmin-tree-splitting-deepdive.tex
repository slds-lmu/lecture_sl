\documentclass[11pt,compress,t,notes=noshow, xcolor=table]{beamer}


\input{../../style/preamble}
\input{../../latex-math/basic-math}
\input{../../latex-math/basic-ml}
\input{../../latex-math/ml-trees} 

\title{Introduction to Machine Learning}

\begin{document}
    
\titlemeta{
Advanced Risk Minimization
}{ 
Loss functions and tree splitting (Deep-Dive)
}{
figure_man/cart_tree_i2ml.png
}{
\item Tree splitting loss vs impurity:
\item Bernoulli loss $\sim$ entropy splitting
\item Brier score $\sim$ gini splitting
}

\begin{framei}[sep=M]{Risk minimization and impurity}

\item Tree fitting: Find best way to split parent node $\Np_0$ into child nodes $\Np_1$ and $\Np_2$ such that $\Np_1 \cup \Np_2 = \Np_0$ and $\Np_1 \cap \Np_2 = \emptyset$
\item Two options for evaluating how good a split is: Per node $\Np$ compute the following:
\begin{enumerate}
\item Compute impurity $\text{Imp}(\Np)$ directly from observations in $\Np$
\item Fit optimal constant using loss function, sum up losses for $\Np$
\end{enumerate}
\item  Two common impurity measures are entropy and Gini index where $\pikN$ are predicted probs for class $k=1,\ldots,g$ in node $\Np$:

\begin{align*}
    \text{Imp}^{\text{ent}}(\Np) &= -\textstyle\sum_{k = 1}^g \pikN \log \pikN\\
    \text{Imp}^{\text{Gini}}(\Np) &= \sum_{k=1}^g \pikN (1-\pikN )
\end{align*}
%$$\text{Imp}^{\text{ent}}(\Np) = -\textstyle\sum_{k = 1}^g \pikN \log \pikN\,,\,\, \text{Imp}^{\text{Gini}}(\Np) = \sum_{k=1}^g \pikN (1-\pikN )$$


\end{framei}


\begin{framei}[sep=M]{Risk minimization and impurity}
\item In the following we will prove that entropy and Gini impurity measures are equivalent to splitting using log loss and Brier score:
\begin{align*}
    \Lpixy &= -\sum_{k = 1}^g \I [y=k] \log (\pi_k(\xv)) \quad \text{(log-loss)}\\
    \Lpixy &= \sum_{k = 1}^g (\I [y=k] - \pi_k(\xv))^2 \quad \text{(Brier)}
\end{align*}


\end{framei}

\begin{frame}{Bernoulli loss min = Entropy splitting}
\textbf{Claim:} Entropy as impurity
$$\text{Imp}^{\text{ent}}(\Np) = -\textstyle\sum_{k = 1}^g \pikN \log \pikN$$
%splitting $\text{Imp}(\Np) = -\textstyle\sum_{k = 1}^g \pikN \log \pikN$ 
is equivalent to mean emp. risk with (multiclass) Bernoulli loss
$$\riskeb(\Np) = %\frac{1}{n_\Np} \sum_{\xy \in \Np} \Lpixy = 
\frac{1}{n_\Np} \sum_{\xy \in \Np} -\sum_{k = 1}^g \I [y=k] \log (\pi_k(\xv))$$  %($\pikN := \frac{1}{n_{\Np}} \textstyle\sum_{(\xv,y) \in \Np} [y = k]$)


\textbf{Proof:} %To prove this we show that the risk related to a subset of observations 
Let $\Np \subseteq \D$ denote the subset of observations in that node and consider $\riskeb(\Np)$ of node $\Np$ with (multiclass) Bernoulli loss  

\vfill

$\Rightarrow$ Optimal constant per node $\pikN = \frac{1}{n_{\Np}} \sum\limits_{(\xv,y) \in \Np} \I [y=k] = \frac{n_{\Np,k}}{n_{\Np }}$\\

\vfill

where $n_{\Np,k}$ is the number of class $k$ observations in node $\Np$ \\

%Entropy of node $\Np$:
%$$
%\text{Imp}(\Np) = -\textstyle\sum_{k = 1}^g \pikN \log \pikN %, \quad \pikN := \frac{1}{n_{\Np}} \textstyle\sum_{(\xv,y) \in \Np} \I [y=k]
%$$

\end{frame}


\begin{framev}[fs=small]{Risk minimization and impurity}

\begin{align*}
\riskeb(\Np) =& \frac{1}{n_{\Np}} \sum_{\xy \in \Np} (- \sum_{k = 1}^g \I [y=k] \log \pi_k(\xv) ) \\
=& \frac{1}{n_{\Np}} -\sum_{k = 1}^g \sum_{\xy \in \Np} \I [y=k]\log \pikN \\
=& \frac{1}{n_{\Np}} -\sum_{k = 1}^g n_{\Np,k} \log \pikN =\frac{1}{n_{\Np}} -\sum_{k = 1}^g (n_\Np \cdot \pikN) \log \pikN  %\underbrace{\sum_{\xy \in \Np} \I [y=k]}_{n_{\Np}\cdot \pikN }
\\
=& - \frac{1}{n_{\Np}} n_{\Np} \sum_{k = 1}^g \pikN \log \pikN = \text{Imp}^{\text{ent}}(\Np)
\end{align*} 

Avg. Bernoulli-risk of node $\Np$ is equal to $\text{Imp}^{\text{ent}}(\Np)$

\end{framev}


\begin{frame}{Brier score minimization = Gini splitting}

\textbf{Claim:} Using Gini as impurity
$$\text{Imp}^{\text{Gini}}(\Np) = \sum_{k=1}^g \pikN (1-\pikN )$$
%splitting $\text{Imp}(\Np) = -\textstyle\sum_{k = 1}^g \pikN \log \pikN$ 
is equivalent to avg. emp. risk using Brier score 
$$\riskeb(\Np)= \frac{1}{n_\Np} \sum_{\xy \in \Np} \Lpixy = \frac{1}{n_\Np} \sum_{\xy \in \Np} \sum_{k = 1}^g (\I [y=k] - \pi_k(\xv))^2$$


%in (2) %($\pikN := \frac{1}{n_{\Np}} \textstyle\sum_{(\xv,y) \in \Np} \I [y=k]$)

\vfill

\textbf{Proof:} Avg. empirical risk $\riskeb(\Np)$ of node $\Np$ using (multiclass) Brier score has optimal constant per node: 
$$\pikN = \frac{1}{n_{\Np}} \sum\limits_{(\xv,y) \in \Np} \I [y=k] = \frac{n_{\Np,k}}{n_{\Np }}$$ \\

\vfill

%Gini index of node $\Np$:
%$$
%\text{Imp}(\Np) = \sum_{k=1}^g \pikN (1-\pikN ) %, \quad \pikN := \frac{1}{n_{\Np}} \textstyle\sum_{(\xv,y) \in \Np} \I [y=k]
%$$


\end{frame}
\begin{framev}[fs=small]{Brier score minimization = Gini splitting}
Inserting the optimal constant, the risk simplifies to
\begin{align*}
\riskeb(\Np) =& \frac{1}{n_\Np} \sum_{\xy \in \Np}  \sum_{k = 1}^g (\I [y=k] - \pikN)^2 
= \frac{1}{n_\Np} \sum_{k = 1}^g \sum_{\xy \in \Np} (\I [y=k] - \frac{n_{\Np,k}}{n_{\Np }})^2\\
% \end{eqnarray*}
% by plugging in the optimal constant prediction w.r.t. the Brier score ($n_{\Np,k}$ is defined as the number of class $k$ observations in node $\Np$): 
% $$\hat \pi_k(\xv)= \pikN = \frac{1}{n_{\Np}} \sum\limits_{(\xv,y) \in \Np} \I [y=k] = \frac{n_{\Np,k}}{n_{\Np }}. $$ 
%  We split the inner sum and further simplify the expression
% \begin{eqnarray*}
=& \frac{1}{n_\Np} \sum_{k = 1}^{g} (\sum_{\xy \in \Np: ~ y = k} (1 - \frac{n_{\Np,k}}{n_{\Np }})^2 + \sum_{\xy \in \Np: ~ y \ne k} (0 - \frac{n_{\Np,k}}{n_{\Np }})^2) \\
=& \frac{1}{n_\Np} \sum_{k = 1}^g n_{\Np,k}(1 - \frac{n_{\Np,k}}{n_{\Np }})^2 + (n_{\Np } - n_{\Np,k})(\frac{n_{\Np,k}}{n_{\Np }})^2
\end{align*}

\vfill

since for $n_{\Np,k}$ observations the condition $y = k$ is met, and for the remaining $(n_\Np - n_{\Np,k})$ observations it is not. 

\end{framev}

\begin{framev}[fs=small]{Brier score minimization = Gini splitting}

We further simplify the expression to

\begin{align*}
\riskeb(\Np) =&  \frac{1}{n_\Np} \sum_{k = 1}^g n_{\Np,k}(\frac{n_{\Np } - n_{\Np,k}}{n_{\Np }})^2 + (n_{\Np } - n_{\Np,k})(\frac{n_{\Np,k}}{n_{\Np }})^2 \\
=& \frac{1}{n_\Np} \sum_{k = 1}^g \frac{n_{\Np,k}}{n_{\Np }} \frac{n_{\Np } - n_{\Np,k}}{n_{\Np }} (n_{\Np } - n_{\Np,k } + n_{\Np,k}) \\
=& \frac{n_{\Np }}{n_\Np} \sum_{k = 1}^g \pikN \cdot (1 - \pikN ) = \text{Imp}^{\text{Gini}}(\Np)
\end{align*} 

Avg. Brier-risk $\riskeb(\Np)$ of the node is equal to its gini-impurity $\text{Imp}^{\text{Gini}}(\Np)$

\end{framev}

\begin{framei}[sep=M]{Weighting for Risk and Impurity}

\item The emp. risk of a \emph{split} is given by sum of per-node risks ($n_0 = n_1 + n_2$ are number of obs in nodes): 
\begin{align*}
\riske(\text{split}) = \riske(\Np_1) + \riske(\Np_2) & = n_1 \riskeb(\Np_1) + n_2 \riskeb(\Np_2) \\
& =  n_1 \text{Imp}(\Np_1) + n_2 \text{Imp}(\Np_2) \\
& = n_0 (\frac{n_1}{n_0} \text{Imp}(\Np_1) + \frac{n_2}{n_0} \text{Imp}(\Np_2))
\end{align*}

\item As you can see above: if working with average risk, we need to reweight in the addition (as the averages are computed on subsets of potentially unequal sizes) 

\item Average risks are used in impurity formulas,we so simply have to adhere to that slight modification in split finding with them

\item The impurity of a split is defined as a weighted average
$$\text{Imp(split)} = \frac{n_1}{n_0} \text{Imp}(\Np_1) + \frac{n_2}{n_0} \text{Imp}(\Np_2)$$

%\item Emp. risk of the split $\riske(\text{split})$ is then proportional to its impurity $\text{Imp(split)}$, i.e., $\argmin_\text{split} \riske(\text{split}) = \argmin_\text{split} \text{Imp(split)}$ 

%\item \textbf{Intuition}: we need to re-weight the per-node impurities as they are already (per-sample) average risks

%\item $\Rightarrow \riske(\text{split}) = \riske(\Np_1) + \riske(\Np_2)  = n_1 \text{Imp}(\Np_1) + n_2 \text{Imp}(\Np_2)
%=n_0 (\frac{n_1}{n_0} \text{Imp}(\Np_1) + \frac{n_2}{n_0} \text{Imp}(\Np_2)) = n_0\text{Imp(split)}$

%\item $\Rightarrow \riske(\text{split}) = \riske(\Np_1) + \riske(\Np_2)  = n_1 \text{Imp}(\Np_1) + n_2 \text{Imp}(\Np_2)
%=n_0 (\frac{n_1}{n_0} \text{Imp}(\Np_1) + \frac{n_2}{n_0} \text{Imp}(\Np_2)) = n_0\text{Imp(split)}$

%\item Brier risk  $\riske(\text{split})$ is proportional to its Gini-impurity $\text{Imp(split)}$, i.e., $\argmin_\text{split} \riske(\text{split}) = \argmin_\text{split} \text{Imp(split)}$

\end{framei}


\endlecture

\end{document}