\documentclass[11pt,compress,t,notes=noshow, xcolor=table]{beamer}
%<<setup-child, include = FALSE>>=
%library(knitr)
%library(qrmix)
%library(mlr)
%library(quantreg)
%library(reshape2)
%set_parent("../style/preamble.Rnw")
%@

\input{../../style/preamble}
\input{../../latex-math/basic-math}
\input{../../latex-math/basic-ml}

\title{Introduction to Machine Learning}

\begin{document}
    
\titlemeta{% Chunk title (example: CART, Forests, Boosting, ...), can be empty
    Advanced Risk Minimization
  }{% Lecture title  
    Maximum Likelihood Estimation vs.
    Empirical Risk Minimization
  }{% Relative path to title page image: Can be empty but must not start with slides/
  figure/residuals_plot_L2_title.png
  }{
  \item Connection between maximum likelihood and risk minimization
  \item Correspondence between Gaussian errors and L2 loss, Laplace errors and L1 loss, and Bernoulli targets and Bernoulli/log loss
  %\item Correspondence between Laplace errors and L1 loss
%\item Correspondence between Bernoulli targets and the Bernoulli / log loss 
}

\begin{vbframe}{Maximum Likelihood}

Let's consider regression from a maximum likelihood perspective. Assume: 

$$
	y~|~ \xv \sim p(y~|~\xv, \thetav)
$$

\vspace{0.5cm}

Common case: 
true underlying relationship $\ftrue$ with additive noise: 

\vspace{0.5cm}

\begin{minipage}{0.5\textwidth}
$$
y = \ftrue(\xv) + \eps
$$
\end{minipage}%
\begin{minipage}{0.5\textwidth}
\includegraphics[width = 0.75\textwidth]{figure/ftrue.pdf}
\end{minipage}

where $\ftrue$ has params $\thetav$ and $\eps$ a RV that follows some distribution $\P_\eps$, with $\E[\eps] = 0$. Also, assume $\epsilon \perp \!\!\! \perp \xv$.


\framebreak 

From a statistics / maximum-likelihood perspective, we assume (or we pretend) we know the underlying distribution family $p(y~|~\xv, \thetav)$.

\begin{itemize}
\item Given i.i.d data $
\D = \Dset
$
 from $\Pxy$ 

the maximum-likelihood principle is to maximize the \textbf{likelihood}

\begin{minipage}{0.5\textwidth}
$$ \LLt = \prod_{i=1}^n \pdfyigxit $$
\end{minipage}%
\begin{minipage}{0.5\textwidth}
\includegraphics[width = 0.75\textwidth]{figure/log_reg_ml.pdf}
\end{minipage}

or equivalently minimize the \textbf{negative log-likelihood (NLL)}

\begin{minipage}{0.5\textwidth}
$$ -\loglt = -\sumin \lpdfyigxit. $$
\end{minipage}%
\begin{minipage}{0.5\textwidth}
\includegraphics[width = 0.75\textwidth]{figure/log_reg_erm.pdf}
\end{minipage}
\end{itemize}


\framebreak 

From an ML perspective we assume our hypothesis space corresponds to the space of the (parameterized) $\ftrue$. 

\begin{itemize}
\item Simply define neg. log-likelihood as \textbf{loss function} 
$$ \Lxyt := - \lpdfygxt $$
\item Then, maximum-likelihood 
 = ERM
$$\risket = \sumin \Lxyit$$

% \item Then the maximum-likelihood estimator $\thetah$, which we obtain by optimizing $\LLt$ is identical
% to the loss-minimal $\thetah$ we obtain by minimizing $\risket$.
\item NB: When we are only interested in the minimizer, we can ignore multiplicative or additive constants.
\item We use $\propto$ as \enquote{proportional up to multiplicative and additive constants}

\end{itemize}

\end{vbframe}


\begin{vbframe}{Gaussian Errors - L2-Loss} 

Assume $y = \ftrue(\xv) + \eps$ with additive Gaussian errors, i.e. $\epsi \sim \mathcal{N}(0, \sigma^2)$. Then $y~|~\xv \sim N\left(\ftrue(\xv), \sigma^2\right)$. The likelihood is then 

{\small
\begin{eqnarray*}
\LL(\thetav) &=& \prodin \pdf\left(\yi ~\bigg|~ \fxit, \sigma^2\right) \\ &\propto& \prodin \exp\left(-\frac{1}{2\sigma^2} \left(\yi - \fxit\right)^2\right)
\end{eqnarray*}
}

%\framebreak 

Easy to see: minimizing Gaussian NLL s is ERM with $L2$-loss:
{\small
\begin{eqnarray*}
- \loglt &=& - \log\left(\LL(\thetav)\right) \\
&\propto& - \log\big(\prodin \exp\big(-\frac{1}{2\sigma^2} \big(\yi - \fxit\big)^2\big)\big) \\
&\propto& \sumin \left(\yi - \fxit\right)^2
\end{eqnarray*}
}

\framebreak 

\begin{footnotesize}
\begin{itemize}
	\item We simulate data $y ~|~\xv \sim \mathcal{N}\left(\ftrue(\xv), 1\right)$ with $\ftrue = 0.2 \cdot \xv$ 
\item Let's plot empirical errors as histogram, after fitting our model with $L2$-loss
\item Q-Q-plot compares empirical residuals vs. theoretical quantiles of Gaussian 
\end{itemize}
\end{footnotesize}

\begin{center}
\includegraphics[width = 0.8\textwidth]{figure/residuals_plot_L2.pdf}
\end{center}
\end{vbframe}

\begin{vbframe}{Laplace Errors - L1-Loss}

Let's consider Laplacian errors $\eps$ now, with density: 

\begin{minipage}{0.5\textwidth}
$$
 \frac{1}{2\sigma} \exp\left(-\frac{|\eps|}{\sigma}\right)\,, \sigma > 0.
$$
\end{minipage}%
\begin{minipage}{0.5\textwidth}
\includegraphics[width = 0.75\textwidth]{figure/laplace_plot.pdf}
\end{minipage}
Then
$$
y = \ftrue(\xv) + \eps 
$$

also follows Laplace distribution with mean $\fxit$ and scale  $\sigma$. 

\framebreak 

The likelihood is then 

\begin{eqnarray*}
\LL(\thetav) &=& \prod_{i=1}^n \pdf\left(\yi ~\bigg|~ \fxit, \sigma\right) \\ &\propto& \exp\left(-\frac{1}{\sigma}\sumin \left|\yi - \fxit\right|\right)\,.
\end{eqnarray*}
The negative log-likelihood is
$$
- \loglt \propto \sumin \left|\yi - \fxit\right|.
$$

MLE for Laplacian errors = ERM with L1-loss. 

\begin{itemize}
\item Some losses correspond to more complex or less known error densities, like the Huber loss \citelink{MEYER2021ALTERNATIVE}

\item Huber density is (unsurprisingly) a hybrid of Gaussian and Laplace

\end{itemize}

\framebreak 

\begin{footnotesize}
\begin{itemize}
	\item We simulate data $y ~|~\xv \sim \text{Laplacian}\left(\ftrue(\xv), 1\right)$ with $\ftrue = 0.2 \cdot \xv$. 
\item We can plot the empirical error distribution, i.e. the distribution of the residuals after fitting a regression model w.r.t. $L1$-loss.
\item With the help of a Q-Q-plot we can compare the empirical residuals vs. the theoretical quantiles of a Laplacian distribution.  
\end{itemize}
\end{footnotesize}
\includegraphics{figure/residuals_plot_L1.pdf}



\end{vbframe}



\begin{vbframe}{Maximum Likelihood in Classification}

Let us assume the outputs $y$ to be Bernoulli-distributed, i.e. $y ~|~ \xv \sim \text{Bern}(\pi_\text{true}(\xv))$. 
The negative log likelihood is
\begin{eqnarray*}
- \loglt &=& -\sumin \lpdfyigxit \\ 
&=& - \sumin \log \big[\pi(\xi)^{y^{(i)}} \cdot (1 - \pi(\xi))^{(1 - y^{(i)})} \big]\\
&=& \sumin -\yi \log[\pi(\xi)] - (1-\yi) \log [1 - \pi(\xi)]. 
\end{eqnarray*}


%\framebreak 

This gives rise to the following loss function 

$$
  L(y, \pix) = -y\log(\pix)-(1-y)\log(1-\pix), \quad y \in \{0, 1\}
$$

which we introduced as \textbf{Bernoulli} loss. 

%\begin{center}
%\includegraphics[width = 0.6\textwidth]{figure/bernoulli_prob.png} \\
%\end{center}



% \framebreak 

% The Bernoulli loss is used in \textbf{logistic regression} in combination with the hypothesis space of linear functions

% \vspace*{-0.3cm}

% \begin{eqnarray*}
%   \Hspace = \left\{f: \Xspace \to \R ~|~\fx = \thetav^\top \xv\right\}
% \end{eqnarray*}

% Scores are afterwards transformed into probabilities by the logistic function $\pix = \left(1 + \exp(- \fx)\right)^{-1}$  

% \begin{center}
%   \includegraphics[width = 0.4\textwidth]{figure_man/logreg-2vars-data.png}~~\includegraphics[width = 0.4\textwidth]{figure_man/logreg-2vars-score-vs-prob.png}
% \end{center}

% \framebreak

% \begin{center}
%   \includegraphics[width=0.8\textwidth]{figure_man/logreg-2vars-surface.png}
% \end{center}


\end{vbframe}




% \begin{vbframe}{Classification Losses: Bernoulli Loss}
% \begin{itemize}
%   \item $L(y, \pix) = -y\ln(\pix)-(1-y)\ln(1-\pix) $
%   % \item Also called \textbf{Bernoulli Loss}
%   \item Convex, differentiable (gradient methods can be used), not robust
%   \item Also called logarithmic loss or cross-entropy loss (which will be motivated later)
% \end{itemize}


% \framebreak 

% \lz 

% The constant model $\pix =< \theta$ that is optimal w.r.t. the empirical risk is the fraction of class $1$ observations

% $$
% \hat \pi (\xv) = \frac{1}{n}\sumin \yi.
% $$

% \textbf{Proof:} Exercise.


% \end{vbframe}



% \begin{vbframe}{Classification Losses: Cross-Entropy Loss}

% \begin{itemize}
%   \item The cross entropy loss 
%   $$
%   \Lxy = -y\ln(\pix)-(1-y)\ln(1-\pix)$$
%   % is equivalent to logistic loss when 
%   \item  The cross entropy loss is closely related to the Kullback-Leibler divergence, which will be introduced later in the chapter.
%   \item Very often used in neural networks with binary output nodes for classification.
% \end{itemize}

% \end{vbframe}

% \section{Outlook}

% \begin{vbframe}{Outlook}

% When introducing different learning algorithms, we will come back to the loss functions introduced in this chapter or even introduce new ones. For example:  

% \begin{itemize}
%   \item Ordinary Linear Regression: L2-loss
%   \item Logistic Regression: Logistic loss
%   \item Support Vector Machine Classification: Hinge-Loss (to be introduced) (see \textbf{SVM} chapter)
%   \item Support Vector Machine Regression: $\epsilon$-insensitive loss (see \textbf{SVM} chapter)
%   \item AdaBoost: Exponential loss (see \textbf{Boosting} chapter)
% \end{itemize}

% Once knowing the theory of risk minimization and properties of loss functions, we can combine model classes and loss functions as needed or even tailor loss functions to our needs. 

% \end{vbframe}

% \framebreak

% we get the risk function

% \begin{eqnarray*}
% \risk(f) &=& \mathbb{E}_x [\max\{0, 1 - \fx\} \pix + \max\{0, 1 + y\fx\} (1-\pix)].
% \end{eqnarray*}

% The minimizer of $\risk(f)$ for the hinge loss function is

% \begin{eqnarray*}
  % $fh(x) =  \footnotesize \begin{cases} 1 \quad \text{ if } \pix > 1/2 \\ -1 \quad \pix < 1/2  \end{cases}$
% \end{eqnarray*}



% \section{Selected methods for regression and classification}

% \begin{vbframe}{Normal linear and additive regression}

% For $i \in \nset$ (simple case and with basis functions):
% \begin{eqnarray*}
% \yi & = & \fxi + \epsi = \theta_0 + \theta^T \xi + \epsi\\
% \yi & = & \fxi + \epsi = \theta_0 + \theta^T \phi(\xi) + \epsi
% \end{eqnarray*}

% \begin{itemize}
% \item basis functions $\phi(x)=(\phi_1(x), \ldots, \phi_m(x))^T$
% \item assumption: $\epsi \iid N(0, \sigma^2)$
% \end{itemize}

% \lz

% Given observed data $\D$  we want to address the questions
% \begin{itemize}
% \item  Given basis functions, how to find $\theta$? ({\bf Parameter estimation})
% \item  How to select basis functions for my problem? ({\bf Model selection} )
% \end{itemize}

% \framebreak

% We'll address the model selection problem later, for now leave it at:
% Doing this \enquote{manually} is rather frowned upon...

% \lz

% A typical ML way to estimate the parameters is to not require the assumption
% $\epsi \iid N(0, \sigma^2)$, but instead assume the that prediction error is measured
% by \emph{squared error} as our \emph{loss function} in \emph{risk minimization}:

% $$
% \riske(\thetav) = SSE(\thetav) = \sumin \Lxyit = \sumin \left(\yi - \theta^T \xi\right)^2
% $$

% NB: We assume here and from now on that $\theta_0$ is included in $\theta$.

% Using matrix notation the empirical risk can be written as
% $$
% SSE(\thetav) = (\ydat - \Xmat\thetav)^T(\ydat - \Xmat\thetav).
% $$


% Differentiating w.r.t $\theta$ yields the so-called \emph{normal equations}:
% $$
% \Xmat^T(\ydat - \Xmat\thetav) = 0
% $$
% The optimal $\theta$ is
% $$
% \thetah = (\Xmat^T \Xmat)^{-1} \Xmat^T\ydat
% $$

% In statistics, we would start from a maximum-likelihood perspective
% $$
% \yi = \fxi + \epsi \sim N\left(\fxi, \sigma^2\right)
% $$
% $$
% \LLt = \prod_{i=1}^n \pdf\left(\yi | \fxit, \sigma^2\right) \propto \exp\left(-\frac{\sumin (\yi - \fxit)^2}{2\sigma^2}\right)
% $$
% It's easy to see that minimizing the neg. log-likelihood is equivalent to the
% loss minimization approach since
% $$
% \llt \propto - \sumin \left(\yi - \fxit\right)^2.
% $$


% \framebreak

% <<lm-mtcars-plot>>=
% data(mtcars)
% regr.task = makeRegrTask(data = mtcars, target = "mpg")
% plotLearnerPrediction("regr.lm", regr.task, features = "disp")
% @
% \end{vbframe}

% \begin{vbframe}{Example: Linear Regr. with L1 vs L2 loss}

% <<l1-vs-l2-loss-prep>>=
% set.seed(123)

% # prediction with f, based on vec x and param vec beta
% f = function(x, beta) {
%   crossprod(x, beta)
% }

% # L1 and L2 loss, based on design mat X, vec, param vec beta, computed with f
% loss1 = function(X, y, beta) {
%   yhat = apply(X, 1, f, beta = beta)
%   sum((y - yhat)^2)
% }
% loss2 = function(X, y, beta) {
%   yhat = apply(X, 1, f, beta = beta)
%   sum(abs(y - yhat))
% }

% # optimize loss (1 or 2) with optim
% # yes, neldermead not really the best, who cares it is 1d
% optLoss = function(X, y, loss) {
%   start = rep(0, ncol(X))
%   res = optim(start, loss, method = "Nelder-Mead", X = X, y = y)
%   res$par
% }

% # plot data and a couple of linear models
% plotIt = function(X, y, models = list()) {
%   gd = data.frame(x = X[, 2],  y = y, outlier = c(TRUE, rep(FALSE, length(y) - 1)))
%   pl = ggplot(data = gd, aes(x = x, y = y, shape = outlier))
%   pl = pl + geom_point(alpha = .8) + guides(shape = FALSE)
%   for (i in seq_along(models)) {
%     m = models[[i]]
%     pl = pl + geom_abline(intercept = m$beta[1], slope = m$beta[2], 
%       col = m$col, lty = m$lty, alpha = .8)
%   }
%   return(pl)
% }


% # generate some data, sample from line with gaussian errors
% # make the leftmost obs an outlier
% n = 10
% x = sort(runif(n = n, min = 0, max = 10))
% y = 3 * x + 1 + rnorm(n, sd = 5)
% X = cbind(x0 = 1, x1 = x)
% y[1] = 50
% @

% L1 loss is less sensitive to outliers than L2 loss:
% <<l1-vs-l2-loss-plot, fig.height=4.5>>=
% # fit l1/2 models on data without then with outlier data
% b1 = optLoss(X[-1,], y[-1], loss = loss1)
% b2 = optLoss(X[-1,], y[-1], loss = loss2)
% b3 = optLoss(X, y, loss = loss1)
% b4 = optLoss(X, y, loss = loss2)

% # plot all 4 models
% pl = plotIt(X, y, models = list(
%   list(beta = b1, col = pal_2[1], lty = "solid"),
%   list(beta = b2, col = pal_2[2], lty = "solid"),
%   list(beta = b3, col = pal_2[1], lty = "dashed"),
%   list(beta = b4, col = pal_2[2], lty = "dashed")
% ))
% print(pl)
% @
% Violet = L2, green = L1 loss.\\
% Solid = fit without, dashed = fit with outlier ($\blacktriangle$ at (\Sexpr{c(x[1], y[1])})).
% \end{vbframe}

\begin{vbframe}{Distributions and losses}
For \textbf{every} error distribution $\P_\eps$ we can derive an equivalent loss function, which leads to the same point estimator for the parameter vector $\thetav$ as maximum-likelihood. Formally, $\thetah \in \argmax_{\thetav} \LL(\thetav) \Leftrightarrow \thetah \in \argmin_{\thetav}-\log(\LL(\thetav))$.  %\implies \thetah \in \argmin_{\thetav}-\log(\LL(\thetav))$
    %\item $\thetah \in \argmax_{\thetav} \log\left(\LL(\thetav)\right) \implies $
    
\vspace{0.2cm}    
\textbf{But}: Other way does not always work: We cannot derive a pdf/error distrib. for every loss -- the Hinge loss is one prominent example (some prob. interpretation is still possible \citelink{SOLLICH1999NINTH}).
%\framebreak

\vspace{0.2cm}
\textbf{When} does the reverse direction hold?

If we can write loss as $L(y,\fx)=L_{\mathbb{P}}(y-\fx)=L_{\mathbb{P}}(r)$ for $r \in \mathbb{R}$, then minimizing $L_{\mathbb{P}}(y-\fx)$ is equiv. to maximizing a conditional log-likelihood $\log(p(y-f(\xv|\thetav))$ if
\begin{enumerate}
 \setlength{\itemsep}{1.0em}
    \item $\log(p(r))$ is affine trafo of $L_{\mathbb{P}}$ (undoing the $\propto$):
% minus due to sign chance when we go from loss to loglik
    $\log(p(r)) = a - bL_{\mathbb{P}}(r),\,\,a \in \mathbb{R}, b>0$
    \item  $p$ is a pdf (non-negative and integrates to one)
\end{enumerate}

%Thus, a loss $L_{\mathbb{P}}$ corresponds to MLE under \textit{some} distribution if there exist $a \in \mathbb{R},\, b>0$ such that
%$$ \int_{\mathbb{R}} \exp(a-bL_{\mathbb{P}}(r))\text{d}r = 1$$
\end{vbframe}


\endlecture
\end{document}
