\documentclass[11pt,compress,t,notes=noshow, xcolor=table]{beamer}
%<<setup-child, include = FALSE>>=
%library(knitr)
%library(qrmix)
%library(mlr)
%library(quantreg)
%library(reshape2)
%set_parent("../style/preamble.Rnw")
%@

\input{../../style/preamble}
\input{../../latex-math/basic-math}
\input{../../latex-math/basic-ml}
\input{../../latex-math/ml-eval}
\input{../../latex-math/ml-trees} % For the comparison of Brier and Gini index

\title{Introduction to Machine Learning}

\begin{document}
    
\titlemeta{
    Advanced Risk Minimization
  }{% Lecture title  
    Bernoulli Loss
  }{% Relative path to title page image: Can be empty but must not start with slides/
  figure/bernoulli_prob.png
  }{
  \item Bernoulli (log, logistic, binomial) loss
  \item Risk minimizer
  \item Optimal constant
  \item Complete separation problem
}

\begin{vbframe}{On probs}

Likelihood of Bernoulli RV:
$$
\LLt = \prodin \pixit^{\yi} \left(1-\pixit\right)^{1-\yi} \,\,, \, y \in \setzo
$$

Transform into NLL:
$$- \loglt = \sumin - \yi \log\left(\pixit\right) - \left(1-\yi\right)\log\left(1-\pixit\right)$$

Bernoulli loss: loss on single sample
  $$
    \Lpixy = - y \log \left(\pix\right) - (1 - y) \log \left(1 - \pix\right)\,\,, \, y \in \setzo 
  $$

\end{vbframe}

\begin{vbframe}{On decision scores}

\begin{itemize}
    \item Transform probs into scores (log-odds): $\fx = \log\left(\frac{\pix}{1 - \pix}\right)$
    \item Then $\pix = \left(1 + \exp(-\fx)\right)^{-1}$
    \item Yields equivalent loss formulation
\end{itemize}
$$\Lxy = - y \cdot \fx + \log(1 + \exp(\fx)) \quad \text{for } y \in \setzo$$

\begin{center}
% \includegraphics[width = 10cm ]{figure_man/bernoulli-loss.png} \\
\begin{figure}
  \subfloat{\includegraphics[width=0.45\textwidth]{figure/bernoulli_prob.png}}
  \subfloat{\includegraphics[width=0.45\textwidth]{figure/logistic.png}}\\
\end{figure}
\end{center}

\end{vbframe}

\begin{vbframe}{Loss in terms of margin}

\begin{itemize}
\item For $y \in \setmp$, loss becomes: 
$$
  \Lxy = \log(1+\exp(-y \cdot \fx)) 
$$
 \item Simply check both cases for y to see
 \item Transformation $y'=(y+1)/2$ and $y'=2y-1$ help to convert between the 2 encodings
 \item All loss variants convex, differentiable
\end{itemize}

\begin{center}
\includegraphics[width = 8cm]{figure/bernoulli_margin.png}
\end{center}

\end{vbframe}


\begin{vbframe}{Risk Minimizer on Probs}

For probs and $y \in \setzo$ is

$$
  \piastxtil = \eta(\xtil) = \P(y = 1 ~|~ \xv = \xtil). 
$$

\textbf{Proof:} We had seen before;

$$
  \riskf = \E_x \left[L(1, \pix) \cdot \eta(\xv) + L(0, \pix) \cdot (1 - \eta(\xv)) \right]
$$

\vfill


For fixed $\xv$, minimize inner part pointwise, use $c$ for best value:

\begin{footnotesize}
\begin{eqnarray*}
  \frac{d }{d c} \left(- \log c  \cdot \eta(\xv)- \log (1 - c) \cdot (1 - \eta(\xv))\right) &=& 0 \\
  - \frac{\eta(\xv)}{c} + \frac{1 - \eta(\xv)}{1 - c} &=& 0 \\
  \frac{- \eta(\xv) + \eta(\xv) c + c - \eta(\xv) c}{c (1 - c)} &=& 0 \\
  c &=& \eta(\xv)
\end{eqnarray*}
\end{footnotesize}

\end{vbframe}

\begin{vbframe}{Risk Minimizer on Scores}


\begin{itemizeS}
    \item For $y \in \{-1, 1\}$ and scores $\fx$: RM is point-wise log-odds
\end{itemizeS}


\begin{minipage}{0.3\textwidth} 
	\centering
	{$\fxbayes =  \log (\frac{\eta(\xv) }{1-\eta(\xv) })$}
\end{minipage}
\hspace{-.03\textwidth}
\begin{minipage}{0.7\textwidth}
	\centering	
	\includegraphics[width=0.7\textwidth]{figure/logistic_inverse.png}
\end{minipage}

\begin{itemizeS}
    \item Undefined for $\eta(\xv) \in \setzo$
    \item Predicts smooth curve increasing in $\P(y=1~|~\xv = \xtil)$ otherwise
    \item Equals $0$ when $\P(y=1~|~\xv = \xtil) = 0.5$
\end{itemizeS}


\framebreak 

\textbf{Proof: } As before we minimize 
\begin{eqnarray*}
  \riskf &=& \E_x \left[L(1, \fx) \cdot \eta(\xv) + L(-1, \fx) \cdot (1 - \eta(\xv)) \right] \\
  &=& \E_x \left[ \log(1 + \exp(- \fx)) \eta(\xv)+ \log(1 + \exp(\fx)) (1 - \eta(\xv)) \right] 
\end{eqnarray*}

For a fixed $\xv$ we compute the point-wise optimal value $c$ by setting the derivative to $0$: 

\begin{footnotesize}
  \begin{eqnarray*}
  \frac{\partial }{\partial c} \log(1 + \exp(-c)) \eta(\xv)+ \log(1 + \exp(c)) (1 - \eta(\xv)) &=& 0 \\
  - \frac{\exp(-c)}{1 + \exp(-c)} \eta(\xv) + \frac{\exp(c)}{1 + \exp(c)} (1 - \eta(\xv)) &=& 0 \\ 
  % - \frac{\exp(-c)}{1 + \exp(-c)} \eta(\xv) + \frac{1}{1 + \exp(- c)} (1 - \eta(\xv)) &=& 0\\ 
  % &=& -  \frac{\exp(-c)}{1 + \exp(-c)} p + \frac{1}{1 + \exp(-c)} - \frac{1}{1 + \exp(-c)} p \\
  - \frac{\exp(-c) \eta(\xv) - 1 + \eta(\xv)}{1 + \exp(-c)} &=& 0 \\
  - \eta(\xv) + \frac{1}{1 + \exp(-c)} &=& 0\\
  % \eta(\xv) &=& \frac{1}{1 + \exp(-c)} \\
   c &=& \log\left(\frac{\eta(\xv)}{1 - \eta(\xv)}\right)
  \end{eqnarray*}
\end{footnotesize}

\end{vbframe}



\begin{vbframe}{Bernoulli: Optimal constant Model}

{\small 

\begin{itemizeS}
    \item Optimal constant probability model $\pix = \theta$ w.r.t. the Bernoulli loss for labels from $\Yspace = \setzo$ is 
    
    $$\thetah = \argmin_{\theta} \risket = \frac{1}{n} \sumin \yi$$
    \item Fraction of class-1 observations in the observed data
    \item Derived by setting derivative of risk to $0$ and solving for $\theta$
    \item Optimal constant score model $\fx = \theta$ w.r.t. the Bernoulli loss labels from $\Yspace = \setmp$ or $\Yspace = \setzo$ is

    $$\thetah = \argmin_{\theta} \risket = \log \frac{\np}{\nn} = \log \frac{\np / n}{\nn /n}$$ 

    where $\nn$ and $\np$ are the numbers of negative and positive observations.%, respectively.

    \item Shows tight (and unsurprising) connection of this loss to log-odds. Proof is left as an exercise.
\end{itemizeS}

}





\end{vbframe}

\begin{vbframe}{Naming Conventions}

We have seen several closely related loss functions: %In the literature, there are different names for the losses: 

%\begin{eqnarray*}
%  L(y, \fx) &=& \log(1+\exp(-y\fx)) \quad \text{for } y \in \setmp \\
%  L(y, \fx) &=& - y \cdot \fx + \log(1 + \exp(\fx)) \quad \text{for } y \in \setzo 
%\end{eqnarray*}

\begin{alignat*}{3} \Lxy    &= \log(1+\exp(-y\fx)) &&\quad \text{for } y \in \setmp \\ \Lxy    &= - y \cdot \fx + \log(1 + \exp(\fx)) &&\quad \text{for } y \in \setzo \\ \Lpixy  &= - y \log \left(\pix\right) - (1 - y) \log \left(1 - \pix\right) &&\quad \text{for } y \in \setzo \\ \Lpixy  &= - \frac{1 + y}{2} \log\left(\pix\right) - \frac{1 - y}{2} \log\left(1 - \pix\right) &&\quad \text{for } y \in \setmp \end{alignat*}



\lz

They are equally referred to as Bernoulli-, Binomial-, logistic-, log-, or cross-entropy loss. % (showing equivalence is a simple exercise).

%  $$
%    L(y, \pix) = - y \log \left(\pix\right) - (1 - y) \log \left(1 - \pix\right) \quad \text{for } y \in \setzo
%  $$

%is referred to as cross-entropy or log-loss. 

%\lz 

%We usually refer to all of them as \textbf{Bernoulli loss}, and rather make clear whether they are defined on labels $y \in \setzo$ or $y \in \setmp$ and on scores $\fx$ or probabilities $\pix$. 

\end{vbframe}


\begin{vbframe}{Optimization Properties: Convergence}

\footnotesize
%The choice of the loss function may also impact convergence behavior.

In case of \textbf{complete separation}, optimization might 
fail, e.g.:
%Consider the following scenario:

\vspace{0.5cm}

\begin{minipage}{0.7\textwidth}
  \begin{itemize}
    \item \textbf{Bernoulli loss} is margin-based and strictly monotonically decreasing in
    $y \cdot \fx$: 
    % \scriptsize
    $$\Lxy = \log \left( 1 + \exp \left( - y  \fx \right)\right)$$
  \end{itemize}
\end{minipage}%
\begin{minipage}{0.05\textwidth}
  \phantom{foo}
\end{minipage}%
\begin{minipage}{0.25\textwidth}
  \includegraphics[width=\textwidth]{figure/bernoulli.png}
\end{minipage}%

\begin{itemize}
  \item $f$ linear in $\thetav$, e.g.,
  \textbf{logistic regression} with $\fxt = \thx$
  %\begin{itemize}
    %\small
  \item Data perfectly separable by our learner, so we can find $\thetav$:
   $$ \yi \fxit = \yi \thetav^T \xi > 0 ~~ \forall \xi$$
%    \\ \vspace{0.1cm} \scriptsize 
    %as every $\xi$ is correctly classified: $\fxit < 0$ for
    %$\yi = -1$, $> 0$ for $\yi = 1$ \small
   % \vspace{0.1cm}
 %   \item[$\Rightarrow$] $y \fxt = $ 
  %\end{itemize}
 % \vspace{0.1cm}
  %   \begin{itemize}
  %   \small
  %   \item[$\Rightarrow$] $f \left( \xv ~|~ a \cdot \thetav \right) = 
  %   a \cdot \fxt$ for $a > 1$
  % \end{itemize}
%\end{itemize}

% \vfill

% With optimization, e.g., gradient descent, we can always find a set of 
% parameters $\thetav^\prime$ that fully separates the data.

%\framebreak

%\begin{itemize} 
  %\small
 % \item In optimization, e.g., with gradient descent, we can always find a set 
  %of parameters $\thetav^\prime$ that classifies all samples perfectly.
   
  %But taking a closer look at $\risket$, we find that the same can be 
  %achieved with $2 \cdot \thetav^\prime$ -- and at lower risk:
  
\item Can now a construct a strictly better $\thetav$

$$    \riske(2 \cdot \thetav) = \sumin  
    % \log \left( 1 + \exp \left( - |2 \thetav^T \xi| \right)\right) \\
    L \left( 2 \yi \thetav^T \xi  
    \right) < \risket
$$

\item As ||$\thetav$|| increases, sum strictly decreases, as argument of L is strictly larger

  %  \begin{flalign*
    %\riske(2 \cdot \thetav) &= \sumin  
    % \log \left( 1 + \exp \left( - |2 \thetav^T \xi| \right)\right) \\
   % L \left( \left \rvert f\left( \xi ~|~ 2  \cdot \thetav \right) \right \rvert  
   % \right)
   %5 = \sumin L \left(2 \cdot \left \rvert %\fxit \right \rvert \right) \\
   %5 &< \sumin L \left( \left \rvert \fxit \right \rvert \right) = \risket
 % \end{flalign*}
\item We can iterate that, so there is no local (or global) optimum, and no numerical procedure can converge
%  \item This actually holds true for every $a \cdot \thetav$ with $a > 1$.
%  \begin{itemize}
%    \small
%    \item[$\Rightarrow$] By increasing $\| \thetav \|$, our loss becomes smaller 
%    and smaller, and optimization runs infinitely.
    % \item[$\Rightarrow$] This is a consequence of the above assumptions and 
    % sometimes encountered in logistic regression and linear support vector 
    % machines.
  \end{itemize}
%\end{itemize}  

\framebreak

\begin{itemize}
  \small
  \item
  Geometrically, this translates to an ever steeper slope of the 
  logistic/softmax function, i.e., increasingly sharp discrimination:
  
  \vspace{0.3cm}
  \begin{minipage}[b]{0.5\textwidth}
    \includegraphics[width=0.8\textwidth]{figure/softmax_1}
  \end{minipage}%
  \begin{minipage}[b]{0.5\textwidth}
    \includegraphics[width=0.8\textwidth]{figure/softmax_2}
  \end{minipage}%
  \item In practice, data are seldomly linearly separable and misclassified 
  examples act as counterweights to increasing parameter values
  \item We can also use \textbf{regularization} for  robust solutions
\end{itemize}

\end{vbframe}




% \vspace*{-0.2cm}

% \begin{eqnarray*}
%   \Lxy = \log \left[1 + \exp \left(-y\fx\right)\right].
% \end{eqnarray*}

% We transform scores into probabilities by

% $$
% \pix = \P(y = 1 ~|~\xv) = s(\fx) = \frac{1}{1 + \exp(- \fx)},
% $$

% with $s(.)$ being the logistic sigmoid function as introduced in chapter 2.

% \framebreak

% As already shown before, an equivalent approach that directly outputs probabilities $\pix$ is minimizing the \textbf{Bernoulli loss}

% \begin{eqnarray*}
% \Lxy = -y \log \left(\pix\right) - (1 - y) \log \left(1 - \pix\right)
% \end{eqnarray*}

% for $\pix$ in the hypothesis space

% \begin{eqnarray*}
%   \Hspace = \left\{\pi: \Xspace \to [0, 1] ~|~\pix = s(\thetav^\top \xv)\right\}
% \end{eqnarray*}

% with $s(.)$ again being the logistic function.

% \framebreak


% Logistic Regression with one feature $\xv \in \R$. The figure shows how $\xv \mapsto \pix$.

% <<fig.height=4>>=
% set.seed(1234)
% n = 20
% x = runif(n, min = 0, max = 7)
% y = x + rnorm(n) > 3.5
% df = data.frame(x = x, y = y)

% model = glm(y ~ x,family = binomial(link = 'logit'), data = df)
% df$score = predict(model)
% df$prob = predict(model, type = "response")
% x = seq(0, 7, by = 0.01)
% dfn = data.frame(x = x)
% dfn$prob = predict(model, newdata = dfn, type = "response")
% dfn$score = predict(model, newdata = dfn)

% p2 = ggplot() + geom_line(data = dfn, aes(x = x, y = prob))
% p2 = p2 + geom_point(data = df, aes(x = x, y = prob, colour = y), size = 2)
% p2 = p2 + xlab("x") + ylab(expression(pi(x)))
% p2 = p2 + theme(legend.position = "none")
% p2
% @

% \framebreak

% Logistic regression with two features:

\endlecture

\end{document}