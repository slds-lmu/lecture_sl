\documentclass[11pt,compress,t,notes=noshow, xcolor=table]{beamer}
%<<setup-child, include = FALSE>>=
%library(knitr)
%library(qrmix)
%library(mlr)
%library(quantreg)
%library(reshape2)
%set_parent("../style/preamble.Rnw")
%@

\input{../../style/preamble}
\input{../../latex-math/basic-math}
\input{../../latex-math/basic-ml}

\title{Introduction to Machine Learning}

\begin{document}
    
\titlemeta{% Chunk title (example: CART, Forests, Boosting, ...), can be empty
    Advanced Risk Minimization
  }{% Lecture title  
    Maximum Likelihood Estimation vs.
    Empirical Risk Minimization
  }{% Relative path to title page image: Can be empty but must not start with slides/
  figure/residuals_plot_L2_title.png
  }{
  \item Understand the connection between maximum likelihood and risk minimization
  \item Learn the correspondence between a Gaussian error distribution and the L2 loss
}

\begin{vbframe}{Maximum Likelihood}

Let's consider regression from a maximum likelihood perspective. Assume: 

$$
	y~|~ \xv \sim p(y~|~\xv, \thetab) 
$$

\vspace{0.5cm}

Common case: 
true underlying relationship $\ftrue$ with additive noise: 

\vspace{0.5cm}

\begin{minipage}{0.5\textwidth}
$$
y = \ftrue(\xv) + \eps
$$
\end{minipage}%
\begin{minipage}{0.5\textwidth}
\includegraphics[width = 0.75\textwidth]{figure/ftrue.pdf}
\end{minipage}

where $\ftrue$ has params $\thetab$ and $\eps$ a RV that follows some distribution $\P_\eps$, with $\E[\eps] = 0$. Also, assume $\epsilon \perp \!\!\! \perp \xv$. 


\framebreak 

From a statistics / maximum-likelihood perspective, we assume (or we pretend) we know the underlying distribution $p(y~|~\xv, \thetab)$. 

\begin{itemize}
\item Then, given i.i.d data $
\D = \Dset
$
 from $\Pxy$ 

the maximum-likelihood principle is to maximize the \textbf{likelihood}

\begin{minipage}{0.5\textwidth}
$$ \LLt = \prod_{i=1}^n \pdfyigxit $$
\end{minipage}%
\begin{minipage}{0.5\textwidth}
\includegraphics[width = 0.75\textwidth]{figure/log_reg_ml.pdf}
\end{minipage}

or equivalently to minimize the \textbf{negative log-likelihood}

\begin{minipage}{0.5\textwidth}
$$ -\loglt = -\sumin \lpdfyigxit. $$
\end{minipage}%
\begin{minipage}{0.5\textwidth}
\includegraphics[width = 0.75\textwidth]{figure/log_reg_erm.pdf}
\end{minipage}
\end{itemize}


\framebreak 

From an ML perspective we assume our hypothesis space corresponds to the space of the (parameterized) $\ftrue$. 

\begin{itemize}
\item Simply define neg. log-likelihood as \textbf{loss function} 
$$ \Lxyt := - \lpdfygxt $$
\item Then, maximum-likelihood 
 = ERM
$$\risket = \sumin \Lxyit$$

% \item Then the maximum-likelihood estimator $\thetah$, which we obtain by optimizing $\LLt$ is identical
% to the loss-minimal $\thetah$ we obtain by minimizing $\risket$.
\item NB: When we are only interested in the minimizer, we can ignore multiplicative or additive constants.
\item We use $\propto$ as \enquote{proportional up to multiplicative and additive constants}

\end{itemize}

\end{vbframe}


\begin{vbframe}{Gaussian Errors - L2-Loss} 

Assume $y = \ftrue(\xv) + \eps$ with additive Gaussian errors, i.e. $\epsi \sim \mathcal{N}(0, \sigma^2)$. Then

$$y~|~\xv \sim N\left(\ftrue(\xv), \sigma^2\right)$$


The likelihood is then 

\begin{eqnarray*}
\LL(\thetab) &=& \prodin \pdf\left(\yi ~\bigg|~ \fxit, \sigma^2\right) \\ &\propto& \prodin \exp\left(-\frac{1}{2\sigma^2} \left(\yi - \fxit\right)^2\right)
\end{eqnarray*}

\framebreak 

Easy to see: minimizing neg. negative log-likelihood with Gaussian errors is the same as ERM with $L2$-loss:

\begin{eqnarray*}
- \loglt &=& - \log\left(\LL(\thetab)\right) \\
&=& - \log\left(\prodin \exp\left(-\frac{1}{2\sigma^2} \left(\yi - \fxit\right)^2\right)\right) \\
&\propto& \sumin \left(\yi - \fxit\right)^2
\end{eqnarray*}



\framebreak 

\begin{footnotesize}
\begin{itemize}
	\item We simulate data $y ~|~\xv \sim \mathcal{N}\left(\ftrue(\xv), 1\right)$ with $\ftrue = 0.2 \cdot \xv$ 
\item Let's plot empirical errors as histogram, after fitting our model with $L2$-loss
\item Q-Q-plot compares empirical residuals vs. theoretical quantiles of Gaussian 
\end{itemize}
\end{footnotesize}

\includegraphics[width = 0.8\textwidth]{figure/residuals_plot_L2.pdf}

\end{vbframe}

\begin{vbframe}{Distributions and losses}

\begin{itemize} 
\item For every error distribution $\P_\eps$ we can derive an equivalent loss function, which leads to the same point estimator for the parameter vector $\thetab$ as maximum-likelihood. Formally,
\begin{itemize}
    \item $\thetah \in \argmax_{\thetab} \LL(\thetab) \implies \thetah \in \argmin_{\thetab}-\log(\LL(\thetab))$  %\implies \thetah \in \argmin_{\thetab}-\log(\LL(\thetab))$ 
    %\item $\thetah \in \argmax_{\thetab} \log\left(\LL(\thetab)\right) \implies $
\end{itemize}
%\lz
\item \textbf{But}: The other way around does not always work: We cannot derive a corresponding pdf or error distribution for every loss function -- the Hinge loss is one prominent example, for which some probabilistic interpretation is still possible however, see \citelink{sollich1999ninth}.

\framebreak

When does the reverse direction hold?
\item If we can write the loss as $L(y,\fx)=L(y-\fx)=L(r)$ for $r \in \mathbb{R}$, then minimizing $L(y-\fx)$ is equivalent to maximizing a conditional log-likelihood $\log(p(y-f(\xv|\thetab))$ if
\begin{itemize}
 \setlength{\itemsep}{1.2em}
    \item $\log(p(r))$ is affine trafo of $L$ (undoing the $\propto$):
% minus due to sign chance when we go from loss to loglik
    $$\log(p(r)) = a - bL(r),\,\,a \in \mathbb{R}, b>0$$
    \item  $p$ is a pdf (non-negative and integrates to one)
\end{itemize}
\lz
Thus, a loss $L$ corresponds to MLE under \textit{some} distribution if there exist $a \in \mathbb{R},\, b>0$ such that
$$ \int_{\mathbb{R}} \exp(a-bL(r))\text{d}r = 1$$
\end{itemize}
    
\end{vbframe}


\endlecture
\end{document}
