\documentclass[11pt,compress,t,notes=noshow, xcolor=table]{beamer}
\input{../../style/preamble}
\input{../../latex-math/basic-math}
\input{../../latex-math/basic-ml}

\newcommand{\titlefigure}{figure_man/equal_decode.png}
\newcommand{\learninggoals}{
  \item Know connection between source coding and (cross-)entropy  
  \item Know that the entropy of the source distribution is the lower bound for the average code length
}

\title{Introduction to Machine Learning}
\date{}

\begin{document}

\lecturechapter{Entropy and Optimal Code Length II}
\lecture{Introduction to Machine Learning}


%%%%%%% CUT HERE SECOND SOURCE CODING CHUNK

\begin{vbframe} {Source coding and (cross-)entropy}

\begin{itemize}
  \item For a random source / distribution $p$, the minimal number of bits to optimally encode messages from is the entropy $H(p)$.
  \item If the optimal code for a different distribution $q(x)$ is instead used to encode messages from $p(x)$, expected code length will grow.
%  (Note: Both distributions are assumed to have the same support.)
\end{itemize}
  \vspace{-0.3cm}
  \begin{figure}
    \centering
      \scalebox{0.5}{\includegraphics{figure_man/shift.png}}
      \scalebox{1}{\includegraphics{figure_man/xent_pq.png}}
      \caption{\footnotesize{$L_p(x)$, $L_q(x)$ are the optimal code lengths for $p(x)$ and $q(x)$}}
  \end{figure}

\framebreak
\textbf{Cross-entropy} is the average length of communicating an event from one distribution with the optimal code for another distribution (assume they have the same domain $\Xspace$ as in KL).
  $$ H(p \| q) = \sum_{x \in \Xspace} p(x) \log\left(\frac{1}{q(x)}\right) = - \sum_{x \in \Xspace} p(x) \log\left(q(x)\right) $$

\begin{figure}
    \centering
      \scalebox{1}{\includegraphics{figure_man/xent_pq.png}}
      \caption{\footnotesize{$L_p(x)$, $L_q(x)$ are the optimal code lengths for $p(x)$ and $q(x)$}}
  \end{figure}
  
We directly see: cross-entropy of $p$ with itself is entropy: $H(p \| p) = H(p)$.
  
\framebreak
  \begin{figure}
    \centering
      \scalebox{0.8}{\includegraphics{figure_man/crossent.png}}
      \tiny{\\ Credit: Chris Olah}
  \end{figure}
  
  \begin{itemize}
    \item \small{In top, $H(p \| q)$ is greater than $H(p)$ primarily because the blue event that is very likely under $p$ has a very long codeword in $q$.
    \item Same, in bottom, for pink when we go from $q$ to $p$.
    \item Note that $H(p \| q) \neq H(q \| p)$}. 
  \end{itemize}

  \framebreak

  \begin{figure}
    \centering
      \scalebox{1}{\includegraphics{figure_man/xent_pq.png}}
      \caption{\footnotesize{$L_p(x)$, $L_q(x)$ are the optimal code lengths for $p(x)$ and $q(x)$}}
  \end{figure}
  
  \begin{itemize}
   \item Let $x^\prime$ denote the symbol "dog". The difference in code lengths is:
  $$ \log \left ( \frac{1}{q(x^\prime)} \right ) - \log \left( \frac{1}{p(x^\prime)} \right) = \log \frac{p(x^\prime)}{q(x^\prime)} $$
  
\item If $p(x^\prime) > q(x^\prime)$, this is positive, if $p(x^\prime) < q(x^\prime)$, it is negative. 
    \item The expected difference is KL, if we encode symbols from $p$:
  $$ D_{KL}(p \| q) = \sum_{x \in \Xspace} p(x) \cdot \log \frac{p(x)}{q(x)} $$
  \end{itemize}

\end{vbframe}


\endlecture
\end{document}

