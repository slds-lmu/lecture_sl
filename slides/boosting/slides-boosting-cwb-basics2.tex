\documentclass[11pt,compress,t,notes=noshow, xcolor=table]{beamer}
\input{../../style/preamble}
\input{../../latex-math/basic-math}
\input{../../latex-math/basic-ml}
\input{../../latex-math/ml-ensembles.tex}
\input{../../latex-math/ml-trees.tex}

\usepackage{dsfont}
\usepackage{transparent}

\newcommand{\titlefigure}{figure/compboost-illustration-2.png}
\newcommand{\learninggoals}{
  \item Concept of CWB and relation to GLM
  \item Built-in feature selection process
  \item Fair base learner selection
}

\title{Introduction to Machine Learning}
\date{}

\begin{document}

\lecturechapter{Componentwise Gradient Boosting}
\lecture{Introduction to Machine Learning}


\begin{vbframe}{Handling of categorical features}

Suppose a feature $x_j \in \{1, \dots, G\}$.
\begin{itemize}
  \item Two common options for encoding the categorical feature $x_j$ in a base learner $b_j$ are:
  \begin{itemize}
    \item 
        One base learner to simultaneously estimate all categories: 
        $$b_j(x_j | \thetab_j) = \sum_{g=1}^G \theta_{j,g}\mathds{1}_{\{g = x_j\}} = (\mathds{1}_{\{x_j = 1\}}, ..., \mathds{1}_{\{x_j = G\}}) \thetab_j$$
        Hence, $b_j$ incorporates a one-hot encoded feature with group means $\thetab\in\R^G$ as estimates. 
    
    \item 
        One binary base learner per category:
        $$b_{j,g}(x_j | \theta_{j,g}) = \theta_{j,g}\mathds{1}_{\{g = x_j\}}$$  
        Including all categories of the feature means adding $G$ base learners $b_{j,1}, \dots, b_{j,G}$ with each accounting for one specific class.
  \end{itemize}
  %\item The \texttt{compboost} package currently implements the first variant.
\end{itemize}

% ------------------------------------------------------------------------------
\framebreak

Advantages within CWB of simultaneously handling all categories: 
\begin{itemize}
    \item 
        Much faster estimation process compared to adding the categories as individual binary base learners.

    \item 
        Explicit solution of $\thetabh = \argmin_{\thetab\in\R^G}\sumin (\yi - b_j(x^{(i)}_j | \thetab))^2$ with:
        $$\thetabh = (\thetah_1, \dots, \thetah_G)^T,\ \thetah_g = n_g^{-1}\sumin \yi \mathds{1}_{\{x^{(i)}_j = g\}}$$

    \item 
        Adding regularization is easy by reframing the base learner as a linear model and using, e.g., Ridge regression for estimating $\thetabh$.  
\end{itemize}

Disadvantage within CWB of simultaneously handling all categories: 
\begin{itemize}
    \item 
        If class $g^\ast\in\{1, \dots, G\}$ is not informative, the respective parameter estimate $\thetah_{g^\ast}$ is (almost sure) not $0$.
\end{itemize}

% ------------------------------------------------------------------------------
\framebreak

Advantages within CWB of including categories individually: 
\begin{itemize}
    \item 
        A finer selection within the feature is possible since non-informative categories are simply not included in the model and hence have an effect of $0$.

    \item 
        Explicit solution of $\thetah_{j,g} = \argmin_{\theta\in\R}\sumin (\yi - b_g(x^{(i)}_j|\theta))^2$ with:
        $$\thetah_{j,g} = n_g^{-1}\sumin \yi \mathds{1}_{\{x^{(i)}_j = g\}}$$
\end{itemize}

Disadvantage within CWB of simultaneously handling all categories:
\begin{itemize}
    \item 
        Fitting CWB is much slower due to the overhead in each iteration (updating the model, pseudo residuals, and doing the base learner selection) after conducting one tiny step into the direction of the 1-dim parameter $\thetah_{j,g}$.

    \item 
        The base learner has exactly one degree of freedom which makes penalization and a fair selection very hard if not impossible.
\end{itemize}

\end{vbframe}

% ------------------------------------------------------------------------------

\begin{vbframe}{intercept handling}

\begin{itemize}
  %\item The loss-optimal constant $\fm[0](\xv)$ is an initial model intercept.
  %\item An intercept is often referred to as part of a model which contains information independent of the features.
  \item Suppose linear base learners $b_j(\xv) = \theta_{j1} + \theta_{j2} x_j$ with one intercept $\theta_{j1}$ per base learner and slope $\theta_{j2}$.
  \item If base learner $\hat{b}_j$ with parameter estimates $\thetamh[1] = (\hat{\theta}_{j1}^{[1]}, \hat{\theta}_{j1}^{[1]})$ is selected in the first iteration consequently updates the model intercept to $\fm[0](\xv) + \hat{\theta}_{j1}^{[1]}$.
  \item Throughout the fitting process, the intercept is adjusted $M$ times to its final form:
    $$
    \fm[0](\xv) + \sum\limits_{m=1}^M \hat{\theta}^{[m]}_{j^{[m]}1}
    $$
\end{itemize}
$\Rightarrow$ All intercepts in the base learners are collected and aggregated to one model intercept.

% ------------------------------------------------------------------------------
\framebreak

Two possible options to handle the intercept in CWB are:

\begin{itemize}

\item Include an intercept base learner:
  \begin{itemize}
    \item Add base learner $b_{\text{int}} = \theta$ as potential candidate considered in each iteration.
    \item At the same time, remove the intercept from all linear base learners to only use $b_j(\xv) = \theta_j x_j$.
    \item The final intercept is given by $\fm[0](\xv) + \hat{\theta}$.
  \end{itemize}
  \item Include an intercept in each linear base learner $b_j(\xv) = \theta_{j1} + \theta_{j2} x_j$ and accumulate all intercepts to one global intercept after the fitting.

\end{itemize}

% ------------------------------------------------------------------------------

\framebreak

The following figure shows a comparison of the parameter updates with different intercept handlings:
\begin{center}
% Recreate figure: rsrc/fig-cwb-intercept-handling.R
\includegraphics[width = \textwidth]{figure/compboost-intercept-handling.png}
\end{center}
The parameter estimates converge to the same value. The used data set is \href{https://github.com/topepo/AmesHousing}{Ames Housing}.


\end{vbframe}

%\newcommand{\phl}{\phantom{---}}

\begin{vbframe}{Feature selection in CWB}

Since only one base learner is selected in iteration $m$, the features are added one by one while the fitting progresses. Therefore, suppose 4 features $x_1, \dots, x_4$ with base learners $b_1, \dots, b_4$ which models these features:
\begin{itemize}
    \item 
        Iteration 1 selects base learner $b_{2}$, $j^{[1]} = 2$.
\end{itemize}
\begin{table}
  \centering
  \scriptsize
    \begin{tabular}{c|cccccccccc|c}
    \multicolumn{1}{c|}{} & \multicolumn{10}{c}{\bfseries Iteration m} \\ 
    $j^{[m]}$ & 1 & 2 & 3 & 4 & 5 & 6 & 7 & 8 & 9 & 10  & {\bfseries \#} \\ \hline\hline
    1 &     & \phl &     &     &     &     &     &     & \phl & \phl & 0\\
    2 & --- &     & \phl &     &     & \phl & \phl & \phl &     &     & 1\\
    3 &     &     &     & \phl & \phl &     &     &     &     &     & 0\\
    4 &     &     &     &     &     &     &     &     &     &     & 0 
    \end{tabular}
\end{table}

\end{vbframe}

\begin{vbframe}{Feature selection in CWB}

Since only one base learner is selected in iteration $m$, the features are added one by one while the fitting progresses. Therefore, suppose 4 features $x_1, \dots, x_4$ with base learners $b_1, \dots, b_4$ which models these features:
\begin{itemize}
    \item 
        Iteration 2 selects base learner $b_{1}$, $j^{[2]} = 1$.
\end{itemize}

\begin{table}
  \centering
  \scriptsize
    \begin{tabular}{c|cccccccccc|c}
    \multicolumn{1}{c|}{} & \multicolumn{10}{c}{\bfseries Iteration m} \\ 
    $j^{[m]}$ & 1 & 2 & 3 & 4 & 5 & 6 & 7 & 8 & 9 & 10  & {\bfseries \#} \\ \hline\hline
    1 &     & --- &     &     &     &     &     &     & \phl & \phl & 1\\
    2 & --- &     & \phl &     &     & \phl & \phl & \phl &     &     & 1\\
    3 &     &     &     & \phl & \phl &     &     &     &     &     & 0\\
    4 &     &     &     &     &     &     &     &     &     &     & 0 
    \end{tabular}
    \addtocounter{framenumber}{-1}
\end{table}

\end{vbframe}

\begin{vbframe}{Feature selection in CWB}

Since only one base learner is selected in iteration $m$, the features are added one by one while the fitting progresses. Therefore, suppose 4 features $x_1, \dots, x_4$ with base learners $b_1, \dots, b_4$ which models these features:
\begin{itemize}
    \item 
        Iteration 3 again selects base learner $b_{2}$, $j^{[3]} = 2$.
\end{itemize}

\begin{table}
  \centering
  \scriptsize
    \begin{tabular}{c|cccccccccc|c}
    \multicolumn{1}{c|}{} & \multicolumn{10}{c}{\bfseries Iteration m} \\ 
    $j^{[m]}$ & 1 & 2 & 3 & 4 & 5 & 6 & 7 & 8 & 9 & 10  & {\bfseries \#} \\ \hline\hline
    1 &     & --- &     &     &     &     &     &     & \phl & \phl & 1\\
    2 & --- &     & --- &     &     & \phl & \phl & \phl &     &     & 2\\
    3 &     &     &     & \phl & \phl &     &     &     &     &     & 0\\
    4 &     &     &     &     &     &     &     &     &     &     & 0 
    \end{tabular}
    \addtocounter{framenumber}{-1}
\end{table}

\end{vbframe}

\begin{vbframe}{Feature selection in CWB}

Since only one base learner is selected in iteration $m$, the features are added one by one while the fitting progresses. Therefore, suppose 4 features $x_1, \dots, x_4$ with base learners $b_1, \dots, b_4$ which models these features:
\begin{itemize}
    \item 
        Iteration 4 selects base learner $b_{3}$, $j^{[4]} = 3$.
\end{itemize}

\begin{table}
  \centering
  \scriptsize
    \begin{tabular}{c|cccccccccc|c}
    \multicolumn{1}{c|}{} & \multicolumn{10}{c}{\bfseries Iteration m} \\ 
    $j^{[m]}$ & 1 & 2 & 3 & 4 & 5 & 6 & 7 & 8 & 9 & 10  & {\bfseries \#} \\ \hline\hline
    1 &     & --- &     &     &     &     &     &     & \phl & \phl & 1\\
    2 & --- &     & --- &     &     & \phl & \phl & \phl &     &     & 2\\
    3 &     &     &     & --- & \phl &     &     &     &     &     & 1\\
    4 &     &     &     &     &     &     &     &     &     &     & 0 
    \end{tabular}
    \addtocounter{framenumber}{-1}
\end{table}

\end{vbframe}

\begin{vbframe}{Feature selection in CWB}

Since only one base learner is selected in iteration $m$, the features are added one by one while the fitting progresses. Therefore, suppose 4 features $x_1, \dots, x_4$ with base learners $b_1, \dots, b_4$ which models these features:
\begin{itemize}
    \item 
        Iteration 10 selects base learner $b_{1}$, $j^{[10]} = 1$.
\end{itemize}

\begin{table}
  \centering
  \scriptsize
    \begin{tabular}{c|cccccccccc|c}
    \multicolumn{1}{c|}{} & \multicolumn{10}{c}{\bfseries Iteration m} \\ 
    $j^{[m]}$ & 1 & 2 & 3 & 4 & 5 & 6 & 7 & 8 & 9 & 10  & {\bfseries \#} \\ \hline\hline
    1 &     & --- &     &     &     &     &     &     & --- & --- & 3\\
    2 & --- &     & --- &     &     & --- & --- & --- &     &     & 5\\
    3 &     &     &     & --- & --- &     &     &     &     &     & 2\\
    4 &     &     &     &     &     &     &     &     &     &     & 0 
    \end{tabular}
    \addtocounter{framenumber}{-1}
\end{table}

\end{vbframe}

\begin{vbframe}{Feature selection in CWB}
The iterative nature of CWB and additivity of the base learners 
\end{vbframe}

\begin{vbframe}{Feature selection in CWB}

\begin{table}
  \centering
  \scriptsize
    \begin{tabular}{c|cccccccccc|c}
    \multicolumn{1}{c|}{} & \multicolumn{10}{c}{\bfseries Iteration m} \\ 
    $j^{[m]}$ & 1 & 2 & 3 & 4 & 5 & 6 & 7 & 8 & 9 & 10  & {\bfseries \#} \\ \hline\hline
    1 &     & --- &     &     &     &     &     &     & --- & --- & 3\\
    2 & --- &     & --- &     &     & --- & --- & --- &     &     & 5\\
    3 &     &     &     & --- & --- &     &     &     &     &     & 2\\
    4 &     &     &     &     &     &     &     &     &     &     & 0 
    \end{tabular}
    \addtocounter{framenumber}{-1}
\end{table}

\textbf{Notes:} 
\begin{itemize}
    \item 
        Feature 4 was not selected at all $\Rightarrow$ Not important to model the data?

    \item
        Number of selections per feature an indicator of the importance of a feature? Is feature 2 most important because it was selected more often than the others?

    \item 
        Due to the additivity of the base learners, 3 feature effects are estimated (see later).
\end{itemize}

\end{vbframe}

% ------------------------------------------------------------------------------

\begin{vbframe}{Example: Life expectancy}

Consider the \texttt{life expectancy} (data set is provided by the WHO and is available on Kaggle: \url{https://www.kaggle.com/datasets/kumarajarshi/life-expectancy-who}) regression task, for which we fit a
CWB model with linear base learners (with intercept) to predict the life expectancy:

\begin{table}
\scriptsize
\begin{tabular}{l|l}
    \textbf{variable} & \textbf{description} \\
    \hline
    \texttt{Life.expectancy} & Life expectancy in years \\	
    \hline
    \texttt{Country}         & The country (just a selection GER, USE, SWE, ZAF, and ETH\\
    \texttt{Year}            & The recorded year\\
    \texttt{BMI}             & Average $\text{BMI} = \frac{\text{body weight in kg}}{(\text{Height in m})^2}$ in a year and country\\
    \texttt{Adult.Mortality} & Adult Mortality Rates of both sexes (probability of dying between 15 \\
                             & and 60 years per 1000 population)
\end{tabular}
\end{table}
Using \texttt{compboost} with $M = 150$ iterations, we can visualize which base learner was selected when and how the estimated feature effects evolve over time.
\end{vbframe}

% tex file and figures are created automatically by: rsrc/fig-cwb-anim.R

\begin{frame}{Example: Life expectancy}
	\begin{figure}
		\centering
		\includegraphics[width=\textwidth]{figure/cwb-anim/fig-iter-0001.png}
	\end{figure}
	\addtocounter{framenumber}{0}
\end{frame}


\begin{frame}{Example: Life expectancy}
	\begin{figure}
		\centering
		\includegraphics[width=\textwidth]{figure/cwb-anim/fig-iter-0002.png}
	\end{figure}
	\addtocounter{framenumber}{-1}
\end{frame}


\begin{frame}{Example: Life expectancy}
	\begin{figure}
		\centering
		\includegraphics[width=\textwidth]{figure/cwb-anim/fig-iter-0005.png}
	\end{figure}
	\addtocounter{framenumber}{-1}
\end{frame}


\begin{frame}{Example: Life expectancy}
	\begin{figure}
		\centering
		\includegraphics[width=\textwidth]{figure/cwb-anim/fig-iter-0010.png}
	\end{figure}
	\addtocounter{framenumber}{-1}
\end{frame}


\begin{frame}{Example: Life expectancy}
	\begin{figure}
		\centering
		\includegraphics[width=\textwidth]{figure/cwb-anim/fig-iter-0015.png}
	\end{figure}
	\addtocounter{framenumber}{-1}
\end{frame}


\begin{frame}{Example: Life expectancy}
	\begin{figure}
		\centering
		\includegraphics[width=\textwidth]{figure/cwb-anim/fig-iter-0016.png}
	\end{figure}
	\addtocounter{framenumber}{-1}
\end{frame}


\begin{frame}{Example: Life expectancy}
	\begin{figure}
		\centering
		\includegraphics[width=\textwidth]{figure/cwb-anim/fig-iter-0020.png}
	\end{figure}
	\addtocounter{framenumber}{-1}
\end{frame}


\begin{frame}{Example: Life expectancy}
	\begin{figure}
		\centering
		\includegraphics[width=\textwidth]{figure/cwb-anim/fig-iter-0030.png}
	\end{figure}
	\addtocounter{framenumber}{-1}
\end{frame}


\begin{frame}{Example: Life expectancy}
	\begin{figure}
		\centering
		\includegraphics[width=\textwidth]{figure/cwb-anim/fig-iter-0037.png}
	\end{figure}
	\addtocounter{framenumber}{-1}
\end{frame}


\begin{frame}{Example: Life expectancy}
	\begin{figure}
		\centering
		\includegraphics[width=\textwidth]{figure/cwb-anim/fig-iter-0038.png}
	\end{figure}
	\addtocounter{framenumber}{-1}
\end{frame}


\begin{frame}{Example: Life expectancy}
	\begin{figure}
		\centering
		\includegraphics[width=\textwidth]{figure/cwb-anim/fig-iter-0050.png}
	\end{figure}
	\addtocounter{framenumber}{-1}
\end{frame}


\begin{frame}{Example: Life expectancy}
	\begin{figure}
		\centering
		\includegraphics[width=\textwidth]{figure/cwb-anim/fig-iter-0070.png}
	\end{figure}
	\addtocounter{framenumber}{-1}
\end{frame}


\begin{frame}{Example: Life expectancy}
	\begin{figure}
		\centering
		\includegraphics[width=\textwidth]{figure/cwb-anim/fig-iter-0090.png}
	\end{figure}
	\addtocounter{framenumber}{-1}
\end{frame}


\begin{frame}{Example: Life expectancy}
	\begin{figure}
		\centering
		\includegraphics[width=\textwidth]{figure/cwb-anim/fig-iter-0110.png}
	\end{figure}
	\addtocounter{framenumber}{-1}
\end{frame}


\begin{frame}{Example: Life expectancy}
	\begin{figure}
		\centering
		\includegraphics[width=\textwidth]{figure/cwb-anim/fig-iter-0116.png}
	\end{figure}
	\addtocounter{framenumber}{-1}
\end{frame}


\begin{frame}{Example: Life expectancy}
	\begin{figure}
		\centering
		\includegraphics[width=\textwidth]{figure/cwb-anim/fig-iter-0117.png}
	\end{figure}
	\addtocounter{framenumber}{-1}
\end{frame}


\begin{frame}{Example: Life expectancy}
	\begin{figure}
		\centering
		\includegraphics[width=\textwidth]{figure/cwb-anim/fig-iter-0140.png}
	\end{figure}
	\addtocounter{framenumber}{-1}
\end{frame}


\begin{frame}{Example: Life expectancy}
	\begin{figure}
		\centering
		\includegraphics[width=\textwidth]{figure/cwb-anim/fig-iter-0150.png}
	\end{figure}
	\addtocounter{framenumber}{-1}
\end{frame}



%%%%% BLIND OUT, include with removing `\if1` and  \fi
\if1

\begin{vbframe}{example: boston housing}

\begin{minipage}[c]{0.4\textwidth}
  \small
  \raggedright
  Consider the \texttt{Boston housing} regression task, for which we fit a
  CWB model with linear base learners (with intercept) to predict median home
  value.
  Using \texttt{compboost} with $M = 100$ iterations, we can
  visualize which base learner was selected when:
\end{minipage}%
\begin{minipage}[c]{0.05\textwidth}
  \phantom{foo}
\end{minipage}%
\begin{minipage}[c]{0.55\textwidth}
  \tiny
  \begin{tabular}{l|l}
    \textbf{variable} & \textbf{description} \\
    \hline
    medv &	median value of owner-occupied homes in k USD \\
    \hline
    crim & per capita crime rate by town \\
    zn &	proportion of residential land zoned for lots $>$ 25k sq.ft \\
    indus &	proportion of non-retail business acres per town \\
    chas &	Charles River dummy (1 if tract bounds river, 0 otherwise) \\
    nox &	nitric oxides concentration (parts per 10m) \\
    rm &	average number of rooms per dwelling \\
    age &	proportion of owner-occupied units built prior to 1940 \\
    dis &	weighted distances to five Boston employment centres \\
    rad &	index of accessibility to radial highways \\
    tax &	full-value property-tax rate per USD 10k \\
    ptratio &	pupil-teacher ratio by town \\
    b &	$1000(B - 0.63)^2$,  $B$ as proportion of blacks by town \\
    lstat &	percentage of lower status of the population \\
  \end{tabular}
\end{minipage}%

\vfill

\begin{center}
\includegraphics[width = \textwidth]{figure/compboost-illustration-1.png}
\end{center}

\framebreak

% ------------------------------------------------------------------------------

The number of features effectively included in the final model depends on the
number of total iterations $M$.

\vfill

$\rightarrow$ A sparse linear regression is fitted.

\vfill

\includegraphics[width = \textwidth]{figure/compboost-illustration-2.png}

\end{vbframe}


\fi

\endlecture
\end{document}
