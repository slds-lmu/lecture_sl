\documentclass[11pt,compress,t,notes=noshow, xcolor=table]{beamer}
\input{../../style/preamble}
\input{../../latex-math/basic-math}
\input{../../latex-math/basic-ml}
\input{../../latex-math/ml-ensembles.tex}
\input{../../latex-math/ml-trees.tex}

\usepackage{dsfont}
\usepackage{transparent}

\newcommand{\titlefigure}{figure/compboost-illustration-2.png}
\newcommand{\learninggoals}{
  \item Concept of of CWB and relation to GLM
  \item Built-in feature selection process
  \item Fair base learner selection
}

\title{Introduction to Machine Learning}
\date{}

\begin{document}

\lecturechapter{Gradient Boosting: CWB Basics 1}
\lecture{Introduction to Machine Learning}

% ------------------------------------------------------------------------------

\begin{vbframe}{Componentwise gradient boosting}

GB (with trees), has strong predictive
performance but is difficult to interpret unless the base learners are stumps.

\lz

The aim of CWB is to find a model that exhibits:

\begin{itemize}
  \item
    strong predictive performance,

  \item
    interpretable components,

  \item
    automatic selection of components,

  \item
    is sparser than a model fitted with maximum-likelihood estimation.
\end{itemize}

\lz

This is achieved by using \enquote{nice} base learners which yield familiar
statistical models
in the end.

\lz

Because of this, CWB is also often referred to as \textbf{model-based boosting}.

\end{vbframe}

% ------------------------------------------------------------------------------

\begin{vbframe}{Base learners}

In GB only one kind of base learner $\mathcal{B}$ is used, e.g., regression trees.

\lz

For CWB we generalize this to multiple base learner sets $\{ \mathcal{B}_1, ... \mathcal{B}_J \}$ with associated parameter spaces
$\{ \bm{\Theta}_1, ... \bm{\Theta}_J \}$,
% $$
%   % b_j^{[m]}(\xv,\pmb\theta^{[m]}) \quad j = 1,\dots, J\,,
%   \{ \bmm_j(\xv, \thetam): j = 1, 2, \dots, J \},
% $$
%
 where $j \in \{ 1, 2, \dots, J \}$ indexes the type of base learner.

\vspace*{0.2cm}
%%% Figures created by: rsrc/fig-compboost-blpools.R
\[
\left\{
\begin{array}{c}
\includegraphics[width=2cm]{figure/boosting-cwb-blpool1.png},
\includegraphics[width=2cm]{figure/boosting-cwb-blpool2.png},
\cdots, 
\includegraphics[width=2cm]{figure/boosting-cwb-blpool3.png},
\includegraphics[width=2cm]{figure/boosting-cwb-blpool4.png}
\end{array}
\right\}
\]
\vspace*{0.2cm}

%Different from GB, in each iteration multiple base learners 
%$b_j\in\mathcal{B}_j$, $j = 1, \dots, J$, are fit to the 
%pseudo residuals $\rmm$ and only best-fitting one
% $\bmm_{\hat{j}}(\xv, \thetam)$
%is selected and updated.
In each iteration, base learners are fitted to the \textbf{pseudo residuals} $\rmm$.


\framebreak

%Common examples of base learners are
We restrict the base learners to additive model components, i.e.,
\begin{minipage}{0.4\textwidth}
    \includegraphics[width=\linewidth]{figure/compboost-base-learner-linear.png}
\end{minipage}\hfill
\begin{minipage}{0.5\textwidth}
  linear effect
\end{minipage}

\begin{minipage}{0.4\textwidth}
    \includegraphics[width=\linewidth]{figure/compboost-base-learner-spline.png}
\end{minipage}\hfill
\begin{minipage}{0.5\textwidth}
  non-linear (spline) effect
\end{minipage}

\begin{minipage}{0.4\textwidth}
    \includegraphics[width=\linewidth]{figure/compboost-base-learner-ridge.png}
\end{minipage}\hfill
\begin{minipage}{0.5\textwidth}
  dummy encoded linear model of a cat. feature
\end{minipage}

\begin{minipage}{0.4\textwidth}
    \includegraphics[width=\linewidth]{figure/compboost-base-learner-tensor.png}
\end{minipage}\hfill
\begin{minipage}{0.5\textwidth}
  tensor product spline for interaction modelling (e.g. spatial effects)
\end{minipage}

\vspace{\baselineskip}

More advanced base learners could also be  Markov random fields, random effects, or trees.

\framebreak

%We restrict these base learners to additive models, i.e., 
Two BLs of the same type can simply be added by adding up their parameter vectors:

$$
 b_j(\xv, \thetab^{[1]}) + b_j(\xv, \thetab^{[2]}) =
 b_j(\xv, \thetab^{[1]} + \thetab^{[2]}).
$$
%% Figure: rsrc/fig-compboost-add.R
\begin{center}
\begin{minipage}{.25\linewidth}
\includegraphics{figure/boosting-cwb-bl-add1.pdf}
\end{minipage}
\begin{minipage}{.05\linewidth}
\vspace*{-0.3cm}
$\bm{+}$
\end{minipage}
\begin{minipage}{.25\linewidth}
\includegraphics{figure/boosting-cwb-bl-add2.pdf} 
\end{minipage}
\begin{minipage}{.05\linewidth}
\vspace*{-0.3cm}
$\bm{=}$
\end{minipage}
\begin{minipage}{.25\linewidth}
\includegraphics{figure/boosting-cwb-bl-add3.pdf}
\end{minipage}
\end{center}
\vspace*{0.1cm}

Thus, if $\{ b_j(\xv, \thetab^{[1]}), b_j(\xv, \thetab^{[2]}) \} \in \mathcal{B}_j$, then $b_j(\xv, \thetab^{[1]} + \thetab^{[2]}) \in \mathcal{B}_j$.

\end{vbframe}



% ------------------------------------------------------------------------------

\begin{vbframe}{Componentwise boosting algorithm}

\footnotesize{Different from GB, multiple base learners 
$b_j\in\mathcal{B}_j$, $j = 1, \dots, J$, are fitted %to the pseudo residuals $\rmm$ 
and only best-fitting one
% $\bmm_{\hat{j}}(\xv, \thetam)$
is selected and updated.}
\vspace{-0.3cm}
\definecolor{algocol}{rgb}{0, 0.125, 0.376}

\begin{algorithm}[H]
  \begin{footnotesize}
  \begin{center}
  \caption{Componentwise Gradient Boosting.}\color{gray}
    \begin{algorithmic}[1]
      \State Initialize $\fm[0](\xv) = \argmin_{\theta_0\in\R} \sum  \limits_{i=1}^n L(\yi, \theta_0)$
      \For{$m = 1 \to M$}
        \State For all $i$: $\rmi = -\left[\pd{L(y, f)}{f}\right]_{f=\fmd(\xi),y=\yi}$
        \color{algocol}
        \For {$j= 1\to J$}
          \State Fit regression base learner $b_j \in \mathcal{B}_j$ to the vector of pseudo-residuals $\rmm$:
          \State $\thetamh_j = \argmin_{\thetav \in \bm{\Theta_j}} \sum  \limits_{i=1}^n
          (\rmi - b_j(\xi, \thetav))^2$
        \EndFor
        \State $j^{[m]} = \argmin_{j} \sum  \limits_{i=1}^n (\rmi - \hat{b}_j(\xi, \thetamh_j))^2$
        \color{lightgray}
        \State Update $\fm(\xv) = \fmd(\xv) + \alpha \hat{b}_{\hat{j}}(\xv, \thetamh_{j^{[m]}})$
      \EndFor
      \State Output $\fh(\xv) = \fm[M](\xv)$
    \end{algorithmic}
    \end{center}
    \end{footnotesize}
    \color{black}
\end{algorithm}

\vspace{-0.3cm}
\scriptsize{({\color{lightgray} Same as for GB, \color{algocol} New inner loop for CWB})}

\end{vbframe}

% ------------------------------------------------------------------------------

\begin{vbframe}{Componentwise boosting algorithm}

\begin{algorithm}[H]
  \begin{footnotesize}
  \begin{center}
  \caption{Componentwise Gradient Boosting (inner loop).}
    \begin{algorithmic}[1]
        \setcounter{ALG@line}{3}

        \For {$j= 1\to J$}
          \State Fit regression base learner $b_j \in \mathcal{B}_j$ to the vector of pseudo-residuals $\rmm$:
          \State $\thetamh_j = \argmin_{\thetab \in \bm{\Theta_j}} \sum  \limits_{i=1}^n
          (\rmi - b_j(\xi, \thetab))^2$
        \EndFor
        \State $j^{[m]} = \argmin_{j} \sum  \limits_{i=1}^n (\rmi - \hat{b}_j(\xi, \thetamh_j))^2$

    \end{algorithmic}
    \end{center}
    \end{footnotesize}
    \color{black}
\end{algorithm}

{\footnotesize Iteration $m$, $j = 1$, $\sum  \limits_{i=1}^n (\rmi - \hat{b}_1(x^{(i)}_1, \thetamh_1))^2 = 24.4$: \phantom{$\Rightarrow$ $j^{[m]} = 1$}}
%% Figures: rsrc/fig-compboost-blpools.R
\begin{center}
\includegraphics[width=0.25\textwidth]{figure/boosting-cwb-bl1-points.png}
\hspace*{0.5cm}
{\transparent{0.3}\includegraphics[width=0.25\textwidth]{figure/boosting-cwb-bl2-points.png}}
\hspace*{0.5cm}
{\transparent{0.3}\includegraphics[width=0.25\textwidth]{figure/boosting-cwb-bl3-points.png}}
\end{center}
\end{vbframe}

% ------------------------------------------------------------------------------

\begin{vbframe}{Componentwise boosting algorithm}

\begin{algorithm}[H]
  \begin{footnotesize}
  \begin{center}
  \caption{Componentwise Gradient Boosting (inner loop).}
    \begin{algorithmic}[1]
        \setcounter{ALG@line}{3}

        \For {$j= 1\to J$}
          \State Fit regression base learner $b_j \in \mathcal{B}_j$ to the vector of pseudo-residuals $\rmm$:
          \State $\thetamh_j = \argmin_{\thetab \in \bm{\Theta_j}} \sum  \limits_{i=1}^n
          (\rmi - b_j(\xi, \thetab))^2$
        \EndFor
        \State $j^{[m]} = \argmin_{j} \sum  \limits_{i=1}^n (\rmi - \hat{b}_j(\xi, \thetamh_j))^2$

    \end{algorithmic}
    \end{center}
    \end{footnotesize}
    \color{black}
\end{algorithm}

{\footnotesize Iteration $m$, $j = 2$, $\sum  \limits_{i=1}^n (\rmi - \hat{b}_2(x^{(i)}_2, \thetamh_2))^2 = 43.2$: \phantom{$\Rightarrow$ $j^{[m]} = 1$}}
\begin{center}
{\transparent{0.3}\includegraphics[width=0.25\textwidth]{figure/boosting-cwb-bl1-points.png}}
\hspace*{0.5cm}
\includegraphics[width=0.25\textwidth]{figure/boosting-cwb-bl2-points.png}
\hspace*{0.5cm}
{\transparent{0.3}\includegraphics[width=0.25\textwidth]{figure/boosting-cwb-bl3-points.png}}
\end{center}
\addtocounter{framenumber}{-1}
\end{vbframe}

% ------------------------------------------------------------------------------

\begin{vbframe}{Componentwise boosting algorithm}

\begin{algorithm}[H]
  \begin{footnotesize}
  \begin{center}
  \caption{Componentwise Gradient Boosting (inner loop).}
    \begin{algorithmic}[1]
        \setcounter{ALG@line}{3}

        \For {$j= 1\to J$}
          \State Fit regression base learner $b_j \in \mathcal{B}_j$ to the vector of pseudo-residuals $\rmm$:
          \State $\thetamh_j = \argmin_{\thetab \in \bm{\Theta_j}} \sum  \limits_{i=1}^n
          (\rmi - b_j(\xi, \thetab))^2$
        \EndFor
        \State $j^{[m]} = \argmin_{j} \sum  \limits_{i=1}^n (\rmi - \hat{b}_j(\xi, \thetamh_j))^2$

    \end{algorithmic}
    \end{center}
    \end{footnotesize}
    \color{black}
\end{algorithm}

{\footnotesize Iteration $m$, $j = 3$, $\sum  \limits_{i=1}^n (\rmi - \hat{b}_3(x^{(i)}_3, \thetamh_3))^2 = 35.2$: \phantom{$\Rightarrow$ $j^{[m]} = 1$}}
\begin{center}
{\transparent{0.3}\includegraphics[width=0.25\textwidth]{figure/boosting-cwb-bl1-points.png}}
\hspace*{0.5cm}
{\transparent{0.3}\includegraphics[width=0.25\textwidth]{figure/boosting-cwb-bl2-points.png}} 
\hspace*{0.5cm}
\includegraphics[width=0.25\textwidth]{figure/boosting-cwb-bl3-points.png}
\end{center}
\addtocounter{framenumber}{-1}
\end{vbframe}

% ------------------------------------------------------------------------------

\begin{vbframe}{Componentwise boosting algorithm}

\begin{algorithm}[H]
  \begin{footnotesize}
  \begin{center}
  \caption{Componentwise Gradient Boosting (inner loop).}
    \begin{algorithmic}[1]
        \setcounter{ALG@line}{3}

        \For {$j= 1\to J$}
          \State Fit regression base learner $b_j \in \mathcal{B}_j$ to the vector of pseudo-residuals $\rmm$:
          \State $\thetamh_j = \argmin_{\thetab \in \bm{\Theta_j}} \sum  \limits_{i=1}^n
          (\rmi - b_j(\xi, \thetab))^2$
        \EndFor
        \State $j^{[m]} = \argmin_{j} \sum  \limits_{i=1}^n (\rmi - \hat{b}_j(\xi, \thetamh_j))^2$

    \end{algorithmic}
    \end{center}
    \end{footnotesize}
    \color{black}
\end{algorithm}

{\footnotesize Iteration $m$: $\Rightarrow$ $j^{[m]} = 1$}
\begin{center}
\includegraphics[width=0.25\textwidth]{figure/boosting-cwb-bl1-points.png}
\hspace*{0.5cm}
\includegraphics[width=0.25\textwidth]{figure/boosting-cwb-bl2-points.png} 
\hspace*{0.5cm}
\includegraphics[width=0.25\textwidth]{figure/boosting-cwb-bl3-points.png}
\end{center}
\addtocounter{framenumber}{-1}
\end{vbframe}
% ------------------------------------------------------------------------------

\begin{vbframe}{Feature selection in CWB}
In CWB, we often define BLs on a single feature

$$
  b_j(x_j, \theta) \quad \text{for } j = 1, 2, \dots, p.
$$
Allows natural form of feature selection:
\begin{itemize}
    \item When we select the best BL in one iter of CWB, we thereby also only select one (associated) feature
    \item Note that a feature (or rather a BL associated with it) can be selected in multiple iters, so $\leq M$ features are selected
    %\item Due to the iterative nature of CWB, this is repeated $M$ times to obtain a selection of $M$ base learners by $j^{[1]}, \dots, j^{[m]}$ with $j^{[M]}\in\{1, \dots, p\}$
    %\item The number $\sum_{m=1}^M \mathds{1}_{[j = j^{[m]}]}$ tells us how often the feature $x_j$ in $b_j$ was selected
\end{itemize}
\vspace{0.5cm}
\begin{center}
%\begin{minipage}{0.3\textwidth}
%    \includegraphics[width=\textwidth]{figure/compboost-base-learner-linear.png}
%\end{minipage}\hfill
\begin{minipage}{0.5\textwidth}
    \includegraphics[width=\textwidth]{figure/compboost-base-learner-spline.png}
\end{minipage}\hfill
\begin{minipage}{0.5\textwidth}
    \includegraphics[width=\textwidth]{figure/compboost-base-learner-ridge.png}
\end{minipage}
\end{center}


%\begin{minipage}{0.4\textwidth}
%    \includegraphics[width=\linewidth]{figure/compboost-base-learner-linear.png}
%\end{minipage}%\hfill
%\begin{minipage}{0.5\textwidth}
%  linear effect
%\end{minipage}

%\begin{minipage}{0.4\textwidth}
%    \includegraphics[width=\linewidth]{figure/compboost-base-learner-spline.png}
%\end{minipage}%\hfill
%\begin{minipage}{0.5\textwidth}
%  non-linear (spline) effect
%\end{minipage}

%The iterative nature of CWB and the additivity of the base learners 
%directly incorporates a variable selection mechanism into the fitting
%process, since in each iteration only the best base learner is selected in
%combination with the associated feature, and each base learner can be
%(substantially) more complex than a stump (e.g., univariate linear effects or 
%splines).
\end{vbframe}


\endlecture
\end{document}
