\documentclass[11pt,compress,t,notes=noshow, xcolor=table]{beamer}
\input{../../style/preamble}
\input{../../latex-math/basic-math}
\input{../../latex-math/basic-ml}
\input{../../latex-math/ml-ensembles.tex}
\input{../../latex-math/ml-trees.tex}

\newcommand{\titlefigure}{figure/compboost-illustration-2.png}
\newcommand{\learninggoals}{
  \item Concept of of CWB and relation to GLM
  \item Built-in feature selection process
  \item Fair base learner selection
}

\title{Introduction to Machine Learning}
\date{}

\begin{document}

\lecturechapter{Componentwise Gradient Boosting}
\lecture{Introduction to Machine Learning}

% ------------------------------------------------------------------------------

\begin{vbframe}{Componentwise gradient boosting}

Gradient boosting, especially when using trees, has strong predictive
performance but is difficult to interpret unless the base learners are stumps.

\lz

The aim of componentwise gradient boosting (CWB) is to find a model that:

\begin{itemize}
  \item
    has strong predictive performance,

  \item
    has components that are still interpretable,

  \item
    does automatic selection of components,

  \item
    is sparser than a model fitted with maximum-likelihood estimation.
\end{itemize}

\lz

This is achieved by using \enquote{nice} base learners which yield familiar
statistical models
in the end.

\lz

Because of this, CWB is also often referred to as \textbf{model-based boosting}.

\end{vbframe}

% ------------------------------------------------------------------------------

\begin{vbframe}{Base learners}

In classical gradient boosting only one kind of base learner $\mathcal{B}$ is used, e.g.,
regression trees.

\lz

For CWB we generalize this to multiple base learner sets $\{ \mathcal{B}_1, ... \mathcal{B}_J \}$ with associated parameter spaces
$\{ \bm{\Theta}_1, ... \bm{\Theta}_J \}$,
% $$
%   % b_j^{[m]}(\xv,\pmb\theta^{[m]}) \quad j = 1,\dots, J\,,
%   \{ \bmm_j(\xv, \thetam): j = 1, 2, \dots, J \},
% $$
%
 where $j \in \{ 1, 2, \dots, J \}$ indexes the type of base learner.
%
\lz

Again, in each iteration, only the best base learner
% $\bmm_{\hat{j}}(\xv, \thetam)$
is selected and updated.

\framebreak

Common examples of base learners are

\begin{minipage}{0.4\textwidth}
    \includegraphics[width=\linewidth]{figure/compboost-base-learner-linear.png}
\end{minipage}\hfill
\begin{minipage}{0.5\textwidth}
  linear effect
\end{minipage}

\begin{minipage}{0.4\textwidth}
    \includegraphics[width=\linewidth]{figure/compboost-base-learner-spline.png}
\end{minipage}\hfill
\begin{minipage}{0.5\textwidth}
  non-linear (spline) effect
\end{minipage}

\begin{minipage}{0.4\textwidth}
    \includegraphics[width=\linewidth]{figure/compboost-base-learner-ridge.png}
\end{minipage}\hfill
\begin{minipage}{0.5\textwidth}
  dummy encoded linear model of a categorical feature
\end{minipage}

\begin{minipage}{0.4\textwidth}
    \includegraphics[width=\linewidth]{figure/compboost-base-learner-tensor.png}
\end{minipage}\hfill
\begin{minipage}{0.5\textwidth}
  product tensor of two base learners for interaction modelling (e.g. two splines)
\end{minipage}

\vspace{\baselineskip}

More advanced base learners could also be  Markov random fields, random effects, or trees.

\end{vbframe}
% ------------------------------------------------------------------------------

\begin{vbframe}{Componentwise boosting algorithm}


\begin{algorithm}[H]
  \begin{footnotesize}
  \begin{center}
  \caption{Componentwise Gradient Boosting.}\color{gray}
    \begin{algorithmic}[1]
      \State Initialize $\fm[0](\xv) = \argmin_{\theta_0\in\R} \sum  \limits_{i=1}^n L(\yi, \theta_0)$
      \For{$m = 1 \to M$}
        \State For all $i$: $\rmi = -\left[\pd{L(y, f)}{f}\right]_{f=\fmd(\xi),y=\yi}$
        \color{algocol}
        \For {$j= 1\to J$}
          \State Fit regression base learner $b_j \in \mathcal{B}_j$ to the vector of pseudo-residuals $\rmm$:
          \State $\thetamh_j = \argmin_{\thetav \in \bm{\Theta_j}} \sum  \limits_{i=1}^n
          (\rmi - b_j(\xi, \thetav))^2$
        \EndFor
        \State $j^{[m]} = \argmin_{j} \sum  \limits_{i=1}^n (\rmi - \hat{b}_j(\xi, \thetamh_j))^2$
        \color{lightgray}
        \State Update $\fm(\xv) = \fmd(\xv) + \alpha \hat{b}_{\hat{j}}(\xv, \thetamh_{j^{[m]}})$
      \EndFor
      \State Output $\fh(\xv) = \fm[M](\xv)$
    \end{algorithmic}
    \end{center}
    \end{footnotesize}
    \color{black}
\end{algorithm}


\end{vbframe}
% ------------------------------------------------------------------------------
\begin{vbframe}{Additive model structure}
\begin{footnotesize}
We restrict these base learners to additive models, i.e., having a base learner $ b_j(\xv, \thetab^{[1]})$ and another base learner $b_j$ of the same type but with a different parameter vector $\thetab^{[2]}$, then it is possible to combine them to a new base learner of the same type:

$$
 b_j(\xv, \thetab^{[1]}) + b_j(\xv, \thetab^{[2]}) =
 b_j(\xv, \thetab^{[1]} + \thetab^{[2]}).
$$
\vspace*{0.1cm}

Thus, if $\{ b_j(\xv, \thetab^{[1]}), b_j(\xv, \thetab^{[2]}) \} \in \mathcal{B}_j$, then $b_j(\xv, \thetab^{[1]} + \thetab^{[2]}) \in \mathcal{B}_j$.
\lz

Often base learners are not defined on the entire feature vector $\xv$ but on
a single feature $x_j$:

$$
  b_j(x_j, \theta) \quad \text{for } j = 1, 2, \dots, p.
$$

This directly incorporates a variable selection mechanism into the fitting
process, since in each iteration only the best base learner is selected in
combination with the associated feature, and each base learner can be
(substantially) more complex than a stump (e.g., univariate linear effects or
splines).
\end{footnotesize}
\end{vbframe}


% ------------------------------------------------------------------------------

\begin{vbframe}{intercept handling}

\begin{itemize}
  \item CWB is initialized with a loss-optimal constant $\fm[0](\xv)$ as initial model intercept.
  \item An intercept is often referred to as part of a model which contains information independent of the features.
  \item Suppose linear base learners $b_j(\xv) = \theta_{j1} + \theta_{j2} x_j$ with data independent intercept $\theta_{j1}$ and slope $\theta_{j2}$.
  \item Adding base learner $\hat{b}_j$ in iteration $m$ with parameter estimates $\thetamh = (\hat{\theta}_{j1}^{[m]}, \hat{\theta}_{j1}^{[m]})$ consequently updates the intercept to $\fm[0](\xv) + \hat{\theta}_{j1}^{[m]}$.
  \item Throughout the fitting process, the intercept is adjusted $M$ times to its final form:
    $$
    \fm[0](\xv) + \sum\limits_{m=1}^M \hat{\theta}^{[m]}_{j^{[m]}1}
    $$
\end{itemize}

% ------------------------------------------------------------------------------

\framebreak

Two possible options to handle the intercept in CWB are:

\begin{itemize}

\item Include an intercept base learner:
  \begin{itemize}
    \item Add base learner $b_{\text{int}} = \theta$ as potential candidate considered in each iteration.
    \item At the same time, remove the intercept from all linear base learners to only use $b_j(\xv) = \theta_j x_j$.
    \item The final intercept is given by $\fm[0](\xv) + \hat{\theta}$.
  \end{itemize}
  \item Include an intercept in each linear base learner $b_j(\xv) = \theta_{j1} + \theta_{j2} x_j$ and accumulate all intercepts to one global intercept after the fitting.

\end{itemize}

% ------------------------------------------------------------------------------

\framebreak

The following figures shows a comparison of the parameter updates with a different intercept handling:
\begin{center}
\includegraphics[width = \textwidth]{figure/compboost-intercept-handling.png}
\end{center}

The used data set is \href{https://github.com/topepo/AmesHousing}{Ames Housing}.


\end{vbframe}

% ------------------------------------------------------------------------------

\begin{vbframe}{handling of categorical features}

\textcolor{red}{@BB}

\begin{itemize}
  \item Basically, there are two options for encoding categorical features:
  \begin{itemize}
    \item Specifying a single base learner for the entire variable
    \item One-hot encoding, i.e., creating a binary feature with associated
    base learner for each category
  \end{itemize}
  \item The \texttt{compboost} package currently implements the latter variant.
\end{itemize}

\end{vbframe}

% ------------------------------------------------------------------------------

\begin{vbframe}{example: boston housing}

\begin{minipage}[c]{0.4\textwidth}
  \small
  \raggedright
  Consider the \texttt{Boston housing} regression task, for which we fit a
  CWB model with linear base learners (with intercept) to predict median home
  value.
  Using \texttt{compboost} with $M = 100$ iterations, we can
  visualize which base learner was selected when:
\end{minipage}%
\begin{minipage}[c]{0.05\textwidth}
  \phantom{foo}
\end{minipage}%
\begin{minipage}[c]{0.55\textwidth}
  \tiny
  \begin{tabular}{l|l}
    \textbf{variable} & \textbf{description} \\
    \hline
    medv &	median value of owner-occupied homes in k USD \\
    \hline
    crim & per capita crime rate by town \\
    zn &	proportion of residential land zoned for lots $>$ 25k sq.ft \\
    indus &	proportion of non-retail business acres per town \\
    chas &	Charles River dummy (1 if tract bounds river, 0 otherwise) \\
    nox &	nitric oxides concentration (parts per 10m) \\
    rm &	average number of rooms per dwelling \\
    age &	proportion of owner-occupied units built prior to 1940 \\
    dis &	weighted distances to five Boston employment centres \\
    rad &	index of accessibility to radial highways \\
    tax &	full-value property-tax rate per USD 10k \\
    ptratio &	pupil-teacher ratio by town \\
    b &	$1000(B - 0.63)^2$,  $B$ as proportion of blacks by town \\
    lstat &	percentage of lower status of the population \\
  \end{tabular}
\end{minipage}%

\vfill

\begin{center}
\includegraphics[width = \textwidth]{figure/compboost-illustration-1.png}
\end{center}

\framebreak

% ------------------------------------------------------------------------------

The number of features effectively included in the final model depends on the
number of total iterations $M$.

\vfill

$\rightarrow$ A sparse linear regression is fitted.

\vfill

\includegraphics[width = \textwidth]{figure/compboost-illustration-2.png}

\end{vbframe}

\endlecture
\end{document}
