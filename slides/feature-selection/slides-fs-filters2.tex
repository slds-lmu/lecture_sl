
\documentclass[11pt,compress,t,notes=noshow, xcolor=table]{beamer}
\input{../../style/preamble}
\input{../../latex-math/basic-math}
\input{../../latex-math/basic-ml}
\input{../../latex-math/ml-feature-sel}

\title{Introduction to Machine Learning}

\begin{document}
    
\titlemeta{% Chunk title (example: CART, Forests, Boosting, ...), can be empty
    Feature Selection
  }{% Lecture title  
    Feature Selection: Filter Methods (Examples and Caveats)
  }{% Relative path to title page image: Can be empty but must not start with slides/
  figure/guyon_example_correlation.png
  }{
  \item Understand how filter methods can be misleading
  \item Understand how filters can be applied and tuned
}

%  \begin{vbframe}{Introduction}
%  \vspace{0.3cm}
%  \begin{itemize}
%  \setlength{\itemsep}{1.4em}
%    \item \textbf{Filter methods} construct a measure that quantifies the dependency between all features and the target variable.
%    \item They yield a numerical score for each feature $x_j$, according to which we rank the features.
%    \item They are model-agnostic and can be applied generically.
%    \item FM is strongly related to methods for variable importance.
%  \end{itemize}
%\vspace{-0.2cm}
%  \begin{center}
%  \begin{figure}
%    \includegraphics[width=0.48\textwidth]{figure/filter_comparison_har_classif.kknn.png}
%  \end{figure}
%  \end{center}

% \end{vbframe}

%\begin{vbframe}{Visualization of filter algorithms}
%
%
%\end{vbframe}

\begin{comment}

  \begin{vbframe}{Minimum Redundancy Maximum Relevancy}
  \begin{itemize}
    \setlength{\itemsep}{2em}
    \item Most filter-type methods rank features based on a univariate association score without considering relationships among the features.
    \begin{itemize}
    \setlength{\itemsep}{1.5em}
      \item Features may be correlated and hence, may cause redundancy.
      \item Selected features cover narrow regions in space.
    \end{itemize}
    \item We want the features to be relevant and maximally dissimilar to each other (minimum redundancy).
    \item Features can be either continuous or categorical.
  \end{itemize}
  \end{vbframe}

  \begin{vbframe}{\MakeLowercase{M}RMR: Criterion functions}
  \begin{itemize}
  \item Let $S \subset \{1, \dots, p \}$ be a subset of features we want to find.

  $$\min \text{Red}(S), \;\;\;\; \text{Red}(S) = \frac{1}{|S|^2} \sum_{j, l \in S} I_{xx}(x_j, x_l)$$

  $$\max \text{Rel}(S), \;\;\;\; \text{Rel}(S) = \frac{1}{|S|} \sum_{j \in S} I_{xy}(x_j, \ydat)$$

  \item $I_{xx}$ measures the strength of the dependency between two features.
  \item $I_{xy}$ measures the strength of the dependency between a feature and the target.
  \item They could be mutual information, correlation, F-statistic, etc.
  \end{itemize}


  \framebreak

  \begin{itemize}
  \item To optimize simultainously, the criteria is combined into a single objective function:
  $$\Psi(S) = (\text{Rel}(S) - \text{Red}(S)) \;\;\;\; \text{ or } \;\;\;\; \Psi(S) = (\text{Rel}(S)/\text{Red}(S))$$

  \item Exact solution requires $ \order (|\Xspace|^{|S|})$ searches, where $|\Xspace|$ is the number of features and $|S|$ is the number of selected features.
  \end{itemize}
  In practice, incremental search methods are used to find near-optimal feature sets defined by $\Psi$:
  \begin{itemize}
  \item Suppose we already have a feature set with $m-1$ features $S_{m-1}$.
  \item Next, we select the $m$-th feature from the set $\bar{S}_{m-1}$ by selecting the feature that maximizes:
  $$\max_{j \, \in \, \bar{S}_{m-1}} [ I_{xy} (x_j, \ydat) - \frac{1}{|S_{m-1}|} \sum_{l \, \in \, S_{m-1}} I_{xx}(x_j, x_l)  ]$$
  \item The complexity of this incremental algorithm is $\mathcal{O}(|p| \cdot| S|)$.
  \end{itemize}
  \end{vbframe}

  \begin{vbframe}{\MakeLowercase{m}RMR: Algorithm}

  \begin{algorithm}[H]
  \footnotesize
    \begin{algorithmic}[1]
      \State Set $S = \emptyset$, $R = \{ 1, \dots, p \}$
      \State Find the feature with maximum relevancy:
      $$j^* := \argmax_{j} I_{xy} (x_j, \ydat)$$
      \State Set $S = \{ j^* \}$ and update $R \leftarrow R \setminus \{j^* \}$
      \Repeat
        \State Find feature $x_j$ that maximizes:
        $$\max_{j \, \in \, R} [ I_{xy} (x_j, \ydat) - \frac{1}{|S|} \sum_{l \, \in \, S} I_{xx}(x_j, x_l)  ]$$
        \State Update $S \leftarrow S \cup \{j^* \}$ and $R \leftarrow R \setminus \{ j^* \}$
      \Until{Expected number of features have been obtained or some other constraints are satisfied.}
      \caption{mRMR algorithm}
    \end{algorithmic}
  \end{algorithm}
  \end{vbframe}
\end{comment}

  \begin{vbframe}{Filter methods can be misleading}

   %%%%%%%%%%%%%%%%%%%
   \begin{figure}
    \centering
    \begin{minipage}{0.45\textwidth}
        \centering
        \includegraphics[width=\textwidth]{figure/guyon_example_presumably_redundant.png} % first figure itself
        %\caption{first figure}
    \end{minipage}\hfill
    \begin{minipage}{0.45\textwidth}
        $\rho_{ACC}$ of log. reg. classifier with:
        \begin{itemize}
        \item{feature $x_1$: 0.76}
        \item{feature $x_2$: 0.78}
        \item{both features: 0.85}
        \end{itemize}
        %\centering
        %\includegraphics[width=\textwidth]{figure/guyon_example_presumably_redundant_rotated.png} % second figure itself
        %\caption{second figure}
    \end{minipage}
    \end{figure}
\vspace{0.3cm}
\footnotesize{\textbf{Information gain from presumably redundant variables}. 2 class problem with indep features. Each class has Gaussian distribution with no covariance. While filter methods suggest redundancy, combination of both vars yields improvement, showing indep vars are not truly redundant. %Right: After 45 degree rotation, showing combination of 2 vars yields separation improvement by factor $\sqrt{2}$, showing i.i.d. vars are not truly redundant. 
For further details, see \citelink{guyon2003introduction}.}
%\footnotesize{\textbf{IG from presumably redundant variables}. Left: 2 class problem with i.i.d. variables. Each class has Gaussian distr. with no covariance. Right: After 45 degree rotation, showing combination of 2 vars yields separation improvement by factor $\sqrt{2}$, showing i.i.d. vars are not truly redundant. For further details, see Guyon and Elisseeff, 2003.}
   %\footnotesize{Isabelle Guyon, André Elisseeff (2003). An Introduction to Variable and Feature Selection.  Journal of Machine Learning Research (3) p. 1157-1182.}

   \framebreak

  %%%%%%%%%%%%%%%%%%%%%%%%
     \begin{figure}
    \centering
    \begin{minipage}{0.45\textwidth}
        \centering
        \includegraphics[width=\textwidth]{figure/guyon_example_intra_class_covariance.png} % first figure itself
        %\caption{first figure}
    \end{minipage}\hfill
    \begin{minipage}{0.45\textwidth}
        \centering
        \includegraphics[width=\textwidth]{figure/guyon_example_intra_class_covariance_perpendicular.png} % second figure itself
        %\caption{second figure}
    \end{minipage}
    \end{figure}
\vspace{0.3cm}
\footnotesize{\textbf{Intra-class covariance}. In projection onto axes, distr. of two variables are same as before. Left: Class conditional distr. have high cov. in direction of the line of two class centers. Right: Class conditional distr. have high cov. in direction perpendicular to line of two class centers. Better separation by using both vars.}

 \framebreak


    %%%%%%%%%%%%%%%%%%%%%%%%%
      %%%%%%%%%%%%%%%%%%%%%%%%
     \begin{figure}
    \centering
    \begin{minipage}{0.45\textwidth}
        \centering
        \includegraphics[width=\textwidth]{figure/guyon_example_correlation.png} % first figure itself
        %\caption{first figure}
    \end{minipage}\hfill
    \begin{minipage}{0.45\textwidth}
        \centering
        \includegraphics[width=\textwidth]{figure/guyon_example_xor.png} % second figure itself
        %\caption{second figure}
    \end{minipage}
    \end{figure}
\vspace{0.3cm}
\footnotesize{\textbf{Variable useless by itself can be useful together with others}. Left: One var has completely overlapping class conditional densities. Still, jointly with other variable separability can be improved. Right: XOR-like chessboard problem. Classes consist of ``clumps" s.t. projection on the axes yields overlapping densities. Single vars have no separation power, only used together.}


\end{vbframe}

\begin{vbframe}{Using Filter Methods}

% \begin{minipage}{0.45\textwidth}
%     \centering
%     \includegraphics[width=\textwidth]{figure/guyon_example_xor.png} % second figure itself
%     %\caption{second figure}
% \end{minipage}


  \begin{columns}
    \begin{column}{0.65\textwidth}
      \begin{enumerate}{}
        \setlength{\itemsep}{1.2em}
          \item Calculate filter score for each feature $x_j$
          \item Rank features according to score values
          \item Choose $\tilde{p}$ best features
          \item Train model on $\tilde{p}$ best features
        \end{enumerate}

        \begin{blocki}{How to choose $\tilde{p}$?}
          \item Could be prescribed by application
          \item Eyeball estimation: read from filter plots
          \item Treat as hyperparameter and tune in a pipeline, based on resampling
        \end{blocki}
    \end{column}

   \begin{column}{0.46\textwidth}
      %\vspace{1cm}
      \begin{figure}
        \centering
        \includegraphics[width=0.89\textwidth]{figure/filter_comparison_har_classif.kknn.png}
      \end{figure}
      \vspace{-0.4cm}
      \begin{figure}
      \includegraphics[width=0.89\textwidth]{figure/fs-filters-scree-plot.png}
      \end{figure}
    \end{column}

  \end{columns}



  \framebreak

  \begin{blocki}{Advantages:}
  \setlength{\itemsep}{1.2em}
    \item Easy to calculate
    \item Typically scales well with the number of features $p$
    \item Generally interpretable
    \item Model-agnostic
  \end{blocki}

  \begin{blocki}{Disadvantages:}
  \setlength{\itemsep}{1.2em}
    \item Univariate analyses may ignore multivariate dependencies
    \item Redundant features will have similar weights
    \item Ignores the learning algorithm
  \end{blocki}
  \end{vbframe}


  % \begin{vbframe}{Filter: Practical example}
  % <<echo=TRUE>>=
  % # Calculate the information gain as filter value
  % fv = generateFilterValuesData(iris.task, method = "information.gain")

  % # We can also pass multiple filter methods at once:
  % fv2 = generateFilterValuesData(iris.task,
  %   method = c("information.gain", "chi.squared"))

  % # fv2 is a FilterValues object and fv2$data holds a data frame
  % # with the importance values for all features
  % fv2$data
  % @

  % <<>>=
  % plotFilterValues(fv2) + ggtitle("Filters for the 4 features of iris dataset")
  % @


  % \framebreak

  % <<size="tiny", echo=TRUE>>=
  % ## Keep the 2 most important features
  % filtered.task = filterFeatures(iris.task, method = "information.gain", abs = 2L)

  % ## Keep the 25% most important features
  % filtered.task = filterFeatures(iris.task, fval = fv, perc = 0.25)

  % ## Keep all features with importance greater than 0.5
  % filtered.task = filterFeatures(iris.task, fval = fv, threshold = 0.5)
  % filtered.task
  % @

  % \framebreak

  % \begin{itemize}
  %   \item Filter methods are often part of the data preprocessing process, wherein a subsequent step, a learning method is applied.
  %   \item In practice we want to automate this, so the feature selection can be part of the validation method of our choice:
  % \end{itemize}

  % <<size="tiny", echo=TRUE>>=
  % lrn = makeFilterWrapper(learner = "classif.kknn",
  %   fw.method = "anova.test", fw.abs = 2L)
  % ps = makeParamSet(
  %   makeDiscreteParam("fw.abs", values = 1:4)
  % )
  % ctrl = makeTuneControlGrid()
  % rdesc = makeResampleDesc("CV", iters = 10L)

  % res = tuneParams(lrn, task = iris.task, resampling = rdesc, par.set = ps,
  %   control = ctrl)
  % res$opt.path
  % @
  % \end{vbframe}

  \endlecture
\end{document}

