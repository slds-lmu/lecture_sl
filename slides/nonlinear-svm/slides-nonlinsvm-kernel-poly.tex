\documentclass[11pt,compress,t,notes=noshow, xcolor=table]{beamer}
\input{../../style/preamble}
\input{../../latex-math/basic-math}
\input{../../latex-math/basic-ml}
\input{../../latex-math/ml-svm}

\title{Introduction to Machine Learning}

\begin{document}

\titlemeta{% Chunk title (example: CART, Forests, Boosting, ...), can be empty
    Nonlinear Support Vector Machines
  }{% Lecture title  
    The Polynomial Kernel
  }{% Relative path to title page image: Can be empty but must not start with slides/
  figure/svm_poly_kernel_deg_9_coef0_1.png
  }{
  \item Know the homogeneous and non-homogeneous polynomial kernel 
  \item Understand the influence of the choice of the degree on the decision boundary
}

\begin{vbframe}{Homogeneous Polynomial Kernel}
$$ k(\xv, \xtil) = (\xv^T \xtil)^d, \text{ for } d \in \N$$
The feature map contains all monomials of exactly order $d$.
  $$\phix = \left( \sqrt{\mat{d \\ k_1, \ldots, k_p}} x_1^{k_1} \ldots x_p^{k_p}\right)_{k_i \geq 0, \sum_i k_i = d}$$
That $\scp{\phix}{\phixt} = k(\xv, \xtil)$ holds can easily be checked by simple calculation
and using the multinomial formula
$$ (x_1 + \ldots + x_p)^d = \sum_{k_i \geq 0, \sum_i k_i = d} \mat{d \\ k_1, \ldots, k_p} x_1^{k_1} \ldots x_p^{k_p}$$
The map $\phix$ has $\mat{p + d - 1 \\ d}$ dimensions.
We see that $\phix$ contains no terms of "lesser" order, so, e.g., linear effects.
As an example for $p=d=2$: $\phix = (x_1^2, x_2^2, \sqrt{2} x_1 x_2)$.
% From the sum-product rules it directly follows that this is a kernel.
\end{vbframe}

\begin{vbframe}{NonHomogeneous Polynomial Kernel}
\small
 $$k(\xv, \xtil) = (\xv^T \xtil + b)^d, \text{ for } b\geq 0, d \in \N$$
The maths is very similar as before, we kind of add a further constant term in the original space, with
$$ (\xv^T \xtil + b)^d = (x_1 \tilde{x}_1 + \ldots + x_p \tilde{x_p} + b)^d$$
The feature map contains all monomials up to order $d$.
  $$\phix = \left( \sqrt{\mat{d \\ k_1, \ldots, k_{p+1}}} x_1^{k_1} \ldots x_p^{k_p} b^{k_{p+1}/2} \right)_{k_i \geq 0, \sum_i k_i = d}$$
The map $\phix$ has $\mat{p + d\\ d}$ dimensions. For $p=d=2$: 
$$(x_1 \tilde{x}_1 + x_2 \tilde{x}_2 + b)^2 = x_1^2\tilde{x}_1^2 + x_2^2 \tilde{x}_2^2 + 2 x_1 x_2 \tilde{x}_1 \tilde{x}_2 + 2b x_1 \tilde{x}_1 + 2b x_2 \tilde{x}_2 + b^2$$
Therefore, 
$$\phix = (x_1^2, x_2^2, \sqrt{2} x_1 x_2, \sqrt{2b} x_1, \sqrt{2b} x_2, b)$$
% From the sum-product rules it directly follows that this is a kernel.
\end{vbframe}


\begin{vbframe}{Polynomial Kernel}


Degree $d = 1$ yields a linear decision boundary. 

\vspace*{0.1cm} 
\begin{center}
\includegraphics[width = 11cm]{figure/svm_poly_kernel_deg_1_coef0_1.png}
\end{center}

\framebreak

The higher the degree, the more nonlinearity in the decision boundary. 

\vspace*{0.1cm}
\begin{center}
\includegraphics[width = 11cm]{figure/svm_poly_kernel_deg_3_coef0_1.png}
\end{center}

\framebreak

The higher the degree, the more nonlinearity in the decision boundary. 
\vspace*{0.1cm}
\begin{center}
\includegraphics[width = 11cm]{figure/svm_poly_kernel_deg_9_coef0_1.png}
\end{center}

\framebreak 

For $k(\xv, \tilde \xv) = (\xv^\top \tilde \xv + 0)^d$ we get no lower order effects. 

\vspace*{0.1cm}
\begin{center}
\includegraphics[width = 11cm]{figure/svm_poly_kernel_deg_3_coef0_0.png}
\end{center}

\end{vbframe}


\endlecture
\end{document}
