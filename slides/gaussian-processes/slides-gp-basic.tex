\documentclass[11pt,compress,t,notes=noshow, xcolor=table]{beamer}
% graphicx and color are loaded via lmu-lecture.sty
% maxwidth is the original width if it is less than linewidth
% otherwise use linewidth (to make sure the graphics do not exceed the margin)
% TODO: Remove once cleared to be superfluous
% \makeatletter
% \def\maxwidth{ %
%   \ifdim\Gin@nat@width>\linewidth
%     \linewidth
%   \else
%     \Gin@nat@width
%   \fi
% }
% \makeatother

% ---------------------------------%
% latex-math dependencies, do not remove:
% - mathtools
% - bm
% - siunitx
% - dsfont
% - xspace
% ---------------------------------%

%--------------------------------------------------------%
%       Language, encoding, typography
%--------------------------------------------------------%

\usepackage[english]{babel}
\usepackage[utf8]{inputenc} % Enables inputting UTF-8 symbols
% Standard AMS suite (loaded via lmu-lecture.sty)

% Font for double-stroke / blackboard letters for sets of numbers (N, R, ...)
% Distribution name is "doublestroke"
% According to https://mirror.physik.tu-berlin.de/pub/CTAN/fonts/doublestroke/dsdoc.pdf
% the "bbm" package does a similar thing and may be superfluous.
% Required for latex-math
\usepackage{dsfont}

% bbm – "Blackboard-style" cm fonts (https://www.ctan.org/pkg/bbm)
% Used to be in common.tex, loaded directly after this file
% Maybe superfluous given dsfont is loaded
% TODO: Check if really unused?
% \usepackage{bbm}

% bm – Access bold symbols in maths mode - https://ctan.org/pkg/bm
% Required for latex-math, preferred over \boldsymbol
% https://tex.stackexchange.com/questions/3238/bm-package-versus-boldsymbol
\usepackage{bm}

% pifont – Access to PostScript standard Symbol and Dingbats fonts
% Used for \newcommand{\xmark}{\ding{55}, which is never used
% aside from lecture_advml/attic/xx-automl/slides.Rnw
% \usepackage{pifont}

% Quotes (inline and display), provdes \enquote
% https://ctan.org/pkg/csquotes
\usepackage{csquotes}

% Adds arg to enumerate env, technically superseded by enumitem according
% to https://ctan.org/pkg/enumerate
% Replace with https://ctan.org/pkg/enumitem ?
% Even better: enumitem is not really compatible with beamer and breaks all sorts of things
% particularly the enumerate environment. The enumerate package also just isn't required
% from what I can tell so... don't re-add it I guess?
% \usepackage{enumerate}

% Line spacing - provides \singlespacing \doublespacing \onehalfspacing
% https://ctan.org/pkg/setspace
% \usepackage{setspace}

% mathtools – Mathematical tools to use with amsmath
% https://ctan.org/pkg/mathtools?lang=en
% latex-math dependency according to latex-math repo
\usepackage{mathtools}

% Maybe not great to use this https://tex.stackexchange.com/a/197/19093
% Use align instead -- TODO: Global search & replace to check, eqnarray is used a lot
% $ rg -f -u "\begin{eqnarray" -l | grep -v attic | awk -F '/' '{print $1}' | sort | uniq -c
%   13 lecture_advml
%   14 lecture_i2ml
%    2 lecture_iml
%   27 lecture_optimization
%   45 lecture_sl
\usepackage{eqnarray}
% For shaded regions / boxes
% Used sometimes in optim
% https://www.ctan.org/pkg/framed
\usepackage{framed}

%--------------------------------------------------------%
%       Cite button (version 2024-05)
%--------------------------------------------------------%

% Superseded by style/ref-buttons.sty, kept just in case these don't work out somehow.

% Note this requires biber to be in $PATH when running,
% telltale error in log would be e.g. Package biblatex Info: ... file 'authoryear.dbx' not found
% aside from obvious "biber: command not found" or similar.
% Tried moving this to lmu-lecture.sty but had issues I didn't quite understood,
% so it's here for now.

\usepackage{hyperref}

% Only try adding a references file if it exists, otherwise
% this would compile error when references.bib is not found
% NOTE: Bibliography packages (usebib, biblatex) are now loaded by ref-buttons.sty when needed
% This keeps all bibliography-related setup in one place

% Legacy \citelink command removed - superseded by ref-buttons.sty

%--------------------------------------------------------%
%       Displaying code and algorithms
%--------------------------------------------------------%

% Reimplements verbatim environments: https://ctan.org/pkg/verbatim
% verbatim used sed at least once in
% supervised-classification/slides-classification-tasks.tex
% Removed since code should not be put on slides anyway
% \usepackage{verbatim}

% Both used together for algorithm typesetting, see also overleaf: https://www.overleaf.com/learn/latex/Algorithms
% algorithmic env is also used, but part of the bundle:
%   "algpseudocode is part of the algorithmicx bundle, it gives you an improved version of algorithmic besides providing some other features"
% According to https://tex.stackexchange.com/questions/229355/algorithm-algorithmic-algorithmicx-algorithm2e-algpseudocode-confused
\usepackage{algorithm}
\usepackage{algpseudocode}

%--------------------------------------------------------%
%       Tables
%--------------------------------------------------------%

% multi-row table cells: https://www.namsu.de/Extra/pakete/Multirow.html
% Provides \multirow
% Used e.g. in evaluation/slides-evaluation-measures-classification.tex
\usepackage{multirow}

% colortbl: https://ctan.org/pkg/colortbl
% "The package allows rows and columns to be coloured, and even individual cells." well.
% Provides \columncolor and \rowcolor
% \rowcolor is used multiple times, e.g. in knn/slides-knn.tex
\usepackage{colortbl}

% long/multi-page tables: https://texdoc.org/serve/longtable.pdf/0
% Not used in slides
% \usepackage{longtable}

% pretty table env: https://ctan.org/pkg/booktabs
% Is used
% Defines \toprule
\usepackage{booktabs}

%--------------------------------------------------------%
%       Figures: Creating, placing, verbing
%--------------------------------------------------------%

% wrapfig - Wrapping text around figures https://de.overleaf.com/learn/latex/Wrapping_text_around_figures
% Provides wrapfigure environment -used in lecture_optimization
\usepackage{wrapfig}

% Sub figures in figures and tables
% https://ctan.org/pkg/subfig -- supersedes subfigure package
% Provides \subfigure
% \subfigure not used in slides but slides-tuning-practical.pdf errors without this pkg, error due to \captionsetup undefined
\usepackage{subfig}

% Actually it's pronounced PGF https://en.wikibooks.org/wiki/LaTeX/PGF/TikZ
\usepackage{tikz}

% No idea what/why these settings are what they are but I assume they're there on purpose
\usetikzlibrary{shapes,arrows,automata,positioning,calc,chains,trees, shadows}
\tikzset{
  %Define standard arrow tip
  >=stealth',
  %Define style for boxes
  punkt/.style={
    rectangle,
    rounded corners,
    draw=black, very thick,
    text width=6.5em,
    minimum height=2em,
    text centered},
  % Define arrow style
  pil/.style={
    ->,
    thick,
    shorten <=2pt,
    shorten >=2pt,}
}

%--------------------------------------------------------%
%       Beamer setup and custom macros & environments
%--------------------------------------------------------%

% Main sty file for beamer setup (layout, style, lecture page numbering, etc.)
% For long-term maintenance, this may me refactored into a more modular set of .sty files
\usepackage{../../style/lmu-lecture}
% Custom itemize wrappers, itemizeS, itemizeL, etc
\usepackage{../../style/customitemize}
% Custom framei environment, uses custom itemize!
\usepackage{../../style/framei}
% Custom frame2 environment, allows specifying font size for all content
\usepackage{../../style/frame2}
% Column layout macros
\usepackage{../../style/splitV}
% \image and derivatives
\usepackage{../../style/image}
% New generation of reference button macros
\usepackage{../../style/ref-buttons}

% Used regularly
\let\code=\texttt

% Not sure what/why this does
\setkeys{Gin}{width=0.9\textwidth}

% -- knitr leftovers --
% These may be used by knitr/R Markdown workflows in other lectures
\makeatletter
\def\maxwidth{ %
  \ifdim\Gin@nat@width>\linewidth
    \linewidth
  \else
    \Gin@nat@width
  \fi
}
\makeatother

% Define colors for syntax highlighting (may be used by knitr)
\definecolor{fgcolor}{rgb}{0.345, 0.345, 0.345}
\definecolor{shadecolor}{rgb}{.97, .97, .97}

% knitr code output environment
\newenvironment{knitrout}{}{}


% Can't find a reason why common.tex is not just part of this file?
\input{../../style/common}

\input{../../latex-math/basic-math}
% machine learning
\newcommand{\Xspace}{\mathcal{X}} % X, input space
\newcommand{\Yspace}{\mathcal{Y}} % Y, output space
\newcommand{\Zspace}{\mathcal{Z}} % Z, space of sampled datapoints
\newcommand{\nset}{\{1, \ldots, n\}} % set from 1 to n
\newcommand{\pset}{\{1, \ldots, p\}} % set from 1 to p
\newcommand{\gset}{\{1, \ldots, g\}} % set from 1 to g
\newcommand{\Pxy}{\mathbb{P}_{xy}} % P_xy
\newcommand{\Exy}{\mathbb{E}_{xy}} % E_xy: Expectation over random variables xy
\newcommand{\xv}{\mathbf{x}} % vector x (bold)
\newcommand{\xtil}{\tilde{\mathbf{x}}} % vector x-tilde (bold)
\newcommand{\yv}{\mathbf{y}} % vector y (bold)
\newcommand{\xy}{(\xv, y)} % observation (x, y)
\newcommand{\xvec}{\left(x_1, \ldots, x_p\right)^\top} % (x1, ..., xp)
\newcommand{\Xmat}{\mathbf{X}} % Design matrix
\newcommand{\allDatasets}{\mathds{D}} % The set of all datasets
\newcommand{\allDatasetsn}{\mathds{D}_n}  % The set of all datasets of size n
\newcommand{\D}{\mathcal{D}} % D, data
\newcommand{\Dn}{\D_n} % D_n, data of size n
\newcommand{\Dtrain}{\mathcal{D}_{\text{train}}} % D_train, training set
\newcommand{\Dtest}{\mathcal{D}_{\text{test}}} % D_test, test set
\newcommand{\xyi}[1][i]{\left(\xv^{(#1)}, y^{(#1)}\right)} % (x^i, y^i), i-th observation
\newcommand{\Dset}{\left( \xyi[1], \ldots, \xyi[n]\right)} % {(x1,y1)), ..., (xn,yn)}, data
\newcommand{\defAllDatasetsn}{(\Xspace \times \Yspace)^n} % Def. of the set of all datasets of size n
\newcommand{\defAllDatasets}{\bigcup_{n \in \N}(\Xspace \times \Yspace)^n} % Def. of the set of all datasets
\newcommand{\xdat}{\left\{ \xv^{(1)}, \ldots, \xv^{(n)}\right\}} % {x1, ..., xn}, input data
\newcommand{\ydat}{\left\{ \yv^{(1)}, \ldots, \yv^{(n)}\right\}} % {y1, ..., yn}, input data
\newcommand{\yvec}{\left(y^{(1)}, \hdots, y^{(n)}\right)^\top} % (y1, ..., yn), vector of outcomes
\newcommand{\greekxi}{\xi} % Greek letter xi
\renewcommand{\xi}[1][i]{\xv^{(#1)}} % x^i, i-th observed value of x
\newcommand{\yi}[1][i]{y^{(#1)}} % y^i, i-th observed value of y
\newcommand{\xivec}{\left(x^{(i)}_1, \ldots, x^{(i)}_p\right)^\top} % (x1^i, ..., xp^i), i-th observation vector
\newcommand{\xj}{\xv_j} % x_j, j-th feature
\newcommand{\xjvec}{\left(x^{(1)}_j, \ldots, x^{(n)}_j\right)^\top} % (x^1_j, ..., x^n_j), j-th feature vector
\newcommand{\phiv}{\mathbf{\phi}} % Basis transformation function phi
\newcommand{\phixi}{\mathbf{\phi}^{(i)}} % Basis transformation of xi: phi^i := phi(xi)

%%%%%% ml - models general
\newcommand{\lamv}{\bm{\lambda}} % lambda vector, hyperconfiguration vector
\newcommand{\Lam}{\Lambda}	 % Lambda, space of all hpos
% Inducer / Inducing algorithm
\newcommand{\preimageInducer}{\left(\defAllDatasets\right)\times\Lam} % Set of all datasets times the hyperparameter space
\newcommand{\preimageInducerShort}{\allDatasets\times\Lam} % Set of all datasets times the hyperparameter space
% Inducer / Inducing algorithm
\newcommand{\ind}{\mathcal{I}} % Inducer, inducing algorithm, learning algorithm

% continuous prediction function f
\newcommand{\ftrue}{f_{\text{true}}}  % True underlying function (if a statistical model is assumed)
\newcommand{\ftruex}{\ftrue(\xv)} % True underlying function (if a statistical model is assumed)
\newcommand{\fx}{f(\xv)} % f(x), continuous prediction function
\newcommand{\fdomains}{f: \Xspace \rightarrow \R^g} % f with domain and co-domain
\newcommand{\Hspace}{\mathcal{H}} % hypothesis space where f is from
\newcommand{\Hall}{\mathcal{H}_{\text{all}}} % unrestricted hypothesis space
\newcommand{\fbayes}{f^{\ast}} % Bayes-optimal model
\newcommand{\fxbayes}{f^{\ast}(\xv)} % Bayes-optimal model
\newcommand{\fkx}[1][k]{f_{#1}(\xv)} % f_j(x), discriminant component function
\newcommand{\fhspace}{\hat f_{\Hspace}} % fhat_H
\newcommand{\fh}{\hat{f}} % f hat, estimated prediction function
\newcommand{\fxh}{\fh(\xv)} % fhat(x)
\newcommand{\fxt}{f(\xv ~|~ \thetav)} % f(x | theta)
\newcommand{\fxi}{f\left(\xv^{(i)}\right)} % f(x^(i))
\newcommand{\fxih}{\hat{f}\left(\xv^{(i)}\right)} % f(x^(i))
\newcommand{\fxit}{f\big(\xv^{(i)} ~|~ \thetav\big)} % f(x^(i) | theta)
\newcommand{\fhD}{\fh_{\D}} % fhat_D, estimate of f based on D
\newcommand{\fhDtrain}{\fh_{\Dtrain}} % fhat_Dtrain, estimate of f based on D
\newcommand{\fhDnlam}{\fh_{\Dn, \lamv}} %model learned on Dn with hp lambda
\newcommand{\fhDlam}{\fh_{\D, \lamv}} %model learned on D with hp lambda
\newcommand{\fhDnlams}{\fh_{\Dn, \lamv^\ast}} %model learned on Dn with optimal hp lambda
\newcommand{\fhDlams}{\fh_{\D, \lamv^\ast}} %model learned on D with optimal hp lambda

% discrete prediction function h
\newcommand{\hx}{h(\xv)} % h(x), discrete prediction function
\newcommand{\hh}{\hat{h}} % h hat
\newcommand{\hxh}{\hat{h}(\xv)} % hhat(x)
\newcommand{\hxt}{h(\xv | \thetav)} % h(x | theta)
\newcommand{\hxi}{h\left(\xi\right)} % h(x^(i))
\newcommand{\hxit}{h\left(\xi ~|~ \thetav\right)} % h(x^(i) | theta)
\newcommand{\hbayes}{h^{\ast}} % Bayes-optimal classification model
\newcommand{\hxbayes}{h^{\ast}(\xv)} % Bayes-optimal classification model

% yhat
\newcommand{\yh}{\hat{y}} % yhat for prediction of target
\newcommand{\yih}{\hat{y}^{(i)}} % yhat^(i) for prediction of ith targiet
\newcommand{\resi}{\yi- \yih}

% theta
\newcommand{\thetah}{\hat{\theta}} % theta hat
\newcommand{\thetav}{\bm{\theta}} % theta vector
\newcommand{\thetavh}{\bm{\hat\theta}} % theta vector hat
\newcommand{\thetat}[1][t]{\thetav^{[#1]}} % theta^[t] in optimization
\newcommand{\thetatn}[1][t]{\thetav^{[#1 +1]}} % theta^[t+1] in optimization
\newcommand{\thetahDnlam}{\thetavh_{\Dn, \lamv}} %theta learned on Dn with hp lambda
\newcommand{\thetahDlam}{\thetavh_{\D, \lamv}} %theta learned on D with hp lambda
\newcommand{\mint}{\min_{\thetav \in \Theta}} % min problem theta
\newcommand{\argmint}{\argmin_{\thetav \in \Theta}} % argmin theta

% densities + probabilities
% pdf of x
\newcommand{\pdf}{p} % p
\newcommand{\pdfx}{p(\xv)} % p(x)
\newcommand{\pixt}{\pi(\xv~|~ \thetav)} % pi(x|theta), pdf of x given theta
\newcommand{\pixit}[1][i]{\pi\left(\xi[#1] ~|~ \thetav\right)} % pi(x^i|theta), pdf of x given theta
\newcommand{\pixii}[1][i]{\pi\left(\xi[#1]\right)} % pi(x^i), pdf of i-th x

% pdf of (x, y)
\newcommand{\pdfxy}{p(\xv,y)} % p(x, y)
\newcommand{\pdfxyt}{p(\xv, y ~|~ \thetav)} % p(x, y | theta)
\newcommand{\pdfxyit}{p\left(\xi, \yi ~|~ \thetav\right)} % p(x^(i), y^(i) | theta)

% pdf of x given y
\newcommand{\pdfxyk}[1][k]{p(\xv | y= #1)} % p(x | y = k)
\newcommand{\lpdfxyk}[1][k]{\log p(\xv | y= #1)} % log p(x | y = k)
\newcommand{\pdfxiyk}[1][k]{p\left(\xi | y= #1 \right)} % p(x^i | y = k)

% prior probabilities
\newcommand{\pik}[1][k]{\pi_{#1}} % pi_k, prior
\newcommand{\pih}{\hat{\pi}} % pi hat, estimated prior (binary classification)
\newcommand{\pikh}[1][k]{\hat{\pi}_{#1}} % pi_k hat, estimated prior
\newcommand{\lpik}[1][k]{\log \pi_{#1}} % log pi_k, log of the prior
\newcommand{\pit}{\pi(\thetav)} % Prior probability of parameter theta

% posterior probabilities
\newcommand{\post}{\P(y = 1 ~|~ \xv)} % P(y = 1 | x), post. prob for y=1
\newcommand{\postk}[1][k]{\P(y = #1 ~|~ \xv)} % P(y = k | y), post. prob for y=k
\newcommand{\pidomains}{\pi: \Xspace \rightarrow \unitint} % pi with domain and co-domain
\newcommand{\pibayes}{\pi^{\ast}} % Bayes-optimal classification model
\newcommand{\pixbayes}{\pi^{\ast}(\xv)} % Bayes-optimal classification model
\newcommand{\piastxtil}{\pi^{\ast}(\xtil)} % Bayes-optimal classification model
\newcommand{\piastkxtil}{\pi^{\ast}_k(\xtil)} % Bayes-optimal classification model for k-th class
\newcommand{\pix}{\pi(\xv)} % pi(x), P(y = 1 | x)
\newcommand{\piv}{\bm{\pi}} % pi, bold, as vector
\newcommand{\pikx}[1][k]{\pi_{#1}(\xv)} % pi_k(x), P(y = k | x)
\newcommand{\pikxt}[1][k]{\pi_{#1}(\xv ~|~ \thetav)} % pi_k(x | theta), P(y = k | x, theta)
\newcommand{\pixh}{\hat \pi(\xv)} % pi(x) hat, P(y = 1 | x) hat
\newcommand{\pikxh}[1][k]{\hat \pi_{#1}(\xv)} % pi_k(x) hat, P(y = k | x) hat
\newcommand{\pixih}{\hat \pi(\xi)} % pi(x^(i)) with hat
\newcommand{\pikxih}[1][k]{\hat \pi_{#1}(\xi)} % pi_k(x^(i)) with hat
\newcommand{\pdfygxt}{p(y ~|~\xv, \thetav)} % p(y | x, theta)
\newcommand{\pdfyigxit}{p\left(\yi ~|~\xi, \thetav\right)} % p(y^i |x^i, theta)
\newcommand{\lpdfygxt}{\log \pdfygxt } % log p(y | x, theta)
\newcommand{\lpdfyigxit}{\log \pdfyigxit} % log p(y^i |x^i, theta)

% probabilistic
\newcommand{\bayesrulek}[1][k]{\frac{\P(\xv | y= #1) \P(y= #1)}{\P(\xv)}} % Bayes rule
\newcommand{\muv}{\bm{\mu}} % expectation vector of Gaussian
\newcommand{\muk}[1][k]{\bm{\mu_{#1}}} % mean vector of class-k Gaussian (discr analysis)
\newcommand{\mukh}[1][k]{\bm{\hat{\mu}_{#1}}} % estimated mean vector of class-k Gaussian (discr analysis)

% residual and margin
\newcommand{\rx}{r(\xv)} % residual 
\newcommand{\eps}{\epsilon} % residual, stochastic
\newcommand{\epsv}{\bm{\epsilon}} % residual, stochastic, as vector
\newcommand{\epsi}{\epsilon^{(i)}} % epsilon^i, residual, stochastic
\newcommand{\epsh}{\hat{\epsilon}} % residual, estimated
\newcommand{\epsvh}{\hat{\epsv}} % residual, estimated, vector
\newcommand{\yf}{y \fx} % y f(x), margin
\newcommand{\yfi}{\yi \fxi} % y^i f(x^i), margin
\newcommand{\Sigmah}{\hat \Sigma} % estimated covariance matrix
\newcommand{\Sigmahj}{\hat \Sigma_j} % estimated covariance matrix for the j-th class
\newcommand{\nux}{\nu(\xv)} % nu(x) = y * f(x)

% ml - loss, risk, likelihood
\newcommand{\Lyf}{L\left(y, f\right)} % L(y, f), loss function
% \newcommand{\Lypi}{L\left(y, \pi\right)} % L(y, pi), loss function
\newcommand{\Lxy}{L\left(y, \fx\right)} % L(y, f(x)), loss function
\newcommand{\Lxyi}{L\left(\yi, \fxi\right)} % loss of observation
\newcommand{\Lxyt}{L\left(y, \fxt\right)} % loss with f parameterized
\newcommand{\Lxyit}{L\left(\yi, \fxit\right)} % loss of observation with f parameterized
\newcommand{\Lxym}{L\left(\yi, f\left(\bm{\tilde{x}}^{(i)} ~|~ \thetav\right)\right)} % loss of observation with f parameterized
\newcommand{\Lpixy}{L\left(y, \pix\right)} % loss in classification
% \newcommand{\Lpiy}{L\left(y, \pi\right)} % loss in classification
\newcommand{\Lpiv}{L\left(y, \piv\right)} % loss in classification
\newcommand{\Lpixyi}{L\left(\yi, \pixii\right)} % loss of observation in classification
\newcommand{\Lpixyt}{L\left(y, \pixt\right)} % loss with pi parameterized
\newcommand{\Lpixyit}{L\left(\yi, \pixit\right)} % loss of observation with pi parameterized
% \newcommand{\Lhy}{L\left(y, h\right)} % L(y, h), loss function on discrete classes
\newcommand{\Lhxy}{L\left(y, \hx\right)} % L(y, h(x)), loss function on discrete classes
\newcommand{\Lr}{L\left(r\right)} % L(r), loss defined on residual (reg) / margin (classif)
\newcommand{\lone}{|y - \fx|} % L1 loss
\newcommand{\ltwo}{\left(y - \fx\right)^2} % L2 loss
\newcommand{\lbernoullimp}{\ln(1 + \exp(-y \cdot \fx))} % Bernoulli loss for -1, +1 encoding
\newcommand{\lbernoullizo}{- y \cdot \fx + \log(1 + \exp(\fx))} % Bernoulli loss for 0, 1 encoding
\newcommand{\lcrossent}{- y \log \left(\pix\right) - (1 - y) \log \left(1 - \pix\right)} % cross-entropy loss
\newcommand{\lbrier}{\left(\pix - y \right)^2} % Brier score
\newcommand{\risk}{\mathcal{R}} % R, risk
\newcommand{\riskbayes}{\mathcal{R}^\ast}
\newcommand{\riskf}{\risk(f)} % R(f), risk
\newcommand{\riskdef}{\E_{y|\xv}\left(\Lxy \right)} % risk def (expected loss)
\newcommand{\riskt}{\mathcal{R}(\thetav)} % R(theta), risk
\newcommand{\riske}{\mathcal{R}_{\text{emp}}} % R_emp, empirical risk w/o factor 1 / n
\newcommand{\riskeb}{\bar{\mathcal{R}}_{\text{emp}}} % R_emp, empirical risk w/ factor 1 / n
\newcommand{\riskef}{\riske(f)} % R_emp(f)
\newcommand{\risket}{\mathcal{R}_{\text{emp}}(\thetav)} % R_emp(theta)
\newcommand{\riskr}{\mathcal{R}_{\text{reg}}} % R_reg, regularized risk
\newcommand{\riskrt}{\mathcal{R}_{\text{reg}}(\thetav)} % R_reg(theta)
\newcommand{\riskrf}{\riskr(f)} % R_reg(f)
\newcommand{\riskrth}{\hat{\mathcal{R}}_{\text{reg}}(\thetav)} % hat R_reg(theta)
\newcommand{\risketh}{\hat{\mathcal{R}}_{\text{emp}}(\thetav)} % hat R_emp(theta)
\newcommand{\LL}{\mathcal{L}} % L, likelihood
\newcommand{\LLt}{\mathcal{L}(\thetav)} % L(theta), likelihood
\newcommand{\LLtx}{\mathcal{L}(\thetav | \xv)} % L(theta|x), likelihood
\newcommand{\logl}{\ell} % l, log-likelihood
\newcommand{\loglt}{\logl(\thetav)} % l(theta), log-likelihood
\newcommand{\logltx}{\logl(\thetav | \xv)} % l(theta|x), log-likelihood
\newcommand{\errtrain}{\text{err}_{\text{train}}} % training error
\newcommand{\errtest}{\text{err}_{\text{test}}} % test error
\newcommand{\errexp}{\overline{\text{err}_{\text{test}}}} % avg training error

% lm
\newcommand{\thx}{\thetav^\top \xv} % linear model
\newcommand{\olsest}{(\Xmat^\top \Xmat)^{-1} \Xmat^\top \yv} % OLS estimator in LM

% ml - Gaussian Process

\newcommand{\fvec}{[f(\xi[1]), \dots, f(\xi[n])]} % function vector
\newcommand{\fv}{\mathbf{f}} % function vector
\newcommand{\mv}{\mathbf{m}} % GP mean vector
\newcommand{\kv}{\mathbf{k}} % cov matrix partition
\newcommand{\kcc}{k(\cdot, \cdot)} % cov of arbitrary inputs
\newcommand{\kxij}[2]{k(\xi, \xi[j])} % cov of x_i, x_j
\newcommand{\Kmat}{\mathbf{K}} % GP cov matrix
\newcommand{\nmk}{\normal(\mv, \Kmat)} % Gaussian w/ mean vec, cov mat
\newcommand{\nzk}{\normal(\zero, \Kmat)} % zero-mean Gaussian
\newcommand{\gpmk}{\mathcal{GP}(m(\cdot), \kcc)} % GP definition
\newcommand{\gpzk}{\mathcal{GP}(\zero, \kcc)} % zero-mean GP
\newcommand{\Xsubset}{\bm{X}} % finite subset from xspace
\newcommand{\fX}{f(\Xsubset)} % Gaussian vector of finite subset
\newcommand{\kXX}{k(\Xsubset, \Xsubset)} % kernel fun for finite subset
\newcommand{\mX}{m(\Xsubset)} % mean fun for finite subset
\newcommand{\ls}{\ell} % length-scale
\newcommand{\xxtnorm}{\| \xv - \xtil\|} % norm of x minus x tilde
\newcommand{\sqexpkernel}{\exp \left(- \frac{\| \xv - \xv^{\prime} \|^2}{2 \ls^2} \right)} % squared exponential kernel

% GP prediction
\newcommand{\xstar}{\xv_\ast} % test obs features
\newcommand{\ystar}{\yv_\ast} % test obs target
\newcommand{\fstar}{\fv_\ast} % test obs fun vector
\newcommand{\Xstar}{\Xmat_\ast} % test design matrix
\newcommand{\fstarvec}{\left[f\left(\xi[1]_{\ast}\right), \dots, f\left(\xi[m]_{\ast}\right) \right]} % pred function vector
\newcommand{\kstar}{\kv_{\ast}} % cov of new obs with x
\newcommand{\kstarstar}{\kv_{\ast \ast}} % cov of new obs
\newcommand{\Kstar}{\Kmat_{\ast}} % cov mat of new obs with x
\newcommand{\Kstarstar}{\Kmat_{\ast \ast}} % cov mat of new obs
\newcommand{\Kmatinv}{\Kmat^{-1}} % inverse cov mat
\newcommand{\Ky}{\Kmat_y} % cov mat of y


\input{../../latex-math/ml-svm}

\title{Introduction to Machine Learning}

\begin{document}

\titlemeta{
Gaussian Processes
}{
Stochastic Processes and Distributions on Functions
}{
figure_man/discrete/marginalization-more.png
}{
\item GPs = distributions over functions 
\item Marginalization property 
\item Mean and covariance function 
}

\begin{framei}[sep=L]{Weight-Space View}
\item Until now: hypothesis space $\Hspace$ of parameterized functions $\fxt$ % (in particular, the space of linear functions)
\item ERM: find risk-minimal parameters (weights) $\thetav$
\item Bayesian paradigm: distribution over $\thetav$ $\Rightarrow$ update prior to posterior belief after observing data according to Bayes' rule
$$
p(\thetav | \Xmat, \yv) 
= \frac{\text{likelihood} \cdot \text{prior}}{\text{marginal likelihood}} 
= \frac{p(\yv | \Xmat, \thetav) \cdot q(\thetav)}{p(\yv|\Xmat)}
$$
\end{framei}

\begin{framei}[sep=L]{Function-Space View}
\item New POV: rather than finding $\thetav$ which parameterizes $\fxt$,  search in space of admissible functions directly
\item Sticking to Bayesian inference, specify prior distribution \textbf{over functions} and update according to observed data points
\end{framei}

\begin{framei}{drawing from function priors}
\item Imagine we could draw functions from some prior distribution

\vfill

\imageC[1]{figure/gp_sample/zeromean_prior_50n.pdf}
\end{framei}

\begin{framei}{drawing from function priors}
\item Restrict sampling to functions consistent with observed data
\vfill
\splitVTT{
\imageC[.9]{figure/gp_sample/zeromean_prior_updates_1.pdf}
}{
\imageC[.9]{figure/gp_sample/zeromean_prior_updates_2.pdf}
}
\vfill
\splitVTT{
\imageC[.9]{figure/gp_sample/zeromean_prior_updates_3.pdf}
}{
\imageC[.9]{figure/gp_sample/zeromean_prior_updates_4.pdf}
}
\item Variety of admissible functions shrinks with seeing more data
\item Intuitively: distributions over functions have ``mean'' \& ``variance''
\end{framei}

\begin{frame2}{Weight-space vs. Function-space View}
\splitVTT{
\textbf{Weight-Space View}
\spacer
\begin{itemizeL}
\item Parameterize functions \\ (e.g., $\fxt = \thx$)
\item Define distributions on $\thetav$
\item Inference in param space $\Theta$
\end{itemizeL}
}{
\textbf{Function-Space View}
\spacer
\begin{itemizeL}
\item Work on functions directly \phantom{(e.g., $\fxt = \thx$)}
\item Define distributions on $f$
\item Inference in fun space $\Hspace$
\end{itemizeL}
}
\end{frame2}

\begin{framei}{discrete functions}
\item Let $\Xspace = \xdat$, $\Hspace = \{f ~|~ f: \Xspace \rightarrow \R\}$
\item Any $f \in \Hspace$ has finite domain with $n < \infty$ elements \\$\Rightarrow$ neat representation with $n$-dim vector 
$$\fv = \fvec^T$$
\item Example functions living in this space for $|\Xspace| \in \{2, 5, 10\}$
\vfill
\splitVThree{
\imageC{figure/discrete/discr_2_expdecay_1n.pdf}
}{
\imageC{figure/discrete/discr_5_expdecay_1n.pdf}
}{
\imageC{figure/discrete/discr_10_expdecay_1n.pdf}
}

\item \footnotesize{NB: The $\xi$ in the above are not really training points, we don't even consider training here. They are the points where we measure our (here: 1D) discrete functions. However, to avoid inventing too many symbols, and since the whole notation leads nicely into what follows next, we accept this ``abuse'' here.}  

\end{framei}

\begin{framei}[sep=L]{Distributions on Discrete Functions}
\item Specify density on vectors / functions with finite domain $f \in \Hspace$ 
\item Natural way: vector representation as $n$-dim RV, e.g.,
$$\fv = \fvec^T \sim \Nmk$$
\item For now: set $\mv = \zero$, assume $\Kmat$ to be given
\end{framei}

\foreach \i [count=\idx from 1] in {2, 5, 10} {
\begin{framei}{example: random discrete functions}
\item Example ctd: $\fv$ on $\i$ points
\item Sample representatives by sampling from a $\i$-dim Gaussian
\ifnum \i=2
$$\fv = [f(\xi[1]), f(\xi[2])]^T \sim \Nzk$$
\else 
$$\fv = [f(\xi[1]), \dots, f(\xi[\i])]^T \sim \Nzk$$
\fi
\item Where points are not (top) or strongly (bottom) correlated
\ifnum \i=2
\item RHS shows 2D density 
\else 
 \item RHS shows correlation matrix / structure
\fi

\vfill
\splitVCompact{.33}{.33}{
\imageL[0.9]{figure/discrete/discr_\i_identity_10n.pdf}
}{
\imageL[.7]{figure/discrete/discr_\i_identity_cov.pdf}
}
\vfill
\splitVCompact{.33}{.33}{
\imageL[0.9]{figure/discrete/discr_\i_expdecay_10n.pdf}
}{
\imageL[.7]{figure/discrete/discr_\i_expdecay_cov.pdf}
}
\end{framei}
}

\begin{framei}{spatial correlation}
\item ``Meaningful'' functions (on numeric $\Xspace$) often have spatial property:
$$\xi, \xi[j] \text{ close in } \Xspace \Rightarrow f(\xi), f(\xi[j]) \text{ close / strongly correlated in } \Yspace$$
\item In other words: fun. values of nearby points should be correlated
\item Enforce this by choosing dist.-based covariance function
$$ \xi[i], \xi[j] \text{ close in } \Xspace \Leftrightarrow \Kmat_{ij} \text{ high }$$
\item E.g., $\Kmat_{ij} = \kxij = \exp \left(- \tfrac{1}{2} \| \xi - \xi[j] \|^2 \right)$ vs identity cov.
\vfill
\splitVCompact{.35}{.35}{
\imageC{figure/discrete/discr_50_squaredexp_1n.pdf}
}{
\imageC{figure/discrete/discr_50_identity_1n.pdf}
}
\item More on covariance function, or \textbf{kernel}, $\kcc$ later on
\end{framei} 

\begin{framei}[sep=L]{From Discrete to Continuous Functions}
\item Until now we used multivar Gaussians to model the outputs of our discrete functions
$$\fv = \fvec^T \sim \Nmk$$
%\item No matter how large $n$ is: still functions over discrete domains
\item Can we simply extend our distribution def to \textbf{continuous}-domain functions by taking $n \rightarrow \infty$?
\item Unclear how to obtain ``infinitely'' long (Gaussian) random vectors
\item Observation: random vectors $\fv$ are collections of RVs enumerated by $\nset$ $\Rightarrow$ \textbf{indexed family} 
\item Can we use more general, infinite index sets?
\end{framei}

\begin{framei}{definition: indexed family}
\item Index $T$ allows us to identify objects in arbitrary sets $\mathcal{S}$
$$s: T \rightarrow \mathcal{S}, \quad t \mapsto s_t = s(t) $$
\item Mapping above is actually the formal definition of writing this notation $\{s_t: t \in T\}$
\item Example: real-valued $\mathcal{S}$ 
\splitV[0.55]{
\begin{itemizeM}
\item $\mathcal{S} = \R$, $t \mapsto s_t$
\item Finite index set, e.g., $T = \nset$ $\Rightarrow$ vector
\item Countable, infinite index set, e.g., $T = \N$ $\Rightarrow$ sequence 
\item Uncountable index set, e.g., $T = \R$ $\Rightarrow$  function %\\example: $\{\sin(t): t \in \R\}$
\end{itemizeM}
}{
\imageR[.8]{figure_man/indexed_family/indexed_family_1.png}
\imageR[.8]{figure_man/indexed_family/indexed_family_2.png}
}
\end{framei}

\begin{framei}[sep=M]{definition: stochastic process}
\item Collection (potentially infinite) of RVs as indexed family $\{Y_t: t \in T\}$; further distributional assumptions give rise to important subclasses
% \item $\{Y_t: t \in T\}$ $\Rightarrow$ often temporal interpretation of $T$
\item Intuition: probability distributions describe random vectors, SP describe random functions
\item Examples
\splitV[0.55]{
\begin{itemizeM}
\item $\mathcal{S}$: space of RVs, $t \mapsto Y_t$
\item Finite index set, e.g., $T = \{1, \dots, m\}$ \\$\Rightarrow$ random vector
\item Countable, infinite index set, e.g., $T = \N$ $\Rightarrow$ discrete-time SP
\item Uncountable index set, e.g., $T = \R$ $\Rightarrow$ continuous-time SP
\end{itemizeM}
}{
\imageR[.8]{figure_man/indexed_family/indexed_family_4.png}
\imageR[.8]{figure_man/indexed_family/indexed_family_3.png}
}
\end{framei}

% \begin{frame}{example: stochastic processes}

\begin{framei}[sep=M]{Def.: Gaussian Process \furtherreading{RASMUSSENWILLIAMS2006GPML}~~ \furtherreading{SNELSON2001THESIS}
}
\item Special kind of SP with index set $\Xspace$; often $\Xspace = \R^p$, but as in SVMs, feature vectors only enter the model via the kernel, so we can work on arbitrary spaces
\item We write formally $f \sim \GPmk$ %~~ \item
% \item Defining  : any finite random vector drawn from a GP is multivariate Gaussian $\Rightarrow$ \textbf{marginalization property}

\item Defining marginalization property: we have a GP 
iff for any finite set of inputs $\bm{X} \subset \Xspace$, 
$$
f(\bm{X}) \sim \normal(\mX, \kXX)
$$
\item With \textbf{mean function} $m: \Xspace \rightarrow \R$ and \\
\textbf{cov function} $k: \Xspace \times \Xspace \rightarrow \R^+_0$
\item With slight abuse of notation, we allow matrix args and write:
\vfill
\begin{itemizeM}
\item $\mv = m(\bm{X}) = [m(\xi[1]), \dots, m(\xi[n])]^T$
\item $\Kmat = k(\bm{X}, \bm{X}) = \left(\kxxt \right)_{\xv, \xtil \in \bm{X}}$
\end{itemizeM}
%\item The output is random with $\fv \sim \normal(\mv, \Kmat)$

\end{framei}

% \foreach \i in {10, 50}{
\begin{framei}[sep=M]{marginalization property}
\item For \textbf{any} finite set of inputs $\bm{X} = \xdat \subset \Xspace:$
    $$
      \fv = f(\bm{X}) = \fvec^T \sim \Nmk
    $$ 
%\item $\mv$ and $\Kmat$ are calculated by a mean and cov function, resp

\splitV{
\imageC[.8]{figure/discrete/discr_50_squaredexp_1n.pdf}
}{
% \ifnum \i=5
% \imageC[.9]{figure_man/discrete/marginalization-5.png}
% \else
\imageC[.8]{figure_man/discrete/marginalization-more.png}
% \fi
}

\item Example with 1D (left) and 2D (right) index set $\Xspace$: Dimension of $\fv$ depends on $n$, not on dimension of $\Xspace$:
\splitV{
\imageC{figure_man/indexed_family/indexed_family_5.png}
}{
\imageC[.8]{figure_man/indexed_family/indexed_family_6.png}
}
\vfill

\end{framei}
% }

% \begin{framei}[sep=L]{definition: gaussian process}
% \end{framei}

\begin{framei}[sep=L]{GP existence theorem}
% \item Idea: for any particular finite-dim distributions, the corresponding GP exists
\item For \textbf{any} 
\begin{itemize}
\item state space $\Xspace$,
\item mean function $m: \Xspace \rightarrow \R$,
\item covariance function $k: \Xspace \times \Xspace \rightarrow \R^+_0$, 
\end{itemize}
\vfill
there \textbf{exists} $f \sim \GPmk$ s.t. $\forall \xv, \xtil \in \Xspace$
\begin{eqnarray*}
\E(\fx) &=& m(\xv) \\
\cov(\fx, f(\xtil)) &=& \kxxt
\end{eqnarray*}
and $f(\bm{X}) \sim \normal(m(\bm{X}), k(\bm{X}, \bm{X}))$ for any $\bm{X} \subset \Xspace$
\item Version of Kolmogorov consistency theorem \\$\Rightarrow$ proof \furtherreading{GRIMMETT2001PROC} (Thm. 8.6.3)
\end{framei}

\begin{framei}[sep=L]{implications of existence theorem}
\item GPs completely specified by their mean \& cov function
\begin{eqnarray*}
m(\xv) &=& \E[f(\xv)] \\
\kxxt &=& \cov(\fx, f(\xtil)) = \E [( f(\xv) - \E[f(\xv)]) ( f(\xtil) - \E[f(\xtil)] )]
\end{eqnarray*}
\item For now, we consider zero-mean GPs with $m(\xv) \equiv \zero$ 

$\Rightarrow$ common, not necessarily drastic assumption 
\item Denote by $\GPzk$ $\Rightarrow$ properties mainly governed by $\kcc$
\item By virtue of existence thm: sampling from GP priors gives us random functions with our properties of choice
\end{framei}

\begin{framei}[sep=L]{sampling from gaussian process priors}
\item Example: $f \sim \GPzk$ with cov function
$$ \kxxt = \exp\left(-\tfrac{1}{2}\xxtnorm^2\right)$$
\item To visualize sample functions, 
\begin{itemize}
\item choose high number $n$ of points $\bm{X} = \xdat$
  \item compute $\Kmat = k(\bm{X}, \bm{X})$ from all pairs $\xi, \xi[j] \in \bm{X}$ 
  \item draw $\fv \sim \Nzk$ 
\end{itemize}
\item 10 randomly drawn functions (note 0 mean)
\vfill
\imageC[.8]{figure/gp_sample/zeromean_prior_10n.pdf}
\end{framei}

\begin{framei}[sep=L]{Further reading}

\item Will go through many details now, but some general refs already

\item The standard book: \furtherreading{RASMUSSENWILLIAMS2006GPML}

\item Good videos can be found here:
\furtherreading{MATHEMATICALMONK2011}
\furtherreading{FREITAS2020}



\end{framei}
\endlecture
\end{document}
