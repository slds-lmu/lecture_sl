\documentclass[11pt,compress,t,notes=noshow, xcolor=table]{beamer}
\input{../../style/preamble}
\input{../../latex-math/basic-math}
\input{../../latex-math/basic-ml}

\newcommand{\titlefigure}{figure_man/bias-variance-ridge.png}
\newcommand{\learninggoals}{
  \item Know alternative interpretations of Ridge regression
  \item Derivation of the bias-variance tradeoff for Ridge regression
}

\title{Introduction to Machine Learning}
\date{}

\begin{document}

\lecturechapter{Ridge Regression Deep-Dive}
\lecture{Introduction to Machine Learning}



\begin{vbframe}{Perspectives on $L2$ regularization}
We already saw that $L2$ regularization is equivalent to a constrained optimization problem:
\begin{eqnarray*}  
  \thetah_{\text{Ridge}} &=& \argmin_{\thetab} \sumin \left(\yi - \thetab^T \xi \right)^2 + \lambda \|\thetab\|_2^2 = ({\Xmat}^T \Xmat  + \lambda \id)^{-1} \Xmat^T\yv\\
  %&=& \argmin_{\thetab} \left(\yv - \Xmat \thetab\right)^\top \left(\yv - \Xmat \thetab\right) + \lambda \thetab^\top \thetab \\
  &=& \argmin_{\thetab} \sumin \left(\yi - \fxit\right)^2 \,
  \text{s.t. } \|\thetab\|_2^2  \leq t
  \end{eqnarray*}
We can also recover the Ridge estimator by performing least-squares on a \textbf{row-augmented} data set: Let \scriptsize{$\tilde{\Xmat}:= \begin{pmatrix} \Xmat \\ \sqrt{\lambda} \id_{p} \end{pmatrix}$ and $\tilde{\yv} := \begin{pmatrix}
    \yv \\ \bm{0}_{p}
\end{pmatrix}$.} \normalsize{Using the augmented data, the least-squares objective becomes}
\small{
$$%\argmin_{\thetab} 
\sum_{i=1}^{n+p} \left(\tilde{\yi} - \thetab^T \tilde{\xi} \right)^2 = %\argmin_{\thetab} 
\sum_{i=1}^{n} \left(\yi - \thetab^T \xi \right)^2 + \sum_{j=1}^{p} \left(0 - \sqrt{\lambda} \theta_j \right)^2 %= \thetah_{\text{Ridge}}
=\sumin \left(\yi - \thetab^T \xi \right)^2 + \lambda \|\thetab\|_2^2
$$
}
\normalsize{Thus the least-squares solution $\thetah$ using $\tilde{\Xmat},\tilde{\yv}$ instead of $\Xmat, \yv$ is $\thetah_{\text{Ridge}}$.}
%$$\thetah_{\text{Ridge}} = ({\Xmat}^T \Xmat  + \lambda \id)^{-1} \Xmat^T\yv$$
\end{vbframe}

\endlecture
\end{document}
