\documentclass[11pt,compress,t,notes=noshow, xcolor=table]{beamer}
\input{../../style/preamble}
\input{../../latex-math/basic-math}
\input{../../latex-math/basic-ml}

\newcommand{\titlefigure}{figure_man/biasvariance_scheme.png}
\newcommand{\learninggoals}{
  \item Understand why overfitting happens
  \item Know how overfitting can be avoided
  \item Know regularized empirical risk minimization 
}

\title{Introduction to Machine Learning}
\date{}

\begin{document}

\lecturechapter{Bias variance}
\lecture{Introduction to Machine Learning}

%\section{Motivation for Regularization}

\begin{vbframe}{Bias variance}

In this slide set, we will visualize the bias-variance trade-off. \\
\lz 

First, we start with the DGP $\mathbb{P}_{xy}$. Assume the true function $$f: [0, 1] \rightarrow \mathbb{R}, x\mapsto +\I_{\{x \geq 0.3\}}(x) - \I_{\{x \geq 0.6\}}(x).$$

Let the feature $x \sim \mathcal{U}([0, 1])$ and the target $y\vert x \sim \mathcal{N}(f(x), \sigma).$
\begin{center}
\includegraphics[width=0.5\textwidth]{slides/regularization/figure_man/bv_true_fun.png}
\end{center}
\framebreak 

Obviously, $f$ is an element of the function family $$\mathcal{H} := \{f: [0, 1] \rightarrow \mathbb{R}\vert\; f \text{ is continuous except for at most 2 jump discontinuities}\}$$

To make our following discussion more formal, we introduce two distance functions.
\begin{itemize}
    \item The distance between two functions $d:\mathcal{H}^2\rightarrow \mathbb{R}_{\geq 0}$ in $\mathcal{H}$ such that $$d(f_1, f_2) = \int_0^1(f_1(x) - f_2(x))^2dx.$$
    \item The distance between a function and $k$ observations $\overline{d}:\mathcal{H}\times([0,1]\times \mathbb{R})^k \rightarrow \mathbb{R}_{\geq 0}$ in $\mathcal{H}$ such that 
    $$\overline d(f, ((x_1, y_1), \dots, (x_k, y_k)) = \frac{1}{k}\sum^k_{i=1}(f_1(x_i) - y_i)^2 $$
\end{itemize}


\framebreak

\center
\vspace*{0.5cm}
\includegraphics[width=0.6\textwidth]{figure_man/biasvariance_scheme.png} \\
\footnotesize{Hastie, The Elements of Statistical Learning, 2009 (p. 225)}


\end{vbframe}



\endlecture
\end{document}
