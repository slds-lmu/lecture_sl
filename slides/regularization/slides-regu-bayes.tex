\documentclass[11pt,compress,t,notes=noshow, xcolor=table]{beamer}
\input{../../style/preamble}
\input{../../latex-math/basic-math}
\input{../../latex-math/basic-ml}

\newcommand{\titlefigure}{figure_man/bayes_reg.png}
\newcommand{\learninggoals}{
  \item Know how regularized risk minimization is same as MAP in Bayesian perspective
  \item Know correspondence of Gaussian/Laplace priors and $L2$/$L1$ regularization
}

\title{Introduction to Machine Learning}
\date{}

\begin{document}

\lecturechapter{Regularization and Bayesian Priors}
\lecture{Introduction to Machine Learning}

%-------------------------------------------------------------------------------

% \section{Regularization from a Bayesian Perspective}

\begin{vbframe} {RRM vs. Bayes}

We already created a link between max. likelihood estimation and ERM.

\lz 

Now we will generalize this for RRM.

\lz

Assume we have a parameterized distribution $p(y | \thetab, \xv)$ for our data and 
a prior $q(\thetab)$ over our parameter space, all in the Bayesian framework.

\lz 

From the Bayes theorem we know:

$$
p(\thetab | \xv, y) = \frac{p(y | \thetab, \xv) q(\thetab) }{p(y | \xv)} \propto 
p(y | \thetab, \xv) q(\thetab)
$$

\framebreak

The maximum a posteriori (MAP) estimator of $\thetab$ is now the minimizer of

$$
- \log p\left(y ~|~ \thetab, \xv\right) - \log q(\thetab).
$$

\begin{itemize}
  \item Again, we identify the loss $\Lxyt$ with $-\log(p(y | \thetab, \xv))$.
  \item If $q(\thetab)$ is constant (i.e., we used a uniform, non-informative 
  prior), the second term is irrelevant and we arrive at ERM.
  \item If not, we can identify $J(\thetab) \propto -\log(q(\thetab))$, i.e., 
  the log-prior corresponds to the regularizer, and the additional $\lambda$, which controls the strength of our
  penalty, usually influences the peakedness / inverse variance / strength of our prior.
\end{itemize}

\framebreak

\begin{figure}
  \centering
  \scalebox{1}{\includegraphics[width=\textwidth]{figure/bayes_prior.png}}
\end{figure}

\begin{itemize}
  \item $L2$ regularization corresponds to a zero-mean Gaussian prior with 
  constant variance on our parameters:
  $\theta_j \sim \mathcal{N}(0, \tau^2)$ 
  % ~ \forall j \in \{1, \dots , d\}$, $\tau > 0$.
  \item $L1$ corresponds to a zero-mean Laplace prior: 
  $\theta_j \sim \mathit{Laplace}(0,b)$.
  $\mathit{Laplace}(\mu, b)$ has density $\frac{1}{2b}\exp(-\frac{|\mu-x|}{b})$, 
  with scale parameter $b$, mean $\mu$ and variance $2b^2$.
  \item In both cases, regularization strength increases as the 
  variance of the prior decreases: a prior probability mass more narrowly 
  concentrated around 0 encourages shrinkage.
\end{itemize}
  
\end{vbframe}

% ------------------------------------------------------------------------------

\begin{vbframe}{Example: Bayesian L2 regularization}

\small We can easily see the equivalence of $L2$ regularization and a Gaussian 
prior:

\begin{itemize}
  \small
  \item We define a Gaussian prior with uncorrelated components for $\thetab$:
  \begin{footnotesize}
    $$q(\thetab) = \mathcal{N}_d(\bm{0}, \mathit{diag}(\tau^2)) 
    = \prod_{j = 1}^d  \mathcal{N}(0, \tau^2)  
    = (2\pi\tau^2)^{-\frac{d}{2}} \exp \left( - \frac{1}{2\tau^2} \sum_{j = 1}^d 
    \theta_j^2 \right).$$
  \end{footnotesize} 
  \item With this, the MAP estimator becomes
  \begin{footnotesize}
  \begin{eqnarray*}
    \thetah^{\text{MAP}} &=& \argmin_{\thetab} \left(
    - \log p\left(y ~|~ \thetab, \xv \right) - \log q(\thetab)
    \right) \\
    &=& \argmin_{\thetab} \left(
    - \log p\left(y ~|~ \thetab, \xv \right) + \tfrac{d}{2} \log(2\pi \tau^2) +
    \frac{1}{2\tau^2} \sum_{j = 1}^d \theta_j^2
    \right) \\
    % &=& \argmin_{\thetab} \left(
    % - \log p\left(\xv ~|~ \thetab\right) + \frac{1}{2\tau^2} {\thetab}^T\thetab 
    % \right) \\
    &=& \argmin_{\thetab} \left(
    - \log p\left(y ~|~ \thetab, \xv \right) + \frac{1}{2\tau^2} \| \thetab \|_2^2
    \right).
  \end{eqnarray*}
  \end{footnotesize} 
  % \item $\frac{1}{\tau^2}$ controls prior precision, i.e., inverse variance, 
  % and thus the amount of shrinkage.
  \item We see how the inverse variance (precision) $1/\tau^2$ controls shrinkage.
  % (e.g., for linear Ridge regression with 
  % $\epsilon \sim \mathcal{N}(0, \sigma^2)$ we set 
  % $\lambda = \frac{\sigma^2}{\tau^2}$).
\end{itemize}
  
  % \item The conditional distribution of $\ydat$ in linear regression with 
  % Gaussian errors $\epsilon^{(i)} \sim \mathcal{N}(0, \sigma^2) ~ \forall i \in 
  % \setn$, $\sigma > 0$, is also Gaussian: 
  % \begin{footnotesize}
  % $$- \log p\left(\ydat ~|~ \xv, \thetab\right) = \frac{n}{2} \log (2\pi 
  % \sigma^2) + \frac{1}{2\sigma^2} \sumin \left(\yi - \fxit \right)^2.$$
  % \end{footnotesize}

% &=& \argmin_{\thetab} \left(
% \tfrac{n}{2} \log (2\pi \sigma^2) + \frac{1}{2\sigma^2} \sumin \left(\yi -
% \fxit \right)^2 + \tfrac{p}{2} \log(2\pi \tau^2) +
% \frac{1}{2\tau^2} \sumjp \theta_j^2
% \right) \\
% &=& \argmin_{\thetab} \left( \frac{1}{\sigma^2} \sumin \left(\yi -
% \fxit \right)^2 + \frac{1}{\tau^2} \sumjp \theta_j^2 \right) \\
% &=& \argmin_{\thetab} \left(
% \frac{1}{\sigma^2} \left(\ydat - \Xmat \thetab\right)^\top \left(\ydat - \Xmat
% \thetab\right) + \frac{1}{\tau^2} {\thetab}^T\thetab
% \right)

% \begin{eqnarray*}
% \thetah^{\text{MAP}} &=& argmin_{\thetab} \left(
% - \log p\left(\ydat ~|~ \xv, \thetab\right) - \log q(\thetab)
% \right) \\
% 
% &=& \argmin_{\thetab} \left(
% \tfrac{n}{2} \log (2\pi \sigma^2) + \frac{1}{2\sigma^2} \sumin \left(\yi -
% \fxit \right)^2 + \tfrac{p}{2} \log(2\pi \tau^2) +
% \frac{1}{2\tau^2} \sumjp \theta_j^2
% \right) \\
% &=& \argmin_{\thetab} \left( \frac{1}{\sigma^2} \sumin \left(\yi -
% \fxit \right)^2 + \frac{1}{\tau^2} \sumjp \theta_j^2 \right) \\
% &=& \argmin_{\thetab} \left(
% \frac{1}{\sigma^2} \left(\ydat - \Xmat \thetab\right)^\top \left(\ydat - \Xmat
% \thetab\right) + \frac{1}{\tau^2} {\thetab}^T\thetab
% \right)
% 
% &=& \argmin_{\thetab} \left(
% \tfrac{n}{2} \log (2\pi \sigma^2) + \frac{1}{2\sigma^2} \sumin \left(\yi -
% \fxit \right)^2 + \tfrac{p}{2} \log(2\pi \tau^2) +
% \frac{1}{2\tau^2} \sumjp \theta_j^2
% \right) \\
% &=& \argmin_{\thetab} \left( \frac{1}{\sigma^2} \sumin \left(\yi -
% \fxit \right)^2 + \frac{1}{\tau^2} \sumjp \theta_j^2 \right) \\
% &=& \argmin_{\thetab} \left(
% \frac{1}{\sigma^2} \left(\ydat - \Xmat \thetab\right)^\top \left(\ydat - \Xmat
% \thetab\right) + \frac{1}{\tau^2} {\thetab}^T\thetab
% \right)
% \end{eqnarray*}

% \footnotesize
% Now, define $\lambda := \frac{\sigma^2}{\tau^2}$ -- note how higher prior 
% precision (i.e., lower variance) increases shrinkage! -- and set the derivative 
% to 0:
% 
% \begin{scriptsize}
% \begin{eqnarray*}
% 0 &=& \frac{1}{\lambda \tau^2} \left( - {\Xmat}^T \ydat + \thetab {\Xmat}^T \Xmat
% \right) + \frac{\lambda}{\sigma^2} \thetab
% \quad \Leftrightarrow \quad 0 = \frac{\sigma^2}{\tau^2} \left( - {\Xmat}^T \ydat 
% + \thetab {\Xmat}^T \Xmat \right) + \lambda^2 \thetab \\
% 0 &=&  - {\Xmat}^T \ydat + \thetab {\Xmat}^T \Xmat + \lambda \thetab 
% \quad \Leftrightarrow \quad 
% \thetab(\Xmat^T \Xmat  + \lambda \id) = {\Xmat}^T \ydat
% \end{eqnarray*}
% \end{scriptsize}
% 
% $\Rightarrow \thetah^{\text{MAP}} = 
% (\Xmat^T \Xmat  + \lambda \id)^{-1} \Xmat^T\ydat  = \thetah^{\text{Ridge}}.$

\end{vbframe}

% ------------------------------------------------------------------------------

% \begin{vbframe} {Minimum Description Length}

% MDL principle

% \begin{itemize}
%   \item (Compress data using using fewer symbols than literal)
%   \item (In the MDL framework, learning is seen as data compression)
%   \item (More we compress, more we learn. Therefore, pick the hypothesis which results in the shortest code.)
%   \item (Occam's razor, principle of parsimony)
%   \item (All else being equal, a simpler explanation is better than a complex one.)
% \end{itemize}

% % \begin{equation}
% %     \begin{aligned}
% %     H_{\mathrm{mdl}} & :=\underset{H \in \mathcal{H}}{\arg \min }\left(L(H)+L_{H}(D)\right) \\
% %     L_{\mathrm{mdl}}(D) & :=\min _{H \in \mathcal{H}}\left(L(H)+L_{H}(D)\right)
% %   \end{aligned}
% % \end{equation}

% \framebreak

% \begin{itemize}
%   \item There is a correspondence between the length of a message L(x) and the distribution P(x)
%   $$P(x)=2^{-L(x)}, \quad L(x)=-\log _{2} P(x)$$
%   \item Two-part code : parameter block and data block
%   \item $L(H)$ is the length of a specific hypothesis in the set.
%   \item $L(D|H)$ is the length of the data encoded under H.
%   \item $L(D,H) = L(H) + L(D|H)$
%   \item (Represents a tradeoff between goodness-of-fit and complexity)
% \end{itemize}

% \framebreak

% Regression example : $Y_{i}=f\left(X_{i}\right)+\epsilon_{i} \text { for } i=1, \ldots, n \text { where } \epsilon_{i} \stackrel{i i d}{\sim} \mathcal{N}\left(0, \sigma^{2}\right)$

% For a given dataset, the length of the encoding is :

% $$\log 1 / p_{Y | X}\left(y^{n} | x^{n}\right) = \log \left(\sqrt{2 \pi \sigma^{2}} e^{\frac{1}{2 \sigma^{2}} \sum_{i=1}^{n}\left(y_{i}-f\left(x_{i}\right)\right)^{2}}\right) \propto \sum_{i=1}^{n}\left(y_{i}-f\left(x_{i}\right)\right)^{2}$$

% \lz

% The two-stage MDL procedure to pick the best hypothesis is :

% $$f_{\gamma}=\arg \min _{f \in F_{\gamma}}\left[L(f)+\frac{1}{n} \sum_{i=1}^{n}\left(Y_{i}-f\left(X_{i}\right)\right)^{2}\right]$$

% This is equivalent to regularized least squares.
% \end{vbframe}




%%%%%%%%%%%%%%%%%%%%%%%%%%%%%%%%%%%%%%%%%%%%%%%%%%%%%%%%%%%%%%%%%%
%%%%%%%%%%%%%%%%%%          REFERENCES          %%%%%%%%%%%%%%%%%%
%%%%%%%%%%%%%%%%%%%%%%%%%%%%%%%%%%%%%%%%%%%%%%%%%%%%%%%%%%%%%%%%%%
% \begin{vbframe}
% \frametitle{References}
% \footnotesize{
% \begin{thebibliography}{99}

% %%%%%%%%%%%%%%%%%%%%%%%%%%%%%%%%%%
% \bibitem[Ian Goodfellow et al., 2016]{5} Ian Goodfellow, Yoshua Bengio and Aaron Courville (2016)
% \newblock Deep Learning
% \newblock \emph{\url{http://www.deeplearningbook.org/}}

% \end{thebibliography}
% }
% \end{vbframe}


\endlecture
\end{document}

