\documentclass[11pt,compress,t,notes=noshow, xcolor=table]{beamer}
\input{../../style/preamble}
\input{../../latex-math/basic-math}
\input{../../latex-math/basic-ml}
% attic content on platt scaling:
%https://github.com/slds-lmu/lecture_sl/blob/fddf79ea7701306085fccd20c102f1418c46749a/attic/xx-posterior-probs/slides.Rnw#L22

\title{Introduction to Machine Learning}

\begin{document}

\titlemeta{% Chunk title (example: CART, Forests, Boosting, ...), can be empty
    Linear Support Vector Machines 
  }{% Lecture title  
    Linear Hard Margin SVM
  }{% Relative path to title page image: Can be empty but must not start with slides/
  figure/svm_geometry.png
  }{
  \item Know that the hard-margin SVM maximizes the margin between data points and hyperplane 
  \item Know that this is a quadratic program
  \item Know that support vectors are the data points closest to the separating hyperplane
}

\begin{vbframe}{Linear classifiers}


\begin{center}
\includegraphics[width =9cm]{figure/linear_classif_1.png} \\
\end{center}


\vspace{-0.5em}

\begin{itemize}
    \item We want study how to build a binary, linear classifier 
      from solid geometrical principles.  
    \item Which of these two classifiers is \enquote{better}?
\end{itemize}
  
  \framebreak
  

\begin{center}
\includegraphics[width =9cm]{figure/linear_classif_2.png} \\
\end{center}


\vspace{-0.5em}

\begin{itemize}
    \item We want study how to build a binary, linear classifier 
      from solid geometrical principles.  
    \item Which of these two classifiers is \enquote{better}?
\end{itemize}
  
    $\quad \rightarrow$ The decision boundary on the right has a larger \textbf{safety margin.}


\end{vbframe}


% \begin{vbframe}{Recall: Hyperplanes}

%   A hyperplane in $\Xspace = \R^p$ is a $p-1$ dimensional linear subspace defined by a normal vector $\thetav$ (usually with $||\thetav|| = 1$), perpendicular to the hyperplane, and an offset $\theta_0$.
%   \lz
% For $\fx := \scp{\thetav}{\xv} + \theta_0$, the hyperplane is defined as
% \vspace{-0.3cm}
%   $$
%   \{\xv \in \Xspace: \scp{\thetav}{\xv} + \theta_0 = 0 \} = \{\xv \in \Xspace ~|~ \fx = 0 \}
%   $$
%   \begin{center}
%   \includegraphics[width=3cm]{figure_man/introduction/hyperplane2d.pdf} ~~~~~
%   \includegraphics[width=3cm]{figure_man/introduction/hyperplane3d.pdf} ~~~~~
%   \includegraphics[width=2cm]{figure_man/introduction/hyperplane_posneg.pdf} \\
%   % \footnotesize{Hyperplane in a 2-/3-dimensional space}
% \end{center}

% % \vspace{-0.5cm}

% \framebreak

% Positive halfspace: $\phantom{i}\{x \in \Xspace : f(x) > 0\}$ (in direction of $\thetav$)\\
% Negative halfspace: $\{x \in \Xspace : f(x) < 0\}$

% % \vspace{-0.5cm}

% The distance between point $x$ and hyperplane is
% $$d(f, x) = \frac{|\scp{\thetav}{x} + \theta_0|}{\|\thetav\|} = \frac{|f(x)|}{||\thetav||},$$\\
% i.e., $d(f, 0) = |\theta_0| / ||\thetav||$.

% For unit length $\thetav$, these simplify to $$d(f, x) = |f(x)|$$ and $d(f, 0) = |\theta_0|$ .

%   % \frac{|\scp{\thetav}{x} + \theta_0|}{\|\thetav\|}.

%   % $f(x)$

% % Consider now labeled data, e. g. $\xyi$ with $\yi \in \{-1, +1\}$.

% % \vspace{1.5cm}

% \end{vbframe}

\begin{vbframe}{Support Vector Machines: Geometry}

For labeled data $\D = \Dset$, with $\yi \in \{-1, +1\}$:
\begin{itemize}
  \item Assume linear separation by $\fx = \thetav^\top \xv + \theta_0$, such that all $+$-observations are in the positive halfspace

  $$
  \phantom{i}\{\xv \in \Xspace: \fx > 0\}
  $$

  and all $-$-observations are in the negative halfspace

  $$
  \phantom{i}\{\xv \in \Xspace : \fx < 0\}.
  $$

  \item For a linear separating hyperplane, we have
  $$
    \yi \underbrace{\left(\thetav^\top \xi + \theta_0\right)}_{= \fxi} > 0 \quad \forall i \in \{1, 2, ..., n\}.
  $$

  \item 
    % For correctly classified points $\left(\xi, \yi\right)$,
  $$
    d \left(f, \xi \right) = \frac{\yi \fxi}{\|\thetav\|} = \yi \frac{\thetav^T \xi + \theta_0}{\|\thetav\|}
  $$
  computes the (signed) distance to the separating hyperplane $\fx = 0$,
    positive for correct classifications, negative for incorrect.
  \item This expression becomes negative for misclassified points.
\end{itemize}


\begin{center}
\includegraphics[width=0.4\textwidth]{figure/svm_geometry.png} \\
\end{center}


\framebreak

\begin{itemize}
    \item The distance of $f$ to the whole dataset $\D$
    is the smallest distance
    $$
    \gamma = \min\limits_i \Big\{ d \left(f, \xi \right) \Big\}.
    $$
    \item This represents the \enquote{safety margin}, it is positive if $f$ separates and we want to maximize it.
\end{itemize}


\begin{center}
\includegraphics[width=0.4\textwidth]{figure/svm_geometry.png} \\
\end{center}


\end{vbframe}

\begin{vbframe}{Maximum margin separation}

  We formulate the desired property of a large \enquote{safety margin} as an optimization problem:
  \begin{eqnarray*}
    & \max\limits_{\thetav, \theta_0} & \gamma \\
    & \text{s.t.} & \,\, d \left(f, \xi \right) \geq \gamma \quad \forall\, i \in \nset.
    \end{eqnarray*}

    \begin{itemize}
      \item The constraints mean: We require that any instance $i$ should have a \enquote{safety} distance of at least $\gamma$ from the decision boundary defined by $f (= \thetav^T \xv + \theta_0) = 0$.
      \item Our objective is to maximize the \enquote{safety} distance.
    \end{itemize}

\end{vbframe}


\begin{vbframe}{Maximum margin separation}

We reformulate the problem:

\begin{eqnarray*}
   & \max \limits_{\thetav, \theta_0} & \gamma \\
   & \text{s.t.} & \,\, \frac{\yi \left( \scp{\thetav}{\xi} + \theta_0 \right)}{\|\thetav\|} \geq \gamma \quad \forall\, i \in \nset.
\end{eqnarray*}

\begin{itemize}
  \item The inequality is rearranged by multiplying both sides with $\|\thetav\|$:
\end{itemize}

\begin{eqnarray*}
   & \max \limits_{\thetav, \theta_0} & \gamma \\
   & \text{s.t.} & \,\,\yi  \left( \scp{\thetav}{\xi} + \theta_0 \right) \geq \|\thetav\| \gamma \quad \forall\, i \in \nset.
\end{eqnarray*}

\framebreak

  \begin{itemize}
    \item Note that the same hyperplane does not have a unique representation:
    $$
      \{\xv \in \Xspace ~|~ \thetav^\top \xv = 0\} = \{\xv \in \Xspace ~|~ c \cdot \thetav^\top \xv = 0\}
    $$
    for arbitrary $c \ne 0$.
    \item To ensure uniqueness of the solution, we make a reference choice -- we only consider hyperplanes with $\|\thetav\| = 1 / \gamma$:
  \end{itemize}

  \begin{eqnarray*}
     & \max \limits_{\thetav, \theta_0} & \gamma \\
     & \text{s.t.} & \,\,\yi  \left( \scp{\thetav}{\xi} + \theta_0 \right) \geq 1 \quad \forall\, i \in \nset.
  \end{eqnarray*}

\framebreak

  \begin{itemize}
    \item Substituting $\gamma = 1 / \|\thetav\|$ in the objective yields:
  \end{itemize}

  \begin{eqnarray*}
     & \max \limits_{\thetav, \theta_0} & \frac{1}{\|\thetav\|} \\
     & \text{s.t.} & \,\,\yi  \left( \scp{\thetav}{\xi} + \theta_0 \right) \geq 1 \quad \forall\, i \in \nset.
  \end{eqnarray*}

 \begin{itemize}
  \item Maximizing $1 / \|\thetav\|$ is the same as minimizing $\|\thetav\|$, which is the same as minimizing $\frac{1}{2}\|\thetav\|^2$:
  \end{itemize}

\begin{eqnarray*}
  & \min\limits_{\thetav, \theta_0} \quad & \frac{1}{2} \|\thetav\|^2 \\
   & \text{s.t.} & \,\,\yi  \left( \scp{\thetav}{\xi} + \theta_0 \right) \geq 1 \quad \forall\, i \in \nset.
\end{eqnarray*}

\end{vbframe}

\begin{vbframe}{Quadratic Program}

We derived the following optimization problem:

  \begin{eqnarray*}
  & \min\limits_{\thetav, \theta_0} \quad & \frac{1}{2} \|\thetav\|^2 \\
  & \text{s.t.} & \,\,\yi  \left( \scp{\thetav}{\xi} + \theta_0 \right) \geq 1 \quad \forall\, i \in \nset.
\end{eqnarray*}

This turns out to be a \textbf{convex optimization problem} -- particularly, a \textbf{quadratic program}: The objective function is quadratic, and the constraints are linear inequalities.

\lz

This is called the \textbf{primal} problem. We will later show that we can also derive a dual problem from it.

\lz

We will call this the \textbf{linear hard-margin SVM}.
\end{vbframe}

% \frame{
% \frametitle{Maximum margin separation}
%    \begin{eqnarray*}
%     \only<1-5>  {& \max \limits_{\thetav, \theta_0} & \gamma}
%     \only<6->{& \min\limits_{\thetav, \theta_0} \quad & \frac{1}{2} \|\thetav\|^2} \\
%     \only<1>{& \text{s.t.} & \,\, d(f, \xi) \geq \gamma \quad \forall\, i \in \nset}
%     \only<2>{& \text{s.t.} & \,\, \frac{\yi \left( \scp{\thetav}{\xi} + \theta_0 \right)}{\|\thetav\|} \geq \gamma \quad \forall\, i \in \nset} \\
%     \only<3-4> {& \text{s.t.} & \,\,\yi  \left( \scp{\thetav}{\xi} + \theta_0 \right) \geq \|\thetav\| \gamma \quad \forall\, i \in \nset} \\
%     \only<5-> {& \text{s.t.} & \,\,\yi  \left( \scp{\thetav}{\xi} + \theta_0 \right) \geq 1 \quad \forall\, i \in \nset}
%    \end{eqnarray*}
%    \vspace{-1em}
%   \pause
%
%    \begin{itemize}
%     % \item<1-> We can get rid of the $\|\thetav\|=1$ constraint by dividing the inequality constraint by $\|\thetav\|$
%     \pause
%     \item<3-> The inequality is rearranged by multiplying both sides with $\|\thetav\|$
%     \pause
%     \item<4-> Remember: As we assume linear separability, any positively scaled $\thetav, \theta_0$ satisfies the constraint, too \\
%     \item<5-> We substitute $\|\thetav\| = \frac{1}{\gamma} \Leftrightarrow \gamma = 1/\|\thetav\|$
%     \item<6-> Maximizing $\gamma$ is the same as minimizing $\|\thetav\|$ which is the same as minimizing $\frac{1}{2}\|\thetav\|^2$
%     % \item There are efficient ``off-the-shelf'' algorithms for solving
%     % such problems.
%    \end{itemize}
%
% }


% \frame{
% \frametitle{Maximum margin separation}
%
%   Now we will reformulate the optimization problem:
%   \begin{eqnarray*}
%     \only<1>  {& \max \limits_{\thetav, \theta_0} & \gamma}
%     \only<2-3>  {& \min\limits_{\thetav, \theta_0} & 1 / (2 \gamma^2)}
%     \only<4->{& \min\limits_{\thetav, \theta_0} \quad & \frac{1}{2} \|\thetav\|^2} \\
%     \only<1-2>{& \text{s.t.} & \,\, \yi  \left( \scp{\thetav}{\xi} + \theta_0 \right) \geq \gamma \quad \forall\, i \in \nset}
%     \only<3->{& \text{s.t.} & \,\, \yi  \left( \scp{\thetav}{\xi} + \theta_0 \right) \geq 1 \quad \forall\, i \in \nset} \\
%     \only<-2>{& \quad & \|\thetav\| = 1}
%     \only<3>{& \quad & \|\thetav\| = 1/\gamma}
%     \only<4->{& \enspace & \enspace}
%   \end{eqnarray*}
%   \vspace{-1em}
%   \pause
%   \begin{itemize}
%     \item Maximizing $\gamma$ is the same as minimizing $1 / (2 \gamma^2)$.
%     \pause
%     \item The solution $(\thetav, \theta_0)$ can be scaled without           changing the classifier. We scale it with the factor $1 / \gamma$.
%     \pause
%     \item The second constraint can be used to eliminate $\gamma$ from
%     the optimization problem.
%     It still holds $\gamma = 1 / \|\thetav\|$.
%     \pause
%     \item This turns out to be a convex optimization problem.
%     This particular form is called a \textbf{quadratic program}:
%     the objective function is quadratic, and the constraints are
%     linear (equalities and) inequalities.
%     % \item There are efficient ``off-the-shelf'' algorithms for solving
%     % such problems.
%   \end{itemize}
%
% }


% --- ADDED FOR MUDS: THIS SHOULD BE REDUNDANT AS IT IS IN SLIDES-2-HARD-MARGIN SVM

% Because we do not cover duality, slides-2-hard-margin-svm would only contain 2 slides. Thus I added it here.

\begin{vbframe}{Support Vectors}
% \vspace{1cm}

% \begin{center}
%   \includegraphics[width=5cm]{figure_man/kernels/separable-f.pdf}
% \end{center}

% \framebreak

  \begin{itemize}
    \item Some $(\xi, \yi)$ will have minimal margin, 
    $\yi  \fxi = 1$, fulfilling the inequality constraints with equality.
    \item Implies a distance of $\gamma = 1 / \|\thetav\|$ from separating hyperplane.
    \item Geometrically obvious 
      that optimal hyperplane doesn't depend on observations with larger distance.
    \item Hence, we call some of these minimal margin vectors (but not necessarily all) {support vectors}.
    \item More formal definition: in upcoming section on duality. 
    
    
\begin{center}
\includegraphics[width=0.3\textwidth]{figure/support_vectors.png} \\
\end{center}


  \end{itemize}


\end{vbframe}


% \begin{vbframe}{Dual Hard-Margin SVM}
% 
% It can be shown that \textbf{dual} representation is an equivalent representation of the problem:
% 
% \vspace*{-0.5cm}
% \begin{eqnarray*}
%     & \max\limits_{\alpha \in \R^n} & \sum_{i=1}^n \alpha_i - \frac{1}{2}\sum_{i=1}^n\sum_{j=1}^n\alpha_i\alpha_j\yi y^{(j)} \scp{\xi}{\xv^{(j)}} \\
%     & \text{s.t.} & \sum_{i=1}^n \alpha_i \yi = 0, \\
%     & \quad & \alpha_i \ge 0~\forall i \in \nset,
% \end{eqnarray*}
% 
% or, defining $\bm{K}:= \Xmat \Xmat^\top$,   equivalently in matrix notation:
% 
% \vspace*{-.5cm}
% \begin{eqnarray*}
%   & \max\limits_{\alpha \in \R^n} & \one^\top \alpha - \frac{1}{2} \alpha^\top \diag(\ydat)\bm{K} \diag(\ydat) \alpha \\
%   & \text{s. t.} & \alpha^\top \ydat = 0, \\
%   & \quad & \alpha \geq 0.
% \end{eqnarray*}
% 
% 
% \framebreak
% 
% It can be shown that the solution of a hard-margin support vector machine can be written as follows:
% 
%   $$
%     \thetah = \sumin \alpha_i \yi \xi \quad \text{ and }\quad \theta_0 = y^{(i^\star)} - \scp{\thetav}{\xv^{(i^\star)}},
%   $$
% 
%   where $(\xv^{(i^\star)}, y^{(i^\star)})$ can be any support vector.
% 
% \lz
% 
% Class predictions for a new observation $\xv$ are constructed by
% $$h(\xv) = \text{sgn}(f(\xv)) = \text{sgn}\left(\sumin \alpha_i \yi {\xi}^{\top} \xv + \theta_0 \right),$$
% 
% which is essentially a weighted sum of the
% \begin{itemize}
% \item dot products ${\xi}^{\top} \xv$ (i.e., the similarity of $\xi$ and $\xv$), with
% \item weights $\alpha_i \yi$, where $\alpha_i = 0$ for all non-SVs.
% \end{itemize}
% \end{vbframe}


\endlecture
\end{document}
