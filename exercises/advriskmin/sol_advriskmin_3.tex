\documentclass[a4paper]{article}
\usepackage[]{graphicx}\usepackage[]{color}
% maxwidth is the original width if it is less than linewidth
% otherwise use linewidth (to make sure the graphics do not exceed the margin)
\makeatletter
\def\maxwidth{ %
  \ifdim\Gin@nat@width>\linewidth
    \linewidth
  \else
    \Gin@nat@width
  \fi
}
\makeatother

\definecolor{fgcolor}{rgb}{0.345, 0.345, 0.345}
\makeatletter
\@ifundefined{AddToHook}{}{\AddToHook{package/xcolor/after}{\definecolor{fgcolor}{rgb}{0.345, 0.345, 0.345}}}
\makeatother
\newcommand{\hlnum}[1]{\textcolor[rgb]{0.686,0.059,0.569}{#1}}%
\newcommand{\hlstr}[1]{\textcolor[rgb]{0.192,0.494,0.8}{#1}}%
\newcommand{\hlcom}[1]{\textcolor[rgb]{0.678,0.584,0.686}{\textit{#1}}}%
\newcommand{\hlopt}[1]{\textcolor[rgb]{0,0,0}{#1}}%
\newcommand{\hlstd}[1]{\textcolor[rgb]{0.345,0.345,0.345}{#1}}%
\newcommand{\hlkwa}[1]{\textcolor[rgb]{0.161,0.373,0.58}{\textbf{#1}}}%
\newcommand{\hlkwb}[1]{\textcolor[rgb]{0.69,0.353,0.396}{#1}}%
\newcommand{\hlkwc}[1]{\textcolor[rgb]{0.333,0.667,0.333}{#1}}%
\newcommand{\hlkwd}[1]{\textcolor[rgb]{0.737,0.353,0.396}{\textbf{#1}}}%
\let\hlipl\hlkwb

\usepackage{framed}
\makeatletter
\newenvironment{kframe}{%
 \def\at@end@of@kframe{}%
 \ifinner\ifhmode%
  \def\at@end@of@kframe{\end{minipage}}%
  \begin{minipage}{\columnwidth}%
 \fi\fi%
 \def\FrameCommand##1{\hskip\@totalleftmargin \hskip-\fboxsep
 \colorbox{shadecolor}{##1}\hskip-\fboxsep
     % There is no \\@totalrightmargin, so:
     \hskip-\linewidth \hskip-\@totalleftmargin \hskip\columnwidth}%
 \MakeFramed {\advance\hsize-\width
   \@totalleftmargin\z@ \linewidth\hsize
   \@setminipage}}%
 {\par\unskip\endMakeFramed%
 \at@end@of@kframe}
\makeatother

\definecolor{shadecolor}{rgb}{.97, .97, .97}
\definecolor{messagecolor}{rgb}{0, 0, 0}
\definecolor{warningcolor}{rgb}{1, 0, 1}
\definecolor{errorcolor}{rgb}{1, 0, 0}
\makeatletter
\@ifundefined{AddToHook}{}{\AddToHook{package/xcolor/after}{
\definecolor{shadecolor}{rgb}{.97, .97, .97}
\definecolor{messagecolor}{rgb}{0, 0, 0}
\definecolor{warningcolor}{rgb}{1, 0, 1}
\definecolor{errorcolor}{rgb}{1, 0, 0}
}}
\makeatother
\newenvironment{knitrout}{}{} % an empty environment to be redefined in TeX

\usepackage{alltt}
\newcommand{\SweaveOpts}[1]{}  % do not interfere with LaTeX
\newcommand{\SweaveInput}[1]{} % because they are not real TeX commands
\newcommand{\Sexpr}[1]{}       % will only be parsed by R




\usepackage[utf8]{inputenc}
%\usepackage[ngerman]{babel}
\usepackage{a4wide,paralist}
\usepackage{amsmath, amssymb, xfrac, amsthm}
\usepackage{dsfont}
\usepackage[usenames,dvipsnames]{xcolor}
\usepackage{amsfonts}
\usepackage{graphicx}
\usepackage{caption}
\usepackage{subcaption}
\usepackage{framed}
\usepackage{multirow}
\usepackage{bytefield}
\usepackage{csquotes}
\usepackage[breakable, theorems, skins]{tcolorbox}
\usepackage{hyperref}
\usepackage{cancel}
\usepackage{bm}


\input{../../style/common}

\tcbset{enhanced}

\DeclareRobustCommand{\mybox}[2][gray!20]{%
	\iffalse
	\begin{tcolorbox}[   %% Adjust the following parameters at will.
		breakable,
		left=0pt,
		right=0pt,
		top=0pt,
		bottom=0pt,
		colback=#1,
		colframe=#1,
		width=\dimexpr\linewidth\relax,
		enlarge left by=0mm,
		boxsep=5pt,
		arc=0pt,outer arc=0pt,
		]
		#2
	\end{tcolorbox}
	\fi
}

\DeclareRobustCommand{\myboxshow}[2][gray!20]{%
%	\iffalse
	\begin{tcolorbox}[   %% Adjust the following parameters at will.
		breakable,
		left=0pt,
		right=0pt,
		top=0pt,
		bottom=0pt,
		colback=#1,
		colframe=#1,
		width=\dimexpr\linewidth\relax,
		enlarge left by=0mm,
		boxsep=5pt,
		arc=0pt,outer arc=0pt,
		]
		#2
	\end{tcolorbox}
%	\fi
}


%exercise numbering
\renewcommand{\theenumi}{(\alph{enumi})}
\renewcommand{\theenumii}{\roman{enumii}}
\renewcommand\labelenumi{\theenumi}


\font \sfbold=cmssbx10

\setlength{\oddsidemargin}{0cm} \setlength{\textwidth}{16cm}


\sloppy
\parindent0em
\parskip0.5em
\topmargin-2.3 cm
\textheight25cm
\textwidth17.5cm
\oddsidemargin-0.8cm
\pagestyle{empty}

\newcommand{\kopf}[2]{
\hrule
\vspace{.15cm}
\begin{minipage}{\textwidth}
%akwardly i had to put \" here to make it compile correctly
	{\sf\bf Introduction to Machine Learning \hfill Exercise sheet #1\\
	 \url{https://slds-lmu.github.io/i2ml/} \hfill #2}
\end{minipage}
\vspace{.05cm}
\hrule
\vspace{1cm}}

\newcommand{\kopfic}[2]{
\hrule
\vspace{.15cm}
\begin{minipage}{\textwidth}
%akwardly i had to put \" here to make it compile correctly
	{\sf\bf Introduction to Machine Learning \hfill Live Session #1\\
	 \url{https://slds-lmu.github.io/i2ml/} \hfill #2}
\end{minipage}
\vspace{.05cm}
\hrule
\vspace{1cm}}

\newcommand{\kopfsl}[2]{
\hrule
\vspace{.15cm}
\begin{minipage}{\textwidth}
%akwardly i had to put \" here to make it compile correctly
	{\sf\bf Supervised Learning \hfill Exercise sheet #1\\
	 \url{https://slds-lmu.github.io/i2ml/} \hfill #2}
\end{minipage}
\vspace{.05cm}
\hrule
\vspace{1cm}}

\newenvironment{allgemein}
	{\noindent}{\vspace{1cm}}

\newcounter{aufg}
\newenvironment{aufgabe}[1]
	{\refstepcounter{aufg}\textbf{Exercise \arabic{aufg}: #1}\\ \noindent}
	{\vspace{0.5cm}}

\newcounter{loes}
\newenvironment{loesung}[1]
	{\refstepcounter{loes}\textbf{Solution \arabic{loes}: #1}\\\noindent}
	{\bigskip}
	
\newenvironment{bonusaufgabe}
	{\refstepcounter{aufg}\textbf{Exercise \arabic{aufg}*\footnote{This
	is a bonus exercise.}:}\\ \noindent}
	{\vspace{0.5cm}}

\newenvironment{bonusloesung}
	{\refstepcounter{loes}\textbf{Solution \arabic{loes}*:}\\\noindent}
	{\bigskip}



\begin{document}
% !Rnw weave = knitr



\input{../../latex-math/basic-math.tex}
\input{../../latex-math/basic-ml.tex}

\kopfsl{3}{Risk Minimization}

\loesung{Connection between MLE and ERM}{

\begin{enumerate}[a)]

  
  
  \item The normal and gamma distribution assume a continuous target variable, however, we do have a discrete target variable.
  Since the target variable (number of cars) represents a count variable, Bernoulli (taking only the values 0 and 1) is not a suitable choice in this context. It follows that the only reasonable choice of the given distributions is the Poisson distribution.
  The Poisson distribution depends on the parameter $\lambda$, where the expected value is given by $E(Y|\xv) = \lambda ( \xv)$.
  The log link function is given by $\log(\lambda(\xv)) = \thetab^T \xv$.
  Following from that the probability function is given by
  $$
  P(Y = y) = \frac{\exp{(-\lambda(\xv))} \cdot (\lambda(\xv))^{y}}{y!} = \frac{\exp{(-\exp{(\thetab^T \xv)})} \cdot \exp({\thetab^T \xv})^{y}}{y!}
  $$
  for $y \in \N_0$.
  
  
  \item We can write the hypothesis space as:
  
  \begin{flalign*}
    \Hspace = \{\fxt = \exp(\thetab^T \xv) ~|~ \thetab \in \R^3 \}
    =  \{\fxt = \exp(\theta_0 + \theta_1 x_1 + \theta_2 x_2) ~|~ 
    (\theta_0, \theta_1, \theta_2) \in \R^3 \}.
    % \Hspace &= \{ f: (\R_{0}^{+})^2 \rightarrow \R ~|~ 
    % \fx = \theta_0 + \thetab^T \xv, ~ (\theta_0, \thetab) \in \R^3 \} \\
    % &=  \{ f: (\R_{0}^{+})^2 \rightarrow \R ~|~ 
    % \fx = \theta_0 + \theta_{\text{age}} x_{\text{age}} + 
    % \theta_{\text{mileage}} x_{\text{mileage}}, ~ (\theta_0, 
    % \theta_{\text{age}}, \theta_{\text{mileage}}) \in \R^3 \},
  \end{flalign*}
  
  Note the \textbf{slight abuse of notation} here: in the lecture, we first 
  define $\thetab$ to only consist of the feature coefficients, with $\xv$ 
  likewise being the plain feature vector. For the sake of simplicity, however, 
  it is more convenient to append the intercept coefficient to the vector of 
  feature coefficients. This does not change our model formulation, but we have 
  to keep in mind that it implicitly entails adding an element 1 at the first 
  position of each feature vector, i.e., $\xi := (1, x_1, x_2)^{(i)} \in 
  \{1\} \cup \Xspace$, constituting the familiar column of ones in the design 
  matrix $\Xmat$.
  
  \item The parameter space is included in the definition of the hypothesis 
  space and in this case given by $\Theta = \R^3$.
  
  
  \item 
  The likelihood for the Poisson distribution is defined by:
   \begin{flalign*}
  \LL(\thetab | \xv) &= 
  \prodin \frac{\exp{(-\exp{(\thetab^T \xi)})} \cdot (\exp{(\thetab^T \xi)})^{\yi}}{\yi!}
  \end{flalign*}
  
  
  The first thing to note is that both MLE and ERM are 
  \textbf{optimization problems}, and both should lead us to the same optimum. 
  Their opposite signs are not a problem: maximizing the likelihood is 
  equivalent to minimizing the negative likelihood. 
  Also, both are defined pointwise.
  The last thing to fix is therefore the product introduced by the independence 
  assumption in the joint likelihood of all observations (recall that we use 
  a \textit{summed} loss in ERM), for which the logarithm is a natural remedy.
  We can thus simply use the \textbf{negative log-likelihood (NLL)} as our loss 
  function (and indeed, many known loss functions can be shown to correspond to 
  certain model likelihoods).
  
  
  
  Let's put these reflections to practice:
  
  \begin{flalign*}
    L_{NLL}\left (\yi, f\left( \xi | \thetab \right) \right) 
    &= - \log \LL(\thetab | \xi) \\
    &= - \ell(\thetab | \xi) \\
    &= - \log \frac{\exp{(-\exp{(\thetab^T \xi)})} \cdot (\exp{(\thetab^T \xi)})^{\yi}}{\yi!} \\
    &= \exp{(\thetab^T \xi)} - \yi (\thetab^T \xi) + \log(\yi!)
  \end{flalign*}
  
  \begin{flalign*}
    \risket &= \sumin - \ell(\thetab | \xi) \\
    &= \sumin L_{NLL}\left (\yi, f\left( \xi | \thetab \right) \right)  \\
    &= \sumin \exp{(\thetab^T \xi)} - \yi (\thetab^T \xi) + \log(\yi!) \\
    &\propto \sumin \exp{(\thetab^T \xi)} - \yi (\thetab^T \xi) \\
  \end{flalign*}  
  
  As we are only interested in the feature coefficients here, we neglect all 
  irrelevant terms that do not depend on $\thetab$ as they have no effect on
  the solution (i.e., the $\argmin$ of $\risket$).
  This is what the proportional sign $\propto$, often used in 
  contexts of optimization and Bayesian statistics, means: we keep 
  only expressions impacted by our parameter of interest because they suffice 
  to yield the intended results or show some property of interest.
  
  %From this we can easily see the correspondence between MLE and ERM:
  %the $L2$ loss is proportional to the negative log-likelihood and hence, the 
  %$\argmax$ of the likelihood (using the assumption of normally distributed 
  %errors) and the $\argmin$ of the risk (using $L2$ loss) are equivalent.
  
  

\end{enumerate}
}
\end{document}
