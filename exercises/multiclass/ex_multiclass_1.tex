\documentclass[a4paper]{article}
\usepackage[]{graphicx}\usepackage[]{color}
% maxwidth is the original width if it is less than linewidth
% otherwise use linewidth (to make sure the graphics do not exceed the margin)
\makeatletter
\def\maxwidth{ %
  \ifdim\Gin@nat@width>\linewidth
    \linewidth
  \else
    \Gin@nat@width
  \fi
}
\makeatother

\definecolor{fgcolor}{rgb}{0.345, 0.345, 0.345}
\makeatletter
\@ifundefined{AddToHook}{}{\AddToHook{package/xcolor/after}{\definecolor{fgcolor}{rgb}{0.345, 0.345, 0.345}}}
\makeatother
\newcommand{\hlnum}[1]{\textcolor[rgb]{0.686,0.059,0.569}{#1}}%
\newcommand{\hlstr}[1]{\textcolor[rgb]{0.192,0.494,0.8}{#1}}%
\newcommand{\hlcom}[1]{\textcolor[rgb]{0.678,0.584,0.686}{\textit{#1}}}%
\newcommand{\hlopt}[1]{\textcolor[rgb]{0,0,0}{#1}}%
\newcommand{\hlstd}[1]{\textcolor[rgb]{0.345,0.345,0.345}{#1}}%
\newcommand{\hlkwa}[1]{\textcolor[rgb]{0.161,0.373,0.58}{\textbf{#1}}}%
\newcommand{\hlkwb}[1]{\textcolor[rgb]{0.69,0.353,0.396}{#1}}%
\newcommand{\hlkwc}[1]{\textcolor[rgb]{0.333,0.667,0.333}{#1}}%
\newcommand{\hlkwd}[1]{\textcolor[rgb]{0.737,0.353,0.396}{\textbf{#1}}}%
\let\hlipl\hlkwb

\usepackage{framed}
\makeatletter
\newenvironment{kframe}{%
 \def\at@end@of@kframe{}%
 \ifinner\ifhmode%
  \def\at@end@of@kframe{\end{minipage}}%
  \begin{minipage}{\columnwidth}%
 \fi\fi%
 \def\FrameCommand##1{\hskip\@totalleftmargin \hskip-\fboxsep
 \colorbox{shadecolor}{##1}\hskip-\fboxsep
     % There is no \\@totalrightmargin, so:
     \hskip-\linewidth \hskip-\@totalleftmargin \hskip\columnwidth}%
 \MakeFramed {\advance\hsize-\width
   \@totalleftmargin\z@ \linewidth\hsize
   \@setminipage}}%
 {\par\unskip\endMakeFramed%
 \at@end@of@kframe}
\makeatother

\definecolor{shadecolor}{rgb}{.97, .97, .97}
\definecolor{messagecolor}{rgb}{0, 0, 0}
\definecolor{warningcolor}{rgb}{1, 0, 1}
\definecolor{errorcolor}{rgb}{1, 0, 0}
\makeatletter
\@ifundefined{AddToHook}{}{\AddToHook{package/xcolor/after}{
\definecolor{shadecolor}{rgb}{.97, .97, .97}
\definecolor{messagecolor}{rgb}{0, 0, 0}
\definecolor{warningcolor}{rgb}{1, 0, 1}
\definecolor{errorcolor}{rgb}{1, 0, 0}
}}
\makeatother
\newenvironment{knitrout}{}{} % an empty environment to be redefined in TeX

\usepackage{alltt}
\newcommand{\SweaveOpts}[1]{}  % do not interfere with LaTeX
\newcommand{\SweaveInput}[1]{} % because they are not real TeX commands
\newcommand{\Sexpr}[1]{}       % will only be parsed by R




\usepackage[utf8]{inputenc}
%\usepackage[ngerman]{babel}
\usepackage{a4wide,paralist}
\usepackage{amsmath, amssymb, xfrac, amsthm}
\usepackage{dsfont}
\usepackage[usenames,dvipsnames]{xcolor}
\usepackage{amsfonts}
\usepackage{graphicx}
\usepackage{caption}
\usepackage{subcaption}
\usepackage{framed}
\usepackage{multirow}
\usepackage{bytefield}
\usepackage{csquotes}
\usepackage[breakable, theorems, skins]{tcolorbox}
\usepackage{hyperref}
\usepackage{cancel}
\usepackage{bm}


\input{../../style/common}

\tcbset{enhanced}

\DeclareRobustCommand{\mybox}[2][gray!20]{%
	\iffalse
	\begin{tcolorbox}[   %% Adjust the following parameters at will.
		breakable,
		left=0pt,
		right=0pt,
		top=0pt,
		bottom=0pt,
		colback=#1,
		colframe=#1,
		width=\dimexpr\linewidth\relax,
		enlarge left by=0mm,
		boxsep=5pt,
		arc=0pt,outer arc=0pt,
		]
		#2
	\end{tcolorbox}
	\fi
}

\DeclareRobustCommand{\myboxshow}[2][gray!20]{%
%	\iffalse
	\begin{tcolorbox}[   %% Adjust the following parameters at will.
		breakable,
		left=0pt,
		right=0pt,
		top=0pt,
		bottom=0pt,
		colback=#1,
		colframe=#1,
		width=\dimexpr\linewidth\relax,
		enlarge left by=0mm,
		boxsep=5pt,
		arc=0pt,outer arc=0pt,
		]
		#2
	\end{tcolorbox}
%	\fi
}


%exercise numbering
\renewcommand{\theenumi}{(\alph{enumi})}
\renewcommand{\theenumii}{\roman{enumii}}
\renewcommand\labelenumi{\theenumi}


\font \sfbold=cmssbx10

\setlength{\oddsidemargin}{0cm} \setlength{\textwidth}{16cm}


\sloppy
\parindent0em
\parskip0.5em
\topmargin-2.3 cm
\textheight25cm
\textwidth17.5cm
\oddsidemargin-0.8cm
\pagestyle{empty}

\newcommand{\kopf}[2]{
\hrule
\vspace{.15cm}
\begin{minipage}{\textwidth}
%akwardly i had to put \" here to make it compile correctly
	{\sf\bf Introduction to Machine Learning \hfill Exercise sheet #1\\
	 \url{https://slds-lmu.github.io/i2ml/} \hfill #2}
\end{minipage}
\vspace{.05cm}
\hrule
\vspace{1cm}}

\newcommand{\kopfic}[2]{
\hrule
\vspace{.15cm}
\begin{minipage}{\textwidth}
%akwardly i had to put \" here to make it compile correctly
	{\sf\bf Introduction to Machine Learning \hfill Live Session #1\\
	 \url{https://slds-lmu.github.io/i2ml/} \hfill #2}
\end{minipage}
\vspace{.05cm}
\hrule
\vspace{1cm}}

\newcommand{\kopfsl}[2]{
\hrule
\vspace{.15cm}
\begin{minipage}{\textwidth}
%akwardly i had to put \" here to make it compile correctly
	{\sf\bf Supervised Learning \hfill Exercise sheet #1\\
	 \url{https://slds-lmu.github.io/i2ml/} \hfill #2}
\end{minipage}
\vspace{.05cm}
\hrule
\vspace{1cm}}

\newenvironment{allgemein}
	{\noindent}{\vspace{1cm}}

\newcounter{aufg}
\newenvironment{aufgabe}[1]
	{\refstepcounter{aufg}\textbf{Exercise \arabic{aufg}: #1}\\ \noindent}
	{\vspace{0.5cm}}

\newcounter{loes}
\newenvironment{loesung}[1]
	{\refstepcounter{loes}\textbf{Solution \arabic{loes}: #1}\\\noindent}
	{\bigskip}
	
\newenvironment{bonusaufgabe}
	{\refstepcounter{aufg}\textbf{Exercise \arabic{aufg}*\footnote{This
	is a bonus exercise.}:}\\ \noindent}
	{\vspace{0.5cm}}

\newenvironment{bonusloesung}
	{\refstepcounter{loes}\textbf{Solution \arabic{loes}*:}\\\noindent}
	{\bigskip}



\begin{document}
% !Rnw weave = knitr



\input{../../latex-math/basic-math.tex}
\input{../../latex-math/basic-ml.tex}

\kopfsl{4}{Multiclass}

\aufgabe{Multiclass and Softmax Regression}{

\begin{enumerate}
\item Read in the MNIST data set included in the data sets in the package \texttt{keras}, using either 
\begin{knitrout}
\definecolor{shadecolor}{rgb}{0.969, 0.969, 0.969}\color{fgcolor}\begin{kframe}
\begin{alltt}
\hlkwd{library}\hlstd{(keras)}
\hlstd{mnist} \hlkwb{<-} \hlkwd{dataset_mnist}\hlstd{()}
\end{alltt}
\end{kframe}
\end{knitrout}
in R or 
\begin{knitrout}
\definecolor{shadecolor}{rgb}{0.969, 0.969, 0.969}\color{fgcolor}\begin{kframe}
\begin{alltt}
import tensorflow as tf
mnist = \hlkwd{tf.keras.datasets.mnist.load_data}(
    path=\hlstr{'mnist.npz'}
)
\end{alltt}
\end{kframe}
\end{knitrout}
in Python. The data set should contain 4 (2x2) list elements for features (pictures) and outcomes (label) for training and testing (as separate list items). 
\item Optional: Check that the data is correctly loaded by visualizing it, e.g., like this:\\

\begin{knitrout}
\definecolor{shadecolor}{rgb}{0.969, 0.969, 0.969}\color{fgcolor}\begin{kframe}


{\ttfamily\noindent\itshape\color{messagecolor}{\#\# Loaded Tensorflow version 2.9.0}}\end{kframe}

{\centering \includegraphics[width=\maxwidth]{figure/unnamed-chunk-4-1} 

}


\end{knitrout}
\item Convert the features to a (\texttt{pandas}) data frame, by flattening the 28x28 images to a 784-entry-long vector, which represents one row in your data frame. Divide the intensity values of each pixel (each column) by 255 to get a value between 0 and 1.  



\item Fit a softmax regression to the given data and interpret the results. Use an existing package by searching for a package that fits \textit{Multinomial Logistic Regression}, which is equal to a softmax regression.

\item Use the deep learning API \texttt{keras} to fit a softmax regression by fitting a network like the following:

\begin{knitrout}
\definecolor{shadecolor}{rgb}{0.969, 0.969, 0.969}\color{fgcolor}\begin{kframe}
\begin{alltt}
\hlkwd{library}(dplyr)
\hlkwd{library}(keras)

\hlcom{# convert outcome using one-hot encoding}
y_train <- \hlkwd{to_categorical}(y_train)
y_test <- \hlkwd{to_categorical}(y_test)

neural_network <- \hlkwd{keras_model_sequential}()

neural_network %>% 
  \hlkwd{layer_dense}(units = ..., \hlcom{# fill in the correct number of units }
              activation = ... \hlcom{# fill in the correct activation function}
              input_shape = \hlkwd{list}(784)) %>% 
  \hlkwd{compile}(
    optimizer = \hlstr{"adam"},
    loss      = \hlstr{"categorical_crossentropy"},
    metric = \hlstr{"accuracy"}
  )

history_minibatches <- \hlkwd{fit}(
  object           = neural_network, 
  x                = \hlkwd{as.matrix}(x_train), 
  y                = y_train,
  batch_size       = 24, 
  epochs           = 80,
  validation_data = \hlkwd{list}(\hlkwd{as.matrix}(x_test), y_test)
)
\end{alltt}
\end{kframe}
\end{knitrout}

and compare the learned \texttt{neural\_network\$weights} with the coefficients from the softmax regression.

\item Compare the performances of the two algorithms using four different multiclass metrics (you are allowed to use existing implementations).

\end{enumerate}
}
\end{document}
