\documentclass[a4paper]{article}
\usepackage[]{graphicx}\usepackage[]{color}
% maxwidth is the original width if it is less than linewidth
% otherwise use linewidth (to make sure the graphics do not exceed the margin)
\makeatletter
\def\maxwidth{ %
  \ifdim\Gin@nat@width>\linewidth
    \linewidth
  \else
    \Gin@nat@width
  \fi
}
\makeatother

\definecolor{fgcolor}{rgb}{0.345, 0.345, 0.345}
\makeatletter
\@ifundefined{AddToHook}{}{\AddToHook{package/xcolor/after}{\definecolor{fgcolor}{rgb}{0.345, 0.345, 0.345}}}
\makeatother
\newcommand{\hlnum}[1]{\textcolor[rgb]{0.686,0.059,0.569}{#1}}%
\newcommand{\hlstr}[1]{\textcolor[rgb]{0.192,0.494,0.8}{#1}}%
\newcommand{\hlcom}[1]{\textcolor[rgb]{0.678,0.584,0.686}{\textit{#1}}}%
\newcommand{\hlopt}[1]{\textcolor[rgb]{0,0,0}{#1}}%
\newcommand{\hlstd}[1]{\textcolor[rgb]{0.345,0.345,0.345}{#1}}%
\newcommand{\hlkwa}[1]{\textcolor[rgb]{0.161,0.373,0.58}{\textbf{#1}}}%
\newcommand{\hlkwb}[1]{\textcolor[rgb]{0.69,0.353,0.396}{#1}}%
\newcommand{\hlkwc}[1]{\textcolor[rgb]{0.333,0.667,0.333}{#1}}%
\newcommand{\hlkwd}[1]{\textcolor[rgb]{0.737,0.353,0.396}{\textbf{#1}}}%
\let\hlipl\hlkwb

\usepackage{framed}
\makeatletter
\newenvironment{kframe}{%
 \def\at@end@of@kframe{}%
 \ifinner\ifhmode%
  \def\at@end@of@kframe{\end{minipage}}%
  \begin{minipage}{\columnwidth}%
 \fi\fi%
 \def\FrameCommand##1{\hskip\@totalleftmargin \hskip-\fboxsep
 \colorbox{shadecolor}{##1}\hskip-\fboxsep
     % There is no \\@totalrightmargin, so:
     \hskip-\linewidth \hskip-\@totalleftmargin \hskip\columnwidth}%
 \MakeFramed {\advance\hsize-\width
   \@totalleftmargin\z@ \linewidth\hsize
   \@setminipage}}%
 {\par\unskip\endMakeFramed%
 \at@end@of@kframe}
\makeatother

\definecolor{shadecolor}{rgb}{.97, .97, .97}
\definecolor{messagecolor}{rgb}{0, 0, 0}
\definecolor{warningcolor}{rgb}{1, 0, 1}
\definecolor{errorcolor}{rgb}{1, 0, 0}
\makeatletter
\@ifundefined{AddToHook}{}{\AddToHook{package/xcolor/after}{
\definecolor{shadecolor}{rgb}{.97, .97, .97}
\definecolor{messagecolor}{rgb}{0, 0, 0}
\definecolor{warningcolor}{rgb}{1, 0, 1}
\definecolor{errorcolor}{rgb}{1, 0, 0}
}}
\makeatother
\newenvironment{knitrout}{}{} % an empty environment to be redefined in TeX

\usepackage{alltt}
\newcommand{\SweaveOpts}[1]{}  % do not interfere with LaTeX
\newcommand{\SweaveInput}[1]{} % because they are not real TeX commands
\newcommand{\Sexpr}[1]{}       % will only be parsed by R



\usepackage[utf8]{inputenc}
\pagenumbering{arabic}
%\usepackage[ngerman]{babel}
\usepackage{a4wide,paralist}
\usepackage{amsmath, amssymb, xfrac, amsthm}
\usepackage{mathtools}
\usepackage{dsfont}
\usepackage[usenames,dvipsnames]{xcolor}
\usepackage{amsfonts}
\usepackage{graphicx}
\usepackage{caption}
\usepackage{subcaption}
\usepackage{framed}
\usepackage{multirow}
\usepackage{bytefield}
\usepackage{csquotes}
\usepackage[breakable, theorems, skins]{tcolorbox}
\usepackage{hyperref}
\usepackage{cancel}
\usepackage{bm}

\input{../../style/common}

\tcbset{enhanced}

%exercise numbering
\renewcommand{\theenumi}{(\alph{enumi})}
\renewcommand{\theenumii}{\roman{enumii}}
\renewcommand\labelenumi{\theenumi}

\font \sfbold=cmssbx10
\setlength{\oddsidemargin}{0cm} \setlength{\textwidth}{16cm}

\sloppy
\parindent0em
\parskip0.5em
\topmargin-2.3 cm
\textheight25cm
\textwidth17.5cm
\oddsidemargin-0.8cm
% \pagestyle{empty}

\newcommand{\kopf}[1] {
\hrule
\vspace{.15cm}
\begin{minipage}{\textwidth}
	{\sf \bf \huge Exercise Collection -- #1}
\end{minipage}
\vspace{.05cm}
\hrule
\vspace{1cm}}

\newcommand{\exlect}
  {\color{black} \hrule \section{Lecture exercises}}
  
\newcommand{\exexams}
  {\color{black} \hrule \section{Further exercises}}
  % rename so it is not immediately clear these are from past exams
  
\newcommand{\exinspo}
  {\color{black} \hrule \section{Ideas \& exercises from other sources}}

\newcounter{aufg}
\newenvironment{aufgabe}[1]
	{\color{black} \refstepcounter{aufg}
	\subsection{Exercise \arabic{aufg}: #1} 
	\noindent}
	{\vspace{0.5cm}}
	
\newenvironment{aufgabeexam}[3] % semester, first or second, question number
	{\color{black} \refstepcounter{aufg}
	\subsection{Exercise \arabic{aufg}: #1, #2, question #3}
	\noindent}
	{\vspace{1.5cm}}

\newcounter{loes}
\newenvironment{loesung}
	{\color{gray} \refstepcounter{loes}\textbf{Solution \arabic{loes}:}
	\\ \noindent}
	{\bigskip}

\setcounter{secnumdepth}{0}



\begin{document}
% !Rnw weave = knitr



\input{../../latex-math/basic-math.tex}
\input{../../latex-math/basic-ml.tex}
\input{../../latex-math/ml-trees.tex}

\kopf{Regularization}

\tableofcontents

% ------------------------------------------------------------------------------
% LECTURE EXERCISES
% ------------------------------------------------------------------------------

\dlz
\exlect
\lz

\aufgabe{Hypothesis Space, Capacity, Regularization}{

\begin{enumerate}
\item Simulate a data set with $n = 100$ observations based on the relationship $Y = \sin(x_1) + \varepsilon$ with noise term $\varepsilon$ following some distribution. Simulate $p=100$ additional covariates $x_2,\ldots,x_{101}$ that are not related to $Y$.
\item On this data set, use different models (and software packages) of your choice to demonstrate
\begin{itemize}
\item overfitting and underfitting;
\item $L1$, $L2$ and elastic net regularization;
\item the underdetermined problem;
\item the bias-variance trade-off;
\item early stopping (use a simple neural network as in Exercise 2).
\end{itemize}
\end{enumerate}
}
\dlz

\aufgabe{Lasso, Subdifferentials}{

Optimization routines for the Lasso use coordinate gradient descent, but instead of using gradients, they resort to subdifferentials. We now try to understand in more detail what subdifferentials are:
\begin{enumerate}
  \item Recall that the Taylor approximation of first order of a function $f(x)$ at point $x_0$ is $$f(x) \approx f(x_0) + f^\prime(x_0)(x-x_0).$$ On the other hand, a differentiable function $f$ is said to be convex on an interval $\mathcal{I}$ if and only if $$f(x) \geq f(x_0) + f^\prime(x_0)(x-x_0)$$ for all points $x,x_0 \in \mathcal{I}$.
  \begin{enumerate}
  \item  What conclusion can we therefore draw if we approximate a convex function with a Taylor approximation of first order? 
  \item Visualize such an approximation for different values $x_0$ for one of the following convex functions on $\mathcal{I} = [-2,2]$. 
  \end{enumerate}
\begin{knitrout}
\definecolor{shadecolor}{rgb}{0.969, 0.969, 0.969}\color{fgcolor}
\includegraphics[width=\maxwidth]{figure/unnamed-chunk-3-1} 
\end{knitrout}

  \item A subdifferential of $f$ is a set of values $\breve{\nabla}_{x_0} f$ defined as $$\breve{\nabla}_{x_0} f = \{ g: f(x) \geq f(x_0) + g \cdot (x-x_0) \, \forall x \in \mathcal{I} \}.$$ Every scalar value $g \in \breve{\nabla}_{x_0}$ is said to be a subgradient of $f$ at $x_0$.  Does a subdifferential have any parallels to the previous question? How can we interpret $g$?
  \item We can make use of subdifferentials for convex but non-differentiable loss functions like the one induced by the Lasso. It holds that:\\
  \begin{center}
  A point $x_0$ is the global minimum of a convex function $f$ $\Leftrightarrow$ $0$ is contained in the subdifferential $\breve{\nabla}_{x_0} f$.\\
  \end{center}
  We can define a subdifferential at point $x_0$ also as a non-empty interval $[x_l,x_u]$ where the lower and upper limit is defined by $$x_l = \lim_{x \to x_0^{-}} \frac{f(x)-f(x_0)}{x-x_0}, \quad x_u = \lim_{x \to x_0^{+}} \frac{f(x)-f(x_0)}{x-x_0}.$$ These resemble the limits of the derivative $\partial f / \partial x$ evaluated at a point very close to $x_0$ when coming from the left or right side, respectively. 
  \begin{enumerate}
  \item Derive $\breve{\nabla}_{x_0} f$ for $f(x) = |x|$ at $x_0 = 0$.
  \item Is $0$ a global minimum? Explain.
  \item What is the subdifferential of the Lasso penalty $\lambda \sum_{j=1}^p |\theta_j|$? Hint: $\breve{\nabla}_{x_0} (f+g) = \breve{\nabla}_{x_0} f + \breve{\nabla}_{x_0} g$. Also, a subdifferential of a constant function is $0$ and at any other differentiable point $x_0$, the subdifferential is equal to the gradient.
  \end{enumerate}
  \item Derive the subdifferential for the Lasso problem  $$\mathcal{R}_{reg} = n^{-1} \sum_{i=1}^n (y^{(i)} - x^{(i)}_{1}\theta_1 - x^{(i)}_{2}\theta_2)^2 + \lambda \sum_{j=1}^2 |\theta_j|$$ w.r.t. $\theta_2$, i.e., for an $L1$-regularized linear model with two linear features $x_1$ and $x_2$.
\end{enumerate}



}

% 
% 
% % ------------------------------------------------------------------------------
% % PAST EXAMS
% % ------------------------------------------------------------------------------
% 
% \dlz
% \exexams
% \lz
% 
% \aufgabeexam{WS2020/21}{first}{1}{
% foo
% }
% 
% \dlz
% \loesung{
% bar
% }
% 
% % ------------------------------------------------------------------------------
% % INSPO
% % ------------------------------------------------------------------------------
% 
% \dlz
% \exinspo
\end{document}
