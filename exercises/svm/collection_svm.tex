\documentclass[a4paper]{article}
\usepackage[]{graphicx}\usepackage[]{color}
% maxwidth is the original width if it is less than linewidth
% otherwise use linewidth (to make sure the graphics do not exceed the margin)
\makeatletter
\def\maxwidth{ %
  \ifdim\Gin@nat@width>\linewidth
    \linewidth
  \else
    \Gin@nat@width
  \fi
}
\makeatother

\definecolor{fgcolor}{rgb}{0.345, 0.345, 0.345}
\makeatletter
\@ifundefined{AddToHook}{}{\AddToHook{package/xcolor/after}{\definecolor{fgcolor}{rgb}{0.345, 0.345, 0.345}}}
\makeatother
\newcommand{\hlnum}[1]{\textcolor[rgb]{0.686,0.059,0.569}{#1}}%
\newcommand{\hlstr}[1]{\textcolor[rgb]{0.192,0.494,0.8}{#1}}%
\newcommand{\hlcom}[1]{\textcolor[rgb]{0.678,0.584,0.686}{\textit{#1}}}%
\newcommand{\hlopt}[1]{\textcolor[rgb]{0,0,0}{#1}}%
\newcommand{\hlstd}[1]{\textcolor[rgb]{0.345,0.345,0.345}{#1}}%
\newcommand{\hlkwa}[1]{\textcolor[rgb]{0.161,0.373,0.58}{\textbf{#1}}}%
\newcommand{\hlkwb}[1]{\textcolor[rgb]{0.69,0.353,0.396}{#1}}%
\newcommand{\hlkwc}[1]{\textcolor[rgb]{0.333,0.667,0.333}{#1}}%
\newcommand{\hlkwd}[1]{\textcolor[rgb]{0.737,0.353,0.396}{\textbf{#1}}}%
\let\hlipl\hlkwb

\usepackage{framed}
\makeatletter
\newenvironment{kframe}{%
 \def\at@end@of@kframe{}%
 \ifinner\ifhmode%
  \def\at@end@of@kframe{\end{minipage}}%
  \begin{minipage}{\columnwidth}%
 \fi\fi%
 \def\FrameCommand##1{\hskip\@totalleftmargin \hskip-\fboxsep
 \colorbox{shadecolor}{##1}\hskip-\fboxsep
     % There is no \\@totalrightmargin, so:
     \hskip-\linewidth \hskip-\@totalleftmargin \hskip\columnwidth}%
 \MakeFramed {\advance\hsize-\width
   \@totalleftmargin\z@ \linewidth\hsize
   \@setminipage}}%
 {\par\unskip\endMakeFramed%
 \at@end@of@kframe}
\makeatother

\definecolor{shadecolor}{rgb}{.97, .97, .97}
\definecolor{messagecolor}{rgb}{0, 0, 0}
\definecolor{warningcolor}{rgb}{1, 0, 1}
\definecolor{errorcolor}{rgb}{1, 0, 0}
\makeatletter
\@ifundefined{AddToHook}{}{\AddToHook{package/xcolor/after}{
\definecolor{shadecolor}{rgb}{.97, .97, .97}
\definecolor{messagecolor}{rgb}{0, 0, 0}
\definecolor{warningcolor}{rgb}{1, 0, 1}
\definecolor{errorcolor}{rgb}{1, 0, 0}
}}
\makeatother
\newenvironment{knitrout}{}{} % an empty environment to be redefined in TeX

\usepackage{alltt}
\newcommand{\SweaveOpts}[1]{}  % do not interfere with LaTeX
\newcommand{\SweaveInput}[1]{} % because they are not real TeX commands
\newcommand{\Sexpr}[1]{}       % will only be parsed by R



\usepackage[utf8]{inputenc}
\pagenumbering{arabic}
%\usepackage[ngerman]{babel}
\usepackage{a4wide,paralist}
\usepackage{amsmath, amssymb, xfrac, amsthm}
\usepackage{mathtools}
\usepackage{dsfont}
\usepackage[usenames,dvipsnames]{xcolor}
\usepackage{amsfonts}
\usepackage{graphicx}
\usepackage{caption}
\usepackage{subcaption}
\usepackage{framed}
\usepackage{multirow}
\usepackage{bytefield}
\usepackage{csquotes}
\usepackage[breakable, theorems, skins]{tcolorbox}
\usepackage{hyperref}
\usepackage{cancel}
\usepackage{bm}

\input{../../style/common}

\tcbset{enhanced}

%exercise numbering
\renewcommand{\theenumi}{(\alph{enumi})}
\renewcommand{\theenumii}{\roman{enumii}}
\renewcommand\labelenumi{\theenumi}

\font \sfbold=cmssbx10
\setlength{\oddsidemargin}{0cm} \setlength{\textwidth}{16cm}

\sloppy
\parindent0em
\parskip0.5em
\topmargin-2.3 cm
\textheight25cm
\textwidth17.5cm
\oddsidemargin-0.8cm
% \pagestyle{empty}

\newcommand{\kopf}[1] {
\hrule
\vspace{.15cm}
\begin{minipage}{\textwidth}
	{\sf \bf \huge Exercise Collection -- #1}
\end{minipage}
\vspace{.05cm}
\hrule
\vspace{1cm}}

\newcommand{\exlect}
  {\color{black} \hrule \section{Lecture exercises}}
  
\newcommand{\exexams}
  {\color{black} \hrule \section{Further exercises}}
  % rename so it is not immediately clear these are from past exams
  
\newcommand{\exinspo}
  {\color{black} \hrule \section{Ideas \& exercises from other sources}}

\newcounter{aufg}
\newenvironment{aufgabe}[1]
	{\color{black} \refstepcounter{aufg}
	\subsection{Exercise \arabic{aufg}: #1} 
	\noindent}
	{\vspace{0.5cm}}
	
\newenvironment{aufgabeexam}[3] % semester, first or second, question number
	{\color{black} \refstepcounter{aufg}
	\subsection{Exercise \arabic{aufg}: #1, #2, question #3}
	\noindent}
	{\vspace{1.5cm}}

\newcounter{loes}
\newenvironment{loesung}
	{\color{gray} \refstepcounter{loes}\textbf{Solution \arabic{loes}:}
	\\ \noindent}
	{\bigskip}

\setcounter{secnumdepth}{0}



\begin{document}
% !Rnw weave = knitr



\input{../../latex-math/basic-math.tex}
\input{../../latex-math/basic-ml.tex}
\input{../../latex-math/ml-svm.tex}
\input{../../latex-math/ml-gp.tex}

\kopf{Support Vector Machine}

\tableofcontents

% ------------------------------------------------------------------------------
% LECTURE EXERCISES
% ------------------------------------------------------------------------------

\dlz
\exlect
\lz

\aufgabe{SVM -- Support Vectors and Separating Hyperplane}{

\begin{minipage}{\textwidth}
\begin{minipage}[c]{0.49\textwidth}
The primal optimization problem for the two-class soft margin SVM classification is given by
\begin{equation*}
\label{eq:softmargin}
\begin{aligned}
& \min_{\mathbf{\theta}, \theta_0, \sli}
& &\frac{1}{2} ||\mathbf{\theta}||^2 + C \sum_{i=1}^n \sli \\
& \text{s.t. :}
& & y^{(i)} (\mathbf{\theta}^\top \mathbf{x}^{(i)} + \theta_0) \geq 1 - \sli, \\
&&& \sli \geq 0, \hspace{8pt} \forall i = 1, \hdots, n.
%&&& \forall i = 1, \hdots, N.
\end{aligned}
\end{equation*}
\end{minipage}
  \hfill
\begin{minipage}[c]{0.51\textwidth}
\centering
\begin{knitrout}
\definecolor{shadecolor}{rgb}{0.969, 0.969, 0.969}\color{fgcolor}

{\centering \includegraphics[width=\maxwidth]{figure/unnamed-chunk-4-1} 

}


\end{knitrout}
\end{minipage}
\end{minipage}

\begin{enumerate}

  \item
    Add the decision boundary to the figure for $\hat\theta = (1, 1)^T, \hat\theta_0 = -2$. (NB: This is the approximate optimum for $C=10$)

  \item
    Identify the coordinates of the support vector(s) and compute the values of their slack variables $\sli$.

  \item
    Compute the Euclidean distance of the non-margin-violating support vector(s) (i.e. support vectors that are located on the margin hyperplanes) to the decision boundary.

  \item
    What needs to be changed in the plot such that a hard margin SVM results into the same decision boundary?

\end{enumerate}
}

\dlz

\aufgabe{SVM -- Optimization}{

Write your own stochastic subgradient descent routine to solve the soft-margin SVM in the primal formulation.\\

Hints:
\begin{itemize}
\item Use the regularized-empirical-risk-minimization formulation, i.e., an optimization criterion without constraints.
\item No kernels, just a linear SVM.
\item Compare your implementation with an existing implementation (e.g., \texttt{kernlab} in R). Are your results similar? Note that you might have to switch off the automatic data scaling in the already existing implementation.
\end{itemize}
}

\dlz

\aufgabe{SVM -- Kernel Trick}{

The polynomial kernel is defined as
$$
k(x, \tilde{x}) = (x^T\tilde{x} + b)^d.
$$
Furthermore, assume $x \in \mathbb{R}^2$ and $d = 2$.

\begin{enumerate}

  \item
    Derive the explicit feature map $\phi$ taking into account that the following equation holds:
    $$
    k(x, \tilde{x}) = \langle \phi(x), \phi(\tilde{x}) \rangle
    $$

  \item
    Describe the main differences between the kernel method and the explicit feature map.

\end{enumerate}
}

% 
% % ------------------------------------------------------------------------------
% % PAST EXAMS
% % ------------------------------------------------------------------------------
% 
% \dlz
% \exexams
% \lz
% 
% \aufgabeexam{WS2020/21}{first}{1}{
% foo
% }
% 
% \dlz
% \loesung{
% bar
% }
% 
% % ------------------------------------------------------------------------------
% % INSPO
% % ------------------------------------------------------------------------------
% 
% \dlz
% \exinspo
\end{document}
