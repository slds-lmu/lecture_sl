\documentclass[a4paper]{article}
\usepackage[]{graphicx}\usepackage[]{color}
% maxwidth is the original width if it is less than linewidth
% otherwise use linewidth (to make sure the graphics do not exceed the margin)
\makeatletter
\def\maxwidth{ %
  \ifdim\Gin@nat@width>\linewidth
    \linewidth
  \else
    \Gin@nat@width
  \fi
}
\makeatother

\definecolor{fgcolor}{rgb}{0.345, 0.345, 0.345}
\makeatletter
\@ifundefined{AddToHook}{}{\AddToHook{package/xcolor/after}{\definecolor{fgcolor}{rgb}{0.345, 0.345, 0.345}}}
\makeatother
\newcommand{\hlnum}[1]{\textcolor[rgb]{0.686,0.059,0.569}{#1}}%
\newcommand{\hlstr}[1]{\textcolor[rgb]{0.192,0.494,0.8}{#1}}%
\newcommand{\hlcom}[1]{\textcolor[rgb]{0.678,0.584,0.686}{\textit{#1}}}%
\newcommand{\hlopt}[1]{\textcolor[rgb]{0,0,0}{#1}}%
\newcommand{\hlstd}[1]{\textcolor[rgb]{0.345,0.345,0.345}{#1}}%
\newcommand{\hlkwa}[1]{\textcolor[rgb]{0.161,0.373,0.58}{\textbf{#1}}}%
\newcommand{\hlkwb}[1]{\textcolor[rgb]{0.69,0.353,0.396}{#1}}%
\newcommand{\hlkwc}[1]{\textcolor[rgb]{0.333,0.667,0.333}{#1}}%
\newcommand{\hlkwd}[1]{\textcolor[rgb]{0.737,0.353,0.396}{\textbf{#1}}}%
\let\hlipl\hlkwb

\usepackage{framed}
\makeatletter
\newenvironment{kframe}{%
 \def\at@end@of@kframe{}%
 \ifinner\ifhmode%
  \def\at@end@of@kframe{\end{minipage}}%
  \begin{minipage}{\columnwidth}%
 \fi\fi%
 \def\FrameCommand##1{\hskip\@totalleftmargin \hskip-\fboxsep
 \colorbox{shadecolor}{##1}\hskip-\fboxsep
     % There is no \\@totalrightmargin, so:
     \hskip-\linewidth \hskip-\@totalleftmargin \hskip\columnwidth}%
 \MakeFramed {\advance\hsize-\width
   \@totalleftmargin\z@ \linewidth\hsize
   \@setminipage}}%
 {\par\unskip\endMakeFramed%
 \at@end@of@kframe}
\makeatother

\definecolor{shadecolor}{rgb}{.97, .97, .97}
\definecolor{messagecolor}{rgb}{0, 0, 0}
\definecolor{warningcolor}{rgb}{1, 0, 1}
\definecolor{errorcolor}{rgb}{1, 0, 0}
\makeatletter
\@ifundefined{AddToHook}{}{\AddToHook{package/xcolor/after}{
\definecolor{shadecolor}{rgb}{.97, .97, .97}
\definecolor{messagecolor}{rgb}{0, 0, 0}
\definecolor{warningcolor}{rgb}{1, 0, 1}
\definecolor{errorcolor}{rgb}{1, 0, 0}
}}
\makeatother
\newenvironment{knitrout}{}{} % an empty environment to be redefined in TeX

\usepackage{alltt}
\newcommand{\SweaveOpts}[1]{}  % do not interfere with LaTeX
\newcommand{\SweaveInput}[1]{} % because they are not real TeX commands
\newcommand{\Sexpr}[1]{}       % will only be parsed by R




\usepackage[utf8]{inputenc}
%\usepackage[ngerman]{babel}
\usepackage{a4wide,paralist}
\usepackage{amsmath, amssymb, xfrac, amsthm}
\usepackage{dsfont}
\usepackage[usenames,dvipsnames]{xcolor}
\usepackage{amsfonts}
\usepackage{graphicx}
\usepackage{caption}
\usepackage{subcaption}
\usepackage{framed}
\usepackage{multirow}
\usepackage{bytefield}
\usepackage{csquotes}
\usepackage[breakable, theorems, skins]{tcolorbox}
\usepackage{hyperref}
\usepackage{cancel}
\usepackage{bm}


\input{../../style/common}

\tcbset{enhanced}

\DeclareRobustCommand{\mybox}[2][gray!20]{%
	\iffalse
	\begin{tcolorbox}[   %% Adjust the following parameters at will.
		breakable,
		left=0pt,
		right=0pt,
		top=0pt,
		bottom=0pt,
		colback=#1,
		colframe=#1,
		width=\dimexpr\linewidth\relax,
		enlarge left by=0mm,
		boxsep=5pt,
		arc=0pt,outer arc=0pt,
		]
		#2
	\end{tcolorbox}
	\fi
}

\DeclareRobustCommand{\myboxshow}[2][gray!20]{%
%	\iffalse
	\begin{tcolorbox}[   %% Adjust the following parameters at will.
		breakable,
		left=0pt,
		right=0pt,
		top=0pt,
		bottom=0pt,
		colback=#1,
		colframe=#1,
		width=\dimexpr\linewidth\relax,
		enlarge left by=0mm,
		boxsep=5pt,
		arc=0pt,outer arc=0pt,
		]
		#2
	\end{tcolorbox}
%	\fi
}


%exercise numbering
\renewcommand{\theenumi}{(\alph{enumi})}
\renewcommand{\theenumii}{\roman{enumii}}
\renewcommand\labelenumi{\theenumi}


\font \sfbold=cmssbx10

\setlength{\oddsidemargin}{0cm} \setlength{\textwidth}{16cm}


\sloppy
\parindent0em
\parskip0.5em
\topmargin-2.3 cm
\textheight25cm
\textwidth17.5cm
\oddsidemargin-0.8cm
\pagestyle{empty}

\newcommand{\kopf}[2]{
\hrule
\vspace{.15cm}
\begin{minipage}{\textwidth}
%akwardly i had to put \" here to make it compile correctly
	{\sf\bf Introduction to Machine Learning \hfill Exercise sheet #1\\
	 \url{https://slds-lmu.github.io/i2ml/} \hfill #2}
\end{minipage}
\vspace{.05cm}
\hrule
\vspace{1cm}}

\newcommand{\kopfic}[2]{
\hrule
\vspace{.15cm}
\begin{minipage}{\textwidth}
%akwardly i had to put \" here to make it compile correctly
	{\sf\bf Introduction to Machine Learning \hfill Live Session #1\\
	 \url{https://slds-lmu.github.io/i2ml/} \hfill #2}
\end{minipage}
\vspace{.05cm}
\hrule
\vspace{1cm}}

\newcommand{\kopfsl}[2]{
\hrule
\vspace{.15cm}
\begin{minipage}{\textwidth}
%akwardly i had to put \" here to make it compile correctly
	{\sf\bf Supervised Learning \hfill Exercise sheet #1\\
	 \url{https://slds-lmu.github.io/i2ml/} \hfill #2}
\end{minipage}
\vspace{.05cm}
\hrule
\vspace{1cm}}

\newenvironment{allgemein}
	{\noindent}{\vspace{1cm}}

\newcounter{aufg}
\newenvironment{aufgabe}[1]
	{\refstepcounter{aufg}\textbf{Exercise \arabic{aufg}: #1}\\ \noindent}
	{\vspace{0.5cm}}

\newcounter{loes}
\newenvironment{loesung}[1]
	{\refstepcounter{loes}\textbf{Solution \arabic{loes}: #1}\\\noindent}
	{\bigskip}
	
\newenvironment{bonusaufgabe}
	{\refstepcounter{aufg}\textbf{Exercise \arabic{aufg}*\footnote{This
	is a bonus exercise.}:}\\ \noindent}
	{\vspace{0.5cm}}

\newenvironment{bonusloesung}
	{\refstepcounter{loes}\textbf{Solution \arabic{loes}*:}\\\noindent}
	{\bigskip}



\begin{document}
% !Rnw weave = knitr



\input{../../latex-math/basic-math.tex}
\input{../../latex-math/basic-ml.tex}
\input{../../latex-math/ml-svm.tex}
\input{../../latex-math/ml-gp.tex}

\kopfsl{5}

\loesung{}{


\begin{enumerate}

  \item
   
    The hyperplane is given by $$\theta_1 x_1^{(i)} + \theta_2 x_2^{(i)} + \theta_0 = 0.$$
    
    Plugging in the values for the $\theta$s and solving for $x_2$, we get the decision boundary as function of $x_1$: $$x_2 = -x_1 + 2.$$ 

  \item
    

    $(0.5, 0.5), (0, 1), (0, 3), (3, 0)$ are support vectors with slack value of $\sli = 0$ as they lie on the margin hyperplanes.

    $(0, 0)$ is also a support vector with slack value of $\sli = 3$.

    Derivation: We use the equation from the constraint
    $y_i (\mathbf{\theta}^\top \mathbf{x}_i + \theta_0) \geq 1 - \sli$ and plug in the values for the margin-violating point $y_i = 1, x_1 = 0, x_2 = 0$:

    $$
    y_i (x_1 + x_2 - 2) = 1 (0 + 0 - 2) \geq 1 - \sli \Rightarrow \sli \geq 3
    $$



  \item

    Using $\xi = \left(\begin{array}{c} 0.5 \\ 0.5 \end{array}\right)$:

    $$
    d(f, \xi) = \frac{\yi f(\xi)}{\| \theta \|_2} = \frac{-1(0.5 + 0.5 - 2)}{\sqrt{2}} = \frac{1}{\sqrt{2}}
    $$

    The distance is the same for all non-margin-violating support vectors.

  \item

    Change point $(0,0)$ from $+$ to $-$ or remove $(0,0)$.

\end{enumerate}

}
\dlz

\loesung{}{

% Paper: https://home.ttic.edu/~nati/Publications/PegasosMPB.pdf
% => Contains, among others, 4 possibilities of how to 
%   estimate theta_0
\begin{itemize}
\item Implementation of the PEGASOS algorithm: 
\begin{knitrout}
\definecolor{shadecolor}{rgb}{0.969, 0.969, 0.969}\color{fgcolor}\begin{kframe}
\begin{alltt}
\hlcom{#' @param y outcome vector}
\hlcom{#' @param X design matrix (including a column of 1s for the intercept)}
\hlcom{#' @param nr_iter number of iterations for the algorithm}
\hlcom{#' @param theta starting values for thetas}
\hlcom{#' @param lambda penalty parameter}
\hlcom{#' @param alpha step size for weight decay}
\hlstd{pegasos_linear} \hlkwb{<-} \hlkwa{function}\hlstd{(}
  \hlkwc{y}\hlstd{,}
  \hlkwc{X}\hlstd{,}
  \hlkwc{nr_iter} \hlstd{=} \hlnum{50000}\hlstd{,}
  \hlkwc{theta} \hlstd{=} \hlkwd{rnorm}\hlstd{(}\hlkwd{ncol}\hlstd{(X)),}
  \hlkwc{lambda} \hlstd{=} \hlnum{1}\hlstd{,}
  \hlkwc{alpha} \hlstd{=} \hlnum{0.01}\hlstd{)}
\hlstd{\{}

  \hlstd{t} \hlkwb{<-} \hlnum{1}
  \hlstd{n} \hlkwb{<-} \hlkwd{NROW}\hlstd{(y)}

  \hlkwa{while}\hlstd{(t} \hlopt{<=} \hlstd{nr_iter)\{}

    \hlstd{f_current} \hlkwb{=} \hlstd{X}\hlopt\hlstd{theta}
    \hlstd{i} \hlkwb{<-} \hlkwd{sample}\hlstd{(}\hlnum{1}\hlopt{:}\hlstd{n,} \hlnum{1}\hlstd{)}

    \hlcom{# update}
    \hlstd{theta} \hlkwb{<-} \hlstd{(}\hlnum{1} \hlopt{-} \hlstd{lambda} \hlopt{*} \hlstd{alpha)} \hlopt{*} \hlstd{theta}
    \hlcom{# add second term if within margin}
    \hlkwa{if}\hlstd{(y[i]}\hlopt{*}\hlstd{f_current[i]} \hlopt{<} \hlnum{1}\hlstd{) theta} \hlkwb{<-} \hlstd{theta} \hlopt{+} \hlstd{alpha} \hlopt{*} \hlstd{y[i]}\hlopt{*}\hlstd{X[i,]}

    \hlstd{t} \hlkwb{<-} \hlstd{t} \hlopt{+} \hlnum{1}

  \hlstd{\}}

  \hlkwd{return}\hlstd{(theta)}

\hlstd{\}}
\end{alltt}
\end{kframe}
\end{knitrout}

\item Check on a simple example

\begin{knitrout}
\definecolor{shadecolor}{rgb}{0.969, 0.969, 0.969}\color{fgcolor}\begin{kframe}
\begin{alltt}
\hlcom{## Check on a simple example}
\hlcom{## --------------------------------------------}

\hlkwd{set.seed}\hlstd{(}\hlnum{2L}\hlstd{)}

\hlstd{C} \hlkwb{=} \hlnum{1}

\hlkwd{library}\hlstd{(mlbench)}
\hlkwd{library}\hlstd{(kernlab)}
\hlstd{data} \hlkwb{=} \hlkwd{mlbench.twonorm}\hlstd{(}\hlkwc{n} \hlstd{=} \hlnum{100}\hlstd{,} \hlkwc{d} \hlstd{=} \hlnum{2}\hlstd{)}


\hlstd{data} \hlkwb{=} \hlkwd{as.data.frame}\hlstd{(data)}
\hlstd{X} \hlkwb{=} \hlkwd{as.matrix}\hlstd{(data[,} \hlnum{1}\hlopt{:}\hlnum{2}\hlstd{])}
\hlstd{y} \hlkwb{=} \hlstd{data}\hlopt{$}\hlstd{classes}
\hlkwd{par}\hlstd{(}\hlkwc{mar} \hlstd{=} \hlkwd{c}\hlstd{(}\hlnum{5}\hlstd{,}\hlnum{4}\hlstd{,}\hlnum{4}\hlstd{,}\hlnum{6}\hlstd{))}
\hlkwd{plot}\hlstd{(}\hlkwc{x} \hlstd{= data}\hlopt{$}\hlstd{x.1,} \hlkwc{y} \hlstd{= data}\hlopt{$}\hlstd{x.2,} \hlkwc{pch} \hlstd{=} \hlkwd{ifelse}\hlstd{(data}\hlopt{$}\hlstd{classes} \hlopt{==} \hlnum{1}\hlstd{,} \hlstr{"-"}\hlstd{,} \hlstr{"+"}\hlstd{),} \hlkwc{col} \hlstd{=} \hlstr{"black"}\hlstd{,}
     \hlkwc{xlab} \hlstd{=} \hlstr{"x1"}\hlstd{,} \hlkwc{ylab} \hlstd{=} \hlstr{"x2"}\hlstd{)}

\hlcom{# recode y}
\hlstd{y} \hlkwb{=} \hlkwd{ifelse}\hlstd{(y} \hlopt{==} \hlstr{"2"}\hlstd{,} \hlnum{1}\hlstd{,} \hlopt{-}\hlnum{1}\hlstd{)}
\hlstd{mod_pegasos} \hlkwb{=} \hlkwd{pegasos_linear}\hlstd{(y,} \hlkwd{cbind}\hlstd{(}\hlnum{1}\hlstd{,X),} \hlkwc{lambda} \hlstd{= C}\hlopt{/}\hlstd{(}\hlkwd{NROW}\hlstd{(y)))}

\hlcom{# Add estimated decision boundary:}
\hlkwd{abline}\hlstd{(}\hlkwc{a} \hlstd{=} \hlopt{-} \hlstd{mod_pegasos[}\hlnum{1}\hlstd{]} \hlopt{/} \hlstd{mod_pegasos[}\hlnum{2}\hlstd{],}
       \hlkwc{b} \hlstd{=} \hlopt{-} \hlstd{mod_pegasos[}\hlnum{2}\hlstd{]} \hlopt{/} \hlstd{mod_pegasos[}\hlnum{3}\hlstd{],} \hlkwc{col} \hlstd{=} \hlstr{"#D55E00"}\hlstd{)}

\hlcom{# Compare to logistic regression:}
\hlstd{mod_logreg} \hlkwb{=} \hlkwd{glm}\hlstd{(classes} \hlopt{~} \hlstd{.,} \hlkwc{data} \hlstd{= data,} \hlkwc{family} \hlstd{=} \hlkwd{binomial}\hlstd{())}
\hlkwd{abline}\hlstd{(}\hlkwc{a} \hlstd{=} \hlopt{-} \hlkwd{coef}\hlstd{(mod_logreg)[}\hlnum{1}\hlstd{]} \hlopt{/} \hlkwd{coef}\hlstd{(mod_logreg)[}\hlnum{2}\hlstd{],}
       \hlkwc{b} \hlstd{=} \hlopt{-} \hlkwd{coef}\hlstd{(mod_logreg)[}\hlnum{2}\hlstd{]} \hlopt{/} \hlkwd{coef}\hlstd{(mod_logreg)[}\hlnum{3}\hlstd{],} \hlkwc{col} \hlstd{=}  \hlstr{"#56B4E9"}\hlstd{,}
       \hlkwc{lty} \hlstd{=} \hlnum{3}\hlstd{,} \hlkwc{lwd} \hlstd{=} \hlnum{2}\hlstd{)}

\hlcom{# decision values}
\hlstd{f_pegasos} \hlkwb{=} \hlkwd{cbind}\hlstd{(}\hlnum{1}\hlstd{,X)} \hlopt \hlstd{mod_pegasos}

\hlcom{# How many wrong classified examples?}
\hlkwd{table}\hlstd{(}\hlkwd{sign}\hlstd{(f_pegasos} \hlopt{*} \hlstd{y))}
\end{alltt}
\begin{verbatim}
## 
## -1  1 
##  5 95
\end{verbatim}
\begin{alltt}
\hlcom{## compare to kernlab. we CANNOT expect a PERFECT match}
\hlcom{## -------------------------------------------------------------}

\hlstd{mod_kernlab} \hlkwb{=} \hlkwd{ksvm}\hlstd{(classes}\hlopt{~}\hlstd{.,}
                   \hlkwc{data} \hlstd{= data,}
                   \hlkwc{kernel} \hlstd{=} \hlstr{"vanilladot"}\hlstd{,}
                   \hlkwc{C} \hlstd{= C,}
                   \hlkwc{kpar} \hlstd{=} \hlkwd{list}\hlstd{(),}
                   \hlkwc{scaled} \hlstd{=} \hlnum{FALSE}\hlstd{)}
\hlstd{f_kernlab} \hlkwb{=} \hlkwd{predict}\hlstd{(mod_kernlab,} \hlkwc{newdata} \hlstd{= data,} \hlkwc{type} \hlstd{=} \hlstr{"decision"}\hlstd{)}
\hlcom{# How many wrong classified examples?}
\hlkwd{table}\hlstd{(}\hlkwd{sign}\hlstd{(f_kernlab} \hlopt{*} \hlstd{y))}
\end{alltt}
\begin{verbatim}
## 
## -1  1 
##  5 95
\end{verbatim}
\begin{alltt}
\hlcom{# compare outputs}
\hlkwd{print}\hlstd{(}\hlkwd{range}\hlstd{(}\hlkwd{abs}\hlstd{(f_kernlab} \hlopt{-} \hlstd{f_pegasos)))}
\end{alltt}
\begin{verbatim}
## [1] 0.00014996 0.38049736
\end{verbatim}
\begin{alltt}
\hlcom{# compare coeffs}
\hlkwd{rbind}\hlstd{(}
  \hlstd{mod_pegasos,}
  \hlkwc{mod_kernlab} \hlstd{=} \hlkwd{c}\hlstd{(mod_kernlab}\hlopt{@}\hlkwc{b}\hlstd{,}
  \hlstd{(params} \hlkwb{<-} \hlkwd{colSums}\hlstd{(X[mod_kernlab}\hlopt{@}\hlkwc{SVindex}\hlstd{, ]} \hlopt{*}
                       \hlstd{mod_kernlab}\hlopt{@}\hlkwc{alpha}\hlstd{[[}\hlnum{1}\hlstd{]]} \hlopt{*}
                       \hlstd{y[mod_kernlab}\hlopt{@}\hlkwc{SVindex}\hlstd{])))}
\hlstd{)}
\end{alltt}
\begin{verbatim}
##                               x.1        x.2
## mod_pegasos -0.05743352 -1.347267 -0.7917586
## mod_kernlab  0.09763532 -1.263707 -0.7747026
\end{verbatim}
\begin{alltt}
\hlcom{# seems we were reasonably close}

\hlcom{# recompute margin}
\hlstd{margin} \hlkwb{=} \hlnum{1} \hlopt{/} \hlkwd{sqrt}\hlstd{(}\hlkwd{sum}\hlstd{(params}\hlopt{^}\hlnum{2}\hlstd{))}

\hlcom{# compute value of intercept shift (the margin shift is in orthogonal direction}
\hlcom{#   to the decision boundary, so this has to be transformed first)}
\hlstd{m} \hlkwb{=} \hlopt{-} \hlstd{params[}\hlnum{1}\hlstd{]} \hlopt{/} \hlstd{params[}\hlnum{2}\hlstd{]}
\hlstd{t_0} \hlkwb{=} \hlstd{margin} \hlopt{*} \hlstd{m} \hlopt{/} \hlstd{(}\hlkwd{cos}\hlstd{(}\hlkwd{atan}\hlstd{(}\hlnum{1}\hlopt{/}\hlstd{m)))}

\hlcom{# add margins to visualization:}
\hlkwd{abline}\hlstd{(}\hlkwc{a} \hlstd{=} \hlopt{-} \hlstd{mod_kernlab}\hlopt{@}\hlkwc{b} \hlopt{/} \hlstd{params[}\hlnum{1}\hlstd{],}
       \hlkwc{b} \hlstd{= m,} \hlkwc{col} \hlstd{=} \hlstr{"#0072B2"}\hlstd{)}
\hlkwd{abline}\hlstd{(}\hlkwc{a} \hlstd{=} \hlopt{-} \hlstd{mod_kernlab}\hlopt{@}\hlkwc{b} \hlopt{/} \hlstd{params[}\hlnum{1}\hlstd{]} \hlopt{+} \hlstd{t_0,}
       \hlkwc{b} \hlstd{= m,} \hlkwc{col} \hlstd{=} \hlstr{"#0072B2"}\hlstd{,} \hlkwc{lty} \hlstd{=} \hlnum{2}\hlstd{)}
\hlkwd{abline}\hlstd{(}\hlkwc{a} \hlstd{=} \hlopt{-} \hlstd{mod_kernlab}\hlopt{@}\hlkwc{b} \hlopt{/} \hlstd{params[}\hlnum{1}\hlstd{]} \hlopt{-} \hlstd{t_0,}
       \hlkwc{b} \hlstd{= m,} \hlkwc{col} \hlstd{=} \hlstr{"#0072B2"}\hlstd{,} \hlkwc{lty} \hlstd{=} \hlnum{2}\hlstd{)}


\hlcom{# add legends}
\hlkwd{legend}\hlstd{(}\hlkwd{par}\hlstd{(}\hlstr{'usr'}\hlstd{)[}\hlnum{2}\hlstd{],} \hlkwd{par}\hlstd{(}\hlstr{'usr'}\hlstd{)[}\hlnum{4}\hlstd{], ,} \hlkwc{bty}\hlstd{=}\hlstr{'n'}\hlstd{,} \hlkwc{xpd}\hlstd{=}\hlnum{NA}\hlstd{,} \hlkwc{legend}\hlstd{=}\hlkwd{c}\hlstd{(}\hlstr{"1"}\hlstd{,}\hlstr{"2"}\hlstd{),}
       \hlkwc{pch}\hlstd{=}\hlkwd{c}\hlstd{(}\hlstr{"-"}\hlstd{,}\hlstr{"+"}\hlstd{),} \hlkwc{title}\hlstd{=}\hlstr{"Classes"}\hlstd{,} \hlkwc{cex} \hlstd{=} \hlnum{0.8}\hlstd{)}
\hlkwd{legend}\hlstd{(}\hlkwd{par}\hlstd{(}\hlstr{'usr'}\hlstd{)[}\hlnum{2}\hlstd{],} \hlnum{1.8}\hlstd{, ,} \hlkwc{bty}\hlstd{=}\hlstr{'n'}\hlstd{,} \hlkwc{xpd}\hlstd{=}\hlnum{NA}\hlstd{,}
       \hlkwc{legend}\hlstd{=}\hlkwd{c}\hlstd{(}\hlstr{"Pegasos"}\hlstd{,}\hlstr{"Logistic"}\hlstd{,}\hlstr{"Kernlab"}\hlstd{,}\hlstr{"Margin"}\hlstd{),}
       \hlkwc{lty}\hlstd{=}\hlkwd{c}\hlstd{(}\hlnum{1}\hlstd{,}\hlnum{3}\hlstd{,}\hlnum{1}\hlstd{,}\hlnum{2}\hlstd{),}
       \hlkwc{col} \hlstd{=} \hlkwd{c}\hlstd{(}\hlstr{"#D55E00"}\hlstd{,}\hlstr{"#56B4E9"}\hlstd{,}\hlstr{"#0072B2"}\hlstd{,}\hlstr{"#0072B2"}\hlstd{),}
       \hlkwc{title}\hlstd{=}\hlstr{""}\hlstd{,} \hlkwc{cex} \hlstd{=} \hlnum{0.8}\hlstd{,} \hlkwc{lwd} \hlstd{=} \hlkwd{c}\hlstd{(}\hlnum{1}\hlstd{,}\hlnum{2}\hlstd{,}\hlnum{1}\hlstd{,}\hlnum{1}\hlstd{))}
\end{alltt}
\end{kframe}
\includegraphics[width=\maxwidth]{figure/unnamed-chunk-7-1} 
\end{knitrout}

\end{itemize}
}
\dlz

\loesung{}{

The polynomial kernel is defined as
$$
k(x, \tilde{x}) = (x^T\tilde{x} + b)^d.
$$
Furthermore, assume $x \in \mathbb{R}^2$ and $d = 2$.

\begin{enumerate}

  \item
    Derive the explicit feature map $\phi$ taking into account that the following equation holds:
    $$
    k(x, \tilde{x}) = \langle \phi(x), \phi(\tilde{x}) \rangle
    $$

    \textbf{Solution:} \\

    \begin{align*}
    k(x, \tilde{x}) &= \left( \left(\begin{array}{c} x_1 \\ x_2 \end{array}\right)^T\left(\begin{array}{c} \tilde{x}_1 \\ \tilde{x}_2 \end{array}\right) + b \right)^2 \\
                    &= \left( x_1\tilde{x}_1 + x_2\tilde{x}_2 + b \right)^2 \\
                    &= (x_1\tilde{x}_1 + x_2\tilde{x}_2)^2 + 2 (x_1\tilde{x}_1 + x_2\tilde{x}_2) b + b^2 \\
                    &= x_1^2 \tilde{x}_1^2 + 2 x_1 \tilde{x}_1 x_2 \tilde{x}_2 + x_2^2 \tilde{x}_2^2 + 2 b x_1 \tilde{x}_1 + 2 b x_2 \tilde{x}_2 + b^2 \\
                    &= \left\langle
                      \left(\begin{array}{c}
                      x_1^2 \\ \sqrt{2}x_1x_2 \\ x_2^2 \\ \sqrt{2b}x_1 \\ \sqrt{2b}x_2 \\ b
                      \end{array}\right),
                      \left(\begin{array}{c}
                      \tilde{x}_1^2 \\ \sqrt{2}\tilde{x}_1\tilde{x}_2 \\ \tilde{x}_2^2 \\ \sqrt{2b}\tilde{x}_1 \\ \sqrt{2b}\tilde{x}_2 \\ b
                      \end{array}\right)
                    \right\rangle \\
                    &= \langle \phi(x), \phi(\tilde{x}) \rangle
    \end{align*}

  \item
    Describe the main differences between the kernel method and the explicit feature map. \\

    \textbf{Solution:} \\

    Using the kernel method reduces the computational costs of computing the scalar product in the higher-dimensional features space after calculating the feature map.

\end{enumerate}
}
\dlz

\loesung{}{


\begin{enumerate}
\item Prior distribution (assuming the same notation as in the lecture): $$\bm{f} \sim \mathcal{N}(\bm{m}, \bm{K})$$ with $\bm{m} = m(\bm{x})$ and $\bm{K}$ defined by the entries $\bm{K}_{ij} = k(x_i,x_j)$. NB: Note the (in-)finite Gaussian property of a GP. 
\item Note that the posterior distribution $\bm{f}|\bm{y},\bm{x}$ in this case is different from the one of $\bm{f}_* | \bm{x}_*, \bm{x}, \bm{y}$ and also from the marginal distribution of $\bm{y} \sim \mathcal{N}(\bm{m},\bm{K} + \sigma^2 \bm{I})$! We have: 
\begin{equation}
\begin{split}
p(\bm{f}|\bm{y}) &\propto p(\bm{y}|\bm{f}) \cdot p(\bm{f}) \\
&\propto \exp(-\frac{1}{2} (\bm{y}-\bm{f})^\top (\sigma^{2} \bm{I})^{-1}(\bm{y}-\bm{f})) \cdot \exp(-\frac{1}{2}(\bm{f}-\bm{m})^\top \bm{K}^{-1} (\bm{f}-\bm{m}))\\
&\propto \exp(-\frac{1}{2} \{ \bm{f}^\top \underbrace{((\sigma^{2} \bm{I})^{-1} + \bm{K}^{-1})}_{=: \bm{K}^{-1}_{post}} \bm{f} -2 \bm{f}^\top \underbrace{((\sigma^{2} \bm{I})^{-1} \bm{y} + \bm{K}^{-1} \bm{m})}_{=: \tilde{\bm{f}}} \})\\
&\propto \exp(-\frac{1}{2} \{ \bm{f}^\top \bm{K}^{-1}_{post} \bm{f} - 2 \bm{f}^\top\tilde{\bm{f}}  \})
\end{split}
\end{equation}
by removing all constant factors that do not depend on $\bm{f}$ as we only need to know the density up to a constant of proportionality. By extending the proportionality, we can get a quadratic form in $\bm{f}$:
\begin{equation}
\begin{split}
p(\bm{f}|\bm{y}) &\propto \exp(-\frac{1}{2} \{ \bm{f}^\top \bm{K}^{-1}_{post} \bm{f} - 2 \bm{f}^\top\tilde{\bm{f}}  \})\\
&\propto \exp(-\frac{1}{2} \{ \bm{f}^\top \bm{K}^{-1}_{post} \bm{f} - 2 \bm{f}^\top \bm{K}^{-1}_{post} \underbrace{\bm{K}_{post} \tilde{\bm{f}}}_{:=\bm{f}_{post}}  \}) \\
&\propto \exp(-\frac{1}{2}  (\bm{f}-\bm{f}_{post})^\top \bm{K}^{-1}_{post} (\bm{f}-\bm{f}_{post}))
\end{split}
\end{equation}
which is the so-called \emph{kernel} of a multivariate normal distribution $\mathcal{N}(\bm{f}_{post},\bm{K}_{post} )$, i.e., $\bm{f}|\bm{y} \sim \mathcal{N}(\bm{f}_{post},\bm{K}_{post} )$. 
\item In order to get the posterior predictive distribution for a new sample $x_*$ from the same data-generating process, we could derive $$p(y_* | x_*, \bm{y}, \bm{x}) = \int p(y_*|x_*, \bm{x}, \bm{y}, \bm{f}) \cdot p(\bm{f}|\bm{y},\bm{x}) \,d\bm{f}.$$ This is feasible but cumbersome. Alternatively, we can make use of the fact that the joint distribution of $\bm{y}$ and $y_*$ is known (cf. slides on noisy GP): 
$$\begin{pmatrix} \bm{y} \\ y_* \end{pmatrix} \sim \mathcal{N}\left( \begin{pmatrix} \bm{m} \\ m_* \end{pmatrix}, 
\begin{pmatrix} 
\bm{K} + \sigma^2 \bm{I} & \bm{K}_*\\
\bm{K}^\top_* & K_{**}\\
\end{pmatrix}
\right),$$
with $m_* = m(x_*)$, $\bm{K}_* = k(x_*, \bm{x})$ and $K_{**} = k(x_*,x_*)$.
The conditional distribution can then be derived using the rule of conditioning for Gaussian distributions: $$y_* | x_*, \bm{x}, \bm{y} \sim \mathcal{N}(m_* + \bm{K}^\top_* (\bm{K} + \sigma^2 \bm{I})^{-1}(\bm{y}-\bm{m}), K_{**} - \bm{K}^\top_* (\bm{K} + \sigma^2 \bm{I})^{-1} \bm{K}_*).$$
\item To implement a GP with squared exponential kernel and $\ls = 1$, we need the inverse of $\bm{K}$. $\bm{x}$ being a vector implies that we have only one feature and thus the entries of our matrix $\bm{K}$ are 
$$
\bm{K} = \begin{pmatrix} 1 & \exp(-0.5 (x^{(1)} - x^{(2)})^2) \\ \exp(-0.5 (x^{(2)} - x^{(1)})^2) & 1 \end{pmatrix}.
$$
The inverse of $\bm{K}$ is then given by $$
\frac{1}{1-\exp(-(x^{(1)} - x^{(2)})^2)} \begin{pmatrix} 1 & -\exp(-0.5 (x^{(1)} - x^{(2)})^2) \\ -\exp(-0.5 (x^{(2)} - x^{(1)})^2) & 1 \end{pmatrix}.
$$
If we have a noisy GP, we would have to add $\sigma^2 \bm{I}_2$ to $\bm{K}$ with resulting inverse 

$$
\bm{K}_y^{-1} = \frac{1}{(1+\sigma^2)^2-\exp(-(x^{(1)} - x^{(2)})^2)} \begin{pmatrix} 1+\sigma^2 & -\exp(-0.5 (x^{(1)} - x^{(2)})^2) \\ -\exp(-0.5 (x^{(2)} - x^{(1)})^2) & 1+\sigma^2 \end{pmatrix}.
$$
Assuming a zero mean GP, we can derive $\frac{\partial \bm{K}_y}{\partial \theta}$ with $\theta = \sigma^2$, which gives us the identity matrix. We can thus maximize the marginal likelihood (slide on \emph{Gaussian Process Training}), by finding $\sigma^2$ that yields 
$$\text{tr}\left( \bm{K}_y^{-1} \bm{y} \bm{y}^\top \bm{K}_y^{-1} - \bm{K}_y^{-1} \right) = 0.$$
This can be solved analytically (though quite tedious). We will use a root-finding function for this. For the posterior predictive distribution we can make use of the results from the previous exercise.

\begin{knitrout}
\definecolor{shadecolor}{rgb}{0.969, 0.969, 0.969}\color{fgcolor}\begin{kframe}
\begin{alltt}
\hlkwd{library}\hlstd{(kernlab)}

\hlcom{# set seed, define n, true (unknown) sigma}
\hlkwd{set.seed}\hlstd{(}\hlnum{4212}\hlstd{)}
\hlstd{n} \hlkwb{<-} \hlnum{2}
\hlstd{sigma} \hlkwb{<-} \hlnum{1}

\hlcom{# define kernel with l = 1}
\hlstd{kernel_fun} \hlkwb{<-} \hlkwa{function}\hlstd{(}\hlkwc{x}\hlstd{)}
  \hlkwd{kernelMatrix}\hlstd{(}\hlkwc{kernel} \hlstd{=} \hlkwd{rbfdot}\hlstd{(}\hlkwc{sigma} \hlstd{=} \hlnum{1}\hlopt{/}\hlnum{2}\hlstd{),}
               \hlkwc{x} \hlstd{= x)}
\hlstd{kernel_fun_pred} \hlkwb{<-} \hlkwa{function}\hlstd{(}\hlkwc{x}\hlstd{,}\hlkwc{y}\hlstd{)}
  \hlkwd{kernelMatrix}\hlstd{(}\hlkwc{kernel} \hlstd{=} \hlkwd{rbfdot}\hlstd{(}\hlkwc{sigma} \hlstd{=} \hlnum{1}\hlopt{/}\hlnum{2}\hlstd{),}
               \hlkwc{x} \hlstd{= x,} \hlkwc{y} \hlstd{= y)}

\hlcom{# draw data according to the generating process:}
\hlstd{x} \hlkwb{<-} \hlkwd{rnorm}\hlstd{(n)}
\hlstd{K} \hlkwb{<-} \hlkwd{kernel_fun}\hlstd{(x)}
\hlstd{K_y} \hlkwb{<-} \hlstd{K} \hlopt{+} \hlkwd{diag}\hlstd{(}\hlkwd{rep}\hlstd{(sigma}\hlopt{^}\hlnum{2}\hlstd{,}\hlnum{2}\hlstd{))}
\hlstd{(y} \hlkwb{<-} \hlkwd{t}\hlstd{(mvtnorm}\hlopt{::}\hlkwd{rmvnorm}\hlstd{(}\hlnum{1}\hlstd{,} \hlkwc{sigma} \hlstd{= K_y)))}
\end{alltt}
\begin{verbatim}
##          [,1]
## [1,] 2.012317
## [2,] 1.866819
\end{verbatim}
\begin{alltt}
\hlcom{# function to find the best sigma^2}
\hlstd{root_fun} \hlkwb{<-} \hlkwa{function}\hlstd{(}\hlkwc{sigmaSq}\hlstd{)\{}
  \hlstd{K_y_inv} \hlkwb{<-} \hlkwd{solve}\hlstd{(K} \hlopt{+} \hlkwd{diag}\hlstd{(}\hlkwd{rep}\hlstd{(sigmaSq,}\hlnum{2}\hlstd{)))}
  \hlnum{0.5}\hlopt{*}\hlkwd{sum}\hlstd{(}\hlkwd{diag}\hlstd{(K_y_inv}\hlopt\hlstd{y}\hlopt\hlkwd{t}\hlstd{(y)}\hlopt\hlstd{K_y_inv} \hlopt{-} \hlstd{K_y_inv))}
\hlstd{\}}

\hlcom{# get the best sigma}
\hlstd{(bestSigmaSq} \hlkwb{<-} \hlkwd{uniroot}\hlstd{(}\hlkwc{f} \hlstd{= root_fun,} \hlkwc{interval} \hlstd{=} \hlkwd{c}\hlstd{(}\hlnum{0}\hlstd{,}\hlnum{20}\hlstd{)))}\hlopt{$}\hlstd{root}
\end{alltt}
\begin{verbatim}
## [1] 1.943684
\end{verbatim}
\begin{alltt}
\hlcom{# plot the optimization problem and best sigma}
\hlstd{possible_sigvals} \hlkwb{<-} \hlkwd{seq}\hlstd{(}\hlnum{0.001}\hlstd{,}\hlnum{20}\hlstd{,}\hlkwc{l}\hlstd{=}\hlnum{1000}\hlstd{)}
\hlkwd{plot}\hlstd{(possible_sigvals,} \hlkwd{sapply}\hlstd{(possible_sigvals, root_fun),}
     \hlkwc{xlab} \hlstd{=} \hlkwd{expression}\hlstd{(sigma}\hlopt{^}\hlnum{2}\hlstd{),} \hlkwc{ylab} \hlstd{=} \hlstr{"marginal likelihood derivative"}\hlstd{,}
     \hlkwc{pch} \hlstd{=} \hlnum{20}\hlstd{)}
\hlkwd{abline}\hlstd{(}\hlkwc{h}\hlstd{=}\hlnum{0}\hlstd{,} \hlkwc{lty}\hlstd{=}\hlnum{2}\hlstd{)}
\hlkwd{abline}\hlstd{(}\hlkwc{v}\hlstd{=bestSigmaSq}\hlopt{$}\hlstd{root,} \hlkwc{lty}\hlstd{=}\hlnum{2}\hlstd{)}
\end{alltt}
\end{kframe}
\includegraphics[width=\maxwidth]{figure/unnamed-chunk-8-1} 
\begin{kframe}\begin{alltt}
\hlcom{# function to draw samples from the predictive posterior}
\hlstd{draw_from_pred_posterior} \hlkwb{<-} \hlkwa{function}\hlstd{(}\hlkwc{number_samples}\hlstd{,} \hlkwc{y}\hlstd{,} \hlkwc{x}\hlstd{,} \hlkwc{xstar}\hlstd{,} \hlkwc{sigmaSq} \hlstd{=} \hlnum{1}\hlstd{)}
\hlstd{\{}

  \hlcom{# invert noisy K}
  \hlstd{K_y_inv} \hlkwb{<-} \hlkwd{solve}\hlstd{(}\hlkwd{kernel_fun}\hlstd{(x)} \hlopt{+} \hlkwd{diag}\hlstd{(}\hlkwd{rep}\hlstd{(sigmaSq,}\hlnum{2}\hlstd{)))}
  \hlcom{# get the other K's for new data}
  \hlstd{Kstar} \hlkwb{<-} \hlkwd{kernel_fun_pred}\hlstd{(x,xstar)}
  \hlstd{Kstarstar} \hlkwb{<-} \hlkwd{kernel_fun}\hlstd{(xstar)}
  \hlcom{# draw samples according to Ex. (d)}
  \hlkwd{rnorm}\hlstd{(number_samples,}
        \hlkwc{mean} \hlstd{=} \hlkwd{as.numeric}\hlstd{(}\hlkwd{t}\hlstd{(Kstar)} \hlopt \hlstd{K_y_inv} \hlopt \hlstd{y),}
        \hlkwc{sd} \hlstd{=} \hlkwd{sqrt}\hlstd{(}\hlkwd{as.numeric}\hlstd{(Kstarstar} \hlopt{-} \hlkwd{t}\hlstd{(Kstar)} \hlopt \hlstd{K_y_inv} \hlopt \hlstd{Kstar))}
  \hlstd{)}

\hlstd{\}}

\hlcom{# draw enough samples to get a feeling for the distribution}
\hlstd{samples_posterior} \hlkwb{<-}
       \hlkwd{draw_from_pred_posterior}\hlstd{(}\hlkwc{number_samples} \hlstd{=} \hlnum{1000}\hlstd{,} \hlkwc{sigmaSq} \hlstd{= bestSigmaSq}\hlopt{$}\hlstd{root,}
                                \hlkwc{y} \hlstd{= y,} \hlkwc{x} \hlstd{= x,} \hlkwc{xstar} \hlstd{=} \hlnum{0}\hlstd{)}
\hlcom{# plot the distribution}
\hlkwd{hist}\hlstd{(samples_posterior,} \hlkwc{breaks}\hlstd{=}\hlnum{50}\hlstd{,} \hlkwc{xlab}\hlstd{=}\hlkwd{expression}\hlstd{(y[}\hlstr{"*"}\hlstd{]}\hlopt{^}\hlstd{b))}
\end{alltt}
\end{kframe}
\includegraphics[width=\maxwidth]{figure/unnamed-chunk-8-2} 
\end{knitrout}

\end{enumerate}
}
\end{document}
